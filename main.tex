\documentclass{book}

\title{CLAST}
\author{Matteo Crespi 748274}
\date{Agosto 2016}

% IMPORT

\usepackage[italian]{babel}
\usepackage[utf8]{inputenc}
\usepackage{csquotes}
\usepackage{graphicx}
\usepackage{listings}
\usepackage{setspace}
\usepackage{color}
\usepackage{float}
\usepackage{url}

% COMMAND

\newcommand{\image}[4]{
  \begin{figure}[H]
  \makebox[\textwidth][c]{
    \includegraphics[scale=#4]{#1}
  }
  \caption{#2}
  \label{#3}
  \end{figure}
}

% LISTING

\renewcommand{\lstlistlistingname}{Elenco dei listati}
\renewcommand{\lstlistingname}{Listato}

\definecolor{gray}{rgb}{0.5,0.5,0.5}

\lstset{
  basicstyle=\ttfamily,
  commentstyle=\color{gray},
  captionpos=b,
}

% FORMATTING

\setstretch{1.3}
\setlength{\oddsidemargin}{36.1pt}
\setlength{\evensidemargin}{0pt}
\setlength{\marginparwidth}{40pt}
\marginparsep 10pt
\topmargin 0pt \headsep .5in
\textheight 8.1in \textwidth 6in
% \brokenpenalty=10000
\linespread{1.3}

% BIBLIOGRAPHY

\usepackage[
  backend=biber,
  sorting=ynt,
  style=ieee-alphabetic,
  defernumbers=true
]{biblatex}
\addbibresource{bib/papers.bib}
\addbibresource{bib/books.bib}
\addbibresource{bib/websites.bib}

% DOCUMENT

\begin{document}

\maketitle

\pagenumbering{roman}
\pagestyle{plain}

\newpage

\tableofcontents
\listoffigures
\lstlistoflistings

\cleardoublepage

\pagenumbering{arabic}
\pagestyle{headings}

\chapter{Introduzione}

% La source code analysis rappresenta il passo preliminare necessario per lo
% svolgimento di attività che si concentrano sulla stima, sul monitoraggio e sul
% miglioramento della qualità di un sistema software. Attività come reverse
% engineering, re-engineering, testing o costruzione di modelli empirici per la
% valutazione della qualità del software prevedono tutte una fase preliminare in
% cui vengono estratte informazioni dal codice sorgente, o per la costruzione di
% un AST che consenta una più semplice interazione con il codice di un programma
% o per la costruzione di una rappresentazione analoga che funga da meccanismo
% per la raccolta e l’accesso alle informazioni espresse dal codice sorgente del
% sistema in analisi.

% Nell’ultimo decennio sono stati sviluppati diversi tipi di linguaggi e toolkits
% per la source code analysis. Alcuni di questi sono particolarmente adatti alla
% comprensione e trasformazione di un sistema <citazione> <citazione> <citazione>
% ; gli strumenti di questo tipo hanno delle potenti capacità dal punto di vista
% dell’analisi, consentono ad esempio di fornire linguaggi per il pattern
% matching o un modo per compiere ricerche all’interno di un AST o operare
% trasformazioni di questo. Altri strumenti, come ad esempio <citazione>
% <citazione> <citazione>, sono invece più orientati verso l’analisi statica e
% verso il calcolo di metriche di valutazione della qualità piuttosto che verso
% la capacità di operare trasformazioni. Ciascuno di questi strumenti per la
% stima, il monitoraggio e l’estrazione di metriche rispetto alla qualità del
% codice sorgente di un programma richiede però, come accennato in precedenza,
% che sia disponibile una qualche tecnologia che consenta il parsing.

Questa tesi presenta le attività svolte durante la progettazione ed
implementazione di una libreria per la generazione di abstract syntax tree nel
contesto di un linguaggio di programmazione ad ambiente dinamico ed estensibile,
il linguaggio Common Lisp, e lo studio delle applicazioni di questa.
Applicazioni come fondamento per la creazione di strumenti di source code
analysis ma soprattutto come infrastruttura in grado di abilitare la definizione
di estensioni del linguaggio stesso che introducano un supporto a funzionalità
anche molto significative.\\

Le principale motivazione alla base della scelta di questo progetto sono legate
alla mancanza di uno strumento analogo nel contesto del linguaggio Common Lisp.
Questo nonostante le applicazioni degli AST siano diverse e documentate nel
contesto di molti altri linguaggi di programmazione. Una seconda motivazione,
non meno significativa, è stata rappresentato dalla volontà di definire un
procedimento che consenta di estendere un linguaggio di programmazione grazie
all’impiego di tecniche di meta programmazione applicate a livello di codice
sorgente.\\

Il principale ambito di ricerca nel quale si colloca questa tesi è quello della
Source Code Analysis and Manipulation, spesso abbreviato con la sigla SCAM nella
letteratura, branca del più ampio settore dell’Ingegneria del Software dedicata
allo studio del processo di estrazione di informazioni relative ad un programma
e della manipolazione di questo a partire dal suo codice sorgente. La tesi
affronta infatti tematiche come la rappresentazione di codice sorgente scritto
utilizzando un particolare linguaggio di programmazione e la definizione di
tecniche di riscrittura del codice sorgente a partire dall’analisi sintattica di
questo. Altri ambiti di ricerca toccati dalla tesi sono quelli della teoria dei
linguaggi di programmazione e dei compilatori.\\

Questo documento si articola in tre macro capitoli: un primo capitolo relativo
ai concetti fondamentali a partire dai quali il lavoro di tesi si articola, una
secondo capitolo che presenta invece la libreria CLAST, soggetto del lavoro di
tesi, ed un terzo capitolo, il cui obiettivo è quello di presentare alcune
applicazioni della libreria CLAST, mostrando ciò che questa consente, sia dal
punto di vista della creazione di strumenti pratici, sia dal punto di vista del
linguaggio di programmazione Common Lisp.\\

La libreria CLAST si pone l’obiettivo si consentire e semplificare la
costruzione di strumenti di source code analysis. Il Capitolo 2 si apre quindi
presentando l’architettura tradizionale di uno strumento di questo tipo e
mostrando come la libreria si integri in questo contesto.

Il capitolo prosegue quindi presentando il concetto di abstract syntax tree, la
struttura dati fondamentale a partire dalla quale la libreria compie il proprio
lavoro di rappresentazione di un programma, e riportando come questo venga
impiegato sia nell’ambito della ricerca, sia nell’ambito dell’industria, aspetti
che consentono di mostrare come uno strumento di generazione di abstract syntax
tree possa risultare abilitante in diversi contesti.

In seguito, viene illustrato il concetto di macro Common Lisp, costrutto che
risulta fondamentale sia dal punto di vista del funzionamento della libreria,
sia dal punto di vista delle applicazioni di questa. Fondamentale rispetto al
funzionamento, in quanto aspetto del linguaggio di programmazione che produce
significative differenze nel modus operandi di uno strumento di source code
analysis come CLAST, specifico per il linguaggio Common Lisp, ed uno strumento
che invece si riferisce a linguaggi di programmazione tradizionali. Fondamentale
rispetto alle applicazioni in quanto costrutto che consente la definizione delle
operazioni di riscrittura del codice che queste sfruttano in maniera molto
importante.

Il capitolo si chiude presentando il concetto di ambiente Common Lisp, strumento
che consente alla libreria CLAST di operare gran parte delle attività di analisi
statica operate da questa, e illustrando come, a differenza di quanto avviene
nella grande maggioranza dei linguaggi di programmazione, questo sia modellato
direttamente da un oggetto specifico, reso accessibile agli utenti del
linguaggio attraverso una specifica interfaccia.\\

Il Capitolo 3 entra invece nel dettaglio rispetto alle attività svolte rispetto
alle attività svolte e alle scelte compiute durante la realizzazione della
libreria CLAST. In particolare, si presenta come questa consenta di
rappresentare un programma attraverso un insieme di strutture definite ad hoc
per il linguaggio target dell’analisi svolta dalla libreria.

In seguito, il capitolo presenta le attività di parsing e analisi operate dalla
libreria, attività che compongono il processo a partire dal quale viene prodotta
la rappresentazione in forma di abstract syntax tree di un programma e che
rappresentano il cuore del funzionamento di CLAST.\\

Infine, il Capitolo 4 discute le potenziali applicazioni di CLAST attraverso
l’analisi di tre diversi casi di studio. Il primo caso di studio approfondisce
come la libreria possa essere utilizzata nel contesto della creazione di uno
strumento di source code analysis, applicazione più facilmente auspicabile,
facendo riferimento all’esempio rappresentato da uno strumento di free variables
analysis.

I successivi casi di studi approfondiscono invece l’applicazione della libreria
CLAST come fondamento per la definizione di estensioni del linguaggio Common
Lisp stesso. In questo contesto viene presentato come, combinando processi di
riscrittura del codice alla capacità di analisi a livello di codice sorgente
consentita dalla libreria, sia possibile introdurre un supporto all’utilizzo di
costrutti di pattern matching e di un modello di valutazione lazy all’interno
del linguaggio. Questo mantenendo caratteristiche di performance analoghe
rispetto ai costrutti tradizionali del linguaggio di programmazione, senza dover
agire a livello compilatore e quindi garantendo portabilità tra le diverse
implementazioni del linguaggio e agendo in maniera del tutto trasparente
rispetto ad un utente del linguaggio.

\begingroup
\let\clearpage\relax

\chapter{Fondamenti}

Lo scopo di questo capitolo è quello di introdurre il lettore all’ambito di
ricerca all’interno del quale questa tesi si colloca, e quindi in particolare
all’ambito della source code analysis, e ai concetti fondamentali a partire dai
quali si articola il progetto soggetto del lavoro di tesi.\\

La \textit{Source Code Analysis} studia il processo di estrazione di un
informazioni riguardo un programma a partire dal codice sorgente di questo,
un’attività di particolare importanza all'interno del più ampio campo
dell'ingegneria del software.

In particolare, la libreria CLAST, soggetto di questa tesi, si pone come
obiettivo quello di abilitare la creazione di strumenti per la source code
analysis nel contesto rappresentato dal linguaggio di programmazione Common
Lisp.

All'interno della Sezione \ref{source-code-analysis}, vengono quindi presentate
l'architettura tipica di uno strumento di source code analysis e le principali
strategie per la creazione di strumenti di questo tipo, indicando e discutendo
come la libreria CLAST si inserisca all'interno di questa architettura e come
possa semplificare la progettazione e costruzione di nuovi strumenti.\\

Le sezioni successive, come precedentemente indicato, affrontano invece i
concetti fondamentali a partire dai quali la libreria CLAST opera e che
risultano particolarmente significativi per la comprensione della progettazione
e realizzazione della libreria, aspetti che verranno approfonditi nel corso del
Capitolo \ref{library}.

Tra questi concetti vengono quindi presenti quello di
\textit{abstract syntax tree} e quelli di \textit{macro} e \textit{ambiente}
Common Lisp.\\

La Sezione \ref{abstract-syntax-tree} illustra quindi il concetto di abstract
syntax tree, in quanto rappresenta la struttura dati fondamentale a partire
dalla quale la libreria svolge il proprio lavoro di modellazione di un programma
Common Lisp. Essendo inoltre gli abstract syntax tree l'output dell'azione della
libreria, vengono presentate le principali applicazioni di strutture dati di
questo tipo, sia nella letteratura e nel mondo della ricerca sia nel mondo
dell'industria. Questo allo scopo di presentare al lettore le possibilità che la
presenza di una libreria come CLAST introduce dal punto di vista della
possibilità di creare nuovi strumenti per la source code analysis.

La Sezione \ref{macro} presenta il concetto di macro Common Lisp. Le macro x
rappresentano infatti l'aspetto più caratteristico del linguaggio di
programmazione Common Lisp, ossia il target dell'analisi svolta dalla libreria.
Risultano inoltre un aspetto del linguaggio particolarmente significativo dal
punto di vista dell'analisi a livello di codice sorgente operata dalla libreria
e il principale strumento a partire dal quale lavorano le applicazioni della
libreria che verranno presentate nel Capitolo \ref{applications}.

Infine, la Sezione \ref{environments} introduce il concetto di ambiente Common
Lisp, strumento che, sia dal punto di vista della progettazione che dal punto di
vista implementativo, rappresenta la principale dipendenza della libreria CLAST,
in quanto fondamento a partire dal quale questa è in grado di agire, operando
un'analisi del codice fornitole in input.

\section{Source Code Analysis}

Data la complessità dei moderni sistemi software, la presenza di strumenti per
il supporto alla creazione e alla manutenzione di questi è fondamentale. Da
questa riflessione emergono le ragioni che portano all’interesse della comunità
informatica per strumenti di source code analysis. Strumenti di questo tipo
forniscono informazioni di estrema importanza agli sviluppatori di un sistema,
informazioni che possono essere utilizzate da questi per coordinare il proprio
lavoro e aumentare la produttività complessiva.\\

Nella letteratura, la source code analysis viene definita come il processo di
estrazione di un informazioni riguardo un programma a partire dal codice
sorgente di questo.

La precedente definizione fa riferimento a due diversi concetti, il concetto di
codice sorgente e il concetto di analisi.

Nel contesto della source code analysis, il termine codice sorgente viene
utilizzato, soprattutto nella letteratura, per indicare una descrizione del
comportamento di un programma in un formato testuale, statico, leggibile e
partire dalla quale è possibile produrre una versione del sistema software
completamente eseguibile. Questa definizione è costruita in maniera tale da
includere i codici macchina, linguaggi di alto livello e anche rappresentazioni
grafiche eseguibili di un sistema.

Il termine analysis viene invece utilizzato per indicare una qualsiasi procedura
automatica o semi-automatica che produce nuova conoscenza rispetto ad un sistema
a partire dal codice sorgente di questo. Infine, il termine “manipolazione"
viene utilizzato per indicare un qualsiasi procedura automatica o semi-
automatica che produce codice sorgente a partire da codice sorgente.
\cite{DBLP:journals/jss/DeanHKV06} \cite{DBLP:conf/icse/Binkley07}\\

Nella letteratura, la source code analysis viene spesso spesso accorpata al
settore che si occupa della manipolazione di codice sorgente, è viene utilizzata
la sigla SCAM, Source Code Analysis and Manipulation, per riferirsi a questo
particolare ambito di ricerca. \cite{DBLP:conf/scam/2001}\\

Come accennato in apertura di questa Sezione, la source code analysis ricopre un
ruolo di grande importanza nel contesto dell'ingegneria del software, tanto che
l'applicazione di tecniche di SCAM possono essere rintracciate in diverse fasi
del ciclo di vita di un sistema software.

Dal supporto per la creazione di un programma all’interno di un IDE, alla
compilazione e deployment, al supporto per la comprensione e correzione di
errori durante la manutenzione, in ciascuno di questi contesti vengono
utilizzate tecniche di Source Code Analysis.\\

Un altro aspetto che contribuisce a rendere importante questo ambito di ricerca
è rappresentato dalla significatività e onnipresenza del soggetto a cui questa
si riferisce: il codice sorgente.

Il progetto di un sistema software ha solitamente inizio con la costruzione di
un modello del funzionamento di questo, come ad esempio diagrammi prodotti
utilizzando il linguaggio di modellazione UML. Questi modelli consentono la
rappresentazione del sistema ad un livello di astrazione alto e gestibile in
modo semplice e veloce. Tuttavia, nelle fasi successive del processo questi
modelli dovranno necessariamente essere “compilati” utilizzando una
rappresentazione che vive ad un livello di astrazione inferiore: il codice
sorgente.

Sfortunatamente, una diretta implementazione dei modelli precederete citati
porta alla creazione di codice tipicamente incompleto e che per questa ragione
richiede aggiunte e modiche. Aggiunte e modifiche non anticipabili al momento
della costruzione dei modelli. Queste modifiche portano alla presenza di
incongruenze tra modelli e codice sorgente ed è proprio la presenza di queste
incongruenze a rendere il codice sorgente l’unica sorgente di informazioni
realmente attendibile rispetto all’effettivo funzionamento di un sistema
software.\\

Questa Sottosezione ha fornito una definizione di source code analysis, allo
scopo di indicare l'ambito di ricerca all'interno del quale lo strumento
discusso da questa tesi si colloca. Ha cercato inoltre di sottolineare
l'importanza di questo ambito di ricerca. Nella prossima Sottosezione, viene
approfondita la struttura di uno strumento di source code analysis, in maniera
tale da fornire al lettore una panoramica rispetto alla tipologia di strumenti
che CLAST consente di creare.

\subsection{Struttura di uno strumento di Source Code Analysis}

I prossimi paragrafi descrivono le tre componenti fondamentali di un
tradizionale sistema per la source code analysis. Questi componenti sono
rappresentate da un parser per il codice sorgente del sistema in analisi, una
struttura dati per la rappresentazione interna allo strumento del codice
sorgente parsato ed un meccanismo che operi la reale analisi del sistema a
partire dalla rappresentazione interna.\\

La prima componente, il parser, opera un'elaborazione del codice sorgente al
fine di produrre una o più rappresentazioni sistema che si desidera analizzare
utilizzando la struttura dati per la rappresentazione del sistema specifica
dello strumento. In sostanza, questo passo opera la conversione dalla sintassi
concreta, utilizzata dal codice sorgente vero e proprio, ad una sintassi
astratta, che meglio si presta alla particolare analisi che si desidera
operare.

La grande maggioranza degli strumenti per la source code analysis fa uso del
parser integrato al compilatore per il linguaggio di programmazione utilizzato
per la costruzione del sistema in analisi. Parser di questo tipo consentono
l’elaborazione dell’intero linguaggio di programmazione. Tuttavia, nel caso in
cui solamente un sottoinsieme della sintassi concreta di un linguaggio sia di
reale interesse per l’analisi, è anche possibile utilizzare tecniche più
semplici.

La definizione di un parser risulta molto spesso un'attività particolarmente
impegnativa nel processo di costruzione di uno strumento di source code
analysis. Nonostante si tratti di un procedimento semplice dal punto di vista
concettuale, in quanto consiste nella semplice lettura e trasformazione di un
testo, la complessità dei linguaggi di programmazione porta molto spesso a
significative difficoltà nella realizzazione di questo componente.\\

La seconda componente di uno strumento per la source code analysis è
rappresentata dalla struttura dati interna utilizzata per la modellazione del
sistema in analisi. La grande maggioranza delle tecniche di rappresentazione e
strutture utilizzate dagli strumenti di source code analysis prende ispirazione
dal o fa riferimento al mondo dei compilatori. Questo perché la quasi totalità
dei compilatori prevede una fase che si pone come obiettivo proprio quello che
il componente analizzato in questo paragrafo si pone: la trasformazione di un
programma in una rappresentazione maggiormente adatta all’analisi da parte di
uno strumento automatico. In particolare però, una trasformazione che conservi
le proprietà significative per il dominio di interesse dell'analisi ma che sia
maggiormente specifica al dominio di interesse.

Diverse tecniche e strutture per la rappresentazione interna ad uno strumento
vengono utilizzate da differenti strumenti di source code analysis. Alcuni
esempi classici di strutture per la rappresentazione del codice sorgente sono
control-flow graphs, call graphs e abstract syntax trees.

In alcuni casi la produzione di queste rappresentazioni interne può essere
compiuta direttamente dalla componente parser dello strumento di source code
analysis. In altri casi, specialmente in contesti più complessi, sono invece
necessarie ulteriori analisi al fine di produrre una rappresentazione definitiva
che possa effettivamente essere utilizzata dalla componente di analisi.\\

Infine, la terza ed ultima componente di uno strumento di source code analysis è
rappresentata dall’effettivo analizzatore.

Nel corso degli anni sono state definiti e impiegati una grande quantità di
meccanismi e tipi di analisi. Un approfondimento di ciascuna di questi esula
dagli obiettivi di questa tesi. All’interno del Capitolo \ref{applications} di
questa tesi, verranno invece analizzati i principali, e anticipati, tipi di
analisi che la rappresentazione offerta dalla libreria CLAST consente con
maggiore dettaglio.\\

Dopo aver presentato le principali componenti per la creazione di uno strumento
di source code analysis, la Sottosezione \ref{sca-approches} approfondisce i
principali approcci alla creazione di uno strumento di questo tipo.

\image{img/sca-tool-architecture}
      {Architettura di uno strumento di Source Code Analysis}
      {fig:sca-tool-architecture}
      {0.5}

Nel contesto dell'architettura appena illustrata, mostrata grafica in Figura
\ref{fig:sca-tool-architecture} la libreria CLAST, soggetto di questa tesi, va a
definire sia la prima componente, un parser per il linguaggio di programmazione
Common Lisp, sia la seconda componente, ossia una struttura dati organizzata in
forma di AST che raccoglie il maggior numero possibile di informazioni
disponibili rispetto al codice del sistema in analisi. I dettagli della
progettazione ed implementazione di queste componenti verranno approfonditi nel
corso dei prossimi capitoli di questa tesi.

La libreria ha quindi lo scopo di semplificare la creazione di strumenti per la
source code analysis, dando la possibilità ad uno sviluppatore di concentrarsi
sulla definizione di un analizzatore specifico per i contenuti e le metriche di
interesse, senza la necessità di dover definire l'infrastruttura di analisi che
sarebbe altrimenti obbligatorio costruire.

\subsection{Approcci per la costruzione di uno strumento di Source Code
Analysis}
\label{sca-approches}

Gli attuali strumenti di source code analysis sono sviluppati utilizzando
principalmente due diverse strategie di sviluppo. Alcuni strumenti realizzano
direttamente la fase di parsing del codice sorgente del sistema in analisi.
Strumenti di questo tipo prevedono quindi una componente parser al loro interno,
componente che deve quindi essere sviluppata e manutenuta parallelamente allo
strumento stesso. Un approccio opposto a quello appena descritto è rappresentato
dalla costruzione di strumenti che operano a partire da uno o più strumenti già
disponibili, come ad esempio un compilatore in grado di produrre un Abstract
Syntax Tree associato ad un programma e esporlo attraverso un’API.

In seguito vengono discussi questi due approcci, indicando vantaggi e
svantaggi, problemi e sfide caratteristici di ciascuno di questi. Questo al
fine di presentare al lettore con maggiore chiarezza il contesto di risorse
all’interno del quale, uno strumento come quello descritto da questa tesi, si
colloca dal punto di vista dello sviluppo e delle risorse preesistenti.\\

Dal punto di vista pratico, la scelta più condizionante per il progettista di
uno strumento per la source code analysis è la seguente: sviluppare
parallelamente al proprio strumento un parser per il linguaggio, o i linguaggi,
di programmazione di riferimento oppure dipendere da un parser già disponibile.

Lo sviluppo dell’intera infrastruttura necessaria per la creazione dello
strumento che si desidera produrre, compreso un parser, ha alcuni vantaggi. In
prima battuta, lo sviluppo del nuovo parser e le attività svolte da questo
potrebbero essere limitate ad un sottoinsieme del linguaggio di riferimento e di
conseguenza ad un sottoinsieme della grammatica di questo. Questo porta ad
ottenere un componente più leggero e facilmente integrabile all'interno
dell'architettura dello strumento che si desidera produrre.

Questo approccio può richiedere la disponibilità di un’infrastruttura per lo
sviluppo di un parser, come ad esempio lexers, generatori di parser e altro
ancora, ma, quanto meno, necessità di un grammatica per il linguaggio che si
desidera analizzare. Uno dei problemi più comuni è che molto spesso
l’implementazione di un linguaggio e la grammatica utilizzata per descriverlo
non sono perfettamente corrispondenti. O la grammatica è incompleta, come ad
esempio è stato per lungo tempo nel caso del linguaggio di programmazione C++,
la cui grammatica non faceva riferimento al costrutto template presente in
questo, oppure la grammatica potrebbe non essere completamente compatibile con i
differenti dialetti di un linguaggio.

Un passo che consente una semplificazione di questo processo può essere
rappresentato dall'estensione o aggiornarnamento una grammatica già esistente.
Questo al fine di non dover necessariamento definire una formalizzazione
completamente nuova dell'intera grammatica di un linguaggio di programmazione.
La scrittura di una nuova grammatica può essere estremamente dispendiosa; il
risultato può essere incompleto, poco robusto, o addirittura accettare un
sovrainsieme del linguaggio. In ogni caso, la continua evoluzione che
caratterizza la maggioranza dei linguaggio di programmazione di attualità
implica la necessità di di costanti aggiornamenti della grammatica sviluppata,
un compito tutt’altro che banale.\\

Fortunatamente, grammatiche precise, robuste e aggiornate, così come parser per
queste, sono incorporate all’interno del compilatore per il linguaggio di
programmazione di interesse e, molto spesso, vengono rese disponibili
all’utilizzo esterno al compilatore stesso. La maggioranza dei frontend dei
compilatori per linguaggi di programmazione attuali può infatti essere estesa
allo scopo di consentire l’analisi e l’estrazione di informazioni dal codice
sorgente fornito in input al compilatore.

In contesti di questo tipo, la maggiore difficoltà è però rappresentata dal
fatto che gli obiettivi di un compilatore e di uno strumento per la source code
analysis sono differenti e, in molti casi, in contrasto.

Lo scopo di un compilatore è infatti quello di produrre codice eseguibile, a
partire dal codice sorgente in input, che sia il più efficiente possibile e con
il minor impiego di risorse possibile. Lo scopo di uno strumento per la source
code analysis è invece quello di produrre la rappresentazione più ricca
possibile del codice sorgente in input rispetto ai particolari parametri di
interesse all’analisi.

Per questa ragione, a seconda del compilatore e a seconda dell’API offerta da
questo, è possibile seguire due strade distinte per poter costruire uno
strumento per la source code analysis a partire da questo:

\begin{itemize}

\item apportare una modifica al compilatore stesso, in maniera tale che sia
possibile l'accesso alla rappresentazione da parte di strumenti esterni a
questo;

\item costruire un wrapper per le funzionalità interessanti offerte dal
compilatore. Nel caso in cui il compilatore memorizzi una rappresentazione del
codice sorgente del programma in input, ad esempio in forma di AST, come formato
di rappresentazione intermedia, è possibile costruire un analizzatore che lavori
a partire da questa rappresentazione, senza la necessità di operare modifiche al
compilatore stesso. Se l’accesso ai formati di rappresentazione intermedi non
fosse disponibile all’esterno, sarebbe però comunque necessario operare un
modifica del codice del compilatore in maniera tale da rendere l’esportazione
possibile.

\end{itemize}

\image{img/sca-compiler-strategies}
      {Strategie per la strutturazione dell'interazione tra uno strumento di
      source code analysis e un compilatore}
      {fig:sca-compiler-strategies}
      {0.5}

Uno degli svantaggi principali del primo approccio citato è rappresentato dal
fatto che, in molti casi, l'architettura del compilatore non viene definita in
maniera tale da facilitare o consentire l'estensione che porta alla produzione
della rappresentazione del sistema utilizzata internamente dal compilatore
stesso e ciò rende particolarmente complesse le attività di implementazione
dell'estensione. Inoltre, un progetto che include l'estensione di un compilatore
richiede necessariamente una fase di studio e comprensione del funzionamento
interno di questo, attività spesso non banali.\\

Uno degli svantaggi principali del secondo approccio è invece rappresentato dal
fatto che, anche strutturando la modifica operata al compilatore in maniera tale
da essere il meno dipendente possibile dalle peculiarità di questo, è sempre
possibile che una nuova versione del compilatore renda le modifiche
precedentemente operate incompatibili, portando a dover riscrivere gran parte
dello strumento. Una seconda problematica significativa che caratterizza il
secondo approccio citato è rappresentata dal fatto che, come riportato in
precedenza, lo scopo di un compilatore e di uno strumento per la source code
analysis sono fondamentalmente diversi e, per questa ragione, è molto probabile
che la rappresentazione che il compilatore offre del codice sorgente, ad esempio
attraverso i nodi di un AST, non riporti informazioni che sarebbero invece di
interesse per lo strumento e, viceversa, riporti una grande quantità di
informazioni di scarso, se non inesistente, interesse per lo strumento, aspetto
che porta alla necessità di introdurre un fase di aggiunta o rimozione di
elementi dalla rappresentazione fornita dal compilatore, aumentando la
complessità dello strumento.\\

Sfortunatamente nessuno dei compilatori attualmente disponibili per il
linguaggio di programmazione Common Lisp fornisce un'API che consenta di
sfruttare la rappresentazione interna che ciascuno di questi produce durante la
propria elaborazione. Inoltre, la presenza di differenti implementazioni del
linguaggio e la grande frammentazione della comunità Lisp derivata da questa,
rendono impossibile utilizzare un approccio di estensione di un compilatore.
Infatti, i vantaggi legati alla possibilità di utilizzare uno strumento source
code analysis costruito attraverso l'estensione di un particolare compilatore
sarebbe limitata solamente alla ristretta porzione di utenti di quella
particolare implementazione.

Si è pertanto scelto di seguire una via diversa, sfruttando la strutturazione
del linguaggio Common Lisp, già molto prossima alla struttura di un AST, e
godendo della possibilità di definire un meccanismo di rappresentazione non
condizionato dalle necessità di un compilatore, andando a strutturare CLAST
come uno strumento completamente distinto da un compilatore e implementare le
strutture necessarie per la creazione di strumenti di source code analysis per
il linguaggio Common Lisp.\\

Questa Sezione ha presentato il campo della source code analysis, ossia l'ambito
di ricerca all'interno del quale questa tesi si colloca, fornendo al lettore
alcuni spunti per comprendere l'importanza che questa ricopre all'interno del
più generale campo dell'ingegneria del software. Si è inoltre presentata la
tradizionale architettura di uno strumento di source code analysis e i
principali approcci alla costruzione di strumenti di questo tipo. Questo al fine
di illustrare la collocazione della libreria CLAST e la tipologia di strumenti
che questa consente di realizzare.

\section{Abstract Syntax Tree}
\label{abstract-syntax-tree}

Un \textit{abstract syntax tree}, nome spesso abbreviato utilizzando la sigla
AST, è uno strumento per la rappresentazione, sotto forma di albero, della
struttura di un programma.

Dal punto di vista formale, un abstract syntax tree è definito come un albero
ordinato ed etichettato, i cui nodi interni rappresentano gli operatori
utilizzati dal codice del programma che si desidera rappresentare e le cui
foglie rappresentano gli operandi soggetto delle operazioni precedentemente
citate. Un arco congiunge un nodo che rappresenta un operatore a ciascuno dei
suoi operandi o, eventualmente, agli operatori innestati a questo. Se si
considera un linguaggio di programmazione tradizionale, esempi di operatori sono
rappresentati da funzioni e operatori booleani, esempi di operandi sono
rappresentati da variabili e constanti.\\

Come affermato dalla precedente definizione, un AST rappresenta la struttura di
un programma. In particolare, a differenza di altri esempi di meccanismi di
rappresentazione, un AST fa riferimento alla \textit{struttura sintattica
astratta} di un programma.

La struttura sintattica astratta di un programma rappresenta un’astrazione
rispetto alla normale struttura, definita solitamente \textit{concreta} per
contrapposizione, di un linguaggio di programmazione. La sintassi astratta di un
linguaggio di programmazione si differenzia da quella concreta, utilizzata per
la scrittura di codice sorgente, in quanto solitamente ne rappresenta una
semplificazione. In particolare, la sintassi astratta non riporta tutte le
regole sintattiche che non influenzano la semantica del programma.\\

Un esempio di regola sintattica che viene trascurata dalla sintassi astratta di
un programma è rappresentata dalle \textit{grouping parentheses}, un costrutto
comune a diversi linguaggi di programmazione che consente di raggruppare un
certo insieme di elementi all'interno di un'unica espressione. La ragione per
cui questa particolare regola viene eliminata, all’interno della sintassi
astratta, è rappresentata dal fatto che questa non aggiunge alcuna informazione
rispetto alla semantica del programma. Infatti, l’informazione che l’utilizzo
delle parentesi trasmette, può essere trasmessa in maniera del tutto equivalente
semplicemente facendo riferimento ad una rappresentazione ad albero del
programma. Rappresentazione in cui elementi correlati possono essere raggruppati
dalla presenza un nodo antenato comune.

Un altro esempio di regola sintattica che viene trascurata per una ragione del
tutto equivalente è la regola di utilizzo del simbolo \texttt{;} come elemento
di separazione di un’istruzione dalla successiva. Anche in questo caso infatti
la struttura stessa dell’albero di rappresentazione suggerisce, attraverso i
nodi, la separazione di una data istruzione dalla successiva e può quindi essere
utilizzata per rappresentare la relazione di sequenzialità tra le istruzioni,
senza la necessità ulteriori modalità di rappresentazione.\\

Da un punto di vista pratico, un abstract syntax tree viene in molti casi
costruito a partire da un parse tree, definito anche \textit{concrete syntax
tree} nella letteratura. Un concrete syntax tree è una rappresentazione della
struttura sintattica concreta di un programma, prodotta tipicamente attraverso
il semplice e diretto parsing del codice sorgente. Il processo di generazione di
un AST a partire da un parse tree procede quindi per eliminazione di nodi e
archi ritenuti ridondanti o non necessari. Lo scopo di queste eliminazioni è
quello di ottenere una rappresentazione che possa servire come strumento più
comodo per le elaborazioni successive, uno strumento che non riporti
informazioni ridondanti o ritenute non interessanti per queste elaborazioni. Gli
AST sono infatti molto spesso degli strumenti per il supporto alla realizzazione
di funzionalità complesse e non un prodotto ultimo dell’elaborazione.

\image{img/concrete-syntax-tree.png}
      {Esempio di Concrete Syntax Tree per l'espressione \texttt{3 + 4 * 5}}
      {fig:concrete-syntax-tree}
      {0.5}

\image{img/abstract-syntax-tree.png}
      {Esempio di Abstract Syntax Tree per l'espressione \texttt{3 + 4 * 5}}
      {fig:abstract-syntax-tree}
      {0.5}

Purtroppo, non esiste una tecnica per la generazione di AST indipendente dal
linguaggio e dal contesto di utilizzo. La causa fondamentale che porta a questa
mancanza di una strategia univoca è data proprio dalla natura di strumento di
supporto degli AST, natura a cui si è fatto riferimento nel paragrafo
precedente. Essendo un AST uno strumento di supporto, risulta infatti
particolarmente importante che questo venga definito a partire dalle necessità
caratteristiche delle funzionalità che si desidera realizzare a partire da
questo.

Molto spesso un AST viene quindi arricchito con informazioni aggiuntive e
caratteristiche del contesto di applicazione. Queste informazioni forniscono
maggiore conoscenza rispetto alla semantica del programma in modo specifico per
la funzionalità di ispezione o analisi per realizzare la quale si fa uso
dell’AST.

Riassumendo, il fatto che funzionalità diverse richiedono la disponibilità di
informazioni diverse e il fatto che gli AST nascono proprio come struttura per
favorire lo sviluppo di una certa funzionalità rendono poco ragionevole la
definizione di una strategia di generazione univoca.\\

La grande varietà di formalismi e costrutti che caratterizzano i diversi
linguaggi di programmazione e la velocità con la quale ne vengono introdotti di
nuovi rappresentano altri due fattori contribuirebbero a rendere un'ipotetica
strategia di generazione degli AST indipendente dal linguaggio poco efficace e
rapidamente obsoleta.\\

Come affermato nel corso dei paragrafi precedenti non risulta possibile, e
nemmeno particolarmente utile, definire una strategia di generazione degli ASTs
univoca. Un AST viene costruito in maniera specifica per un linguaggio di
programmazione e per la realizzazione di una specifica funzionalità.

Tuttavia, nel caso in cui si abbia una famiglia di linguaggi di programmazione
molto simili tra loro, ossia che condividono gran parte della propria sintassi
astratta, e si desideri realizzare una stessa funzionalità o uno stesso
strumento di analisi per ciascuno di questi, è possibile utilizzare un
rappresentazione unificata, comune a ciascuno dei linguaggi appartenenti alla
famiglia. Un AST di questo tipo, comune a diversi linguaggi di programmazione,
viene detto \textit{AST unificato}.

Formalmente, un AST unificato è una rappresentazione ad albero della struttura
sintattica astratta del codice sorgente di un programma scritto utilizzando un
linguaggio facente parte di una data famiglia di linguaggi di programmazione.

Un esempio di AST unificato è rappresentato dal clang AST, ossia l’Abstract
Syntax Tree che viene prodotto dallo strumento clang \footnote
{http://clang.llvm.org}, uno strumento che si pone come obiettivo quello di
definire un modulo di frontend per compilatore LLVM \footnote{http://llvm.org/},
compilatore comune alla famiglia di linguaggi di programmazione C, C++,
Objective C and Objective C++. \cite{DBLP:conf/lcpc/LattnerA04}\\

L’obiettivo più comune per il quale vengono impiegati gli AST è la
semplificazione e disaccoppiamento dei diversi passi che compongono il processo
di compilazione. Nelle prossime sezioni verrà brevemente illustrato il processo
con il quale un AST viene prodotto e utilizzato da un compilatore. Le
applicazioni degli AST non si limitano però al solo contesto dei compilatori;
nella letteratura è infatti possibile trovare applicazioni degli AST anche nel
contesto degli strumenti per il refactoring \cite{jscodeshift2016} e nel campo
della clone-detection \cite{DBLP:conf/saci/LazarB14}. Anche solamente
all’interno di un compilatore tuttavia, l’applicazione degli AST non si limita
al solo supporto della generazione di codice macchina. Altre applicazioni
all’interno di un compilatore vanno dal supporto per l'aggiunta di una
tipizzazione statica per il linguaggio target della compilazione a quello per
l'utilizzo di costrutti di pattern matching.\\

La libreria CLAST, soggetto di questa tesi, utilizza agli abstract syntax tree
come meccanismo per la rappresentazione di un programma al fine di consentire la
creazione di strumenti di source code analysis. Data quindi l'importanza, sia
dal punto di vista teorico che dal punto di vista pratico, di queste strutture
dati all'interno di CLAST, si è scelto di presentare, attraverso questa sezione,
un approfondimento rispetto alla teoria alla base degli AST e alle applicazioni
di questi nel contesto di strumenti per l'analisi ed elaborazione di sistemi
software.

La Sottosezione \ref{ast-applications} di questo capitolo fornisce una
panoramica delle principali applicazioni degli AST nei diversi contesti che
fanno un utilizzo di questo meccanismo di rappresentazione. Questo allo scopo di
introdurre il lettore ai requisiti che una libreria per la generazione di AST
per un linguaggio di programmazione, come la libreria CLAST descritta da questa
tesi, deve essere in grado di soddisfare.

\subsection{Applicazioni degli AST}
\label{ast-applications}

Le applicazioni degli abstract syntax tree sono diverse, sia nel mondo della
ricerca, sia nel mondo dei compilatori che nel mondo dell’industria. All’interno
di questa sezione vengono presentati alcuni esempi provenienti da questi
differenti mondi per sottolineare l’importanza della disponibilità di uno
strumento per la generazione di AST per un dato linguaggio di programmazione.
Questo consente quindi di mostrare al lettore l’utilità, i requisiti e la
tradizionale collocazione di uno strumento per la generazione di ASTs come
quello che viene trattato da questa tesi.

Essendo il maggiore campo di applicazione degli ASTs quello dei compilatori, nei
prossimi paragrafi se ne presenta un approfondimento.

\subsubsection{Nei compilatori}

L'obiettivo di un compilatore è quello di trasformare il proprio input,
rappresentato un programma scritto utilizzando un dato linguaggio di alto
livello, producendo come output una programma equivalente scritto in linguaggio
macchina. Tipicamente, il lavoro di un compilatore può essere scomposto in un
certo insieme di fasi in consecutive. Ciascuna di queste fasi riceve in input
una certa rappresentazione del programma originale e ne produce in output una
differente, arricchita e modificata al fine di consentire o semplificare le fasi
successive. In seguito, vengono presentate le fasi del processo di compilazione
che tipicamente portano alla generazione di un AST.\\

Le prime operazioni svolte durante il processo di compilazione sono quelle
relative all’analisi lessicale del codice sorgente, detta anche
\textit{scanning}. Lo scopo di questa fase è quello di leggere i caratteri che
compongono il codice sorgente del programma target della compilazione e riunirli
all’interno di gruppi logici di caratteri correlati tra loro detti
\textit{token}. Esempi di token comuni alla maggior parte dei linguaggi di
programmazione sono rappresentati dalle keyword utilizzate dal linguaggio, ossia
le sequenze di caratteri riservate dal linguaggio per identificare l'utilizzo di
particolari costrutti, i numeri interi e l’operatore di assegnamento. Una volta
identificati, i token vengono forniti come input alla fase successiva del
processo di compilazione in forma di stream.\\

La fase successiva alla fase di analisi lessicale è rappresentata dalla fase di
analisi sintattica o parsing, la quale fornisce, a partire dallo stream di token
prodotto dalla fase precedente, un insieme di entità sintattiche, come ad
esempio espressioni e istruzioni.

Le entità sintattiche sopraccitate vengono quindi poste, sempre durante questa
fase, all’interno di un parse tree, struttura a partire dalla quale viene
prodotto l’AST del programma, procedendo per eliminazione degli elementi
ridondanti o poco interessanti alle fasi successive, come affermato in apertura
di questa sezione nel corso della definizione di abstract syntax tree.

Nel caso di alcuni linguaggi di programmazione, la struttura sintattica target
del parsing risulta particolarmente semplice da trattare. In questi casi, alcuni
strumenti scelgono di non realizzare una fase di definizione del parse tree,
procedendo direttamente alla costruzione di un AST a partire dall’output della
fase di analisi sintattica.

All’interno di un compilatore, un abstract syntax tree rappresenta quindi molto
spesso l’output della fase di elaborazione compiuta dalla componente parser, o
più generalmente l’output della fase di parsing, e l’input per le successive
fasi di analisi semantica e generazione di codice macchina.

\image{img/compiler-architecture}
      {Architettura di un compilatore per un linguaggio di programmazione
      tradizionale}
      {fig:compiler-architecture}
      {0.5}

Tipicamente, un compilatore fa riferimento a diverse altre strutture dati
durante la sua azione. Tuttavia l’AST esegue ricopre un ruolo unico in quanto
utilizzato in moltissime fasi differenti, tra le quali:

\begin{itemize}

\item l'AST viene utilizzato molto intensamente durante la fase di analisi
semantica, durante la quale un compilatore verifica il corretto utilizzo degli
elementi del linguaggio e del programma.

\item Durante la fase di analisi semantica, un compilatore genera tabelle dei
simboli utilizzati da un programma a partire dall’AST.

\item Dopo essere stato utilizzato per la verifica della correttezza sintattica,
l’AST viene utilizzato come base per la generazione di codice macchina o, come
più spesso accade, viene utilizzato per la generazione di una ”intermediate
representation” o ”IR”, a cui spesso nella letteratura si fa riferimento come a
un linguaggio intermedio, apposito per la generazione di codice.

\end{itemize}

Dato l’elenco di operazioni svolte da un compilatore a partire da un AST,
riportato all’interno del paragrafo precedente, è possibile presentare alcuni
dei requisiti che vengono tipicamente a cui la progettazione e l’implementazione
di uno strumento per la generazione di AST sono soggette. É importante
sottolineare che, data la già citata natura di strumento di supporto degli AST,
specifici requisiti di un AST sono fortemente dipendenti dalla specifica
applicazione. Per questa ragione i requisiti presentati in seguito sono
solamente rappresentativi dell’applicazione del caso d'uso rappresentato dal
processo di compilazione di un linguaggio di programmazione tradizionale.

\begin{itemize}

\item La presenza di operatori n-ari in un linguaggio di programmazione rende
necessario che l’AST per tale linguaggio supporti la presenza di nodi con un
numero arbitrario di figli.

\item L’ordine di esecuzione delle istruzioni di un programma deve poter essere
correttamente identificato, conservato ed rappresentato in modo esplicito
dall’AST, come anche quello degli operandi di eventuali operazioni n-arie.

\item Gli identificativi e valori utilizzati dalle istruzioni di assegnamento
presenti all’interno del codice devono essere memorizzati e ricercabili tramite
ispezione dei nodi.

\item Durante la creazione dei nodi, i tipi delle variabili esplicitati dal
programmatore devono essere preservati, così come l'occorrenza di ciascuna
dichiarazione all’interno del codice sorgente.

\item A partire da un AST deve sempre essere possibile ricostruire il codice
sorgente originale nella sua interezza. Il codice prodotto in questo modo
dovrebbe essere sufficientemente simile al codice originale da conservarne il
funzionamento in fase di esecuzione, una volta ricompilato.\\

\end{itemize}

Data la complessità dei requisiti appena proposti in riferimento alla
progettazione di un AST per il compilatore di un dato linguaggio di
programmazione, l’applicazione di noti design pattern può risultare di grande
aiuto alla progettazione e  realizzazione di un sistema per la generazione di
ASTs.

Ad esempio, è fortemente probabile che un compilatore debba procedere diverse
volte alla visita dei nodi che compongono l’AST. Inoltre, molto spesso è
necessario che il compilatore svolga operazioni differenti a seconda dello
specifico tipo di nodo incontrato durante il processo di visita o in base al
valore di particolari attributi di questo. Infine, essendo un AST utilizzato da
diverse componenti di un compilatore, è importante che questo fornisca
un’interfaccia per la visita standard particolarmente facile da comprendere ed
utilizzare per tutti i differenti gruppi di progettisti e sviluppatori di queste
diverse componenti.

Un esempio di pattern che tradizionalmente risulta particolarmente appropriato
in questo contesto è rappresentato dal design pattern Visitor, presentato in
\cite{gamma1995design}. Il pattern Visitor risulta particolarmente appropriato a
questo contesto in quanto di soddisfare le necessità appena elencate, fornendo
delle linee guida che consentano facilmente l’implementazione delle operazioni
di visita dei singoli nodi e attraversamento dell’albero in modo efficiente,
fornendo accesso a tutte le informazioni relative a tipo e attributi a ciascuno
nodo, ed esponendo allo stesso tempo uninterfaccia tipicamente già familiare
agli utenti.\\

Dopo aver illustrato il principale campo di applicazione degli ASTs, e di
riflesso di uno strumento per la generazione di ASTs, la prossima sottosezione
di questo capitolo presenta alcuni dei principali lavori presenti nella
letteratura del settore rispetto a strumenti per la creazione, manipolazione ed
analisi degli AST, sottolineando alcune delle principali linee di ricerca in
questo ambito.

\subsubsection{Nella ricerca}
\label{ast-research}

In questa sottosezione vengono presentati alcuni lavori correlati allo strumento
presentato all’interno di questa tesi. Lo scopo di questa sottosezione è quindi
quello di fornire al lettore una panoramica rispetto ad alcuni dei campi di
applicazione di uno strumento per la generazione di AST nell'ambito della
ricerca. A questo scopo vengono elencati alcuni articoli relativi al mondo degli
abstract syntax tree, lavori che presentano il problema della progettazione di
AST e meccanismi di generazione per ASTs per la realizzazione di sistemi che
realizzano svariate funzionalità a partire dall’elaborazione delle informazioni
a cui questi danno accesso.\\

In \cite{martinez2014accurate}, Martinez et al. propongono una tecnica che
lavora a partire da un AST e, in particolare, basata sull’utilizzo di un
algoritmo di calcolo della distanza tra istanze di AST. Questa tecnica consente
la correzione di errori presenti a livello di codice sorgente mediante
l’identificazione di pattern di correzione precedentemente applicati durante lo
sviluppo.

La tecnica opera in prima battuta andando a ricercare all’interno di un sistema
per il controllo delle versioni tutte quelle revisioni che contengono la
correzione di un errore. Dopo questa prima fase di ricerca, il progetto viene
monitorato allo scopo di identificare nuove occorrenze di errori precedentemente
corretti. Ricerca e monitoraggio vengono entrambi operate a livello di AST,
utilizzando il sopraccitato algoritmo di calcolo della distanza tra AST. Una
volta identificata l’occorrenza di un errore precedentemente risolto, sempre
lavorando a livello di AST, viene applicata nuovamente la modifica identificata
come correzione per quel particolare errore.\\

ASTLOG è uno strumento sviluppato da Crew, durante il suo lavoro come
ricercatore presso Microsoft Inc. Si tratta di uno strumento che consente di
operare ricerche, anche molto complesse, all’interno del codice di un programma
scritto utilizzando i linguaggi C e C++, programmi anche di dimensioni molto
significative. \cite{DBLP:conf/dsl/Crew97}

Questo strumento si pone come alternativa ai generali metodi di ricerca in
sistemi UNIX come
grep\footnote{https://www.gnu.org/software/grep/manual/grep.html} e
awk\footnote{https://www.gnu.org/s/gawk/manual/gawk.html}, consentendo la
ricerca di pattern complessi come ad esempio: l’utilizzo di un particolare nome
di variabile, dichiarata specificando un dato tipo all’interno di un metodo che
prende in input un dato numero di parametri, presente all’interno di una classe
dichiarata all’interno di un file che importa una data libreria.

Lo strumento è stato realmente applicato anche al di fuori della ricerca e, in
particolare, \cite{DBLP:conf/dsl/Crew97} presenta come casi di studio ricerche
operate all’interno del codice sorgente di Microsoft SQL Server, descritto come
uno programma di 450mila righe di codice, e di Microsoft Word, ai tempi indicato
come un programma da più di due milioni di righe di codice.\\

In \cite{DBLP:conf/kbse/Welty97}, Welty presenta un’ontologia per la
rappresentazione di conoscenza a livello di codice sorgente basata sull’utilizzo
di ASTs. L’obiettivo di questa ontologia è quello di minimizzare lo sforzo
necessario ai singoli membri di un team di sviluppo per la documentazione e la
ricerca di informazioni relative all’implementazione di un sistema software,
strumenti grazie ai quali risulta più semplice l’aggiunta di nuovi membri ad un
team.\\

Bulychev e Minea descrivono, in \cite{peter2008duplicate}, un approccio,
indipendente dal linguaggio, per l’identificazione occorrenze di codice
duplicato, definite formalmente code clones dagli autori, all’interno di grandi
sistemi software. L’approccio viene quindi illustrato attraverso la
presentazione un algoritmo che consente di confrontare, a livello di AST, due o
più frammenti di codice al fine di ricercare sequenze di istruzioni che possono
essere riottenute, applicando le dovute sostituzioni di sottoalberi agli AST, a
partire dalle sequenze di istruzioni presenti all’interno di altri frammenti di
codice.\\

Utilizzando come fondamento il lavoro di Bulychev e Minea, Lazar e Banias
presentano una metodologia \cite{DBLP:conf/saci/LazarB14} per l’identificazione
di episodi di plagio nello specifico contesto di sistemi software sviluppati
utilizzando il linguaggio di programmazione C. Uno degli elementi di maggiore
interesse esposti da questo lavoro, rispetto ai contenuti di questa tesi, è
rappresentato dallo studio che i due autori descrivono rispetto alle riflessioni
operate per la definizione del meccanismo di generazione degli AST. Gli autori
presentano infatti la descrizione di un meccanismo, esterno al processo di
compilazione, costruito in maniera tale da riportare solamente gli elementi
strettamente necessari all’analisi che la metodologia si pone come obiettivo.\\

In \cite{DBLP:conf/acsac/YamaguchiLR12}, Yagamaguchi et al. descrivono una
tecnica per l’analisi di AST al fine di identificare pattern riconosciuti
durante lo studio di vulnerabilità software note. Questo al fine di poter
verificare la presenza di occorrenze di queste stesse vulnerabilità ed
eventualmente segnalare le modifiche necessarie che lo sviluppatore del sistema
dovrà applicare.\\

Infine, Neamtiu et al. descrivono in \cite{DBLP:journals/sigsoft/NeamtiuFH05}
uno strumento che lavora combinando un sistema di controllo delle versioni ad
un’analisi degli AST e che consente di tracciare e studiare l’evoluzione nel
tempo di un sistema software, scritto utilizzando il linguaggio C, soprattutto
per quanto concerne gli aspetti architetturali del sistema.

\subsubsection{Nell’industria}

Mentre la precedente sottosezione ha elencato alcuni degli utilizzi degli AST
nel mondo della ricerca, all’interno di questa sottosezione vengono presentate
alcune delle applicazioni degli AST che possono essere rintracciate nel mondo
dell’industria. Questo al fine di completare la panoramica che questo capitolo
ha offerto sui requisiti e sulle applicazioni degli abstract syntax tree e dei
sistemi per la loro generazione.\\

Dopo alla più tradizionale applicazione degli AST, illustrata nella sottosezione
di questo capitolo dedicata all’utilizzo degli AST nel contesto di un
compilatore, la seconda applicazione più significativa degli AST è all’interno
di strumenti per il supporto allo sviluppo, in particolare all’interno di
sistemi per l'analisi e modifica di codice sorgente.\\

Per molti programmatori, la possibilità di utilizzare all’interno del proprio
IDE una funzionalità che consenta di selezionare un insieme di istruzioni e
costruire un metodo a partire da queste, tipicamente chiamata \texttt{Extract
Method}, è una funzionalità la cui presenza viene data sostanzialmente per
scontata.

Perché sia possibile realizzare questa funzionalità, e più generalmente perché
sia possibile realizzare un sistema che consenta di svolgere una qualsiasi
operazione di refactoring, è però di fondamentale importanza avere a
disposizione un meccanismo di rappresentazione del codice del sistema in
analisi. Gli AST rappresentano la struttura dati più appropriata, e più
impiegata \cite{eclipse2006} \cite{netbeans2007}, a questo scopo.

Eclipse\footnote{https://www.eclipse.org}, l’IDE più utilizzato dagli utenti del
linguaggio di programmazione Java, realizza la funzionalità appena descritta e,
più generalmente la grande maggioranza delle operazioni di modifica di codice
sorgente automatica, a partire da ASTParser, una libreria Java che consente la
generazione e visita dell’abstract syntax tree di un programma. Libreria che è
possibile utilizzare anche per la generazione di plugin che estendono le
funzionalità dell’ambiente Eclipse stesso da parte di sviluppatori di terze
parti.\\

JSCodeshift \cite{jscodeshift2016} è un altro esempio di strumento che utilizza
come fondamento per il suo funzionamento gli abstract syntax tree. JSCodeshift
è uno strumento open-source, nato come progetto interno a Facebook\texttrademark,
per la manutenzione di grandi quantità di codice JavaScript.

In particolare, JSCodeShift viene utilizzato per operare \textit{codemod},
termine con il quale gli sviluppatori dello strumento fanno riferimento ad un
insieme di cambiamenti su larga scala che coinvolgono grandi porzioni di del
codice di un sistema. Cambiamenti che possono consistere di semplici modifiche,
come ad esempio la sostituzione del nome di una variabile o metodo, o anche di
modiche molto complesse, come l'aggiornamento delle chiamate ad una libreria o
interazione con un framework a seguito di cambiamenti critici, \textit{source
breaking changes}, nell'interfaccia o funzionamento di questo.\\

Da un punto di vista pratico, il funzionamento di JSCodeshift può essere
semplificato come segue: lo strumento prende in input un frammento di codice
JavaScript ed il mapping associato alla modifica che si desidera venga operata
dallo strumento, e produce in output un nuovo frammento di codice JavaScript in
cui il mapping è stato applicato a tutte le occorrenze del pattern di input
associato alla modifica. Mapping che descrive, a livello di AST, la modifica che
si desidera venga applicata al codice matchato da una particolare espressione di
origine specificata per esso.

Lo strumento lavora quindi completamente a livello di AST andando dapprima a
generare l’AST del codice in input, andando poi ad identificare i nodi
all’interno di questo che sono matchati dall’espressione di origine del mapping,
generando codice di rimpiazzo adeguato a partire dal nodo matchato e dal mapping
fornito in input, operando il rimpiazzo del nodo, e andando infine a rigenerare
codice JavaScript a partire dall’AST all’interno del quale è stata operata la
sostituzione.\\

Nonostante la relativa semplicità dello strumento dal punto di vista
concettuale, la sua utilità risulta molto significativa. Consente infatti di
operare modifiche anche molto complesse a sistemi software di grandi dimensioni
\cite{jsconf2016}. Modifiche che risulterebbero eccessivamente dispendiose in
termini di tempo, qualora operate in modo manuale, o addirittura impossibili
utilizzando approcci più tradizionali come ad esempio quelli basati su
espressioni regolari, questo soprattutto a causa della presenza di costrutti
ambigui nella, sempre in evoluzione, grammatica del linguaggio JavaScript.\\

\section{Macro}

Il costrutto macro in Common Lisp consente ad un utente di definire funzione in
grado di convertire una form Lisp in una differente form prima che queste
vengano valutate e compilate.\\

A differenza di quanto avviene nella grande maggioranza degli altri linguaggi,
questa conversione viene però operata a livello di espressione, e non a livello
i caratteri o stringhe. Piuttosto che fornire semplicemente la possibilità di
definire una sostituzione che consente l’utilizzo di una sintassi banalmente
più breve, una macro Lisp consente di estendere in maniera nativa e reale il
linguaggio di programmazione Lisp stesso.

Nel contesto del linguaggio di programmazione Common Lisp, il costrutto macro
risulta di fondamentale importanza per la scrittura di codice di qualità: la
presenza di questo costrutto consenti infatti di poter scrivere codice chiaro
ed elegante a livello di utilizzo (API), aspetto che consente avere una
maggiore comprensione del sistema che si sta costruendo e di ragionare in modo
più semplice rispetto alla correttezza e alle caratteristiche di questo,
andando poi però ad eseguire una trasformazione ad versione interna di questo
codice più efficiente ma tipicamente più complessa.

La presenza del costrutto macro consente infatti di produrre codice
particolarmente conciso, chiaro ed elegante a livello di utilizzo ed API,
consentendo di lavorare ad un livello di astrazione maggiore rispetto a quello
della reale implementazione, implementazione che potrà poi essere
arbitrariamente complessa ma tipicamente anche molto più efficiente. Il
costrutto macro consente quindi di incapsulare la complessità
dell’implementazione di una certa funzionalità, rendendo il sistema complessivo
potenzialmente molto più semplice e consentendo a chi implementa di avere una
migliore comprensione rispetto alle caratteristiche di correttezza e
funzionamento del sistema che sta costruendo.\\

La caratteristica fondamentale che distingue una form macro dalle altre form
Lisp é che, come accennato in apertura di questa sezione, una macro non produce
in output un valore. Ogni macro produce come output una nuova form costruita a
partire da quella in input, rappresentata dal parametro body specificato in
fase di invocazione della macro.

Proprio perché una macro consiste in un processo di generazione di una nuova
form, ossia codice Lisp, a partire da altre form, ossia codice Lisp, molto
spesso può risultare comodo pensare ad una macro come ad un programma in grado
di generare un altro programma.

Nel caso più semplice, una macro sostituisce una form a partire da un template,
a partire da una chiara corrispondenza, anche a livello visivo, tra il codice
generante e il codice generato, consentita dalla combinazione tra costrutto
\texttt{BACKQUOTE} e costrutto \texttt{COMMA}. Macro più complesse possono
invece essere utilizzate per accedere all’intera potenza espressiva del
linguaggio Lisp e poter generare codice a partire dai parametri forniti in
input a queste durante l’esecuzione di un programma.\\

\subsection{Come vengono utilizzate}

Il principale strumento per la definizione di macro in Common Lisp è
rappresentato una macro stessa: \texttt{DEFMACRO}. Una form \texttt{DEFMACRO}
ricalca molto da vicino la struttura di una form \texttt{DEFUN}, il più comune
costrutto utilizzato per la definizione di funzioni. Infatti, sia
\texttt{DEFUN} che \texttt{DEFMACRO} specificano:

\begin{itemize}

\item un nome con il quale verrà internata la macro in definizione all’interno
  dell’environment Lisp,

\item una lista di nomi dei parametri in input alla macro,

\item il corpo della macro stessa, che nel caso di \texttt{DEFUN} opera una
  computazione generica a partire dai parametri di input, nel caso di
  \texttt{DEFMACRO} opera necessariamente una trasformazione tra form.

\end{itemize}

\begin{lstlisting}[caption=Signature della macro DEFMACRO]
(defmacro name (argument ...) body)
\end{lstlisting}

A differenza di \texttt{DEFUN}, per la quale è ammesso che vengano ritornati
uno o più valori, \texttt{DEFMACRO} è vincolata a ritornare un unico elemeneto:
un form che rappresenta il risultato dell'espansione del parametro body fornito
in input. Un’ulteriore differenza, più complessa dal punto di vista
concettuale, è rappresentata dal fatto che i parametri che vengono forniti ad
una macro non vengono valutati, a differenza di quanto avviene nel caso
dell’invocazione di una funzione. Ad esempio, fornendo in input la form
\texttt{(+ 1 2)} ad una macro, il parametro body della macro verrà definito
come la lista \texttt{(+ 1 2)} e non come il valore 3.\\

Il metodo più semplice per generare una form nel corpo di una macro è
utilizzando la backquote reader macro \texttt{`}. Questa macro si comporta in
maniera molto simile al più noto quote \texttt{‘}, ad eccezione dei casi in cui
si ha un’occorrenza del simbolo comma \texttt{,}.

Il simbolo \texttt{COMMA} è ammesso dalla sintassi Lisp tradizionale solamente
all’interno di una backquoted form, ossia una form prefissa dal simbolo
backquote. In caso di utilizzo del simbolo nel contesto di una quoted form,
ossia una form prefissa dal simbolo quote, Lisp segnalerà un errore al momento
della lettura della form.

In maniera analoga a \texttt{QUOTE}, \texttt{BACKQUOTE} previene la valutazione
di una form. L’utilizzo del simbolo \texttt{COMMA} consente invece di fare in
modo che la valutazione di una subform avvenga nonostante l’utilizzo del
simbolo \texttt{BACKQUOTE}.

\begin{lstlisting}
? `(1 plus 2 is ,(+ 1 2)) (1 PLUS 2 IS 3)
\end{lstlisting}

Se si confronta il precedente esempio, che utilizza una combinazione dei
simboli \texttt{BACKQUOTE} e \texttt{COMMA}, e il seguente esempio, definito in
maniera molto simile, che utilizza però il simbolo \texttt{QUOTE}, è piuttosto
semplice comprendere il significato di quanto appena esposto.

\begin{lstlisting}
? '(1 plus 2 is ,(+ 1 2)) (1 PLUS 2 IS (+ 1 2))
\end{lstlisting}

Risulta abbastanza naturale immaginare come \texttt{BACKQUOTE} e \texttt{COMMA}
possano quindi fornire uno strumento per la definizione di template che
consentano la sostituzione di elementi.\\

Quanto appena riportato raccoglie le informazioni fondamentali alla definizione
di una macro. In seguito viene invece analizzato il momento successivo alla
definizione, ossia il processo di espansione di una macro.

\subsection{Espansione di una macro}

Quando la funzione \texttt{EVAL} riceve in input una lista il cui CAR (primo
elemento) è un simbolo, Lisp procede con la verifica della presenza del simbolo
tra le definizioni locali all’esecuzione (\texttt{FLET}, \texttt{LABEL},
\texttt{MACROLET}). Se questa ricerca non ottiene risultati, Lisp verifica
invece la presenza di definizioni a livello globale per il simbolo. Se una
delle due ricerche ha successo e il simbolo viene matchato con l’identificativo
associato ad una macro, la form viene identificata come una chiamata a macro.

A questo punto viene utilizzata una funzione definita a livello di linguaggio,
detta funzione di espansione. Questa funzione viene invocata con l’intera macro
come primo parametro e con un environment come secondo parametro. Come
affermato in precedenza, questa funzione deve ritornare una nuova form Lisp,
che viene detta espansione della chiamata a macro.

Una volta che la funzione di espansione ha prodotto in output la nuova form,
questa viene valutata al posto della form originale e il risultato di questa
valutazione viene ritornato come risultato della chiamata a macro complessiva
originale.

% Come detto, il corpo della macro ritorna una form, form che verrà valutata
% solo in un momento successivo all’esecuzione della macro, ossia al processo 
% di espansione. Quindi, quando Lisp valuta una chiamata ad una macro viene 
% prima di tutto espanso il corpo della macro definita, come illustrato nei 
% paragrafi precedenti, e solamente in seguito, una volta definito il corpo 
% espanso, verrà valutato il risultato complessivo dell’esecuzione.

Il processo appena illustrato rappresenta una semplificazione del reale
processo di espansione di una macro ma rimane comunque corretto dal punto di
vista concettuale e consente comunque di comprendere il funzionamento generale
del processo in maniera più semplice e soprattutto evidenzia delle potenzialità
di una macro come strumento per la programmazione.\\

L’aspetto probabilmente più importante trascurato dalla precedente spiegazione
riguarda il momento in cui viene espansa una form Lisp. Lisp prevede infatti
che sia possibile ritardare il processo di espansione delle macro anche ad un
momento successivo alla compilazione. Nel prossimo paragrafo vengono indicati i
momenti in cui è possibile che una macro venga espansa.

Come detto, una macro potrebbe essere espansa una sola volta, quando il
programma viene compilato. Altrimenti potrebbe essere espansa al primo utilizzo
di questa durante dell’esecuzione del programma, e questa espansione potrebbe
essere memorizzata per consentirne un riutilizzo più efficiente nel caso in cui
si dovessero verificare chiamate successive. Infine, una macro potrebbe essere
espansa ad ogni invocazione. Una macro definita in modo corretto dovrebbe
essere in grado di agire in modo corretto in ciascuna queste diverse
situazioni.

\begin{itemize}

\item Essendo che una macro potrebbe essere espansa in momenti diversi nel ciclo di vita di un programma, questa dovrebbe essere scritta in maniera tale da dipendere il meno possibile dall’ambiente in cui viene eseguita per ottenere una corretta espansione.

\item Inoltre, per assicurarsi un comportamento consistente in diverse esecuzioni, sarebbe meglio assicurarsi del fatto che tutte le definizioni di macro siano disponibili, all’interprete o compilatore, prima che si verifichi una qualsiasi chiamata a queste nel codice del programma.

\item Essendo che una macro potrebbe potenzialmente dover essere espansa già in fase di compilazione, è necessario che un compilatore Lisp, abbia già internato la definizione di una macro prima del primo utilizzo di questa all’interno del codice. Se così non dovesse essere si verificherebbe infatti un errore, in quanto il compilatore sarebbe in grado di espandere la macro al momento corretto.

\item Si è scelto di sottolineare questa differenza in quanto questa è particolarmente caratteristica del costrutto macro e non in linea con le modalità di compilazione e valutazioni del resto del linguaggio.

\end{itemize}

Un aspetto particolarmente interessante rispetto al funzionamento delle macro
Common Lisp è rappresentato dal fatto che macro e funzioni in Lisp non sono in
alcun modo interscambiabili. Come presentato nei paragrafi precedente infatti,
il processo di valutazione di una macro si compone di passaggi diversi rispetto
a quelli che contraddistinguono la valutazione di una funzione, differenza che
rende impossibile il riutilizzo delle une al posto delle altre.

Una conseguenza rilevante di questa impossibilità di sostuire macro a funzioni
e vicersa è il fatto che, nonostante possa sembrare sensato in prima battuta,
una macro non può essere utilizzata come parametri di una higher order
function, come ad esempio \texttt{APPLY}, \texttt{FUNCALL} o \texttt{MAP}. In
queste situazioni, la lista che rappresenta la chiama a macro originale infatti
non esisterà, e non può esistere, questo perché, in una certa misura, è come se
i parametri di questa fossero già stati valutati.

\section{Environments}

Un \textit{environment} Common Lisp è definito come un oggetto che raccoglie
un insieme di binding, e più generalmente informazioni riguardo a variabili,
funzioni, simboli e altri elementi utilizzati da un programma Lisp. Le
informazioni contenute in un environment possono essere utili a diversi scopi
tra cui espansione di macro e, più in generale, valutazione e compilazione di
form. L’operatore macroexpand, ad esempio, prende in input un parametro
opzionale che rappresenta l’environment all’interno del quale una macro verrà
e spansa.\footnote{http://franz.com/support/documentation/current/doc/environments.htm}\\

La definizione riportata dal paragrafo precedente fa riferimento alla
documentazione di una delle più note ed impiegate implementazioni dello
standard Common Lisp. Si è scelto di riportare questa particolare definizione
in quanto risulta essere sufficientemente generica ed articolata da riassumere
e approssimare in modo ragionevole il funzionamento generale degli
environments a prescindere dalla particolare implementazione. Purtroppo, non è
possibile citare una definizione standard di environment, in quanto la
standard Common Lisp non tratta in alcun modo questa tematica. In seguito
vengono spiegate le motivazioni di questa lacuna.\\

L’ANSI X3J13 è il comitato tecnico, nato nel 1986, responsabile per la
definizione dello standard ANSI Common Lisp. La formalizzazione operata del
comitato si è basata in gran parte sul contenuto del libro “Common Lisp: the
Language” di G.L. Steele \cite{steele1984common}, spesso abbreviato nella
comunità Lisp con la sigla \textit{CLtL1}. Prima che cominciassero i lavori
del comitato, il libro, pubblicato nel 1984, ha infatti per anni rappresentato
uno standard \textit{de facto} per le diverse implementazioni del linguaggio.
Ad oggi, la più recente ri-edizione del libro \cite{steele1990common}, datata
1990, rappresenta uno dei riferimenti di maggiore importanza per la comunità,
tanto che la versione definitiva dello standard, pubblicata nel 1994,
attribuisce tanta importanza al libro da suggerire agli implementatori
l’utilizzo dei simboli \texttt{:ctlt1} e \texttt{:ctlt2} per consentire
l’interoperabilità tra implementazioni ANSI Common Lisp e altri dialetti che
fanno invece maggiore riferimento alle diverse edizioni del libro.

Rispetto alla seconda edizione del libro, lo standard definitivo riporta delle
significative variazioni, sia in termini additivi e che sottrattivi. Data la
rilevanza che il libro ha all’interno della comunità Common Lisp, la gran
parte delle implementazioni del linguaggio più diffuse supportano l’API
descritta dal libro. Tuttavia, la mancanza di una formalizzazione all’interno
dello standard ha portato all’introduzione di variazioni e peculiarità tipiche
di ciascuna implementazione, variazioni che complicano un lavoro che voglia
mantenere una portabilità da implementazione a implementazione e che verranno
approfondite nelle Sezioni a seguire.\\

Tra gli elementi non approvati dallo standard ANSI per l’aggiunta al
linguaggio ma presenti all’interno di \textit{CLtL2}, si trova proprio la
definizione di environment Common Lisp e la proposta di un'API per
l’interazione con questo tipo di oggetto, uno degli elementi più importanti
per la creazione della libreria CLAST. Nei prossimi paragrafi viene presentato
il concetto di environment in Common Lisp a partire da quanto descritto da
CLtL2 e dalla documentazione disponibile in accompagnamento alle diverse
implementazioni del linguaggio.\\

ESSENDO CHE OGNI TANTO SI DICE CHE LO STANDARD NON FA ALCUN RIFERIMENTO AGLI
ENVIRONMENT, IN ALTRI CONTESTI PERÒ SE NE PARLA E SI DICE CHE UN UTENTE NON
DEVE ESSERE IN GRADO DI ACCEDERE DIRETTAMENTE.

\subsection{Environments API}

Uno degli aspetti formalizzati dallo standard ANSI Common Lisp rispetto agli
environment specifica che un utente del linguaggio non debba essere in grado
in alcun modo di accedere o modificare direttamente un oggetto environment.
Per questa ragione, molte implementazioni utilizzano degli oggetti immutabili
come meccanismo di rappresentazione di un environment. Questo aspetto risulta
perfettamente in linea con l’API descritta da CLtL2, la quale descrive un
insieme di funzioni che consentono l’accesso ai contenuti memorizzati
dall’environment e una funzione per l’aggiunta controllata di informazioni a
questo.

Le funzioni che compongono l’API appena citata sono sette, in particolare,
CLtL2 descrive come insieme minimo di funzioni quattro funzioni che consentono
l’interazione con oggetti environment: tre funzioni che consentono l’accesso
alle informazioni presenti all’interno di un oggetto environment, \texttt
{VARIABLE-INFORMATION}, \texttt{FUNCTION-INFORMATION} e \texttt
{DECLARATION-INFORMATION}, e un costruttore \texttt{AUGMENT-ENVIRONMENT}. Le
altre tre funzioni che compongono l’interfaccia suggerita dal libro sono
chiamate \texttt {PARSE-MACRO}, \texttt{ENCLOSE} e \texttt{DEFINE-
DECLARATION}. In seguito viene presentata un breve descrizione relativa al
funzionamento di ciascuna delle funzioni indicate, in maniera tale da
consentire al lettore di meglio comprendere il supporto tipicamente offerto
dal linguaggio alla creazione delle libreria soggetto di questa tesi.\\

Le funzioni \texttt{VARIABLE-INFORMATION}, \texttt{FUNCTION-INFORMATION} e
\texttt{DECLARATION-INFORMATION} forniscono l’accesso alle informazioni
relative alle dichiarazioni attualmente presenti all’interno di un
environment, memorizzate all’interno dell’oggetto che rappresenta questo
attraverso l’utilizzo della funzione \texttt{AUGMENT-ENVIRONMENT}, o anche
automaticamente aggiunte dall’interprete o dal compilatore del linguaggio.
Ciascuna di queste funzioni può essere eseguita fornendo in input un parametro
environment opzionale. Nel caso in cui questo non non venga specificato
infatti, la funzione utilizzerà semplicemente l’environment lessicale vuoto
come valore di default. In seguito viene presentato il funzionamento di
ciascuna di queste funzioni ad un livello di dettaglio maggiore.

\subsubsection{Funzioni di base}

\texttt{VARIABLE-INFORMATION} ritorna le informazioni riguardanti
l’interpretazione del simbolo fornito in input come simbolo associato ad una
variabile presente all’interno dell'environment lessicale in input. Queste
informazioni sono rappresentate da un valore che indica il tipo della
definizione, o del binding, presente per la variabile all’interno
dell’environment (speciale, lessicale, costante, \dots), un valore che indica
se sia stato trovato o meno un binding per la variabile ed un valore che
riporta informazioni aggiuntive rispetto alla variabile in analisi, come ad
esempio il tipo e se questa sia stata dichiarata utilizzando \texttt
{DYNAMIC-EXTENT} o \texttt{IGNORE}.\\

\texttt{FUNCTION-INFORMATION} ritorna informazioni relative
all’interpretazione della simbolo fornito in input come nome di funzione, nel
caso in cui questo appaia come funzione all’interno dell’environment in input.
Anche questa funzione ritorna tre valori: un primo valore utilizzato per
indicare il tipo di definizione o binding presente per la funzione all’interno
dell’environment (funzione, macro, form speciale o assente, un secondo valore
che specifica se la funzione sia locale o globale ed un terzo valore che
specifica informazioni aggiuntive rispetto alla funzione, identificate da Lisp
all’interno dell’ambiente, come ad esempio se questa sia dichiarata
utilizzando \texttt {DYNAMIC-EXTENT}, se sia stato richiesto l’inling o meno
della funzione e il tipo di questa.\\

\texttt{DECLARATION-INFORMATION} ritorna le informazioni relative ad una
dichiarazione che utilizza come nome il simbolo fornito in input presenti
all’interno del environment in input. Questa funzione consente di analizzare
tutte le definizioni relative ad elementi diversi da variabili e funzioni che
possono essere presenti all’interno di un environment.\\

La funzione \texttt{AUGMENT-ENVIRONMENT} rappresenta lo strumento fondamentale
per l’analisi necessaria alla libreria CLAST. \texttt{AUGMENT-ENVIRONMENT}
consente di produrre un nuovo oggetto environment a partire da un input
rappresentato da un environment opzionale, a partire dal quale si desidera che
venga generato il nuovo oggetto, e da una o più liste di definizioni di
variabili, funzioni, macro e dichiariazioni. La funzione produrrà quindi in
output un oggetto environment che raccoglierà le informazioni contenute
dall'environment fornito in input, nel caso in cui questo sia stato utilizzato
un parametro di questo tipo in fase di invocazione, e le informazioni generate
della nuove dichiarazioni specificate.\\

La funzione \texttt{AUGMENT-ENVIRONMENT} rappresenta lo strumento fondamentale
per l’analisi necessaria alla libreria CLAST. \texttt{AUGMENT-ENVIRONMENT}
consente sia di produrre un nuovo oggetto environment, sia di aggiungere
informazioni ad un'instanza di environment già esistente. La funzione lavora a
partire da un input rappresentato da un environment opzionale, presente nel
caso in cui si desideri aggiungere informazioni ad un environment piuttosto
che crearne uno nuovo, e da una o più liste di definizioni di variabili,
funzioni, macro e dichiariazioni. La funzione produce quindi in output un
oggetto environment che raccoglie le informazioni contenute dall'environment
fornito in input, nel caso in cui questo sia stato utilizzato un parametro di
questo tipo in fase di invocazione, e le informazioni generate dall'analisi
delle dichiarazioni specificate.\\

\subsubsection{Macro accessorie}

A differenza delle funzioni appena presentate, che hanno come obiettivo quello
di consentire ad un utente l'interazione con un oggetto environment, in questo
paragrafo vengono presentate delle macro che hanno il particolare obiettivo di
fornire un'implementazione di alcune delle funzionalità fondamentali per la
creazione di programmi in grado di analizzare programmi Lisp e per
l'estensione dell'API offerta dalle funzioni descritte dal paragrafo
precedente.\\

\texttt{DEFINE-DECLARATION} é una macro che consente di estendere e introdurre
un supporto a nuove tipologie di dichiarazione per la memorizzazione
attraverso \texttt {AUGMENT-ENVIRONMENT}; operazione che viene compiuta ad
esempio dall’implementazione Allegro Common Lisp, la quale introduce due nuove
ulteriori definizioni \texttt{BLOCK} e \texttt{TAG}. In generale, questa
funzione risulta utile soprattutto internamente agli implementatori del
linguaggio, o a quegli utenti che desiderano estendere il linguaggio con nuovi
costrutti.\\

La funzione \texttt{PARSE-MACRO} consente di destrutturare i parametri di una
macro, un compito molto spesso necessario e generalmente realizzabile a
partire dalla macro \texttt{DESTRUCTURING-BIND}, la quale consente utilizzare
gli elementi di una lista per inizializzare un insieme di variabili. L’autore
indica come ragione per cui si è scelto di aggiungere questa funzione il fatto
che qualsiasi programma desideri analizzare del codice Lisp dovrà molto
probabilmente definire una funzione analoga e, nonostante la presenza di
\texttt{DESTRUCTURING-BIND}, l’implementazione di questa funzionalità non è
del tutto banale. \cite{steele1990common} Dal punto di vista pratico, \texttt
{PARSE-MACRO} è strutturata in maniera del tutto analoga ad una
\texttt{DEFMACRO}, accetta infatti i medesimi parametri nel medesimo ordine, a
livello di output invece ritorna una lambda- expression che prende in input
due valori, una form ed un environment, e che potrà essere utilizzata per
aggiungere informazioni ad un ambiente fornito in input alla funzione
lambda.\\

Infine, la funzione \texttt{ENCLOSE} consente di espandere funzioni definite
all’interno di un lexical environment e consente quindi l’analisi, da parte di
un programma Lisp, di un programma Lisp che fa utilizzo di una form del tipo
\texttt{(eval-when (:compile-toplevel) …)} attraverso l’esecuzione all’interno
dell’environment che contiene questa. Dal punto di vista pratico,
\texttt{ENCLOSE} è una funzione che lavora a partire da un input,
rappresentato da una coppia lambda-expression ed environment, e ritorna un
oggetto di tipo funzione equivalente a quello che si otterebbe valutando la
form \texttt{`(function ,lambda-expression)} nell'environment sintattico
fornito in input. Affinchè la valutazione possa avere successo è necessario
che la form faccia riferimento solamente a simboli definiti all'interno
dell'environment fornito in input assieme ad essa.


\endgroup

\begingroup
\let\clearpage\relax

\chapter{CLAST}
\label{library}

Lo scopo di questo capitolo è quello dipresentare al lettore il funzionamento e
l'interfaccia della libreria CLAST, mostrando quindi sia l'architettura interna
della libreria sia le funzionalità offerte da questa ad un utente.

Internamente la libreria è organizzata in moduli, in particolare, all'interno di
tre diversi moduli chiamati rispettivamente: \textit{modulo di
rappresentazione}, \textit{modulo di parsing e analisi} e \textit{modulo di
traversal}.\\

Il modulo di rappresentazione è responsabile per la definizione delle strutture
necessarie alla modellazione di un programma Common Lisp. Si tratta quindi del
modulo fondamentale a partire dal quale la libreria è in grado di realizzare una
rappresentazione in forma di AST del programma in input, a partire anche
dall'azione del modulo di parsing e analisi. La progettazione e la struttura
interna di questo modulo vengono illustrate e discusse all'interno della Sezione
\ref{representation}.\\

All'interno della Sezione \ref{parsing} viene invece presentato il modulo di
parsing ed analisi. Questo modulo rappresenta il cuore del funzionamento della
libreria in quanto realizza la funzionalità vera e propria di generazione
dell'AST di un programma scritto utilizzando il linguaggio Common Lisp a partire
dalle strutture di rappresentazione offerte dal modulo di rappresentazione
sopraccitato.

Lo scopo del modulo è quindi quello di svolgere le attività di parsing del
codice sorgente del programma in input, in maniera tale da organizzare i diversi
elementi di questo all'interno di un AST, e di svolgere parallelamente
un'attività di analisi del funzionamento di questo, mirata ad arricchire con
maggiori informazioni i nodi che compongono la struttura ad albero producendo
quindi la rappresentazione più ricca possibile del programma in input. Nel corso
della sezione vengono quindi presentate le principali componenti del modulo e
l'interazione tra queste attraverso un esempio.\\

L'ultima sezione di questo capitolo presenta invece il modulo di traversal
offerto dalla libreria, il quale raccoglie le diverse funzionalità costruite a
partire a partire dalla rappresentazione prodotta in output dal modulo di
parsing ed analisi. La responsabilità principale del modulo è quella di
consentire agli utenti l'ispezione e l'esecuzione di elaborazioni dell'AST,
prodotto a partire dalle strutture denifinite dal modulo di rappresentazione, di
un programma Common Lisp. Questo modulo costituisce quindi la principale
interfaccia attraverso la quale un sistema costruito a partire dalla libreria è
in grado di accedere ai servizi offerti da questa.

\image{img/clast-architecture.png}
      {Componenti della libreria CLAST}
      {fig:clast-architecture}
      {0.5}

\section{Modulo di rappresentazione}
\label{representation}

In questa sezione viene approfondita la struttura e il contenuto del modulo di
rappresentazione presente all'interno della libreria CLAST. In particolare,
vengono presentate e discusse le scelte compiute e il procedimento utilizzato
per la definizione delle diverse classi utilizzate per la rappresentazione di un
programma Common Lisp. Classi che verranno poi utilizzate dal modulo di parsing
ed analisi in fase di generazione dell'AST associato al programma target
dell'azione della libreria.\\

La struttura del modulo di rappresentazione ricalca molto da vicino il processo
di valutazione del codice sorgente di un programma Common Lisp. Questo allo
scopo di ottenere delle strutture che forniscano una rappresentazione molto
vicina a quella utilizzata internamente da un reale sistema Common Lisp, ossia
la più ricca possibile. Tutto questo mantenendo sempre il riferimento al codice
sorgente vero e proprio, in maniera tale da ottenere sia informazioni rispetto
alla semantica del programma a livello macchina, sia informazioni rispetto alla
semantica desiderata dall’autore del programma.

\subsection{Processo di valutazione}
\label{CL-valutazione}

Come affermato nel precedente paragrafo, le componenti facenti parte del
modulo di rappresentazione sono particolarmente legate al processo di
valutazione utilizzato dal linguaggio. Per questa ragione, allo scopo di
fornire al lettore una migliore comprensione degli elementi che verranno
presentati nel dettaglio in seguito, in questa sottosezione viene presentato
un breve sunto del processo di valutazione e delle strutture fondamentali di
un programma scritto utilizzando il linguaggio di programmazione Common
Lisp.\\

La struttura fondamentale alla base del processo di valutazione di un programma
Lisp è la form. Una \textit{form} viene infatti definita formalmente dallo
standard ANSI Common Lisp proprio come \textit{“an object meant to be
evaluated”}. Dal punto di vista pratico una form può essere costituita sia da un
atomo che da una lista. A partire da questo elemento particolarmente generico
vengono poi definite delle particolari tipologie di form, a seconda delle
modalità con le quali procede la valutazione operata da un sistema Common Lisp
nel momento in cui queste vengono incontrate.

\subsubsection{Form atomiche: self-evaluating objects}

Il caso in cui una form è costituita da un atomo rappresenta il caso più
semplice del processo di valutazione. Una volta identificato un atomo infatti,
il sistema di valutazione deve semplicemente verificare se questo rappresenti un
oggetto auto-valutante, \textit{self-evaluating object}, oppure un simbolo.

Nel caso di un self-evaluating object, il sistema Lisp si limita a produrre
come risultato l’oggetto stesso, come il nome stesso di questa struttura
suggerisce. Esempi di self-evaluating objects sono rappresentati da numeri,
sia interi che complessi, stringhe, pathnames e array.

\begin{lstlisting}[caption=Esempi di form di tipo self-evaluating object]

CL-USER > 3
3
CL-USER > #c(2/3 5/8)
#C(2/3 5/8)
CL-USER > #p"S:[BILL]OTHELLO.TXT"
#P"S:[BILL]OTHELLO.TXT"
CL-USER > #(a b c)
#(A B C)
CL-USER > "fred smith"
"fred smith"

\end{lstlisting}

\subsubsection{Form atomiche: simboli}

Nel caso in cui un atomo non sia un self-evaluating object, il sistema Lisp lo
identifica automaticamente come simbolo. A sua volta, un simbolo può
rappresentare o una symbol macro, una form utilizzata in sostituzione di
un’altra form, o una variabile.

La prima operazione compiuta da un sistema Common Lisp aderente allo standard in
questo caso è una verifica della presenza o meno di un binding relativo ad una
macro all’interno dell'ambiente lessicale attivo al momento della valutazione.
Se viene effettivamente identificato un binding che utilizza il simbolo come
nome, viene applicata la funzione associata dalla definizione di macro. Questo
al fine di produrre una form che verrà valutata al posto del simbolo stesso,
come descritto durante la discussione del processo di macro-espansione
illustrato nella Sottosezione \ref{macroexpansion}.

Nel caso in cui non sia presente una definizione di macro all’interno
dell’ambiente, il sistema Lisp assume che il simbolo rappresenti una variabile.
Il valore della variabile associata al simbolo viene quindi ricercato e prodotto
come output del processo di valutazione.\\

\begin{lstlisting}[caption=Esempio di form di tipo symbol]

CL-USER > *STANDARD-INPUT*
#<SWANK/GRAY::SLIME-INPUT-STREAM {1004F2D653}>

\end{lstlisting}

Come affermato, nel caso in cui il simbolo non sia stato identificato come
symbol-macro, il sistema assume che questo rappresenti una variabile, nel caso
in cui questo non lo sia, o più generalmente in cui non sia stato associato
alcun valore alla variabile referenziata dal simbolo, un sistema Common Lisp
aderente allo standard si limita a segnalare un errore di tipo \texttt
{UNBOUND-VARIABLE}.

Sia il modulo di rappresentazione che il modulo di parsing, seguono questa
stessa convenzione, il che porta ad ottenere una rappresentazione corretta del
comportamento di un programma che, a runtime, manifesterà un comportamento
scorretto e un fallimento. Questo significa però che il risultato prodotto dal
lavoro del modulo di parsing e analisi sarà una rappresentazione che presenterà
sufficienti informazioni da consentire l’individuazione e la segnalazione del un
comportamento potenzialmente scorretto; esattamente ciò che risulta più
interessante per gli strumenti di source code analysis per i quali CLAST si pone
come infrastruttura abilitante.\\

\subsubsection{Form composte: function, macro, lambda e special forms}

Nel caso invece in un una form non sia costituita da una variabile, ma piuttosto
da una lista, questa prende il nome di compound form. In questo caso il processo
di valutazione prosegue in modo significativamente più approfondito. Ciascuna
compound form viene infatti scomposta in due componenti: un operatore,
rappresentato dal simbolo di testa della lista, detto \texttt{CAR} della lista,
e una lista di parametri, rappresentata da una nuova lista contenente tutti gli
elementi della lista originale ad eccezione dell’operatore, detta \texttt{CDR}.
Il sistema Lisp procede quindi all'analisi del simbolo operatore, ricercando
eventuali associazioni tra questo ed elementi all’interno ambiente lessicale
corrente alla valutazione e all'interno dell'ambiente globale. In base al
risultato prodotto da questa ricerca la compound form viene classificata
all’interno di una delle seguenti 4 categorie: \textit{special form},
\textit{macro form}, \textit{function form} o \textit{lambda form}.

\begin{lstlisting}[
  caption=Esempio di estrazione degli elementi operatore e parametri di una
  compound form,
  label={lst:car-cdr}
]

CL-USER > (car '(+ 1 2))
+                        ; Operator: CAR is a symbol
CL-USER > (cdr '(+ 1 2))
(1 2)                    ; Params: CDR is a list

\end{lstlisting}

Il processo di valutazione si ramifica quindi a seconda di come la form sia
stata classificata al passo precedente.

\begin{itemize}

\item Nel caso in cui l’operatore venga identificato come il nome di una
funzione e quindi la compound form sia stata classificata come function form,
questa viene invocata utilizzando come input la lista dei parametri identificata
al passo precedente.

\item Nel caso in cui l’operatore venga identificato come nome di una macro e
quindi la compound form sia stata classificata come macro form, la valutazione
prosegue dando inizio al processo di valutazione delle macro illustrato nel
corso del capitolo precedente, Sottosezione \ref{macroexpansion}.

\item Nel caso in cui l’environment in cui viene eseguita la ricerca non
contenga alcuna definizione per il simbolo operatore, la compound form viene
identificata come lambda form. L’operatore viene quindi utilizzato come una
funzione e la valutazione avviene in maniera analoga a quanto riportato nel caso
di una function form.

\end{itemize}

Il caso più particolare dal punto di vista del processo di valutazione si
verifica quando, a partire dall’analisi dell’operatore, una compound form viene
identificata dal sistema come una special form. Il maggior interesse per
compound form di tipo special form è legato al fatto che form di questo tipo
possono utilizzare o una sintassi speciale, o regole di valutazione speciali, o
entrambi, oltre ad essere potenzialmente in grado di compiere modifiche
dell’environment all’interno del quale vengono valutate, oltre che più
generalmente del flusso di controllo stesso del programma.

Questo significa che la valutazione di una compound form di questo tipo può
produrre nuovi environment lessicali o dinamici all’interno dei quali verranno
valutate eventuali sotto-form innestate a questa.

Ad esempio, una compound form che utilizza l’operatore \texttt{LET}, dichiara un
nuovo environment lessicale, identico a quello di invocazione, all’interno del
quale vengono aggiunti nuovi binding rispetto a nomi di variabile. Questo
significa che form di tipo special form avranno un impatto particolarmente
significativo sul funzionamento del modulo di parsing e analisi, in quanto, ad
esempio, avendo la possibilità di utilizzare una sintassi speciale, e
rappresenterà quindi potenzialmente una nuova regola della grammatica target del
parser.\\

Questo conclude la breve panoramica del processo di valutazione di un programma
Common Lisp e delle strutture fondamentali che caratterizzano questo processo
offerta da questa sottosezione. Panoramica che ha mostrato le modalità con il
quale opera il processo di valutazione di un programma Lisp rispetto alle
principali componenti di questo.

Nel corso della prossima sottosezione verrano quindi approfonditi meccanismi che
la libreria CLAST utilizza per la rappresentazione di queste strutture al fine
di facilitare i compiti di analisi del codice sorgente di un programma.

\subsection{Struttura del modulo}

Dopo aver brevemente presentato il processo di valutazione utilizzato da un
sistema Common Lisp e le strutture fondamentali utilizzate da questo processo,
si presentano i meccanismi che vengono forniti e utilizzati dalla libreria CLAST
allo scopo di rappresentare un programma Lisp.

Come precedentemente riportato, i meccanismi di rappresentazione ricalcano
molto da vicino le informazioni prodotte e raccolte da un sistema Lisp durante
il processo di valutazione. Questo perché che la rappresentazione che la
libreria vuole offrire è la più ricca possibile e quello della valutazione è
il momento in cui viene prodotto e raccolto il maggior numero di informazioni
rispetto alla semantica del programma in analisi.\\

Dal punto di vista pratico, la rappresentazione fornita dalla libreria è
strutturata all’interno di un grande insieme di classi. Classi organizzate
all’interno di una gerarchia, sfruttando il supporto offerto dal linguaggio
Common Lisp all’ereditarietà e in particolare all’ereditarietà multipla.

\image{img/clast-representation.png}
      {Modulo di rappresentazione - Diagramma delle classi (ridotto)}
      {fig:clast-representation}
      {0.5}

\subsubsection{CLAST-ELEMENT e FORM}

La classe fondamentale, al vertice della gerarchia delle strutture esposte da
CLAST, è la classe \texttt{CLAST-ELEMENT}. Questa classe ha il semplice scopo
di raccogliere le diverse strutture offerte dalla libreria all'interno di un
unico tipo, in maniera tale da facilitare l’ispezione e l’analisi di oggetti
prodotti dalla libreria. Per questa ragione la classe non dichiara alcun
attributo e nessun metodo viene specializzato rispetto a questa.\\

La prima classe concreta all’interno della gerarchia è la classe \texttt{FORM}.
Lo scopo di questa classe è quello di rappresentare i dettagli fondamentali di
qualsiasi elemento presente all’interno di un programma Common Lisp e, in
particolare ,fungere da nodo, ossia unità fondamentale, della rappresentazione
mediante AST fornita dalla libreria.\\

Per fare questo, la classe \texttt{FORM} espone quindi tre slot, attributi,
fondamentali.

\begin{itemize}

\item Un attributo \texttt{SOURCE} che riporta il codice sorgente associato al
nodo, form, in analisi. La presenza di questo attributo ha lo scopo di
facilitare il compito di un analizzatore che lavora a partire dalla libreria,
interessato in particolare modo ad aspetti testuali del codice sorgente.

\item Un attributo \texttt{TOP} indica il nodo \texttt{FORM} all’interno della
quale questa istanza è innestata. Questo attributo risulta di fondamentale
importanza, sia per gli utenti della libreria, sia per la libreria stessa, in
quanto è l'elemento che consente di rappresentare un programma all’interno di
una struttura ad albero, un AST, come anticipato dal Capitolo
\ref{abstract-syntax-tree}, e di implementare una navigazione all'interno di
questa struttura. Questo consente di ottenere una rappresentazione
universalmente nota e per la quale le operazioni di traversal risultano
particolarmente semplici.

\item Un attributo \texttt{TYPE} riporta invece il tipo dichiarato, o
potenzialmente inferito, del nodo o form in analisi. Questo allo scopo di
facilitare il lavoro di strumenti come type checkers, che cerchino di aggiungere
una tipizzazione statica ad un linguaggio di programmazione dinamico come il
Common Lisp.

\end{itemize}

\subsubsection{Mixin}

Come precedentemente affermato, il modulo fa utilizzo del supporto
all'ereditarietà multiplo offerto dal linguaggio. In particolare, questo il
meccanismo viene sfruttato allo scopo di definire un insieme di classi in grado
di agire da mixin.\\

Un \textit{mixin} o \textit{trait} viene definito come una classe che definisce
un insieme i metodi e/o attributi allo scopo di facilitare il riuso di questi da
parte di altre classi, senza però forzare la definizione di una relazione di
ereditarietà diretta tra questa e le classi che operano il riuso. Lo scopo di
soluzioni di questo tipo, implementate da diversi linguaggi di programmazione
con modalità e nomi differenti, è quindi fondamentalmente quello di facilitare
il riuso di codice, evitando allo stesso tempo i problemi legati alle ambiguità
che possono essere causate dall’impiego dell’ereditarietà multipla, legate ad
esempio al Diamond Problem\cite{martin1997}.\\

A differenza di quanto avviene nel contesto di altri linguaggi di
programmazione, come ad esempio il linguaggio Scala, un linguaggio che
specifica un costrutto dedicato esclusivamente alla definizione di mixin, in
Common Lisp un mixin viene definito attraverso una semplice definizione di
classe, in maniera del tutto analoga a quanto avverrebbe per la definizione di
una classe tradizionale.\\

Alla base della libreria vengono quindi definite altre due classi, oltre alla
classe \texttt{FORM} che, come anticipato, hanno lo scopo di agire da mixin.
La prima di queste classi è chiamata \texttt{IMPLICIT-PROGN}, la seconda è
chiamata \texttt{EXPANSION-COMPONENT}.

\subsubsection{IMPLICIT-PROGN e EXPANSION-COMPONENT}

Il costrutto \texttt{PROGN} è il costrutto fondamentale alla definizione di
codice imperativo in Common Lisp: valuta l’insieme di form fornite in input in
sequenza e ritorna il risultato prodotto dalla valutazione dell’ultima di
queste, scartando il risultato di tutte le precedenti. Lo scopo della classe
\texttt{IMPLICIT-PROGN} è quello di raccogliere attributi e metodi necessari
all’analisi di compound form che contengono una form di tipo \texttt{PROGN}
implicita. Essendo infatti l’utilizzo del costrutto \texttt{PROGN} presente in
modo implicito alla base del funzionamento di diversi altri costrutti, come ad
esempio \texttt{DEFUN} e \texttt{DEFMACRO}, si è scelto di isolare le
responsabilità e le strutture fondamentali al parsing e all’analisi di form di
questo tipo all’interno di questo mixin, allo scopo di facilitare il riuso
all’interno delle funzioni e delle strutture dedicate al parsing dei diversi
costrutti che utilizzano un \texttt{IMPLICIT-PROGN}.

\begin{lstlisting}[caption=Esempio di costrutto che compie un utilizzo del
costrutto \texttt{PROGN} in modo implicito]

(defun sum-and-log (x y)
  ;; Forms on these next two lines will be implicitly wrapped in a
  ;; PROGN form together and thus executed sequentially. When the
  ;; SUM-AND-LOG function the evaluation process will encounter the
  ;; first form, execute it and discard the result. Then, it will
  ;; execute the second form, and since this is the last form in the
  ;; implicit progn wrapper it will return the result of its
  ;; evaluation as the function evalutation result.
  (format t "~a plus ~a equals..." x y)
  (+ x y)
  )

\end{lstlisting}

I due slot fondamentali esposti da questa classe sono i seguenti.

\begin{itemize}

\item \texttt{IPROGN-FORMS} è un attributo utilizzato allo scopo di tenere
traccia delle form innestate all’interno di quella rappresentata da questa
istanza di CLAST-ELEMENT, ossia delle form che verranno implicitamente
eseguite all’interno di una form di tipo \texttt{PROGN}. Un secondo scopo di
questa classe, non meno importante del primo, è quello di mantenere la
relazione tra una form e le sue sotto-form. Ciò consentirà infatti la
navigazione all'interno del codice, in quanto mantiene traccia delle form
innestate all'interno dell'istanza corrente.

\item \texttt{BODY-ENV} è invece un attributo che riporta un oggetto di tipo
ambiente, il quale rappresenta l'ambiente all’interno del quale verrà eseguita
la valutazione delle form memorizzate dall’attributo \texttt{IPROGN-FORMS}. È
importante notare che, nel caso di molte special forms, questo attributo risulta
di fondamentale importanza agli scopi di uno strumento di analisi. Questo
consente infatti di osservare il reale ambiente all'interno del quale verrà
eseguita una determinata form.

\end{itemize}

La classe \texttt{EXPANSION-COMPONENT} è invece responsabile per la
definizione delle strutture e dei metodi che consentono il parsing di form
soggette al processo di valutazione tipico di una macro, presentato nel corso
dalla Sottosezione \ref{macroexpansion}.

In particolare, questa classe espone un altro attributo, chiamato \texttt
{FORM-EXPANSION}, fondamentale ad uno strumento per l’analisi di un programma
Lisp, ossia la risultato dell’espansione della macro rappresentata dal nodo.\\

\subsubsection{Strutture di dettaglio}

Le tre classi appena riportate rappresentano il substrato fondamentale della
rappresentazione offerta dalla libreria CLAST. Il livello di dettaglio offerto
dalla libreria è però molto maggiore rispetto a quello possibile utilizzando
solamente queste tre classi. La libreria dichiara infatti più di cento classi
che vengono utilizzate per rappresentare istruzioni ad un livello di dettaglio
di singolo operatore. Una discussione puntuale rispetto alla rappresentazione
offerta permessa da ciascuna di queste risulterebbe troppo estesa per essere
riportata all’interno di questa tesi e viene pertanto rimandata alla
documentazione della libreria.\\

Non sarebbe possibile e nemmeno utile, avere un classe specifica per la
rappresentazioni qualsiasi possibile operatore definito da un utente e
linguaggio. In particolare, vengono la libreria espone strutture per la
rappresentazione nel dettaglio di tutti gli operatori speciali indicati dallo
standard ANSI Common Lisp, Sezione 3.1.2.1.2 Listato 2, più tutti gli operatori
ritenuti di particolare interesse dal punto di vista dell’analisi non presentati
all’interno di quella lista.

L'insieme di questi operatori aggiuntivi consiste principalmente degli
operatori legati al CLOS, al meccanismo delle dichiarazioni e al costrutto
loop, considerati di maggiore interesse rispetto ad altri in quanto
particolarmente utilizzati dagli utenti del linguaggio Common Lisp e quindi
significativamente più rilevanti dal punto di vista dell’analisi.\\

Qualsiasi compound form viene tuttavia rappresentata con il maggiore grado di
precisione possibile, a seconda dello specifico caso in analisi. In presenza di
operatori non trattati in modo specifico dalla libreria, come un'applicazione
dell'operatore della compound form la libreria è in grado di rappresentare,
distinguendo, se si tratti dell’applicazione di una funzione o di una macro a
partire dalla ricerca dell'operatore all'interno dell'environment in uso.\\

Questa sezione ha presentato nel dettaglio la struttura e le componenti
fondamentali del modulo di rappresentazione presente nella libreria CLAST.
Nella prossima sezione verrà invece approfondito il funzionamento del modulo
dedicato a parsing e analisi del codice sorgente di un programma, mostrando
quindi come gli elementi del modulo di rappresentazione vengano realmente
impiegati dalla libreria stessa.

\section{Modulo di parsing}
\label{parsing}

Dopo aver approfondito, nel corso della Sezione \ref{representation}, le
strutture che la libreria CLAST offre per la rappresentazione di codice
sorgente, questa sezione ha l’obiettivo di presentare e discutere la
progettazione e l'implementazione del secondo modulo facente parte libreria. Il
modulo di parsing ed analisi è responsabile dello svolgimento delle reali
operazioni di analisi sintattica e semantica del codice sorgente fornito in
input alla libreria. Attività che produrranno come output la rappresentazione
del programma Common Lisp soggetto in forma di abstract syntax tree da cui la
libreria prende nome.\\

\subsubsection{Attività parsing}

Dal punto di vista strutturale, il modulo consiste di un parser a discesa
ricorsiva, una particolare tipologia di parser che verrà brevemente discussa nel
corso dei prossimi paragrafi, opportunamente esteso in maniera tale da
supportare le attività di analisi necessarie.

Formalmente, un \textit{parser a discesa ricorsiva} è definito come un parser
che opera in modo top-down, costruito a partire da un insieme di procedure
mutuamente esclusive, o equivalenti, in cui tipicamente ciascuna procedura
implementa una delle produzioni della grammatica associata al linguaggio target
dell’attività di parsing.

Il particolare parser presente all’interno della libreria CLAST può inoltre
essere definito, dal punto di vista degli analizzatori sintattici, come un
\textit{recursive predictive parser}: un parser a discesa ricorsiva in grado di
svolgere la propria attività di analisi e scomposizione senza necessità di
operare backtracking o, in altri termini, senza la necessità di svolgere e
abbandonare computazioni parziali.\\

I principali vantaggi legati alla scelta di organizzare il processo di parsing
come un parser di questo tipo sono principalmente due. Da un punto di vista
computazionale, la proprietà più interessante di un recursive predictive parser
è rappresentata dal fatto che questo è in grado di lavorare in tempo lineare
nella dimensione del suo input.

Un secondo vantaggio legato alla scelta di organizzare le attività di parsing
come un recursive predictive parser è invece più legato ad aspetti di qualità
del software e manutenibilità dell’implementazione prodotta. La forte
strutturazione del parser all'interno un insieme di procedure specifiche per
ciascuna produzione della grammatica del linguaggio in analisi, guida infatti
l’implementazione nella definizione di un insieme di funzioni altamente coese
nel rispetto dei principi di single responsibility e separation of concerns che
consentono di ottenere codice particolarmente espressivo, manutenibile e
facilmente testabile.\\

È importante sottolineare che, affinché sia possibile definire un predictive
parser per un dato linguaggio, è necessario che la grammatica di questo sia un
grammatica di tipo context-free, ossia una grammatica non ambigua e per la quale
vale la proprietà per cui è sempre possibile osservare un numero finito di token
dallo stream di input per poter determinare la corretta produzione della
grammatica da applicare. Fortunatamente, la grammatica del linguaggio Common
Lisp è una grammatica di tipo context-free.

\subsubsection{Strategie per la costruzione di un parser}

Un parser come quello appena descritto può essere costruito utilizzando un
approccio di tipo automatico oppure utilizzando un approccio manuale.\\

Approcci automatici alla creazione di un parser lavorano tipicamente a partire
da uno strumento specializzato nella generazione di sistemi software di questo
tipo, detti appunto generatori di parser. Un generatore di parser prende in
input una descrizione formale della grammatica del linguaggio target e produce
in output un programma in grado di operare il parsing di tale linguaggio.
Alternativamente, alcuni generatori di parser producono in output una struttura,
solitamente chiamata parse table, a partire dalla quale un programma driver
generico è in grado di operare il parsing.\\

Il vantaggio principale legato all'utilizzo di strumenti automatici è
chiaramente rappresentato dall’immediatezza con la quale questi consentono di
produrre strumenti efficienti. Un secondo vantaggio particolarmente
significativo è rappresentato dal fatto che, in molti casi, il parser generato
dallo strumento è corretto in modo formalmente dimostrabile.

Lo svantaggio fondamentale legato all'utilizzo di un generatore di parser è
dovuto al fatto che la loro estensione, al fine di supportare la raccolta di
informazioni aggiuntive, risulta molto spesso difficile se non impossibile.
Inoltre, l’utilizzo di questi strumenti non è possibile nel caso in cui non sia
disponibile una descrizione formale della grammatica del linguaggio target delle
operazioni di parsing, una descrizione che possa essere fornita in input allo
strumento.\\

Il requisito e lo svantaggio appena citati rendono l’applicazione di un
approccio automatico non applicabile al contesto del progetto soggetto di questa
tesi. Infatti, in prima battuta, non sono disponibili definizioni formali della
grammatica completa del linguaggio Common Lisp. Inoltre, ponendosi CLAST come
una soluzione per la creazione di strumenti di source code analysis di
molteplici tipologie, la possibilità di estendere il processo di parsing al fine
di raccogliere il maggior numero di informazioni possibili rispetto al codice
del programma in analisi risulta di fondamentale importanza. Informazioni che un
parser generato in modo automatico da uno strumento generico non è tipicamente
in grado di fornire.\\

Per queste ragioni, la scelta intrapresa è stata quella di implementare un
parser in modo manuale. Da un punto di vista prettamente legato all’analisi
sintattica che deve essere operata da un parser, il linguaggio Common Lisp ben
si presta alla costruzione di strumenti di questo tipo, essendo il codice Lisp
organizzato all’interno di liste. Tuttavia, questa attività ha comunque
necessitato di uno sforzo particolarmente significativo dal punto di vista
implementativo e del testing. Questo soprattutto a causa del grande numero di
costrutti che utilizzano una sintassi speciale utilizzati dal linguaggio e della
mancanza di una definizione formale che potesse automatizzare, almeno in parte,
la generazione del codice di implementazione e la ricerca di casi di test.

\subsubsection{Attività analisi}

Come precedentemente affermato la complessità di CLAST come strumento non è
solamente rappresentata dalla fase di parsing, ma soprattutto dalla fase di
raccolta di informazioni aggiuntive, fase che viene svolta parallelamente al
parsing.

Infatti, molte delle informazioni utili ed interessanti per strumenti di source
code analisys non sono immediatamente ottenibili dalla rappresentazione in forma
di abstract syntax tree prodotta dalla sola attività di parsing di un programma
e realizzata, nel contesto di CLAST, come descritto dalla precedente
sottosezione.

Esempi di informazioni di particolare utilità e interesse per uno strumento di
source code analysis che non possono essere ricavate da un solo AST sono le
informazioni relative all'utilizzo da parte del programma di variabili libere,
particolarmente interessanti per la realizzazione di strumenti per il linting, o
informazioni relative al tipo delle form che rappresentano i nodi dell'AST,
particolarmente interessati per la realizzazione type checkers.\\

Per questa ragione, parallelamente all'attività di parsing vera a propria, il
modulo descritto da questa sezione opera un'attività di analisi e raccolta di
informazioni riguardo all'ambiente di valutazione del codice del programma
soggetto.

Questa attività di analisi viene svolta a partire dal gestione di oggetti di
tipo ambiente, oggetti che consentono la memorizzazione di definizioni e
dichiarazioni valide al momento della valutazione di un particolare frammento di
codice, e che ricalca molto da vicino l’attività svolta dalla componente di
valutazione di un sistema Common Lisp.\\

La progettazione e realizzazione dell'attività appena descritta rappresentano
uno degli aspetti di maggiore complessità all'interno della libreria; richiedono
infatti sia la comprensione del meccanismo interno di valutazione caratteristico
del linguaggio Common Lisp e del funzionamento di ciascun costrutto del
linguaggio, sia la manipolazione, utilizzando un'API particolarmente ristretta e
relativamente poco adatta allo scopo, di un oggetto di tipo ambiente. Tutto
questo in modo parallelo al processo di navigazione del codice sorgente operato
dal parsing, aspetti che non sempre possono essere combinati con facilità.

\subsection{Implementazione del modulo}

I precedenti paragrafi hanno descritto, ad alto livello, le principali
funzionalità e il funzionamento del modulo di parsing. La prossima sottosezione
entrerà nel dettaglio dell'API offerta dal modulo agli utenti della libreria e
del funzionamento interno al modulo, mostrando come le attività di parsing ed
analisi siano state definite.

\subsubsection{La funzione \texttt{PARSE}}

La funzionalità fondamentale realizzata dal modulo di parsing e analisi viene
esposta agli utenti della libreria attraverso un'interfaccia rappresentata da
un’unica funzione, la funzione \texttt{PARSE}. La funzione \texttt{PARSE} è una
funzione che prende in input una form Lisp e produce in output due valori.

\begin{itemize}

\item Il primo valore ritornato consiste nel nodo radice della rappresentazione
in forma di AST del programma Common Lisp, rappresentato dalla form fornita in
input. Un AST in cui nodi sono oggetti definiti all’interno del modulo di
rappresentazione, presentati nella Sezione \ref{representation}.

\item Il secondo valore ritornato consiste invece di un oggetto ambiente, il
quale rappresenta il particolare ambiente prodotto dalla valutazione della form
e arricchito durante l'attività di analisi svolta dal modulo in modo parallelo
al parsing.

\end{itemize}

La coppia prodotta come output rappresenta il risultato delle attività parallele
di parsing e analisi precedentemente descritte e il massimo insieme di
formazioni che risulta possibile raccogliere, rispetto ad un dato frammento di
codice Common Lisp, utilizzando la libreria CLAST. Il Listato
\ref{lst:parse-function} mostra un estratto del codice della libreria in cui
viene presentata la definizione della funzione \texttt{PARSE}.

\begin{lstlisting}[
  caption=Definizione e documentazione relative alla funzione \texttt{PARSE},
  label={lst:parse-function}
]

(defgeneric parse (form &rest keys
                        &key
                        enclosing-form
                        macroexpand
                        environment
                        &allow-other-keys)
  (:documentation "Parses a form in a given 'environment'.

The methods of this generic function return a AST 'node' (a
CLAST-ELEMENT) and the - possibly modified - ENVIRONMENT.

The methods of PARSE dispatch on 'atomic' and on 'compound' (i.e.,
CONS) forms.  Atomic forms - numbers, string, arrays, and symbols -
are dealt with directly.  Compound forms are dispatched to PARSE-FORM.

Arguments and Values:

form : the form to be parsed.
keys : the collection of key-value pairs passed to the call.
enclosing-form : the form that \"contains\" the form beling parsed.
environment : the environment in which the form is being parsed.
element : a CLAST-ELEMENT representing the AST node just parsed.
environment1 : the environment resulting from parsing the FORM.
")
  )

\end{lstlisting}

I precedenti paragrafi hanno discusso la funzione \texttt{PARSE} dal punto di
vista della funzionalità che viene esposta da questa, ossia dal punto di vista
dell'utente della libreria. Dal punto di vista del funzionamento interno alla
libreria, la funzione \texttt{PARSE} rappresenta il punto di inizio del processo
di parsing e analisi operato dalla libreria. In particolare, la funzione ha la
responsabilità di applicare le regole più generali della grammatica del
linguaggio Common Lisp.

Questo significa che la funzione si limita a verificare se la form fornita in
input rappresenti un self-evaluating object, un simbolo o una compound form: i
tre elementi fondamentali della grammatica e del processo di valutazione del
Common Lisp presentato nel corso della Sezione \ref{representation}.\\

Nel caso in cui la form rappresenti un self-evaluating symbol, il parsing e la
valutazione risultano relativamente semplici. Viene infatti immediatamente
ritornata un’istanza della classe \texttt{CLAST-ELEMENT} appartenente alla
sottoclasse più appropriata alla rappresentazione del self-evaluating object in
analisi e il parsing termina immediatamente, riportando, insieme all’istanza
appena descritta, un ambiente lessicale vuoto. Questo in quanto la valutazione
di self-evaluating objects non produce alcuna variazione rispetto al contenuto
dell'ambiente.\\

Nel caso in cui la form rappresenti un simbolo questo viene risolto all’interno
dell’oggetto ambiente opzionale specificato in fase di invocazione. A questo
punto, il processo di parsing e analisi prendono due possibili percorsi diversi.

\begin{enumerate}

\item Nel caso in cui il simbolo sia associato ad una variabile, il parsing
agisce in maniera analoga a quanto riportato in precedenza. Viene ritornata
un’istanza della classe \texttt{CLAST-ELEMENT} adeguata alla rappresentazione
del simbolo e il valore associato a questo identificato nell’ambiente, assieme
ad un ambiente lessicale vuoto. Il processo ha quindi fine.

\item Nel caso in cui invece il simbolo rappresenti una macro, la funzione
\texttt{PARSE} svolge un procedimento più complesso, andando ad operare la
macro-espansione del simbolo all’interno dell’ambiente opzionale fornito in
input alla funzione per poi ritornare, come primo valore un’istanza di \texttt
{EXPANSION-COMPONENT} che riporti il simbolo originale, il corpo ottenuto
tramite macro espansione e il valore ottenuto dalla valutazione di questo corpo,
e come secondo valore l’ambiente lessicale prodotto dall’espansione del simbolo.

\end{enumerate}

Infine, nel caso in cui invece la form fornita in input alla funzione
\texttt{PARSE} sia una compound form, il parsing e l’analisi proseguono in modo
significativamente più complesso. Questo in quanto, come riportato nella
Sottosezione \ref{CL-valutazione} relativamente al processo di valutazione
caratteristico del linguaggio Common Lisp, questo è il caso in cui si ha la
possibilità di incontrare diversi costrutti che utilizzano una sintassi speciale
e regole di valutazione speciali.\\

Tenendo fede alla struttura precedentemente citata, in cui si ha una definizione
di una funzione per ciascuna produzione della grammatica del linguaggio target e
il parsing viene organizzato costruendo un predictive parser, la funzione
\texttt{PARSE} delega ad un’altra funzione il processo di ricerca, all’interno
dell’insieme delle diverse regole della grammatica, della corretta produzione da
applicare per proseguire nel processo di parsing.\\

La funzione \texttt{PARSE} si limita quindi ad operare una destrutturazione
dalla compound form all'interno di un coppia formata dell’elemento \texttt{CAR}
e del \texttt{CDR}, la testa e la coda della lista che costituisce la compond
form, come mostra il Listato \ref{lst:car-cdr}. A partire da questi due elementi
la funzione opera quindi un'invocazione della funzione generica
\texttt{PARSE-FORM}, responsabile per la prosecuzione del parsing e per l'avvio
dello schema di discesa ricorsiva.\\

La funzione \texttt{PARSE-FORM} è, come anche la funzione \texttt{PARSE}, una
funzione generica. Il meccanismo delle funzioni generiche è uno dei meccanismi
fondamentali attraverso i quali il CLOS, ossia il Common Lisp Object System,
implementa il supporto alla programmazione object-oriented offerto dal
linguaggio Common Lisp. Come verrà presentato in seguito, le funzioni generiche
rappresentano anche uno degli strumenti che la libreria CLAST utilizza in modo
più intensivo. Per questa ragione, la prossima sezione illustra brevemente il
meccanismo delle funzioni generiche Common Lisp.

\subsubsection{Funzioni generiche}
\label{generic-functions}

Formalmente, una funzione generica Common Lisp può essere definita come una
funzione il cui comportamento viene determinato a partire dalle classi e dalle
identità dei parametri che le vengono forniti in input.\\

Una funzione generica specifica un nome di funzione e una lista di parametri ma,
a differenza di una funzione ordinaria, non specifica alcun corpo e quindi
nessun comportamento che debba essere eseguito in risposta ad una sua
invocazione. La reale implementazione del comportamento di una funzione generica
viene specificata attraverso la definizione di metodi.

\begin{lstlisting}[caption=Confronto tra definizione di funzioni generiche e
ordinarie]

;; A generic function definition. Note that no body for the function
;; needs to be specified.
(defgeneric id (x))

;; An ordinary function definition. A body needs to be specified,
;; otherwise an empty body is implicit nil return form is assumed.
(defun id (x)
  x)

\end{lstlisting}

Un metodo Common Lisp è una funzione che specifica l’implementazione di una
funzione generica per un determinato insieme di specializzazioni dei parametri
di questa. Ciascun parametro può essere specializzato da un metodo in due
diversi modi:

\begin{itemize}

\item indicando la classe di appartenenza di questo, come ad esempio
\texttt{NUMBER},

\item indicando l’identità di uno specifico oggetto, come ad esempio il valore
\texttt{9}.

\end{itemize}

\begin{lstlisting}[caption=Esempi definizione di metodi Common Lisp]

CL-USER > (defgeneric some-method (x))
#<STANDARD-GENERIC-FUNCTION COMMON-LISP-USER::SOME-METHOD (0)>
CL-USER > (defmethod some-method ((x number))
     "NUMBER")
#<STANDARD-METHOD COMMON-LISP-USER::SOME-METHOD (NUMBER) {10037A9953}>
CL-USER > (defmethod some-method ((x (eql 42)))
     "This method is eql specialized")
#<STANDARD-METHOD COMMON-LISP-USER::SOME-METHOD ((EQL 42)) {1003843553}>

\end{lstlisting}

Quando una funzione generica viene invocata, un sistema Common Lisp analizza i
parametri specificati dall'invocazione e ricerca, dall’elenco dei metodi
associati alla funzione, quale metodo riporti specializzazioni compatibili con i
tipi e le identità di questi, per poi eseguire il codice associato al metodo,
determinando così il reale comportamento della funzione generica.

\begin{lstlisting}[caption=Esempi di invocazione di metodi Common Lisp]

CL-USER > (some-method 9)
"NUMBER"
CL-USER > (some-method 42)
"This method is eql specialized"
CL-USER > (some-method "invalid")
;; Evaluation aborted on #<SIMPLE-ERROR "~@<There is no applicable
;; method for the generic function ~2I~_~S~~I~_when called with
;; arguments ~2I~_~S.~:>" {1003B3E8F3}>.

\end{lstlisting}

Da un punto di vista teorico, il meccanismo delle funzioni generiche Common Lisp
rappresenta un’implementazione del concetto di polimorfismo tipico della
programmazione Object Oriented, realizzato solitamente nel contesto di altri
linguaggi di programmazione con tecniche di message passing e funzioni come
\texttt{SEND} in Smalltalk e \texttt{objc\_msgSend} in Objective-C.\\

\begin{lstlisting}[caption=Signature della funzione \texttt{PARSE-FORM}]

(defgeneric parse-form (op form &rest keys
                        &key
                        enclosing-form
                        macroexpand
                        environment
                        &allow-other-keys)
  (:documentation "Parses a form in a given 'ENVIRONMENT' given its 'op'.

The methods of PARSE-FORM descend recursively in a form, by
dispatching on the form 'operator'.  Each sub-form is passed,
recursively to PARSE.

Arguments and Values:

form : the form to be parsed.
keys : the collection of key-value pairs passed to the call.
enclosing-form : the form that \"contains\" the form beling parsed.
environment : the environment in which the form is being parsed.
element : a CLAST-ELEMENT representing the AST node just parsed.
environment1 : the environment resulting from parsing the FORM.
")
  )

\end{lstlisting}

\subsubsection{La funzione \texttt{PARSE-FORM}}

Si è detto che la funzione \texttt{PARSE-FORM} è una funzione generica. Il
comportamento di questa funzione viene quindi specificato andando a definire un
certo insieme di metodi. In particolare, la libreria riporta sostanzialmente la
dichiarazione di un metodo per ciascun operatore Common Lisp, uno per ciascuna
special form che utilizza una sintassi caratteristica, la quale a sua volta
identifica una particolare regola della grammatica.

Dal punto di vista pratico, ciascuno di questi metodi opera una specializzazione
rispetto all’identità del primo parametro che riceve in input, il parametro
\texttt{OP} o \texttt{OPERATOR}, in particolare, opera una specializzazione
rispetto al particolare operatore che identifica una delle regole della
grammatica. Questo porta ad ottenere la struttura di predictive parser, già
citata in precedenza in questo capitolo, in cui si ha la definizione di una
funzione mutuamente esclusiva per ciascuna produzione della grammatica. Il
Listato \ref{lst:metodi-parse-form} mostra alcuni esempi di signature dei metodi
associati alla funzione generica \texttt{PARSE-FORM}.

\begin{lstlisting}[
  caption=Esempi di specializzazione operata dai metodi \texttt{PARSE-FORM},
  label={lst:metodi-parse-form}
]

(defmethod parse-form ((op (eql 'block)) form |# ... #|)
  ;; ...
  )

(defmethod parse-form ((op (eql 'declare)) form |# ... #|)
  ;; ...
  )

(defmethod parse-form ((op (eql 'let*)) form |# ... #|)
  ;; ...
  )

\end{lstlisting}

Dal punto di vista delle performance e della complessità algoritmica, è
importante sottolineare che una delle operazioni più significative nel contesto
di un predictive parser è rappresentata dall’identificazione della corretta
regola da applicare in risposta ad un particolare input. L’aver strutturato il
processo di parsing in maniera tale che questo vada a delegare l'attività di
ricerca di questa regola interamente al sistema Lisp in utilizzo, sistema già
largamente ottimizzato e studiato da questo punto di vista, tramite l’utilizzo
di funzioni generiche in fase di implementazione, è un aspetto che rende il
modulo performante e significativamente meno complesso di quanto sarebbe
altrimenti necessario, specialmente nel caso in cui il processo di
ottimizzazione della ricerca dovesse essere svolto interamente alla libreria
stessa.

\subsection{Parsing e analisi in dettaglio}

Per illustrare in maniera più approfondita il funzionamento della libreria, in
questa sezione vengono discusse nel dettaglio le attività di parsing e analisi
operate dalla libreria attraverso la discussione di un esempio, rappresentato da
una semplificazione di uno dei metodi \texttt{PARSE-FORM} descritti nel corso
dei paragrafi precedenti. In particolare, viene analizzato il funzionamento e
l'implementazione del metodo \texttt{PARSE-FORM} responsabile del parsing e
dell'analisi di compound form che utilizzano l’operatore \texttt{LET*}.\\

In seguito viene riportato un frammento della specifica ANSI in cui viene
definita la sintassi del costrutto \texttt{LET*} e vengono specificati i
dettagli rispetto al processo di valutazione di form che utilizzano questo
operatore.

\begin{lstlisting}[caption=Estratto della documentazione relativa al costrutto
\texttt{LET*}]

let ({var | (var [init-form])}*) declaration* form* => result*

(1) LET and LET* create new variable bindings and (2) execute a
series of forms that use these bindings. LET performs the
bindings in parallel (3) and LET* does them sequentially.

The form

 (let ((var1 init-form-1)
       (var2 init-form-2)
       ...
       (varm init-form-m))
   declaration1
   declaration2
   ...
   declarationp
   form1
   form2
   ...
   formn)

first evaluates the expressions init-form-1, init-form-2, and so
on, in that order, saving the resulting values. Then all of the
variables varj are bound to the corresponding values; (4) each
binding is lexical unless there is a special declaration to the
contrary. The expressions formk are then evaluated in order; the
values of all but the last are discarded (that is, the body of a
let is an implicit progn).

For both LET and LET*, if there is not an init-form associated
with a var, var is initialized to NIL.

\end{lstlisting}

Il Listato \ref{lst:parse-form-let} presenta invece un'approssimazione del
metodo \texttt{PARSE-FORM} specializzato rispetto all'operatore \texttt{LET*}
implementato dalla libreria. I dettagli rispetto alle diverse operazioni
compiute dal metodo verranno approfondite nei prossimi paragrafi.

\begin{lstlisting}[
  caption=Semplificazione del metodo \texttt{PARSE-FORM} per il parsing del
  costrutto \texttt{LET*},
  label={lst:parse-form-let}
]

(defmethod parse-form ((op (eql 'let*)) params top
                       :optional env internal-env)

  ;; 1. Extract all the various syntactic elements from the compound
  ;; form based on the syntax rules associated with the special form
  ;; denoted by the let* operator.
  (destructuring-bind (bindings declarations :rest body-forms) params

    ;; 2. A copy of the provided environment is used in order to be
    ;; able to reproduce both the environment that is produced by the
    ;; evaluation of the form and the environment `body-env` in which
    ;; the form is evaluated. For LET* forms this difference is very
    ;; significant, since the evaluation environment will contain
    ;; variable definitions specified by the bindings, but the returned
    ;; environment will not.
    (let ((body-env (clone (or internal-env body-env))))

      ;; 3. For each binding in the list of bindings invoke the
      ;; parse-binding function on it. The parse binding function
      ;; simply returns a new environment object, based on the
      ;; invocation one, that records the addition of variable
      ;; declaration with the name specified by `var` portion of
      ;; the binding.
      (dolist (binding bindings)
        (setf body-env (parse-binding binding body-env)))

      ;; 4. For each declaration parse-declaration generete a new
      ;; environment object, based on the provided one, adding
      ;; informations as specified by the declaration.
      (when declarations
        (multiple-value-bind (declaration-form body-env-with-declarations)
            (parse 'declare (cdr declarations) body-env)

          ;; 5. Update the body-env with information from all
          ;; declarations. Since the declarations themselves are not
          ;; tracked directly the representation object, a CLAST-ELEMENT
          ;; instance is discarded.
          (declare (ignore declaration-form))
          (setf body-env body-env-with-declarations)))

      ;; 6. Execute parsing of body forms. If a value for
      ;; internal-env was specified, each parsing methods uses it to
      ;; lookup definitions, but only augments `env`; otherwise it
      ;; defaults to using `env` for both lookup and recording
      ;; definitions. This step represents the recursive descent
      ;; portion of the parsing algorithm.
      (multiple-value-bind (iprogn-forms augmented-env)
          (if body-forms
              (parse-iprogn-forms body-forms env body-env)
            (values nil env))

        ;; 7. Return a CLAST-ELEMENT representation of the input
        ;; form and the provided environment augmented only with
        ;; definitions from body forms (i.e. that does not contain
        ;; any information about local bindings).
        (values
         (make-instance 'let*-form
            :source (cons operator params) :top top :type t
            :iprogn-forms iprogn-forms :body-env body-env
            :bindings bindings)
         augmented-env)))))

\end{lstlisting}

Il metodo prende in input cinque valori, tre dei quali opzionali. Il parametro
\texttt{OP} rappresenta il parametro rispetto al quale il metodo specializza la
funzione generica \texttt{PARSE-FORM}. Questo in maniera tale da consentire
l'identificazione della particolare regola nella grammatica Common Lisp di cui
il metodo di occupa di compiere il parsing.

Il parametro \texttt{PARAMS} rappresenta invece la lista dei parametri che,
insieme all'operatore, forma la compound form originale a partire dalla quale è
si è originato il più recente passo della discesa ricorsiva operata dal parser.

Il parametro opzionale \texttt{ENCLOSING-FORM} rappresenta invece un puntatore
alla form all'interno della quale è contenuta la form attualmente in analisi.
Questo attributo consente, in fase di inizializzazione del nodo che rappresenta
la form all'interno dell'AST prodotto in output dalla libreria, di specificare
un valore per l'attributo \texttt{TOP} del nodo. Come sottolineato in
precedenza, questo attributo risulta di fondamentale importanza in quanto
rappresenta un arco all'interno dell'AST e consente di organizzare i nodi,
oggetti del modulo di rappresentazione, all'interno di un albero.

Il parametro \texttt{ENV} rappresenta l'ambiente principale utilizzato dal
metodo, ossia l'oggetto ambiente a partire dal quale verranno aggiunte le
informazioni prodotte dalla valutazione della form e identificate durante
l'attività di analisi. Si tratta di un valore opzionale, in fase di analisi
della radice dell'AST non potrà infatti presente alcun ambiente che da
arricchire. In questo caso il comportamento di default è quello di definire un
ambiente lessicale nullo.

Infine, il parametro \texttt{INTERNAL-ENV} rappresenta un ambiente secondario,
il quale viene utilizzato per la ricerca delle informazioni necessarie al
processo di analisi. Questa divisione risulta particolarmente utile all'interno
dell'intero modulo in quanto consente di mantenere l'ambiente di valutazione
della form distinto dall'ambiente che verrà arricchito con le informazioni
prodotte dalla valutazione stessa e ritornato. In molti casi infatti, come ad
esempio nel caso del costrutto \texttt{LET*} analizzato dall'esempio presentato
in questa sezione, l'ambiente di definizione presenta informazioni aggiuntive,
non accessibili esternamente alla form, e che quindi non dovranno fare parte
dell'ambiente ritornato dal parsing.

La prima operazione svolta dal metodo è legata all'attività di parsing.
\textit{1.} Il valore \texttt{PARAMS}, il quale rappresenta la lista dei
parametri della compound form originale, viene infatti destrutturato all’interno
di tre diversi elementi in base a quanto specificato dalla regola della
grammatica del linguaggio associata al costrutto \texttt{LET*}. La regola
specifica infatti che ciascuna form che abbia \texttt{LET*} come operatore
si compone di:

\begin{itemize}

\item una lista di bindings come primo parametro;

\item una lista di dichiarazioni come secondo parametro opzionale;

\item un certo numero di form successive che rappresentano il corpo della
compound form complessiva, una sequenza di form che dovranno essere eseguite
nell’ambiente prodotto dalla valutazione di bindings e dichiarazioni.

\end{itemize}

La seconda operazione svolta dal metodo è \textit{2.} la generazione di una
copia dell’oggetto ambiente fornito in input, copia che verrà utilizzata in
seguito. Questa copia risulta fondamentale in quanto è necessario che la
libreria mantenga due oggetti ambiente distinti durante le attività di analisi:

\begin{itemize}

\item un ambiente che verrà arricchito con le informazioni relative ai binding
specificati localmente dalla form e all'interno del quale avverrà la valutazione
del corpo di questa,

\item un ambiente all'interno del quale verranno aggiunte dichiarazioni e
definizioni presenti nel corpo della form relativamente a dichiarazioni di
variabili, funzioni, macro e altro; ambiente che verrà ritornato al termine
dell'attività di parsing.

\end{itemize}

La presenza di questi due environment distinti, uno per la scrittura e uno per
la lettura, rappresenta una convenzione comune all’intero modulo e di
particolare importanza e uno dei concetti più complessi alla base del
funzionamento della libreria. Questa convenzione consente infatti di
implementare il parsing di costrutti come \texttt{LET*}, in cui l’environment in
cui avviene la valutazione di una form di questo tipo contiene informazioni
aggiuntive rispetto quello in prodotto in output dalla valutazione stessa.\\

Il metodo opera quindi \textit{3.} un’invocazione del metodo

\texttt{PARSE-BINDING} a partire da ciascun binding. La funzione
\texttt{PARSE-BINDING} è responsabile del parsing delle regole specificate dalla
grammatica in relazione alla sintassi dei binding, ossia la regola \textit{var
| (var [init- form])}, e della creazione un nuovo environment, a partire da
quello fornito in input, contenente la definizione della variabile specificata
dal binding in analisi. Ogni invocazione della funzione \texttt{PARSE-BINDING}
ritorna quindi un nuovo oggetto environment che viene utilizzato per aggiornare
l’environment interno al metodo.\\

Il metodo prosegue con \textit{4.} l’analisi delle dichiarazioni specificate
dalla form relativamente alle variabili dichiarate dai binding. Chiaramente,
solo nel caso in cui queste siano presenti. Questa analisi viene operata
procedendo in modo ricorsivo e invocando la funzione \texttt{PARSE} a partire
dalla form di specifica delle dichiarazioni. Una particolarità di questa
operazione è rappresentata dal fatto che, in questo caso, \texttt{5.} la
rappresentazione della form \texttt{DECLARE} viene scartata e viene considerato
solamente l'environment prodotto dalla valutazione di questa come aggiornamento
dell'environment interno al metodo. Questo perché le informazioni prodotte dalle
dichiarazioni vengono già conservate all'interno dell'environment ritornato.
Sarebbe quindi ridondante e relativamente poco utile tenere traccia della
presenza di dichiarazioni in due modi differenti ma equivalenti.

Il passo successivo è rappresentato dal parsing delle form innestate alla form
\texttt{LET*}, operazione che viene delegata alla funzione
\texttt{PARSE-IPROGN-FORMS}. Questa funzione ha lo scopo di compiere la
valutazione delle form specificate in input utilizzando i due environment
indicati e produrre in output due valori, una lista contenente la
rappresentazione di ciascuna delle form, utilizzando le strutture del modulo di
rappresentazione, e un environment aumentato con le definizioni che occorrono
all’interno delle form. Questa funzione rappresenta l'effettiva operazione di
discesa ricorsiva operata dal modulo. Dal punto di vista pratico, la funzione si
limita infatti ad operare una chiamata ricorsiva alla funzione \texttt{PARSE}
per ciascuna form innestata, aggiornando ad ogni passo l'environment di
scrittura.\\

Infine, a partire da quanto prodotto dai passi precedenti, il metodo si limita
a \texttt{7.} ritornare i due valori descritti in apertura di questa
sottosezione e comuni a qualsiasi metodo \texttt{PARSE-FORM}: un'istanza di
\texttt{CLAST-ELEMENT} che riporta tutti dettagli relativi all'utilizzo della
form \texttt{LET*} fornita in input, che rappresenta il nodo radice del
sottoalbero del quale il metodo ha effettuato il parsing, e un secondo valore
rappresentato dall'oggetto environment contenente tutte le informazioni
raccolte durante la discesa ricorsiva attraverso la form operata dal parser.\\

\image{img/parse-form-let*.png}
      {Diagramma di flusso del metodo PARSE-FORM}
      {fig:parse-form-let*}
      {0.5}

Questa sottosezione ha presentato, attraverso un esempio, il funzionamento e
l’analisi svolta dalla libreria. Analisi che consente di produrre una
rappresentazione estremamente precisa del codice fornito in input a questa.
Questo conclude la discussione rispetto al modulo di parsing e analisi offerto
dalla libreria CLAST.

\section{Modulo di traversal}
\label{traversal}

Il terzo e ultimo modulo che compone la libreria CLAST è rappresentato dal
modulo di traversal. Lo scopo di questo modulo è quello di consentire agli
utenti della libreria l'ispezione e l'analisi delle strutture di
rappresentazione prodotte in output dal modulo di parsing ed analisi descritto
dalla Sezione \ref{parsing}. Si tratta quindi del modulo più critico dal punto
di vista dell'API, in quanto rappresenta la principale interfaccia tra CLAST ed
un analizzatore costruito a partire da questa.

In seguito viene fornita al lettore una presentazione ad alto livello del
funzionamento interno del modulo, per poi proseguire presentando le diverse
funzionalità che vengono offerte dal modulo di traversal agli utenti della
libreria.

\subsection{Struttura del modulo}

La struttura interna del modulo ricalca molto da vicino il design pattern
Visitor, presentato da Gamma et al. in \cite{gamma1995design}. In seguito, viene
presentato il funzionamento di base del design pattern e come questo sia stato
esteso per adattarsi allo specifico contesto in analisi.

Si è scelto di sfruttare il design pattern Visitor principalmente per come
questo consenta di ottenere una separazione tra una dato insieme di operazioni e
gli oggetti a partire dai quali queste operazioni vengono eseguite. Separazione
che, all'interno della libreria CLAST, consente di isolare all'interno di un
modulo specifico per il traversal dell'AST tutte le operazioni relative alla
visita dei nodi e isolare all'interno di un modulo differente, come presentato
nella Sezione \ref{representation}, tutte le strutture e gli oggetti utili alla
costruzione dell'AST stesso.

\image{img/visitor-classes}
      {Design Pattern Visitor - Diagramma delle classi}
      {fig:visitor-classes}
      {0.5}

\subsubsection{Design pattern Visitor}

Come mostrato dal diagramma in Figura \ref{fig:visitor-classes}, diagramma
presentato in \cite{gamma1995design}, il pattern descrive una collaborazione tra
5 diversi elementi:

\begin{itemize}

\item \texttt{Visitor} dichiara un'operazione di visita per ciascuna classe che
sarà toccata dal processo di visita, ossia ciascun \texttt{ConcreteElement} che
compone la \texttt{ObjectStructure}.

\item \texttt{ConcreteVisitor} implementa ciascuna operazione dichiarata dal
\texttt{Visitor}. Ciascuna operazione realizza lo specifico frammento della
logica di visita associata alla particolare classe soggetto dell'operazione.

\item \texttt{Element} definisce un'operazione \texttt{accept} che riceve in
input un oggetto di tipo \texttt{Visitor}.

\item \texttt{ConcreteElement} implementa l'operazione definita da \texttt
{Element}.

\item \texttt{ObjectStructure} incapsula la logica di enumerazione dei propri
elementi, fornendo ad un visitor un'interfaccia semplificata per la visita di
ciascuno dei propri elementi.

\end{itemize}

Le modalità di interazione vengono quindi riassunte dal diagramma di attività
in Figura \ref{fig:visitor-sequence}:

\begin{enumerate}

\item l'interazione ha inizio con l'invocazione del metodo della classe
\texttt{ObjectStructure} che raccoglie la conoscenza rispetto all'ordine di
visita dei diversi elementi che compongono la propria struttura;

\item il metodo opera quindi un'invocazione del metodo \texttt{accept} di
ciascun elemento di tipo (\texttt{ConcreteElement}) secondo le modalità specifiche della classe di appartenenza;

\item ciascun elemento opera quindi un'invocazione del metodo \texttt{visit},
fornendo tipicamente sé stesso come parametro dell'invocazione.

\end{enumerate}

\image{img/visitor-sequence}
      {Design Pattern Visitor - Diagramma di sequenza}
      {fig:visitor-sequence}
      {0.5}

Dopo aver brevemente presentato il design pattern a cui la struttura del modulo
fa riferimento, le prossime sottosezioni presentano l'API del modulo e alcuni
dettagli relativi al funzionamento interno di questo.

\subsubsection{\texttt{CLAST-ELEMENT-SUBFORMS}}

La funzione \texttt{CLAST-ELEMENT-SUBFORMS} è una funzione generica. Questa
funzione può essere considerata come il corrispondente della classe
\texttt{ObjectStructure} e dell'interfaccia \texttt{Element} descritte dal
design pattern. Lo scopo di questa funzione è infatti quello di raccogliere
l'informazione rispetto alla struttura di ciascun elemento interessato dal
processo di visita e guidare il processo di visita all'interno di una dato
elemento; scopo che, nel contesto della libreria CLAST, si traduce
nell'identificazione della struttura interna di ciascuna form e nella guida del
processo di visita attraverso ciascuna form. La funzione opera quindi un mapping
tra una form e le sue sottoform. Dal punto di vista pratico, data una form la
funzione produce in output una lista contenente le diverse sottoform contenute
in questa, ordinate in accordo a quanto richiesto dal processo di visita.

\begin{lstlisting}[caption=Definizione della funzione \texttt
{CLAST-ELEMENT-SUBFORMS}]

(defgeneric clast-element-subforms (form)
  (:documentation "Returns a list of 'subforms' of a given FORM.

The methods of this generic form operate on the different kinds of
AST nodes that are of class FORM.  Other Common Lisp objects have
NULL subforms and LISTs are returned as they are.

Arguments and Values:

form : an instance of class FORM or LIST or a Common Lisp object.
result : a list of 'subforms' (or NIL).
"))

\end{lstlisting}

La funzione generica \texttt{CLAST-ELEMENT-SUBFORMS} viene quindi specializzata
attraverso la definizione di un metodo per ciascun \texttt{CLAST-ELEMENT}, ossia
per ciascuna potenziale form Common Lisp nodo dell'AST. Dal punto di vista
teorico, questo significa che ciascun metodo può essere visto come l'estensione
di ciascun \texttt{ConcreteElement} per la definizione del metodo
\texttt{accept}.

Chiaramente, essendo il metodo comune a tutte le diverse form, una data
implementazione potrebbe ritornare un certo numero di elementi, ad esempio nel
caso dell'implementazione relativa ad una form \texttt{LET*}, o una lista vuota,
ad esempio nel caso di un self-evalutating object come la form \texttt{9}.

\begin{lstlisting}[caption=Esempi di implementazione del metodo
\texttt{CLAST-ELEMENT-SUBFORMS}]

;; Simple implementation example
(defmethod clast-element-subforms ((ce constant-form)) ())

;; Complex implementation example
(defmethod clast-element-subforms ((df do-form))
  (list (form-binds df)
        (form-test df)
        (return-form df)
        (form-body df)))

\end{lstlisting}

\subsubsection{WALK}
\label{walk}

A partire dalla funzione \texttt{CLAST-ELEMENT-SUBFORMS} lavora la funzione
probabilmente più importante a livello di API offerta dell'intero modulo, la
funzione \texttt{WALK}. Lo scopo di questa funzione è quello di consentire ad un
utente della libreria di operare il reale traversal della rappresentazione in
forma di AST prodotta dal modulo di parsing ed analisi. Da un punto di vista
pratico, data un'istanza di \texttt{CLAST-ELEMENT}, la funzione \texttt{WALK}
opera una visita dell'AST radicato in questa in modo ricorsivo, ossia
depth-first.

Per continuare il parallelo tra struttura della libreria e il design pattern
Visitor, precedentemente illustrato, la funzione \texttt{WALK} può essere vista,
con una certa approssimazione, come la classe astratta \texttt{Visitor}, la
quale può essere estesa per implementare un particolare processo di visita.

\subsection{Funzionalità offerte}

La funzione \texttt{WALK} è una funzione di ordine superiore, ossia una funzione
che prende in input o restituisce altre funzioni. Proprio questo aspetto
rappresenta il meccanismo di estensione attraverso il quale un utente della
libreria può adattare il processo di visita alle proprie necessità e specificare
le operazioni che desidera svolgere in risposta alla visita di un nodo o
porzione dell'AST.

Le diverse funzioni che vengono fornite in input alla funzione \texttt{WALK}
possono quindi invece essere viste come classi \texttt{ConcreteVisitor}.

\begin{lstlisting}[
  caption=Definizione della funzione \texttt{WALK},
  label={lst:walk}
]

(defgeneric walk (clast-element &rest keys
                                &key
                                key ; #'identity
                                result-type
                                map-fun
                                reduce-fun
                                initial-value
                                environment
                                op-first
                                &allow-other-keys)
  (:documentation "The 'visiting' engine used to traverse a form.

The WALK generic function methods recursively traverse the tree
corresponding to a form (i.e., CLAST-ELEMENT) using a map/reduce
scheme.

The function MAP-FUN is applied to each (sub)form and their respective
subforms are WALKed over.  WALK uses MAP-SUBFORMS internally,
therefore it generates sequences (of type RESULT-TYPE) as output. Once
the traversing of subforms is completed the function REDUCE-FUN is
applied, via REDUCE to the resulting sequence.
")
  )

\end{lstlisting}

Come il listato \ref{lst:walk} mostra, la funzione prende in input in
particolare due funzioni, una funzione chiamata \texttt{MAP-FUN} e una funzione
chiamata \texttt{REDUCE-FUN}. Come il nome stesso di questi due parametri
suggerisce, le funzioni associate a questi vengono invocate seguendo uno schema
di esecuzione di tipo \textit{map-reduce}. Questo significa che durante il
processo di visita, la funzione \texttt{WALK} applicherà la funzione \texttt
{MAP-FUN} a ciascun nodo dell'AST incontrato, e utilizzerà la funzione \texttt
{REDUCE-FUN} a partire dalla lista di valori ritornati dall'applicazione di
\texttt{MAP-FUN} a ciascun nodo per ottenere il risultato finale della
computazione.

L'utilizzo di questo schema di funzionamento risulta particolarmente appropriato
per l'implementazione di funzioni che operano un'interrogazione dell'AST allo
scopo di identificare l'occorrenza d particolari tipologie di nodi. Operazioni
di interrogazione che risultano particolarmente utili a strumenti per la source
code analysis, come verrà illustrato nel corso della Sezione
\ref{free-variables-analysis}.

Oltre alla funzionalità di visita vera e propria esposta dalla funzione
\texttt{WALK}, la quale rappresenta il cuore del funzionamento del modulo, il
modulo di traversal offre diverse altre funzionalità, ciascuna delle quali viene
costruita proprio a partire dall'interfaccia esposta dalla funzione
\texttt{WALK}.

\subsubsection{AS-STRING}

\texttt{AS-STRING} è un'altra delle funzioni più importanti offerte dal modulo.
La funzione funzione prende in input un'istanza di \texttt{CLAST-ELEMENT} e una
rappresentazione testuale dei dati relativi a questa. Può quindi risultare come
una funzione analoga a metodi come \texttt{toString} Java e \texttt{description}
Objective-C. La particolarità di questa funzione generica è costituita dal fatto
che ciascuno dei suoi metodi agisce in modo ricorsivo, restituendo una
rappresentazione completamente componibile, strutturata e leggibile da un
sistema automatico, dell'AST radicato nel nodo specificato come input.

Oltre a rappresentare una funzione particolarmente utile in fase di debugging e
costruzione della libreria, questa funzione risulta particolarmente importante
in quanto, producendo un rappresentazione testuale, consente l'analisi dell'AST
programmi Common Lisp, e più in generale della rappresentazione di un programma
offerta dalla libreria, anche da parte di sistemi scritti utilizzando un
linguaggio di programmazione differente.

Esempi di sistemi che vengono abilitati dalla presenza di questo metodo sono per
esempio rappresentati da compilatori source-to-source, ossia sistemi per la
traduzione di codice sorgente da un linguaggio di programmazione ad un altro.
Uno strumento di questo tipo potrebbe consentire la produzione di codice in un
qualsiasi linguaggio di programmazione a partire dalla ricca rappresentazione di
una programma Common Lisp prodotta in output dal modulo di parsing e analisi
della libreria. Un'altra classe di sistemi la cui creazione viene abilitata
dalla presenza della funzionalità offerta dai metodi \texttt{AS-STRING} è da
estensioni per il supporto alla programmazione, come un syntax highlighter o un
linter, all'interno di IDE o editor di testo, programmi in molti casi scritti
utilizzando linguaggi di programmazione diversi dal Common Lisp.


\endgroup

\begingroup
\let\clearpage\relax

\chapter{Applicazioni di CLAST}
\label{applications}

Il precedente capitolo ha presentato la libreria CLAST, uno strumento per la
generazione di AST per programmi scritti utilizzando il linguaggio di
programmazione Common Lisp. Essendo questo CLAST uno strumento del tutto nuovo
nel contesto di questo linguaggio di programmazione, questo capitolo
approfondisce ciò che viene reso possibile dalla disponibilità di questo
strumento.

Vengono quindi presentate alcune delle potenziali applicazioni della libreria
CLAST. In particolare, si descrive come questa possa essere utilizzata come
fondamento per la costruzione di strumenti di source code analysis e per la
definizione di estensioni, anche molto complesse, del linguaggio di
programmazione Common Lisp.\\

La Sezione \ref{lint} presenta come la libreria consenta, in modo estremamente
semplice, la definizione di operazioni di ricerca all’interno di particolari
elementi all’interno di un frammento di codice Common Lisp. Questo viene fatto
considerando come esempio una particolare tipologia di analisi, chiamata free
variables analysis, un’analisi mirata all’identificazione della presenza di
variabili libere. Si mostra quindi come la libreria offra un supporto a questo
tipo di analisi e più in generale allo svolgimento di moltissime diverse
tecniche di analisi statica basate sul codice sorgente, grazie alla facilità con
cui è possibile, a partire da questa, identificare particolari elementi
all’interno di un programma. Si presenta quindi come queste analisi possano
andare a rappresentare la base per la costruzione di uno strumento di linting
per il linguaggio Common Lisp, una particolare categoria di strumenti di
supporto alla programmazione, attualmente non disponibili per il linguaggio.\\

La Sezione \ref{pattern-matching} presenta un secondo caso di studio, relativo
all’applicazione della libreria CLAST come strumento in grado di abilitare
l’estensione del linguaggio Common Lisp in maniera tale da supportare l’utilizzo
di costrutti di pattern matching in modo efficiente. Si discute dapprima come
sia possibile introdurre un costrutto di pattern matching nel contesto del
linguaggio combinando macro Common Lisp all’algoritmo per la compilazione del
pattern matching presentato da L. Augustsson in … per poi mostrare poi come
questa modalità di implementazione sia soggetta a problemi di efficienza.
Problemi che si mostrerà possono essere risolti a partire dall'utilizzo di
CLAST.\\

Infine, la Sezione \ref{lazy-evaluation} presenta il caso di studio
rappresentato da una seconda estensione del linguaggio Common Lisp, questa volta
mirata ad introdurre un supporto alla lazy evaluation all’interno di questo.
Anche in questo caso di analizza come questo supporto possa essere realizzato a
partire dall’impiego di operazioni di riscrittura del codice basate
sull’utilizzo del costrutto macro ed elementi dallo stato dell’arte, per poi
presentare come i problemi che emergono possano essere risolti utilizzando le
funzionalità afrore dalla libreria CLAST.

\section{Strumenti di linting}
\label{lint}

Lo scopo di questa sezione è quello di presentare come la libreria CLAST
consenta e faciliti la costruzione di strumenti di source code analysis
relativamente a programmi scritti utilizzando il linguaggio Common Lisp.

La Sezione \ref{sca-architecture} ha presentato come la libreria possa
integrarsi all'interno dell'architettura di uno strumento di source code
analysis, andando a fornire due delle tre componenti fondamentali per la
creazione di uno strumento di questo tipo, un parser e il generatore di una
rappresentazione del programma che consenta lo svolgimento di analisi,
componenti comuni ad ogni strumento di source code analysis per il particolare
linguaggio target dell'analisi.

L'avere a disposizione queste due componenti off-the-shelf rende la creazione di
uno strumento di source analysis significativamente più semplice, consentendo ad
un utente di potersi focalizzare sulla creazione del solo analizzatore,
responsabile per lo studio delle proprietà di interesse.

Questa sezione presenta quindi la definzione di free variables analysis,
un'analisi mirata all'indentificazione di variabili libere all'interno di un
frammento di codice, e mostra come un'analisi di questo tipo possa essere
operata in modo estremamente semplice a partire dalla libreria. Si discute
quindi come la capacità di svolgere questa analisi possa portare alla
costruzione di uno strumento di analisi statica che consenta di prevenire
un'intera categoria di errori nel contesto dell'utilizzo del linguaggio Common
Lisp.

La sezione si conclude presentando come, la possibilità di identificare
l'utilizzo di particolari costrutti o la presenza di dati elementi all'interno
di un programma consente la creazione di una particolare categoria di strumenti,
chiamati linter, per il linguaggio Common Lisp. Strumenti attualmente non
disponibili per il supporto alla programmazione utilizzando questo linguaggio ma
particolarmente diffusi ed utilizzati con ottimi risultati nel contesto di altri
linguaggi di programmazione. \cite{DBLP:conf/sac/PotocnikCS14}

\subsection{Free Variables Analysis}
\label{free-variables-analysis}

Formalmente, una variabile libera è una notazione che specifica una posizione
all'interno di un'espressione in cui una sostituzione potrebbe avere luogo.
L'idea è collegata al concetto di placeholder, un simbolo che verrà rimpiazzato
in seguito da un certo valore, o di carattere wildcard, ossia un simbolo che
rappresenta un qualsiasi simbolo, aspetto che verrà approfondito maggiormente
dalla Sezione \ref{pattern-matching}, relativamente al concetto di pattern
matching.

In particolare, nel contesto della programmazione e dei linguaggi di
programmazione, il termine variabile libera viene utilizzato per riferirsi a
variabili che vengono utilizzate all'interno di una funzione senza che queste
siano definite all'interno di questa o specificate come parametro. In questo
ambito, il termine risulta essere molto spesso sinonimo di variabile non
locale.

Per contrasto, viene definita variabile occupata una variabile che è
precedemente stata libera ma è attualmente è associata ad uno specifico valore o
insieme di valori.\\

Dal punto di vista logico, il valore utilizzato come nome di una particolare
variabile non risulta di particolare importanza. Tuttavia, il riutilizzo dello
stesso nome associato ad una variabile attualmente occupata potrebbe risultare
contraddittorio o, più generalmente, generare confusione. Per questa ragione,
nel momento in cui una variabile libera viene occupata, in un linguaggio di
programmazione tradizionale come anche nel contesto del linguaggio Common Lisp,
il nome associato a questa viene ritirato dallo spazio dei nomi validi per
l'associazione ad una qualsiasi variabile.

L'analisi di un frammento di codice mirata ad identificare la presenza e
l'utilizzo di variabili libere all'interno di un dato frammento di codice viene
definita \textit{free variables analysis}.

I prossimi paragrafi presentano come la libreria CLAST sia in grado di agire
come fondamento per lo svolgimento di free variables analysis.\\

\subsubsection{Free Variables Analysis e CLAST}

Nel corso della Sezione \ref{traversal}, si è mostrato come la libreria CLAST
consenta ad un utente di operare una navigazione all'interno della
rappresentazione offerta da questa di un programma Common Lisp. In particolare,
nella Sottosezione \ref{walk}, si è mostrato come questa operazione possa essere
estesa modo molto semplice allo scopo di ricercare e raccogliere informazioni da
un particolare frammento di codice. In seguito si presenta quindi come questa
potenzialità della libreria possa essere sfruttata allo scopo di svolgere free
variable analysis.

La libreria modella l'utilizzo di una variabile libera all'interno di una data
form utilizzando la classe \texttt{FREE-VARIABLE-REF}. Il diagramma in Figura
\ref{fig:free-variable-ref} mostra la porzione del modulo di rappresentazione
interessata dalla definizione di questa classe.\\

\image{img/free-variable-ref.png}
      {Diagramma della classi per la rappresentazione di variabili libere}
      {fig:free-variable-ref}
      {0.5}

Dal punto di vista del processo di valutazione, si ha un'occorrenza di una
variabile libera nel caso in cui, una volta che il sistema Common Lisp incontra
una symbol form e che questo identifica la symbol form come l'utilizzo di una
variabile, una ricerca all'interno dell'ambiente lessicale attivo non produce
alcun risultato relativamente al simbolo.

Come illustrato nel corso della Sottosezione \ref{parsing}, la libreria CLAST
esegue le attività di parsing e analisi in modo del tutto coerente al processo
di valutazione e, in particolare, anche relativamente ai passaggi appena
descritti.

% AGGIUNGI FLOWCHART DEL PROCESSO DI IDENTIFICAZIONE DI UNA VARIABILE LIBERA

Questo significa che, analizzando l'AST prodotto dalla libreria allo scopo di
rappresentare un programma, è sufficiente ricercare la presenza di nodi di tipo
\texttt{FREE-VARIABLE-REF} in maniera tale da identificare l'utilizzo e la
presenza di variabili libere all'interno di un programma.

Per svolgere questa ricerca la libreria offre la funzione \texttt{WALK},
esportata dal modulo di traversal e presentata nel corso della Sezione
\ref{traversal}. Questa funzione consente l'esecuzione del traversal di un AST
prodotto dalla libreria secondo lo schema map-reduce, dando quindi la
possibilità ad un utente di definire operazioni di ricerca e raccolta di
informazioni all'interno dell'AST.\\

Il Listato \ref{lst:free-variables} mostra quindi come la funzione \texttt{WALK}
possa essere estesa, in maniera particolarmente semplice, allo scopo di svolgere
free variables analysis su di un particolare frammento di codice.

\begin{lstlisting}[
  caption=Query functions per l'identificazione di variabili libere all'interno
  di una form Common Lisp,
  label={lst:free-variables}
]

(defun free-variables (form)
  "Returns all the 'free' variables present in FORM.

Arguments and Values:

form   : a CLAST-ELEMENT.
result : a LIST of SYMBOLS."
  (walk form
        :map (lambda (e)
               (typecase e
                 (free-variable-ref (list (form-symbol e)))))
        ))

\end{lstlisting}

La funzione \texttt{FREE-VARIABLES} opera una semplice specializzazione della
funzione \texttt{WALK}. Come descritto dalla Sottosezione \ref{walk},
\texttt{WALK} prende in input due funzioni che vengono eseguite secondo uno
schema map-reduce. \texttt{FREE-VARIABLES} specifica quindi una funzione che
implementa la componente di mapping della funzione. In particolare, la funzione
specificata viene applicata ad ogni nodo e ritorna il nodo stesso, nel caso in
cui il nodo in input sia di tipo \texttt{FREE-VARIABLE}, e il valore
\texttt{NIL} in qualsiasi altro. Non specificando un valore per la componente di
reducing della funzione \texttt{WALK}, la funzione \texttt{FREE-VARIABLES}
sfrutta del comportamento di default di questa, ossia quello di ritornare la
lista degli elementi prodotti in output da ciascuna applicazione della funzione
di mapping.\\

La Free variable analysis è quindi una delle tipologie di analisi che viene
significativamente semplificato dalla possibilità di avere a disposizione una
libreria per la generazione di AST come CLAST.

Dal punto di vista della programmazione, la disponibilità di uno strumento di
questo tipo risulta molto significativa. In particolare, nel contesto di un
linguaggio non tipizzato e fortemente dinamico come il Common Lisp, questo la
possibilità di definire strumenti come questo, e come potenzialmente molti altri
strumenti di analisi statica, porta alla produzione di codice più sicuro e alla
prevenzione di diversi potenziali errori a runtime, come ad esempio gli errori
\texttt{UNBOUND-VARIABLE} generati dall'accesso ad una variabile libera.

La prossima sezione presenta alcuni esempi di strumenti source di code analysis
abilitati dalla presenza funzionalità appena citata sono rappresentati da linter
per il linguaggio di programmazione Common Lisp.

\subsection{Linting}

Un linter è uno strumento che ha come scopo quello di segnalare ad uno
sviluppatore potenziali errori ed inconsistenze all'interno del codice sorgente
di un programma. Questa categoria di strumenti prende il proprio nome da un
strumento di questo tipo specifico per il linguaggio C, chiamato Lint, e
disponibile come parte del sistema operativo Unix. \cite{johnson1977lint}

Generalmente, viene definito \textit{strumento di linting} un qualsiasi
programma in grado di svolgere un'attività di analisi statica del codice
sorgente di un programma alla ricerca di potenziali inconsistenze a livello
funzionamento, potenziali errori nell'utilizzo dei costrutti offerti dal
linguaggio di programmazione o frammenti di codice non aderente alle particolare
linee guida stilistiche utilizzate dal team di sviluppo, producendo a partire da
quanto identificato segnalazioni di errori o warning, potenzialmente già in fase
di scrittura del codice.

Esempi di potenziali errori sono rappresentati dall'utilizzo di variabili
libere, descritto dalla precedente sottosezione, divisioni per zero, l'utilizzo
di condizioni costanti all'interno di istruzioni di branching e potenziali
overflow aritmetici. A differenza delle ottimizzazioni e delle segnalazioni di
errori che vengono tipicamente operate da un compilatore. Gli strumenti di
linting si riferiscono molto spesso ad utilizzi del linguaggio consentiti da un
compilatore e dalla sintassi del linguaggio ma spesso sono indice di
imprecisioni da parte di uno sviluppatore. Inoltre, in alcuni casi, strumenti di
questo tipo fanno riferimento ad livello di superiore un livello superiore
rispetto ad un compilatore, andando a segnalare potenziali problemi di
portabilità di un programma nel passaggio dall'utilizzo di un compilatore ad un
altro.

La verifica compiuta da un linter è quindi tipicamente più stringente rispetto a
quella operata da un compilatore e consente di evidenziare l'utilizzo di bad
practice nello sviluppo, di costrutti che portano ad uno spreco di risorse, o in
generale di operazioni che sono tipicamente problematiche.\\

È importante sottolineare il fatto che molti degli errori che risulta impossibile
identificare a livello di compilazione risultano impossibili da identificare
anche a livello di linting. Problemi di questo tipo sono rappresentati
da tutti quei problemi che richiedono l'utilizzo di tecniche di analisi
dinamica. Questo rappresenta il limite più significativo delle analisi che è
possibili svolgere a partire dalla libreria CLAST e in particolare a partire da
qualsiasi strumento per la source code analysis. Non risulta infatti possibile
identificare tutti quegli errori che possono essere identificati solamente in
seguito all'esecuzione del codice.

Inoltre, l'approccio utilizzato da strumenti di linting è quello accettare un
compromesso e cerca di segnalare errori rispetto alla cui presenza si ha più
confidenza. Ad esempio, un linter a cui viene richiesto di verificare l'assenza
di chiamate ad una particolare funzione potrebbe lavorare assumendo che in
assenza di occorrenze di una chiamata alla funzione questa non possa essere mai
chiamata, mentre nel aso in cui sia presente almeno un'occorrenza che questa
possa essere effettivamente invocata. Si tratta di un approccio ragionevole ma
che produce una quantità significativa di falsi positivi. Per questa ragione i
linter sono tipicamente strumenti con un alto grado di configurabilità.

Alla scrittura di questa tesi, non è disponibile alcuno strumento di linting
compatibile con codice Common Lisp. Questo a causa dell'assenza di strumenti che consentano di operare un'analisi del codice sorgente di programmi Common Lisp.

La libreria CLAST rappresenta quindi infrastruttura particolarmente abilitante
da questo punto di vista, consentendo possibilità di definire strumenti di
questo tipo, come mostrato  questa sezione nel particolare caso della free
variable analysis.

La libreria rende infatti immediata la ricerca di un qualsiasi specifico
elemento all'interno di un programma Common Lisp. Aspetto che a sua volta
consente la definizione di strumenti di linting che compiano un monitoraggio e
una validazione del codice, potenzialmente già nel momento in cui questo viene
scritto, andando semplicemente verificare particolari condizioni in seguito ad
ogni modifica ad un file, andando a verificare, ad esempio, che un predicato
simile a \texttt {FREE-VARIABLES} ritorni sempre una lista vuota.

\section{Pattern Matching}

Formalmente, il pattern matching viene definito come un meccanismo che consente
di identificare e scomporre le componenti di un dato pattern a partire dal
confronto con i valori presenti all’interno di una data struttura dati. Nel
caso in cui siano presenti variabili all’interno del pattern specificato, il
meccanismo prevede che queste vengano inizializzate al valore corrispondente
specificato all’interno della struttura dati.\\

Il pattern matching è considerato come uno strumento di fondamentale importanza
nel contesto dei linguaggi di programmazione funzionali. Proprio per questa
ragione, nell’ambito dei linguaggi di programmazione, è proprio questa classe
di linguaggi quella in cui si è vista la maggiore diffusione e adozione di
questo meccanismo. Da un punto di vista storico, i primi linguaggi funzionali
ad introdurre un supporto al pattern matching sono stati il linguaggio SASL, un
linguaggio funzionale puro definito nel 1972 da D. Turner
\cite{DBLP:journals/spe/Turner79}, e il linguaggio Hope
\cite{DBLP:conf/lfp/BurstallMS80}, sviluppato nel 1980 da un gruppo di
ricercatori presso l'università di Edimburgo. Il primo utilizzo nel contesto di
un linguaggio di programmazione più diffuso risale però all’utilizzo nel
contesto del linguaggio di programmazione ML nel 1973, prima all’interno del
dialetto Standard-ML (SML) \cite{milner1997definition} e poi con il dialetto
Lazy-ML (LML) \cite{DBLP:conf/lfp/Augustsson84}.

Nel corso degli anni, il pattern matching è divenuto un elemento comune
all'interno di molti linguaggi di programmazione. Nel linguaggio Haskell
\cite{DBLP:conf/hopl/HudakHJW07}, un linguaggio anch'esso funzionale puro nato
nel 1990, il pattern matching svolge addirittura un ruolo centrale per il
funzionamento complessivo del linguaggio, rappresentando lo strumento
fondamentale, e più appropriato, per la gestione dei tipi di dato algebrici che
contraddistinguono Haskell. Per concludere questo riferimento storico, è
importante sottolineare che anche linguaggi molto recenti, come Rust
\cite{rust2016}, Elixir \cite{laurent2014introducing} ed Elm \cite{elm2016},
hanno scelto di introdurre un supporto al pattern matching.

Per meglio illustrare la definizione di fornita nel paragrafo precedente, in
seguito viene presentato un esempio di costrutto, realmente presente
all'interno di un linguaggio di programmazione, che introduce all'interno del
linguaggio un supporto al pattern matching. In particolare, l'esempio di
implementazione proposto e descritto in seguito fa riferimento al linguaggio
LML, precedentemente citato, e in particolare al costrutto \texttt{CASE}
offerto da questo. Si è scelto di presentare questo particolare esempio di
implementazione dei principi del pattern matching in quanto particolarmente
rappresentativo dei principi generali alla base questo.

Il costrutto \texttt{CASE} consente ad un utente del linguaggio ML di esaminare
il valore di un elemento ed eseguire codice in modo condizionale a partire da
questa valutazione.

\begin{lstlisting}[caption=Il costrutto \texttt{CASE} presente in LML
  rappresenta un esempio di supporto al pattern matching.]
case e in
    p1 : e1
||  p2 : e2
...
||  pn : en
end
\end{lstlisting}

Nell'esempio i $p_i$ rappresentano i pattern che dovranno essere verificati
all’esecuzione del costrutto, $e$ rappresenta un’espressione, detta
\textit{discriminante}, ossia un valore o insieme di valori a partire dai quali
devono essere verificati i pattern $p_i$ per identificare la presenza di
un match, infine ciascun $e_i$ rappresenta l'espressione che deve essere
ritornata nel caso in cui il pattern $p_i$ venga individuato come match.

Un pattern è rappresentato da un insieme di dimensione arbitraria ed
eterogeneo, in cui ciascun elemento può essere classificato o come una costante
o come una variabile. Un pattern composto da una sola variabile o costante
viene detto \textit{semplice}.

Gli elementi che compongono un'espressione, come anche i pattern che vanno ad
essere verificati a partire da questa, possono essere organizzati all’interno
di una lista o di un albero, costruito a partire da liste innestate.

Nel caso invece di valori costanti, il matching operato tra valore specificato
dal pattern e valore corrispondente presente all'interno della struttura dati è
di tipo esatto. Nel caso in cui un pattern $p_i$ riporti una variabile al suo
interno, questa viene matchata in modo incondizionato, e il valore
corrispondente a questa all’interno dell’espressione e viene utilizzato per
effettuare l’inizializzazione di questa nel contesto della valutazione
dell'espressione $e_i$. Una convenzione che le diverse varianti di ML adottano,
come anche molti altri linguaggi, è quella di riservare il nome di variabile
\texttt{\_} per indicare una variabile al cui valore non si è interessati. Nel
caso di utilizzo di un \texttt{\_} all’interno di un pattern, si effettuerà
sempre un matching incondizionato, ma il valore associato a questo match verrà
semplicemente scartato invece che essere utilizzato per l’inizializzazione
della variabile.

Il valore prodotto quindi dalla valutazione di un'istanza del costrutto
\texttt{CASE} è quello prodotto dalla valutazione dell’espressione $e_i$
associata al primo pattern $p_i$ identificato come match rispetto
all’espressione discriminante $e$.\\

Durante la valutazione di un'istanza del costrutto \texttt{CASE}, valgono le
seguenti proprietà:

\begin{itemize}
\item i diversi pattern vengono verificati a partire dall’ordine di
  definizione, ossia dall’alto verso il basso, così come anche le diverse
  componenti interne a ciascun pattern, ossia da sinistra verso destra;
\item nel caso in cui i pattern non siano sufficienti a coprire l’intero
  insieme di variabilità dell’espressione discriminante, il compilatore
  aggiunge un ulteriore pattern in grado di matchare un qualsiasi valore e
  associa a questo un’espressione che porterà alla generazione di un errore a
  runtime;
\item l’espressione discriminante $e$ viene valutata solamente quanto basta per
  poter identificare il primo pattern con il quale si ha un match;
\item é ammesso che i pattern si sovrappongano tra di loro, ma non è ammesso
  che un pattern si sovrapponga completamente ad uno dei pattern successivi ad
  esso. Questo perché il pattern successivo non potrebbe mai essere matchato e
  il linguaggio considera questo come un errore di programmazione da parte
  dell'utente.\\
\end{itemize}

La scelta del linguaggio ML di definire un ordinamento totale tra i diversi
pattern, utilizzando l’ordine di definizione è una scelta arbitraria ma
condivisa da molte altre implementazioni del pattern matching e che consente di
identificare univocamente l'espressione che dovrà essere ritornata nel caso in
cui più pattern vadano a matchare l'espressione discriminante. Una verifica
compiuta in assenza di un ordinamento totale richiederebbe infatti la
definizione un altro meccanismo per la risoluzione dei conflitti nel caso in
siano presenti più pattern per il quale si ha un match con l’espressione
discriminante. Tuttavia la definizione di un meccanismo generale per risolvere
questo genere di problemi non risulta particolarmente intuitiva e lavorerebbe
necessariamente per singoli casi, richiedendo inoltre tipicamente la
definizione di una funzione di distanza che consenta di identificare il match
più specifico rispetto all’espressione discriminante in grado di agire a
prescindere dal tipo di espressione e pattern in analisi. Molte implementazioni
ritengono che questo processo agguinga un overhead significativo e
difficilmente giustificabile alla programmazione.\\

È importante sottolineare che, nonostante gran parte delle implementazioni di
strumenti per il supporto al pattern matching all’interno di diversi linguaggi
di programmazione utilizzino un’interpretazione delle idee fondamentali molto
simile a quella appena esposta, non tutte queste implementazioni hanno
necessariamente lo stesso funzionamento e/o processo di esecuzione.

Tuttavia, come accennato in precedenza, i principi del funzionamento del
costrutto \texttt{CASE} appena riportato possono essere rintracciati in maniera
del tutto analoga anche nel contesto di implementazioni dei principi alla base
del pattern matching che utilizzano linguaggi di programmazione diversi e che
si pongono obiettivo completamente diversi.

Ad esempio, nel linguaggio Haskell, sviluppato da un gruppo di sviluppatori tra
cui L. Augustsson, coautore di LML e della definizione del costrutto
\texttt{CASE} presente in questo, il processo di definizione e valutazione di
una funzione, elemento centrale al funzionamento del linguaggio, lavora in modo
molto simile a quanto appena riportato.

\begin{lstlisting}[caption=Esempio di definizione di una funzione semplificata
  per il calcolo del fattoriale utilizzando il linguaggio Haskell.]
factorial 0 = 1
factorial n = n * factorial(n-1)
\end{lstlisting}

Il precedente esempio definisce una prima versione della funzione factorial, il
cui input ammesso è il valore intero 0, e che ritorna il valore intero 1. La
seconda definizione specifica invece come input il nome di una variabile, la
quale andrà ad essere matchata in modo incondizionato e quindi a rappresentare
un qualsiasi parametro fornito in input alla funzione, e ritorna il risultato
di una computazione ricorsiva costruita partire dal valore del parametro in
input.

All’invocazione della funzione factorial vengono applicate delle regole del
tutto analoghe a quelle definite in precedenza:

\begin{itemize}
\item le diverse definizioni della funzione vengono valutare in ordine di
  definizione come anche i parametri specificati da ciascuna di queste, nel
  caso in si abbia un match tra i parametri specificati dall'invocazione e
  quelli specificati dalla definizione in analisi, il codice associato a quella
  particolare definizione viene eseguito.
\item nel caso non sia presente una definizione una della funzione in grado di
  matchare i parametri di invocazione, il runtime del linguaggio produrrà un
  errore nel quale indicherà che i pattern associati alla funzione f non sono
  sufficienti per poter valutare la chiamata (\texttt{Non-exhaustive patterns
  in function 'factorial'});
\item i parametri di input vengono valutati solamente quanto basta per poter
  identificare la prima definizione della funzione con la quale si ha un match.
\end{itemize}

Nel contesto del linguaggio Haskell, il pattern matching viene quindi
utilizzato anche come meccanismo alla base del dispatching delle chiamate ad
una funzione. Un’aspetto di questa implementazione che è significativo
sottolineare è rappresentato dal fatto che, a differenza dei più tradizionali
meccanismi per il dispatching di chiamate a funzione utilizzati da linguaggi di
programmazione più comuni, come ad esempio il linguaggio Java, il dispatching
delle chiamate viene effettuato andando a verificare l’effettivo valore dei
parametri specificati dall’invocazione, e non solamente il tipo di questi.\\

Il fatto che il pattern matching possa essere utilizzato come strumento che
consente di introdurre aspetti come l’esecuzione condizionata e dynamic
dispatch all’interno di un linguaggio di programmazione può fornire un’idea
della potenza espressivo del meccanismo

Dopo aver presentato nei precedenti paragrafi una definizione di pattern
matching, i principi alla base di questo e alcuni esempi di utilizzo di questo
meccanismo, in seguito vengono analizzati i principali vantaggi che un
linguaggio di programmazione può ottenere introducendo un supporto al pattern
matching e come il linguaggio Common Lisp si relazioni con il pattern
matching.\\

Alcuni dei vantaggi che l'introduzione di un supporto al pattern matching porta
all’interno di un linguaggio di programmazione sono legati in prima battuta al
fatto che questo consente, in maniera semplificata, l’utilizzo di un paradigma
di programmazione dichiarativo piuttosto che imperativo. Questo consente di
evidenziare in modo chiaro l’intento di un frammento di codice utilizzando una
forma leggibile. L’utilizzo di un approccio come quello descritto in precedenza
per la definizione di una funzione rende immediatamente visibili quali siano le
precondizioni e le supposizioni rispetto ai parametri di input ad una funzione;
un compito per il quale un approccio imperativo necessiterebbe di una struttura
apposita per la formulazione di asserzioni rispetto ai parametri in input, che
infatti molto spesso viene aggiunta a linguaggi di questo tipo. Strumento che,
quello delle asserzioni, comunque non è in grado di fornire le garanzie che è
in grado di fornire un approccio dichiarativo, in cui la validità delle
precondizioni, o asserzioni, rispetto ai parametri può essere garantita già in
fase di compilazione, prevenendo il potenziale generarsi di errori a runtime.\\

Il vantaggio più significativo che la disponibilità di un supporto al pattern
matching porta all’interno di un linguaggio di programmazione è però
rappresentato dal fatto che questo consente di trasferire lavoro dall’utente
del linguaggio al compilatore. Lavori, come ad esempio la validazione di
asserzioni e precondizioni e la gestione del branching, che risultano
particolarmente soggetti ad errori di programmazione, e per i quali un
compilatore è tipicamente in grado di produrre con maggior successo un codice
corretto. Oltre alla correttezza, un ulteriore aspetto che un utilizzo più
intensivo del compilatore porta ad ottenere è relativo alle prestazioni di un
programma: avendo a disposizione maggiori informazioni rispetto al
funzionamento e ad i meccanismi di rappresentazione interni del linguaggio di
programmazione, un compilatore è tipicamente in grado di produrre codice più
efficiente rispetto a quello che potrebbe essere prodotto da un utente del
linguaggio.\\

Common Lisp non fornisce un supporto diretto al pattern matching.
Tuttavia il concetto di pattern matching può essere rintracciato in diversi
aspetti del linguaggio.

Ad esempio, il costrutto \texttt{DESTRUCTURING-BIND} è una macro fornita
nativamente dal linguaggio per la destrutturazione di liste, anche
arbitrariamente innestate, all’interno delle loro componenti elementari.

\begin{lstlisting}[caption=Esempio utilizzo della macro
  \texttt{DESTRUCTURING-BIND}.]
(destructuring-bind (a (b) c)
    (list 1 (list 2) 3)
  (values a b c))             ; prints 1 2 3
\end{lstlisting}

\begin{lstlisting}[caption=Signature della macro \texttt{DESTRUCTURING-BIND}.]
destructuring-bind lambda-list expression declaration* form*
=> result*
\end{lstlisting}

\texttt{DESTRUCTURING-BIND} rappresenta una forma elementare di pattern
matching in cui viene ammesso un singolo pattern, il quale può
solamente essere composto da variabili, e che essendo unico dovrà
necessariamente matchare correttamente l’espressione in input affinché non si
generino errori. Oltre alle limitazioni appena presentate, il costrutto non
ammette l'utilizzo di costanti e wildcard. \footnote{In sostanza il costrutto
non opera un reale matching degli elementi di una struttura, ma solamente una
scomposizione di questi all'interno dell'insieme di variabili specificato.}

Tuttavia, nonostante l'assenza di un supporto nativo al pattern matching, la
forte estensibilità che contraddistingue Common Lisp rende possibile
l'introduzione di un costrutto per il pattern matching in maniera naturale.
Come verrà approfondito all'interno delle prossime Sezioni.

\section{Supporto alla Lazy Evaluation}
\label{lazy-evaluation}

All'interno di questa sezione viene presentato un terzo caso di studio relativo
alle principali applicazioni della libreria CLAST. Questa sezione mostra quindi
come la libreria possa essere utilizzata per definire una seconda estensione al
linguaggio di programmazione Common Lisp. In particolare, un'estensione che
consenta ad un utente del linguaggio di utilizzare un modello di valutazione
lazy, al contrario del tradizionale modello adottato dal linguaggio, detto di
tipo eager o strict.\\

A questo scopo viene dapprima presentato il modello di valutazione lazy,
discutendo le principali applicazioni, attraverso un esempio, e riportando
vantaggi e svantaggi legati all'utilizzo di questo.

La Sottosezione \ref{lazy-eval-cl} riporta quindi come sia possibile introdurre
un supporto a questo modello di valutazione nel contesto del linguaggio di
programmazione Common Lisp, discute le principali problematiche che si
incontrano durante un'operazione di questo tipo e le soluzioni presenti nella
letteratura rispetto relativamente a questi problemi.

Infine, si riporta come la libreria CLAST possa essere utilizzata per rendere
l'utilizzo di un modello di valutazione lazy in Common Lisp sostanzialmente
indistinguibile, dal punto di vista sintattico, dall'utilizzo del modello
originale adottato dal linguaggio stesso, attraverso la definizione di
operazioni di riscrittura automatica del codice.

\subsection{Lazy Evaluation}

Ciascun linguaggio di programmazione, funzionale o meno, può essere classificato
o come linguaggio strict o come linguaggio lazy, in base alla modalità di
valutazione che adotta.

La fase di passaggio dei parametri ad una funzione rappresenta il momento in cui
le differenze tra questi due modelli di valutazione risultano più evidenti:

\begin{itemize}

\item in un linguaggio che adotta un modello di valutazione strict, i parametri
di una funzione vengono valutati subito prima dell'esecuzione del corpo di
questa e tipicamente già in fase di invocazione.

\item In un linguaggio che adotta un modello di valutazione lazy, i parametri
vengono invece valutati \textit{on-demand}.

\end{itemize}

In un modello di valutazione lazy, i parametri vengono quindi inizialmente
passati al corpo della funzione come espressioni non valutate e vengono valutati
solamente nel momento in cui la computazione necessita realmente del risultato
della loro valutazione per poter proseguire. Questo significa anche che, se la
computazione non dovesse mai avere necessità del valore prodotto dalla
valutazione di una data un'espressione, questa espressione non verrà mai
effettivamente valutata.\\

Linguaggi di programmazione funzionali puri che adottano questo secondo modello
di valutazione molto spesso combinano a questo l'utilizzo di tecniche di
memoizzazione \cite{Geyer-Schulz:1989:MEM:379209.379211}. Nel contesto di un
linguaggio di questo tipo quindi, una volta che un dato parametro viene
valutato, il valore prodotto dalla valutazione viene memorizzato. Questo in
maniera tale che, nel caso in cui il programma richieda nuovamente la sua
valutazione dell'espressione associata al parametro, il sistema possa limitarsi
a restituire il valore precedentemente memorizzato piuttosto che operare la
stessa computazione nuovamente. Si tratta di un'ottimizzazione che in molti casi
può portare ad ottimi risultati, ma che può essere applicata in modo sicuro
solamente all'interno di un linguaggio puro, in cui la valutazione di una data
espressione produce sempre lo stesso risultato.\\

Per meglio illustrare la differenza tra i modelli di valutazione appena
descritti, il Listato \ref{lst:si} mostra un frammento di codice il cui
comportamento varia in modo significativo nel passaggio dall'utilizzo di un
modello all'altro.

\begin{lstlisting}[
  caption=Codice di esempio,
  label={lst:si}
]

(defun si (condicio ergo alternatio)
  (if condition
      ergo
      alternatio))

\end{lstlisting}

Se si considera l'invocazione \texttt{(si t 9 (loop))} della funzione riportata
come esempio dal Listato \ref{lst:si}, relativamente ad un linguaggio che adotta
un modello di valutazione strict, si osserva che l'invocazione produce un loop
infinito. Prima dell'esecuzione della funzione infatti, il linguaggio opera una
valutazione dei parametri forniti a questa e, in particolare, la valutazione
dell'espressione \texttt{(loop)}. La valutazione di questa espressione causa
quindi la non terminazione del programma. Se si considera invece la medesima
invocazione all'interno di un linguaggio che adotta un modello di valutazione
lazy, si osserva invece che l'esecuzione termina, producendo come output il
valore \texttt{9}. Questo perché i parametri forniti in input alla funzione non
vengono valutati fino al momento in cui il valore associato a questo risulta
realmente necessario all'esecuzione ed, essendo che il terzo parametro non
risulta mai effettivamente necessario a questa, non sia avrà mai una valutazione
del parametro \texttt{(loop)} né quindi il generarsi di un loop infinito.\\

Il vantaggio fondamentale introdotto dalla disponibilità di un supporto alla lazy
evaluation è principalmente legato a questa ottimizzazione. A sua volta, la
presenza di questa ottimizzazione introduce però nuove possibilità, come ad
esempio la possibilità di rappresentare strutture dati infinite o circolari e
operazioni su sequenze di valori particolarmente lunghe in maniera
significativamente più efficiente rispetto a quanto si otterrebbe con un
linguaggio che adotta un modello di valutazione strict.

Lo svantaggio fondamentale del modello di valutazione lazy è rappresentato dal
fatto che, a differenza di quanto avviene nel contesto del modello di
valutazione strict, risulta particolarmente complesso ragionare rispetto alle
performance di un dato programma. Nei linguaggi che utilizzano un modello
strict, ciascuna sotto-espressione viene valutata nel momento stesso in cui
questa compare a livello sintattico. Questo semplifica in maniera sostanziale il
processo di ragionamento rispetto alle risorse di tempo e spazio utilizzate da
un programma di questo tipo in un dato istante. Nel contesto di un linguaggio
che utilizza un modello di valutazione lazy, anche utenti esperti del linguaggio
possono avere difficoltà nel predire quando e se una data sotto-espressione
verrà effettivamente valutata.

Entrambi i modelli di valutazione hanno quindi i propri vantaggi e svantaggi.
Per questa ragione, un linguaggio di programmazione ideale dovrebbe consentire
ad un utente un supporto per l'utilizzo di entrambi questi modelli, senza
introdurre overhead significativi dal punto di vista sintattico e della
complessità di un sistema, consentendo quindi ad un utente di scegliere quale
modello adottare in relazione alle necessità del sistema in sviluppo.\\

Linguaggi di programmazione tradizionali, come ad esempio il linguaggio C e il
linguaggio Java, adottano un modello di valutazione strict, e non prevedono, a
livello di linguaggio stesso, un supporto alla definizione di meccanismi che
consentano di influenzare il processo di valutazione in maniera tale da ottenere
un comportamento lazy.

Esempi di linguaggi di programmazione che adottano un modello lazy come default
sono rappresentati dal linguaggio Haskell \cite{DBLP:conf/hopl/HudakHJW07}, dal
linguaggio Miranda \cite{DBLP:journals/eatcs/Turner87} e dalla variante del
linguaggio ML chiamata LML \cite{DBLP:conf/lfp/Augustsson84}.

Altri linguaggi invece utilizzano un modello di valutazione strict come default
ma consentono all'utente di specificare che un particolare frammento di codice
debba essere eseguito utilizzando secondo il modello lazy attraverso l'utilizzo
di una sintassi speciale. Esempi di linguaggi appartenenti a quest'ultima
categoria sono il linguaggio Scheme, con i costrutti \texttt{delay} e
\texttt{force} e OCaml, con \texttt{lazy} e \texttt{Lazy.force}.

\subsection{Lazy Evalutation e Common Lisp}
\label{lazy-eval-cl}

Il linguaggio di programmazione Common Lisp fa parte della categoria dei
linguaggi che adottano un modello di valutazione strict e non forniscono alcun
costrutto che consenta l'utilizzo del modello lazy.

Esistono tuttavia diverse strategie che possono essere impiegate allo scopo di
introdurre un supporto all'utilizzo di un modello di valutazione lazy nel
contesto di un linguaggio di programmazione.

Tra queste strategie, la più nota e più adatta al contesto del linguaggio Common
Lisp è probabilmente quella illustrata da Abelson et al. in \cite{Abelson1996} e
parallelamente da Okasaki in \cite{DBLP:conf/afp/Okasaki96}, basata sul concetto
di stream e sulle primitive \texttt{delay} e \texttt{force}.\\

La primitiva \texttt{delay} genera a partire da una data espressione una
funzione che può essere invocata ottenendo come risultato il valore prodotto
dalla valutazione dell'espressione. Nel contesto dei linguaggi funzionali, la
funzione prodotta dalla primitiva \texttt{delay} viene spesso chiamata
\textit{thunk}. Il Listato \ref{lst:delay} mostra un'implementazione della
primitiva \texttt{delay} in Common Lisp.

\begin{lstlisting}[
  caption=Implementazione della primitiva delay in Common Lisp,
  label={lst:delay}
]

(defmacro delay (expr) `(lambda () ,expr))

\end{lstlisting}

La primitiva \texttt{force} consente, dato il risultato di una funzione prodotta
utilizzando la primitiva \texttt{delay} ossia un thunk, di ottenere il risultato
della valutazione dell'espressione a partire dalla quale questa è stata creata.

\begin{lstlisting}[
  caption=Implementazione della primitiva force in Common Lisp,
  label={lst:force}
]

(defun force (thunk) (funcall thunk))

\end{lstlisting}

Il limite di un sistema all'interno del quale un modello di valutazione lazy
viene implementato solamente a partire da queste due primitive è rappresentato
dal fatto che la definizione di una qualsiasi funzione, che adotta il modello
lazy, prevederà un utilizzo particolarmente intenso di questi due costrutti. Ad
esempio, l'operatore discusso dalla precedente Sottosezione e presentato dal
Listato \ref{lst:si} potrebbe essere implementato in modo lazy come mostrato dal
Listato \ref{lst:si-delay-force}, combinando il meccanismo delle macro Common
Lisp alle due primitive.

\begin{lstlisting}[
  caption=Implementazione della primitiva force in Common Lisp,
  label={lst:si-delay-force}
]

(defmacro si (condicio ergo alternatio)
       `(if (force ,condition)
            (force (delay ,ergo))
            (force (delay ,alternatio))))

\end{lstlisting}

Si può osservare che, nonostante la semplicità dell'esempio, l'implementazione
dell'operatore risulta quasi nascosta dall'utilizzo delle due primitive. Mentre
l'implementazione originale dell'operatore utilizzava 67 caratteri, la nuova
implementazione ne richiede 115.

Oltre a questo aspetto relativo alla leggibilità e manutenibilità del codice,
un'altra problematica particolarmente significativa è rappresentato dal costante
overhead introdotto alla fase di sviluppo del codice vero e proprio. Per un
utente del linguaggio diventa infatti importante tenere sempre presente quando e
dove sia necessario ritardare l'esecuzione di ciascun elemento, un aspetto
significativo soprattutto all'aumentare della complessità.

Una terza problematica è invece rappresentata dal fatto che, avendo utilizzato
il costrutto macro, il prodotto della definizione non è una funzione. Questo
significa che l'operatore \texttt{SI}, come qualsiasi altro operatore definito
in maniera analoga, non potrà essere utilizzato come una funzione e quindi
utilizzato in combinazione a funzioni come \texttt{FUNCALL}, \texttt{MAP} e in
generale qualsiasi funzione di ordine superiore; aspetto che limita in maniera
significativa l'utilità di questa definizione.\\

Per risolvere questi due problemi è possibile introdurre un nuovo costrutto,
discusso nel dettaglio nei prossimi paragrafi.

\subsubsection{DEFLAZY}

Data la definizione di una qualsiasi funzione è sempre possibile produrre una
nuova versione di questa, del tutto equivalente ma che utilizza un modello di
valutazione lazy, attraverso l’utilizzo di una macro Common Lisp. Una macro di
questo tipo dovrebbe:

\begin{enumerate}

\item utilizzare la definizione di funzione fornita dall'utente per definire la
normale versione strict della funzione all'interno dell'ambiente;

\item generare un thunk per ciascuna parametro della funzione;

\item generare una versione della funzione fornita che lavora secondo il modello
lazy a partire dai thunk prodotti per ciascun parametro e dal corpo della
versione originale della funzione.

\end{enumerate}

Un'implementazione della macro appena descritta viene fornita dalla libreria
CLAZY \cite{DBLP:journals/corr/Antoniotti14} e viene riportata nel listato
\ref{lst:deflazy}.

\begin{lstlisting}[
  caption=Definizione della macro \texttt{DEFLAZY},
  label={lst:deflazy}
]

(defmacro deflazy (name args &body body)
  "Defines a function while ensuring that a lazy version exists as well."
  (declare (type cons args body)
           (type symbol name))
  (multiple-value-bind (renamed-arglist _ vars-thunk-names)
      (rename-lambda-vars args)
    (declare (type list renamed-arglist)
             (ignore _))
    (let ((new-name (lazy-name name)))
      (declare (type symbol new-name))
      `(progn
         (defun ,new-name ,renamed-arglist
           (symbol-macrolet ,(mapcar #'create-var-thunk-call-expansion vars-thunk-names)
             ;; (format t "Calling the lazy version...~%")
             ,@body))
         (setf (get-lazy-version ',name) (function ,new-name))
         (cl:defun ,name ,args ,@body)
         ))))

\end{lstlisting}

Questo primo costrutto, chiamato \texttt{DEFLAZY}, risulta essere una soluzione
ai problemi descritti in precedenza.

\begin{itemize}

\item Consente di mantenere uno stile di definizione del tutto analogo a quello
che si utilizzerebbe per la definizione di una funzione secondo il modello di
valutazione strict.

\item Produce funzioni vere e proprie e che pertanto possono essere utilizzate
come parametri in qualsiasi ambito del linguaggio lo richieda.

\item Contente di ottenere due definizioni della funzione, una per ciascun
modello di valutazione.

\end{itemize}

\begin{lstlisting}[
  caption=Esempio di utilizzo della macro \texttt{DEFLAZY},
  label={lst:deflazy-use}
]

(deflazy si (condicio ergo alternatio)
       (if condicio ergo alternatio))

\end{lstlisting}

L'avere a disposizione di entrambe le versioni di una stessa funzione permette
ad un utente di poter scegliere il modello di esecuzione più adatto a seconda
dallo specifico contesto in analisi.\\

La libreria affronta quindi il problema delle definizione di una funzione lazy a
partire dalla soluzione appena presentata. Dal punto di vista dell'invocazione
di una funzione, la scelta operata dalla libreria CLAZY è invece la seguente.

Il nome specificato dall'utente in fase di definizione tramite \texttt{DEFLAZY},
viene utilizzato dalla libreria per identificare la versione strict della
funzione. Questo consente infatti di mantenere un comportamento di default di
una chiamata a funzione del tutto analogo a quello che si avrebbe operando una
normale definizione della funzione, rimanendo quindi in linea con il resto del
linguaggio di programmazione.

Per quanto riguarda invece la versione lazy della funzione, come mostrato dal
Listato \ref{lst:deflazy}, questa non viene invece definita all'interno
dell'ambiente utente ma piuttosto all'interno di un ambiente specifico alla
libreria. La libreria prevede quindi che un'utente utilizzi un particolare
operatore, chiamato \texttt{LAZY:CALL}, in fase di chiamata per indicare la
volontà di utilizzare la versione lazy della funzione. In risposta all'utilizzo
di questo operatore, la libreria estrae la versione lazy della funzione definita
all'interno del proprio ambiente ed riflette la chiamata operata dall'utente su
di questa.

Queste due modalità di invocazione vengono presentate dal Listato
\ref{lst:strict-lazy-calls}.

\begin{lstlisting}[
  caption=Confronto tra invocazione delle versione strict e lazy di una funzione,
  label={lst:strict-lazy-calls}
]

;; Strict invocation
(si 42 (loop))

;; Lazy invocation
(lazy:call 'si 42 (loop))

\end{lstlisting}

\subsection{Lazy Evalutation e CLAST}

La soluzione realizzata dalla libreria CLAZY risulta particolarmente appropriata
al contesto del linguaggio Common Lisp ma risulta ancora soggetta ad un
problema.

Tipicamente, nel caso in cui si voglia adottare un modello di valutazione di
tipo lazy, si vuole che questo venga utilizzato all'interno di un intero modulo
e non semplicemente nel contesto di un numero ristretto di chiamate a funzione.

Infatti, combinare utilizzo di entrambi i modelli all'interno della stessa
porzione di sistema può portare a risultati inattesi, sia in termini di
performance che in termini correttezza. Questo specialmente nel contesto di un
linguaggio non puro, come il Common Lisp, in cui è possibile che una funzione
produca cambiamenti di stato del un sistema complessivo come effetto
collaterale.

Tipicamente quello che si desidera è quindi avere determinate parti all'interno
di un sistema in cui il modello di valutazione adottato è un modello lazy, e
altre parti in cui si adotta un modello di valutazione strict.

In un contesto come quello appena descritto, quello che si ottiene utilizzando
la soluzione descritta nel corso della precedente sezione, è che in alcuni
moduli del sistema, quelli che adottano il modello di valutazione lazy, ogni
chiamata a funzione sarà mediata dall'utilizzo del costrutto \texttt{LAZY:CALL},
attraverso il quale l'utente specifica ogni volta la volontà di applicare la
definizione lazy di una data funzione.

Questo porta ad un problema analogo a quanto presentato dal Listato
\ref{lst:si-delay-force}, in cui si aveva una situazione in cui il codice che
consente l’utilizzo di un modello di valutazione lazy nasconde la semantica vera
e propria del programma e introduce un pesante overhead all'attività di
sviluppo, come viene mostrato dall'esempio riportato nel Listato
\ref{lst:messy-lazy-call}.

\begin{lstlisting}[
  caption=Confronto tra definizione di funzione con chiamate strict e con chiamate lazy,
  label={lst:messy-lazy-call}
]

(defun mean-confidence-interval (data confidence)
  (let* ((initial-array  (fill-array data 1))
   (count          (length a))
   (mean           (mean a))
   (standard-error (standard-error initial-array))
   (percent-point  (percent-point (/ (1+ confidence) 2.0)
          (- count 1)))
   (h              (* standard-error percent-point)))
    (values mean (- mean h) (+ mean h))))

(deflazy mean-confidence-interval (data confidence)
  (let* ((initial-array  (lazy:call 'fill-array data 1))
   (count          (lazy:call 'length a))
   (mean           (lazy:call 'mean a))
   (standard-error (lazy:call 'standard-error initial-array))
   (percent-point  (lazy:call 'percent-point
            (/ (lazy:call 'plus-one
              confidence)
               2.0)
            (lazy:call 'subtract count 1)))
   (h              (lazy:call 'multiply
            standard-error
            percent-point))
   )
    (values mean
      (lazy:call 'subtract mean h)
      (lazy:call 'sum mean h))))

\end{lstlisting}

Anche in questo caso il problema potrebbe essere risolto automatizzando la
sostituzione delle chiamate a funzioni lazy attraverso un processo di
riscrittura del codice sorgente della data porzione di programma per la quale
l'utente del linguaggio desidera adottare un modello di valutazione lazy.

I passi previsti da questo processo di riscrittura automatico sarebbero i
seguenti:

\begin{enumerate}

\item identificare tutte le chiamate a funzione presenti all'interno di un dato
frammento di codice

\item identificare per quali di queste funzioni è disponibile una definizione
lazy;

\item Operare un riscrittura di ciascuna delle chiamate a funzione identificate
al passo precedente utilizzando il costrutto \texttt{LAZY:CALL}.

\end{enumerate}

Come riportato, risulta necessario identificare l'elenco di tutte le chiamate a
funzione all'interno di un frammento di codice, operazione che, prima della
definizione della libreria CLAST, risultava impossibile a causa della mancaza di
uno strumento in grado di compiere analisi di codice sorgente scritto
utilizzando il linguaggio Common Lisp.

Come presentato nel corso capitoli precedenti, la libreria CLAST è però in grado
di compiere questa operazione di ricerca in modo particolarmente semplice ed
efficiente. La libreria consente quindi la definizione del processo di
riscrittura appena descritto attraverso la definizione di una macro, risolvendo
il problema discusso nel corso di questa Sottosezione.\\

Questo consente quindi ad un utente del linguaggio di programmazione Common Lisp
la possibilità di adottare un modello di valutazione lazy, per un certa porzione
di codice, in maniera particolarmente semplice e nativa rispetto al linguaggio
di programmazione, come mostrato nel Listato \ref{lst:lazily}.

\begin{lstlisting}[
  caption=Modalità di utilizzo della lazy evalutation abilitata da CLAST,
  label={lst:lazily}
]

(lazily
 (defun mean-confidence-interval (data confidence)
   (let* ((initial-array  (fill-array data 1))
    (count          (length a))
    (mean           (mean a))
    (standard-error (standard-error initial-array))
    (percent-point  (percent-point (/ (1+ confidence) 2.0)
           (- count 1)))
    (h              (* standard-error percent-point)))
     (values mean (- mean h) (+ mean h)))))

\end{lstlisting}

In questa sezione si è quindi come la libreria CLAST possa essere utilizzata
come strumento per definire un estensione nativa del linguaggio Common Lisp,
estensione che consenta ad un utente di scegliere ed utilizzare in maniera
semplice e naturale rispetto al linguaggio stesso il modello di valutazione lazy
durante la definizione del proprio sistema.


\endgroup

\chapter{Conclusioni}

Lo scopo della tesi svolta è stato quello di progettare ed implementare una
libreria per la generazione di abstract syntax tree nel contesto di un
linguaggio di programmazione ad ambiente dinamico ed estensibile, il linguaggio
Common Lisp, studiando quindi le possibili applicazioni di questa. Applicazioni
sia come fondamento per la creazione di strumenti di source code analysis, sia
come infrastruttura in grado di abilitare la definizione di estensioni del
linguaggio Common Lisp stesso.\\

L'obiettivo primario della tesi è quindi stato rappresentato dalla progettazione
della libreria CLAST. Tra gli utilizzi anticipati della libreria, uno dei più
significativi è sicuramente quello rappresentato dalla possibilità di costruire
uno strumento di source code analysis a partire da questa. Per questa ragione,
si è mostrato come questa si integri alla tradizionale architettura a tre
componenti di uno strumento di source code analysis. Si è presentato come
specifiche scelte progettuali abbiano cercato di rendere il più semplice
possibile questo procedimento di integrazione e come questa consente ad un
utente di potersi dedicare interamente allo sviluppo della componente di analisi
relativa allo studio delle proprietà di proprio interesse.

Si è quindi illustrata la struttura interna della libreria, le sue componenti e
l'API offerta da questa. Durante questa presentazione sono state descritte le
principali difficoltà del processo di definizione di un insieme di elementi che
consentano la rappresentazione di un programma Common Lisp ad un livello che
consente lo svolgimento di analisi anche molto complesse. Difficoltà dovute sia
alla mancanza di riferimenti relativi allo specifico linguaggio sia alla novità
dello studio svolto da questa tesi. Infine, si è presentato il funzionamento
della libreria, le scelte progettuali ed implementative legate alla definizione
di questo.\\

Il secondo obiettivo fondamentale della tesi è stato quello di studiare le
possibili applicazioni della libreria CLAST, sia alla creazione di strumenti per
la source code analysis, sia all'utilizzo di questa come fondamento per la
definizione di estensione del linguaggio Common Lisp stesso.

In particolare, sono stati presentati tre casi di studio relativamente ai quali
si è analizzato l'utilizzo della libreria come strumento abilitante: la
costruzione di strumenti per la source code analysis, l'estensione del
linguaggio allo scopo di introdurre un supporto al pattern matching ed infine
l'estensione del linguaggio allo scopo di introdurre un supporto nativo alla
lazy evaluation.\\

Si è quindi presentato come la libreria consenta di definire strumenti di
analisi statica come ad esempio strumenti di linting, strumenti in grado di
identificare potenziali inconstenze all'interno di un programma e produrre
segnalazioni relative a queste. Il caso di strumento di linting analizzato è
stato quello di uno strumento per la free variable analysis, la ricerca di
variabili libere all'interno di un dato frammento di codice, rappresentato da
una form nel contesto del linguaggio Common Lisp. Si è quindi mostrato come
questo possa essere realizzato in maniera estremamente semplice a partire dalle
funzionalità di analisi del codice di un programma offerte dalla libreria.\\

La successiva applicazione della libreria analizzata dalla tesi è stata quella
in funzione della definizione di un'estensione del linguaggio Common Lisp allo
scopo di introdurre un supporto nativo al pattern matching. Sono stati
presentati i vantaggi che la disponibilità di questo meccanismo introduce a
livello di linguaggio di programmazione e come questo possa essere introdotto a
partire dall'applicazione di un algoritmo offerto dallo stato dell'arte.

Si è quindi descritto come questo algoritmo possa essere implementato in maniera
tale da produrre operazioni di riscrittura del codice sorgente a partire dal
costrutto macro Common Lisp e come un'implementazione diretta di questo
algoritmo porti alla produzione di codice con un grande numero di variabili
libere, codice quindi meno efficiente rispetto a quello che si potrebbe ottenere
utilizzando i costrutti nativi del linguaggio.

La libreria CLAST è stata applicata in questo contesto come strumento che
consente l'identificazione delle variabili libere introdotte dalla riscrittura
operata a partire dall'algoritmo e permette quindi di operare una riscrittura
del codice sorgente che elimini queste variabili. Questo ha consentito di
definire un processo a partire dal quale è possibile introdurre un supporto al
pattern matching garantendo un livello di performance di questo del tutto
equivalente a quelle che si otterrebbe utilizzando i costrutti nativi del
linguaggio.\\

L'ultimo caso di studio nel contesto del quale è stata studiata l'applicazione
della libreria CLAST è stato quello dell'aggiunta di un supporto alla lazy
evaluation al Common Lisp. Si sono quindi presentate le caratteristiche di
questo modello e come questo si contrapponga al modello di valutazione strict
adottato dal linguaggio, presentando vantaggi e svantaggi di ciascuno di questi.

Anche in questo caso di è presentato come l'introduzione di questo modello di
valutazione sia possibile combinando operazioni di riscrittura, definite a
partire da elementi dello stato dell'arte, alle funzionalità di introspezione a
livello di codice sorgente offerte dalla libreria CLAST. Sono inoltre stati
discussi i diversi problemi che si incontrano nel corso della realizzazione di
questa seconda estensione, soprattutto dal punto di vista dell'API offerte ad un
utente del linguaggio per l'accesso a questa funzionalità.\\

L'aspetto più caratteristico di queste estensioni è rappresentato dal fatto che
introducono nuove funzionalità ad un linguaggio di programmazione agendo in
maniera del tutto trasparente rispetto ad un utente del linguaggio, che non ha
quindi la necessità di utilizzare strumenti esterni, come preprocessors o
transpilers, senza dover apportare modifiche ad un compilatore e, anzi, in
maniera del tutto portabile da un'implementazione all'altra, un aspetto
particolarmente significativo nel contesto del linguaggio analizzato dalla tesi.

Gli approcci mostrati durante la realizzazione delle estensioni del linguaggio
descritte nel corso della tesi possono essere riutilizzati allo scopo di
consentire l'introduzione, con le stesse caratteristiche di efficienza e
portabilità, anche di altre funzionalità al linguaggio Common Lisp, non
analizzate nel corso di questa tesi.

Inoltre, l'approccio e l'idea di procedere all'estensione di un linguaggio di
programmazione attraverso la definizione di operazioni di riscrittura e
meta-programmazione a livello del codice sorgente di un programma sono anch'esse
riutilizzabili nel contesto di qualsiasi altro linguaggio di programmazione,
presente o futuro, che disponga di un costrutto per la riscrittura di codice
sorgente, equivalente al costrutto macro Common Lisp, e all'introspezione a
livello di codice sorgente, ossia una funzionalità equivalente all'Environments
API.

\subsection{Sviluppi Futuri}

In questa sezione vengono presentati alcuni potenziali sviluppi futuri del
lavoro presentato da questa tesi, alcuni dei quali presi in considerazione
durante il progetto di tesi stesso ma non affrontati per mancanza di risorse,
altri invece semplicemente auspicati ma dei quali si è tenuto conto sia in fase
di progettazione che di implementazione della libreria.\\

In questa tesi, e in particolare nel corso del Capitolo \ref{applications}, sono
state presentate alcune applicazioni della libreria allo scopo di costruire
sturmenti di analisi. Un primo aspetto potrebbe risultare interessante
approfondire è rappresentato dallo studio dell'applicazione della libreria CLAST
allo scopo di realizzare uno strumento di type checking, uno strumento che
consenta di ottenere i vantaggi e le garanzie caratteristiche della tipicazzione
statica all'interno di un linguaggio dinamico come il Common Lisp.

Il problema dell'aggiunta di una tipizzazione statica ad un linguaggio dinamico
rappresenta un problema di grande attualità nel contesto dei linguaggi di
programmazione. La libreria CLAST può consentire, a partire dalle stesse analisi
mostrate nel contesto del Capitolo \ref{applications} relativamente alla free
variables analysis e al pattern matching, la creazione di un moto di inferenza
relativamente ai tipi di un programma che permetta di definire un processo
alternativo a quello di annotazione e compilazione source-to-source attualmente
adottato dalla grande maggioranza delle soluzioni in questo ambito.\\

Un secondo aspetto che potrebbe essere approfondito ulteriormente rispetto a
quanto mostrato nel contesto di questa tesi riguarda il procedimento descritto
per l'estensione del linguaggio Common Lisp relativamente al pattern matching.
Questa tesi ha descritto un procedimento grazie al quale questo meccanismo
potrebbe essere aggiunto al linguaggio operando una traduzione diretta rispetto
al costrutto \texttt{CASE} LML. Il costrutto \texttt{CASE} rappresenta infatti
un costrutto di base al quale può essere ricondotto qualsiasi altro costrutto di
pattern matching. Potrebbe però essere interessante studiare come il pattern
matching possa essere adattato allo specifico contesto del linguaggio Common
Lisp, attraverso la definizione di una struttura ed una sintassi che possa
risultare maggiormente naturali rispetto al linguaggio.\\

Infine, un ultima ma non meno interessante tematica che potrebbe essere
approfondita relativamente alle quanto presentato in questa tesi è rappresentata
dalla reinterpretazione delle tecniche descritte nel contesto di linguaggi di
programmazione tradizionali. In particolare, potrebbe essere interessante
studiare l'implementazione e l'applicazione delle tecniche di meta
programmazione a livello di codice sorgente di linguaggi di programmazione
largamente diffusi e fortemente fortemente dinamici, come ad esempio il
linguaggio JavaScript e il linguaggio Python. Il forte dinamismo che li
contraddistingue e la disponbilità di strumenti che consentano la generazione di
AST rende infatti entrambi questi linguaggi candidati ideali allo studio
dell'applicazione delle tecniche descritte da questa tesi.

\defbibfilter{papers}{
  type=article or
  type=inproceedings
}
\printbibliography[filter=papers,title={Articoli Citati}]
\printbibliography[type=book,title={Bibliografia}]
\printbibliography[type=misc,title={Sitografia}]

\end{document}
