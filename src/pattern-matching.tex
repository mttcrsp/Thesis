\section{Supporto al Pattern Matching}
\label{pattern-matching}

All'interno di questa sezione viene presentata una delle principali applicazioni
della libreria CLAST, l'utilizzo delle potenzialità di source code analysis
offerte da questa allo scopo di introdurre un supporto al pattern matching
all'interno del linguaggio Common Lisp stesso.

Nel corso di questa sezione viene quindi dapprima introdotto il concetto di
pattern matching e il funzionamento di questo meccanismo, comune a diversi
linguaggi di programmazione funzionali, come ad esempio il linguaggio ML e il
linguaggio Haskell.

Viene poi mostrato un esempio di implementazione di questo costrutto nel
contesto di un linguaggio di programmazione, rappresentato in particolare dal
costrutto \texttt{CASE} del linguaggio LML.

Il linguaggio Common Lisp non offre un supporto nativo al pattern matching,
anche se i concetti alla base del funzionamento di questo meccanismo possono
essere rintracciati in diversi aspetti del linguaggio.

Viene quindi mostrato come sia possibile operare un'estensione del linguaggio
allo scopo di introdurre un supporto al pattern matching a partire dal lavoro di
L. Augustsson, co-autore del linguaggio LML ed esponente di spicco della
comunità Haskell, non che autore di uno dei primi e più noti per lo stesso
linguaggio di programmazione.

Viene quindi presentato il problema dell'inefficienza introdotta da
un'implementazione diretta dei principi descritti da Augustsson in
\cite{DBLP:conf/lfp/Augustsson84} e si mostra infine come la libreria CLAST
consenta di risolvere questi problemi, abilitando l'estensione del linguaggio
Common Lisp in maniera tale da introdurre un supporto del tutto nativo al
pattern matching.

\subsection{Pattern Matching}

Formalmente, il \textit{pattern matching} viene definito come un meccanismo che
consente di identificare e scomporre le componenti di un certo pattern a partire
dal confronto con i valori presenti all’interno di una data struttura dati. Nel
caso in cui siano presenti variabili all’interno del pattern specificato, il
meccanismo prevede che queste vengano inizializzate al valore corrispondente
specificato all’interno della struttura dati.

Il pattern matching è considerato essere uno strumento di fondamentale
importanza nel contesto dei linguaggi di programmazione funzionali. Proprio per
questa ragione, nell’ambito dei linguaggi di programmazione, è proprio questa
classe di linguaggi quella in cui si è vista la maggiore diffusione e adozione
di questo meccanismo. Da un punto di vista storico, i primi linguaggi funzionali
ad introdurre un supporto al pattern matching sono stati il linguaggio SASL, un
linguaggio funzionale puro definito nel 1972 da D. Turner
\cite{DBLP:journals/spe/Turner79}, e il linguaggio Hope
\cite{DBLP:conf/lfp/BurstallMS80}, sviluppato nel 1980 da un gruppo di
ricercatori presso l'università di Edimburgo. Il primo utilizzo nel contesto di
un linguaggio di programmazione più diffuso risale però all’utilizzo nel
contesto del linguaggio di programmazione ML nel 1973, prima all’interno del
dialetto Standard-ML (SML) \cite{milner1997definition} e poi con il dialetto
Lazy-ML (LML) \cite{DBLP:conf/lfp/Augustsson84}.

Nel corso degli anni, il pattern matching è divenuto un elemento comune
all'interno di molti linguaggi di programmazione. Nel linguaggio Haskell
\cite{DBLP:conf/hopl/HudakHJW07}, un linguaggio anch'esso di tipo funzionale e
puro nato, nel 1990. Nel contesto dell'Haskell il pattern matching svolge
addirittura un ruolo centrale per il funzionamento complessivo del linguaggio,
rappresentando lo strumento fondamentale, e più appropriato, per la gestione dei
tipi di dato algebrici che lo contraddistinguono. Per concludere questo
riferimento storico, è importante sottolineare che anche linguaggi molto
recenti, come ad Rust \cite{rust2016}, Elixir \cite{laurent2014introducing} ed
Elm \cite{elm2016}, hanno scelto di introdurre un supporto al pattern
matching.

Per meglio illustrare la definizione fornita nel paragrafo precedente, viene
presentato un esempio di costrutto che implementa un supporto al meccanismo di
pattern matching all'interno di un reale linguaggio di programmazione. In
particolare, l'esempio di implementazione proposto in seguito fa riferimento al
linguaggio LML e al costrutto \texttt{CASE} definito da questo. Si è scelto di
presentare questo particolare esempio di implementazione dei principi del
pattern matching in quanto particolarmente rappresentativo dei principi generali
alla base del meccanismo.

Il costrutto \texttt{CASE} consente ad un utente del linguaggio ML di esaminare
il valore di un elemento ed eseguire codice in modo condizionale a partire da
questa valutazione.

\newpage

\begin{lstlisting}[
  caption=Esempio di utilizzo del costrutto \texttt{CASE} LML,
  label={lst:case}
]

case e in
    p1 : e1
||  p2 : e2
...
||  pn : en
end

\end{lstlisting}

Nel listato \ref{lst:case}, i $p_i$ rappresentano i pattern che devono essere
verificati durante l'esecuzione, $e$ rappresenta un’espressione, detta
\textit{discriminante}, ossia un valore o insieme di valori a partire dai quali
deveno essere verificati i pattern $p_i$ per identificare la presenza di un
match. Infine, ciascun $e_i$ rappresenta l'espressione che deve essere ritornata
nel caso in cui il pattern $p_i$ venga individuato come match rispetto al
discriminante.

Un \textit{pattern} è rappresentato da un insieme di dimensione arbitraria ed
eterogeneo, in cui ciascun elemento può essere classificato o come una costante
o come una variabile. Un pattern composto da una sola variabile o costante viene
detto \textit{semplice}.

Gli elementi che compongono un'espressione, come anche i pattern che vanno ad
essere verificati a partire da questa, possono essere organizzati all’interno di
una lista o di un albero, costruito a partire da liste innestate.

Nel caso invece di valori costanti, il matching operato tra valore specificato
dal pattern e corrispondente valore presente all'interno della struttura dati è
di tipo esatto. Nel caso in cui un pattern $p_i$ riporti una variabile al suo
interno, questa viene matchata in modo incondizionato, e il valore
corrispondente a questa all’interno dell’espressione $e$ viene utilizzato per
effettuare l’inizializzazione di questa nel contesto della valutazione
dell'espressione $e_i$.

Una convenzione che le diverse varianti di ML adottano, come anche molti altri
linguaggi, è quella di riservare il nome di variabile \texttt{\_} per indicare
una variabile al cui valore non si è interessati. Nel caso di utilizzo di un
\texttt{\_} all’interno di un pattern, si effettuerà un matching incondizionato,
come nel caso di una qualsiasi variabile, ma il valore associato a questo match
verrà semplicemente scartato invece che essere utilizzato per l’inizializzazione
della variabile.

Il valore prodotto quindi dalla valutazione di un'istanza del costrutto
\texttt{CASE} è quello prodotto dalla valutazione dell’espressione $e_i$
associata al primo pattern $p_i$ identificato come match rispetto
all’espressione discriminante $e$.

Durante la valutazione di un'istanza del costrutto \texttt{CASE}, valgono le
seguenti proprietà:

\begin{itemize}

\item i diversi pattern vengono verificati a partire dall’ordine di definizione,
ossia dall’alto verso il basso, così come anche le diverse componenti interne a
ciascun pattern, ossia da sinistra verso destra;

\item nel caso in cui i pattern non siano sufficienti a coprire l’intero insieme
di variabilità dell’espressione discriminante, il compilatore aggiunge un
ulteriore pattern in grado di matchare un qualsiasi valore e associa a questo
un’espressione che porterà alla generazione di un errore a runtime;

\item l’espressione discriminante $e$ viene valutata solamente quanto basta per
poter identificare il primo pattern con il quale si ha un match;

\item é ammesso che i pattern si sovrappongano tra di loro, ma non è ammesso che
un pattern si sovrapponga completamente ad uno dei pattern successivi ad esso.
Questo perché altrimenti il pattern successivo non potrebbe mai essere matchato
e il linguaggio considera questo come un errore di programmazione da parte
dell'utente.

\end{itemize}

La scelta del linguaggio ML di definire un ordinamento totale tra i diversi
pattern utilizzando l’ordine di definizione è una scelta arbitraria ma condivisa
da molte altre implementazioni del pattern matching e che consente di
identificare univocamente l'espressione che dovrà essere ritornata nel caso in
cui più pattern vadano a matchare l'espressione discriminante.

Una verifica compiuta in assenza di un ordinamento totale richiederebbe infatti
la definizione un altro meccanismo per la risoluzione dei conflitti nel caso in
siano presenti più pattern per il quale si ha un match con l’espressione
discriminante. Tuttavia la definizione di un meccanismo generale per risolvere
questo genere di problemi non risulta particolarmente intuitiva e lavorerebbe
necessariamente per singoli casi, richiedendo inoltre la definizione di una
funzione di distanza che consenta di identificare il match più specifico
rispetto all’espressione discriminante in grado di agire a prescindere dal tipo
di espressione e pattern in analisi. Molte implementazioni ritengono che questo
processo aggiunga un overhead alla programmazione significativo e difficilmente
giustificabile.

\subsubsection{Pattern matching come strumento}

È importante sottolineare che nonostante gran parte delle implementazioni di
strumenti per il supporto al pattern matching all’interno di diversi linguaggi
di programmazione utilizzino un’interpretazione delle idee fondamentali molto
simile a quella appena esposta, non tutte queste implementazioni hanno
necessariamente lo stesso funzionamento e/o processo di esecuzione descritto
all'interno dei paragrafi precedenti.

Tuttavia, come accennato in precedenza, i principi del funzionamento del
costrutto \texttt{CASE} appena riportato possono essere rintracciati in maniera
del tutto analoga anche nel contesto di implementazioni dei principi alla base
del pattern matching che utilizzano linguaggi di programmazione diversi e che si
pongono obiettivo completamente diversi.

Ad esempio, nel linguaggio Haskell, sviluppato da un gruppo di sviluppatori tra
cui L. Augustsson, coautore di LML e della definizione del costrutto
\texttt{CASE} presente in questo, il processo di definizione e valutazione di
una funzione lavora in modo molto simile a quanto appena riportato e rappresenta
un elemento centrale al funzionamento del linguaggio.

\begin{lstlisting}[caption=Esempio di definizione di una funzione Haskell naïve
per il calcolo del fattoriale]

factorial 0 = 1
factorial n = n * factorial(n-1)

\end{lstlisting}

Il precedente esempio definisce una prima versione della funzione fattoriale, il
cui input ammesso è il valore intero 0, e che ritorna il valore intero 1. La
seconda definizione specifica invece come input il nome di una variabile, la
quale andrà ad essere matchata in modo incondizionato e quindi a rappresentare
un qualsiasi parametro fornito in input alla funzione, e ritorna il risultato di
una computazione ricorsiva operata a partire dal valore del parametro in input.

All’invocazione della funzione \texttt{factorial} vengono applicate delle regole
del tutto analoghe a quelle definite in precedenza:

\begin{itemize}

\item le diverse definizioni della funzione vengono valutate in ordine di
definizione, come anche i parametri specificati da ciascuna di queste. Nel caso
in si abbia un match tra i parametri specificati dall'invocazione e quelli
specificati dalla definizione in analisi, il codice associato a quella
particolare definizione viene eseguito.

\item Nel caso in cui non sia presente una definizione una della funzione in
grado di matchare i parametri di invocazione, il runtime del linguaggio produrrà
un errore nel quale indicherà che i pattern associati alla funzione
\texttt{factorial} non sono sufficienti per poter valutare la chiamata
\footnote{\texttt{Non-exhaustive patterns in function 'factorial'}}).

\item i parametri di input vengono valutati solamente quanto basta per poter
identificare la prima definizione della funzione con la quale si ha un match.

\end{itemize}

Nel contesto del linguaggio Haskell, il pattern matching viene quindi utilizzato
anche come meccanismo alla base del dispatching delle chiamate ad una funzione.
Un’aspetto di questa implementazione che è significativo sottolineare è
rappresentato dal fatto che, a differenza dei più tradizionali meccanismi per il
dispatching di chiamate a funzione utilizzati da linguaggi di programmazione più
comuni, come ad esempio il linguaggio Java, il dispatching delle chiamate viene
effettuato andando a verificare l’effettivo valore dei parametri specificati
dall’invocazione, e non solamente il tipo di questi. Questo in maniera molto
siimile a quanto presentato dalla Sezione \ref{generic-functions} relativamente
al meccanismo delle funzioni generiche Common Lisp.

Il fatto che il pattern matching possa essere utilizzato come strumento che
consente di introdurre aspetti come l’esecuzione condizionata e dynamic dispatch
all’interno di un linguaggio di programmazione può fornire un’idea della potenza
espressivo del meccanismo e di quanto l'accesso a questo possa arricchire un
linguaggio di programmazione.

Dopo aver presentato nei precedenti paragrafi una definizione di pattern
matching, i principi alla base di questo ed alcuni esempi di utilizzo di questo
meccanismo, in seguito vengono analizzati i principali vantaggi che un
linguaggio di programmazione può ottenere introducendo un supporto al pattern
matching e come il linguaggio Common Lisp si relazioni con il pattern
matching.

\subsubsection{Vantaggi legati al Pattern Matching}

Alcuni dei vantaggi che l'introduzione di un supporto al pattern matching porta
all’interno di un linguaggio di programmazione sono legati in prima battuta al
fatto che questo consente, in maniera semplificata, l’utilizzo di un paradigma
di programmazione dichiarativo piuttosto che imperativo. Questo permette di
evidenziare in modo chiaro l’intento di un frammento di codice, utilizzando una
forma particolarmente leggibile.

L’utilizzo di un approccio come quello descritto in precedenza per la
definizione di una funzione rende immediatamente visibili quali siano le
precondizioni e le supposizioni rispetto ai parametri di input ad una funzione;
un compito per il quale un approccio imperativo necessiterebbe di una struttura
apposita per la formulazione di asserzioni rispetto ai parametri in input.
Struttura che infatti molto spesso viene aggiunta a linguaggi di questo tipo ma
che, molto spesso, non è in grado di fornire le garanzie che è in grado di
fornire un approccio dichiarativo, in cui la validità delle precondizioni, o
asserzioni, rispetto ai parametri può essere garantita già in fase di
compilazione, prevenendo il potenziale generarsi di errori a runtime.

Il vantaggio più significativo che la disponibilità di un supporto al pattern
matching porta all’interno di un linguaggio di programmazione è però
rappresentato dal fatto che questo consente di trasferire lavoro dall’utente del
linguaggio al compilatore. Lavoro, come ad esempio la validazione di asserzioni
e precondizioni e la gestione del branching, che risultano particolarmente
soggetti ad errori di programmazione e disattenzione, e per i quali un
compilatore è tipicamente in grado di produrre codice corretto con miglior tasso
di successo. Oltre alla correttezza, un ulteriore aspetto che un utilizzo più
intensivo del compilatore porta ad ottenere è relativo alle prestazioni di un
programma: avendo a disposizione maggiori informazioni rispetto al funzionamento
e ad i meccanismi di rappresentazione interni del linguaggio di programmazione,
un compilatore è tipicamente in grado di produrre codice più efficiente rispetto
a quello che potrebbe essere prodotto da un utente del linguaggio.

\subsection{Common Lisp e Pattern Matching}

Il linguaggio Common Lisp non fornisce un supporto diretto al pattern matching.
Tuttavia il concetto di pattern matching può essere rintracciato in diversi
aspetti del linguaggio.

Ad esempio, il costrutto \texttt{DESTRUCTURING-BIND} è una macro fornita
nativamente dal linguaggio per la destrutturazione di liste, anche
arbitrariamente innestate, all’interno delle loro componenti elementari.

\begin{lstlisting}[
  caption=Esempio utilizzo della macro \texttt{DESTRUCTURING-BIND},
  label={lst:destructuring-bind}]

CL-USER> (destructuring-bind (a (b) c)
       (list 1 (list 2) 3)
     (values a b c))
1
2
3

\end{lstlisting}

\begin{lstlisting}[caption=Sintassi della macro \texttt{DESTRUCTURING-BIND}]

destructuring-bind lambda-list expression declaration* form*
=> result*

\end{lstlisting}

\texttt{DESTRUCTURING-BIND} rappresenta una forma elementare di pattern matching
in cui viene ammesso un singolo pattern, il quale può solamente essere composto
da variabili, e che essendo unico dovrà necessariamente matchare correttamente
l’espressione in input affinché non si generino errori. Oltre alle limitazioni
appena presentate, il costrutto non ammette l'utilizzo di costanti e wildcard.
\footnote{In sostanza il costrutto non opera un reale matching degli elementi di
una struttura, ma solamente una scomposizione di questi all'interno dell'insieme
di variabili specificato. Tuttavia l'effetto prodotto dall'utilizzo di questo
costrutto risulta molto simile, come mostrato dal Listato
\ref{lst:destructuring-bind}.}

Un altro aspetto del linguaggio in cui è possibile rintracciare principi del
pattern matching è rappresentato dal meccanismo di dispatching dei metodi,
presentato nella Sezione \ref{generic-functions}, in cui il corretto metodo, o
insieme di metodi, che dovrà essere eseguito in risposta ad una particolare
invocazione di una funzione generica viene identificato dal sistema Common Lisp
utilizzando un meccanismo di matching molto simile a quello del pattern
matching.

Nonostante l'assenza di un supporto nativo al pattern matching, la forte
estensibilità che contraddistingue il Common Lisp, dovuta soprattutto alla
presenza del costrutto macro descritto nel corso dalla Sezione \ref{macro},
rende possibile l'introduzione di un costrutto per il pattern matching in
maniera nativa.

Nel corso del progetto di tesi, si è studiato come questo potesse essere
possibile soprattutto a partire dal lavoro presentato da L. Augustsson in
\cite{DBLP:conf/fpca/Augustsson85}. All'interno di questo documento, l'autore
descrive infatti l'implementazione e la compilazione un costrutto per il pattern
matching, in particolare relativamente al costrutto \texttt{CASE} discusso in
precedenza. Mentre gli aspetti di generazione di codice macchina non risultano
di particolare interesse nel contesto di questa tesi, l'aspetto più
significativo del documento è rappresento dal procedimento che l'autore descrive
per la traduzione del codice di pattern matching all'interno di un insieme di
funzioni mutuamente ricorsive.

Questo processo di traduzione risulta di grande interesse in quanto, essendo un
procedimento puramente funzionale, rappresenta un meccanismo particolarmente
adatto all'introduzione di un supporto al pattern matching nel contesto del
linguaggio Common Lisp; al contrario di differenti approcci alla compilazione
del pattern matching, come ad esempio quello descritto da Maranget et al. in
\cite{DBLP:conf/icfp/FessantM01} e \cite{DBLP:conf/lfp/Maranget92}, che
risultano significamente più complessi a livello di implementazione utilizzando
il linguaggio Common Lisp.

Una descrizione approfondita del funzionamento dell'algoritmo presentato in
\cite{DBLP:conf/fpca/Augustsson85} esula dagli obiettivi di questa tesi. Si
rimanda infatti al documento stesso per i dettagli relativi all'implementazione
di questo. Tuttavia, l'intuizione alla base dell'algoritmo è la seguente: il
pattern matching può sempre essere riscritto come una più semplice sequenza di
operazioni di verifica, if statements, rispetto al tipo o all'identità di una
data variabile.

In particolare, questa operazione di riscrittura avviene in due passi distinti.

\begin{enumerate}

\item Tutti i pattern che utilizzano uno stesso costruttore al più alto livello
di grouping possono essere raccolti in un unico pattern, rimpiazzato i diversi
sotto-pattern con delle variabili.

\item Vengono definiti nuove case expression, più semplici, che analizzano i
sotto-pattern, espressioni che, procedendo in modo ricorsivo, verranno trattate
in maniera analoga durante la prossima iterazione dell'algoritmo.

\end{enumerate}

Come affermato nel paragrafo precedente, l'algoritmo procede ad una riscrittura
ricorsiva del frammento di codice contenente una case expression. Mentre il
procedimento appena descritto rappresenta il passo ricorsivo dell'algoritmo, il
caso base di questo è rappresentato dal caso in cui ciascun pattern presente nel
frammento di codice è rappresentato da un pattern semplice; pattern che può
quindi essere riscritto come un semplice if statement.

Il Listato \ref{lst:compiling-pattern-matching} mostra l'azione dell'algoritmo a
partire da un esempio.

\newpage

\begin{lstlisting}[
  caption=Esempio di funzionamento dell'algoritmo per la compilazione del pattern matching di L. Augustsson (continua),
  label={lst:compiling-pattern-matching}
]
(* Original pattern matching expression *)

case e in
    [x;true] : x
||  [false] : true
||  z : false
end

(* First rewrite *)
case e in
    cons(x, cons(true, nil)) : x
||  cons(false, nil) : true
||  z : false
end

(* Second rewrite *)
case e in
    cons(x, v2) :
        case x in
            false :
                case v2 in
                    nil : true
                ||  _ : default
                end
        ||  v3 : case v2 in
                     cons(true, nil) : x
                 ||  _ : default
                 end
        end
||  z : false
end

\end{lstlisting}

\newpage

\begin{lstlisting}[
  caption=Esempio di funzionamento dell'algoritmo per la compilazione del pattern matching di L. Augustsson,
  label={lst:compiling-pattern-matching-continues}
]
(* Final rewrite *)
case e in
    cons(x, v2) :
        case x in
            false :
                case v2 in
                    nil : true
                ||  _ : default
                end
        ||  v3 : case v2 in
                     cons(v4, v5) :
                         case v4 in
                             true :
                                 case v5 in
                                     nil : x
                                 ||  _ : default
                                 end
                         ||  _ : default
                         end
                 ||  _ : default
                 end
        end
||  z : false
end
\end{lstlisting}

La riscrittura del codice prodotta in output dall'algoritmo può quindi essere
operata in maniera del tutto analoga utilizzando il linguaggio Common Lisp,
utilizzando i costrutti nativi \texttt{CASE}, molto simile al costrutto
\texttt{switch} diffuso nei linguaggio di programmazione tradizionali, e
\texttt{LET}, costrutto utilizzato per la definizione di variabili locali. Il
Listato \ref{lst:cl-pm-compiled} mostra come il risultato del processo di
riscrittura appena descritto.

\newpage

% (case e
%   ((eql 'cons (type-of e))
%    (let ((x (car e))
%    (v2 (cdr v2)))
%      (ecase x
%        (false
%   (case v2
%     (nil t)
%     (otherwise (error "Unmatched expression"))))
%        (otherwise
%   (let ((v3 x))
%     (case v2
%       ((eql 'cons (type-of v2))
%        (let ((v4 (car v2))
%        (v5 (cdr v2)))
%          (case v4
%      (true
%       (case v5
%         (nil x)
%         (otherwise (error "Unmatched expression"))))
%      (otherwise (error "Unmatched expression")))))
%       (otherwise (error "Unmatched expression"))))))))
%   (otherwise nil))

\begin{lstlisting}[
  caption=Risultato del processo di riscrittura del pattern matching in Common Lisp,
  label={lst:cl-pm-compiled}
]

(match e
  ((cons x t) x)
  (nil t)
  (z nil))

(case e
  ((eql 'cons (type-of e))
   (let ((x (car e))
   (v2 (cdr v2)))
     (case x
       (false
  (let ((ev2 t))
    (case v2
      (nil t)
      (ev2 (error "Unmatched expression")))))
       (otherwise
  (let ((v3 x)
        (ev3 t))
    (case v2
      ((eql 'cons (type-of v2))
       (let ((v4 (car v2))
       (v5 (cdr v2))
       (ev4 t))
         (case v4
     (true
      (let ((ev5 t))
        (case v5
          (nil x)
          (ev5 (error "Unmatched expression")))))
     (t (error "Unmatched expression")))))
      (ev3 (error "Unmatched expression"))))))))
  (otherwise nil))

\end{lstlisting}

\newpage

Si possono quindi compiere diverse osservazioni rispetto al funzionamento
dell'algoritmo. Una prima osservazione è che la quantità di codice generata da
questo è molto maggiore rispetto alla quantità di codice originale. Questo
tuttavia risulta poco significativo, in quanto si trattarebbe di codice generato
in maniera automatica in fase di compilazione attraverso l'utilizzo del
costrutto macro, e che quindi risulterebbe sostanzialmente invisibile ad un
utente del linguaggio.

Una seconda osservazione è invece relativa all'efficienza dell'algoritmo. Si può
infatti osservare che l'algoritmo porta alla generazione di un grande numero di
variabili, un aspetto che incide sia dal punto di vista dei tempi di
compilazione sia dal punto di vista dell'efficienza a runtime di un programma
che utilizza questo meccanismo. Si evidenzia inoltre che molte di queste
variabili non vengono poi utilizzate effettivamente dal programma ma risultano
essere variabili libere per tanto potenzialmente eliminabili dal programma.

Teoricamente, la fase di ottimizzazione di svolta da un compilatore dovrebbe
essere in grado di compiere un'operazione di rimozione di queste variabili. Cosa
che consentirebbe di ottenere codice di pattern matching con la stessa
efficienza che caratterizza gli altri costrutti offerti dal linguaggio.

Sfortunatamente i compilatori per il linguaggio Common Lisp, come anche la gran
parte dei compilatori per linguaggi analoghi, non sono solitamente in grado di
compiere questo tipo di ottimizzazione.

Tuttavia, grazie alla presenza del costrutto macro non è richiesto che sia il
compilatore stesso a compiere questo passo di ottimizzazione. Un programma
Common Lisp stesso è infatti in grado, a partire dal meccanismo rappresentato
dalle macro, di operare una riscrittura del proprio codice sorgente e quindi,
potenzialmente, anche di operare una riscrittura in cui si compie
un'eliminazione delle variabili precedentemente descritte.

Questo, fino a prima della costruzione della libreria CLAST risultava
impossibile, non era infatti disponibile alcun componente che consentisse
l'analisi di codice sorgente scritto in Common Lisp, e risultava quindi
impossibile costruire uno strumento che consentisse l'identificazione di
particolari elementi all'interno di questo, come ad esempio le variabili
precedentemente citate, operando in sostanza operazioni di meta- programmazione
direttamente a livello di codice sorgente.

La libreria CLAST, che come mostrato nel corso della precedente sezione di
questo capitolo consente di compiere free variables analysis nel contesto del
linguaggio Common Lisp, e consente quindi la costruzione di una soluzione al
problema appena discusso.

Attraverso l'utilizzo della libreria risulta possibile compiere una ricerca che
consenta di identificare le variabili libere all'interno del codice generato da
un'applicazione dell'algoritmo. I risultati potranno quindi essere utilizzati
per compiere una trasformazione \textit{source-to-source} attraverso la quale
queste variabili vengono eliminate, consentendo quindi la generazione di codice
maggiormente efficiente e rendendo possibile l'introduzione di un meccanismo
nativo di pattern matching nel contesto del linguaggio per estensione di
questo.

\image{img/pattern-matching-rewrite.png}
      {Pattern Matching - Diagramma di flusso del processo di riscrittura}
      {fig:pattern-matching-rewrite}
      {0.5}

È inoltre importante sottolineare che la definizione di un supporto al pattern
matching attraverso il procedimento appena descritto risulta del tutto portabile
rispetto alle diverse implementazioni del linguaggio. Aspetto che non si
otterrebe agendo a livello dell'ottimizzazione operata dal compilatore e che
risulta particolarmente significativo nel contesto della frammentata comunità
Common Lisp.
