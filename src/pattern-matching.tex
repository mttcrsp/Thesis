\section{Pattern Matching}

% - A partire dalle informazioni rispetto all'AST e in particolare dal fatto che
% si è particolarmente semplificato il problema della free variable analysis,
% l'aggiunta di un supporto al pattern matching e alla compilazione del pattern
% matching per il linguaggio Common Lisp risulta particolarmente semplificata.
% - Cosa è il pattern matching
% - Il professore ha fatto un discorso immenso sul fatto che la compilazione del
% pattern matching, come descritto dal paper semplice, porta all'introduzione di
% un grande insieme di variabili libere.
% - Questo può essere risolto dalla libreria che è in grado di identificare la
% presenza di variabili libere e ottimizzare la compilazione del pattern matching.

Formalmente, il pattern matching viene definito come un meccanismo che consente
di identificare e scomporre le componenti di un dato pattern a partire dal
confronto con i valori presenti all’interno di una data struttura dati. Nel
caso in cui siano presenti variabili all’interno del pattern specificato, il
meccanismo prevede che queste vengano inizializzate al valore corrispondente
specificato all’interno della struttura dati.\\

Il pattern matching è considerato come uno strumento di fondamentale importanza
nel contesto dei linguaggi di programmazione funzionali. Proprio per questa
ragione, nell’ambito dei linguaggi di programmazione, è proprio questa classe
di linguaggi quella in cui si è vista la maggiore diffusione e adozione di
questo meccanismo. Da un punto di vista storico, i primi linguaggi funzionali
ad introdurre un supporto al pattern matching sono stati il linguaggio SASL, un
linguaggio funzionale puro definito nel 1972 da D. Turner
\cite{DBLP:journals/spe/Turner79}, e il linguaggio Hope
\cite{DBLP:conf/lfp/BurstallMS80}, sviluppato nel 1980 da un gruppo di
ricercatori presso l'università di Edimburgo. Il primo utilizzo nel contesto di
un linguaggio di programmazione più diffuso risale però all’utilizzo nel
contesto del linguaggio di programmazione ML nel 1973, prima all’interno del
dialetto Standard-ML (SML) \cite{milner1997definition} e poi con il dialetto
Lazy-ML (LML) \cite{DBLP:conf/lfp/Augustsson84}.

Nel corso degli anni, il pattern matching è divenuto un elemento comune
all'interno di molti linguaggi di programmazione. Nel linguaggio Haskell
\cite{DBLP:conf/hopl/HudakHJW07}, un linguaggio anch'esso funzionale puro nato
nel 1990, il pattern matching svolge addirittura un ruolo centrale per il
funzionamento complessivo del linguaggio, rappresentando lo strumento
fondamentale, e più appropriato, per la gestione dei tipi di dato algebrici che
contraddistinguono Haskell. Per concludere questo riferimento storico, è
importante sottolineare che anche linguaggi molto recenti, come Rust
\cite{rust2016}, Elixir \cite{laurent2014introducing} ed Elm \cite{elm2016},
hanno scelto di introdurre un supporto al pattern matching.

Per meglio illustrare la definizione di fornita nel paragrafo precedente, in
seguito viene presentato un esempio di costrutto, realmente presente
all'interno di un linguaggio di programmazione, che introduce all'interno del
linguaggio un supporto al pattern matching. In particolare, l'esempio di
implementazione proposto e descritto in seguito fa riferimento al linguaggio
LML, precedentemente citato, e in particolare al costrutto \texttt{CASE}
offerto da questo. Si è scelto di presentare questo particolare esempio di
implementazione dei principi del pattern matching in quanto particolarmente
rappresentativo dei principi generali alla base questo.

Il costrutto \texttt{CASE} consente ad un utente del linguaggio ML di esaminare
il valore di un elemento ed eseguire codice in modo condizionale a partire da
questa valutazione.

\begin{lstlisting}[caption=Il costrutto \texttt{CASE} presente in LML
  rappresenta un esempio di supporto al pattern matching.]
case e in
    p1 : e1
||  p2 : e2
...
||  pn : en
end
\end{lstlisting}

Nell'esempio i $p_i$ rappresentano i pattern che dovranno essere verificati
all’esecuzione del costrutto, $e$ rappresenta un’espressione, detta
\textit{discriminante}, ossia un valore o insieme di valori a partire dai quali
devono essere verificati i pattern $p_i$ per identificare la presenza di
un match, infine ciascun $e_i$ rappresenta l'espressione che deve essere
ritornata nel caso in cui il pattern $p_i$ venga individuato come match.

Un pattern è rappresentato da un insieme di dimensione arbitraria ed
eterogeneo, in cui ciascun elemento può essere classificato o come una costante
o come una variabile. Un pattern composto da una sola variabile o costante
viene detto \textit{semplice}.

Gli elementi che compongono un'espressione, come anche i pattern che vanno ad
essere verificati a partire da questa, possono essere organizzati all’interno
di una lista o di un albero, costruito a partire da liste innestate.

Nel caso invece di valori costanti, il matching operato tra valore specificato
dal pattern e valore corrispondente presente all'interno della struttura dati è
di tipo esatto. Nel caso in cui un pattern $p_i$ riporti una variabile al suo
interno, questa viene matchata in modo incondizionato, e il valore
corrispondente a questa all’interno dell’espressione e viene utilizzato per
effettuare l’inizializzazione di questa nel contesto della valutazione
dell'espressione $e_i$. Una convenzione che le diverse varianti di ML adottano,
come anche molti altri linguaggi, è quella di riservare il nome di variabile
\texttt{\_} per indicare una variabile al cui valore non si è interessati. Nel
caso di utilizzo di un \texttt{\_} all’interno di un pattern, si effettuerà
sempre un matching incondizionato, ma il valore associato a questo match verrà
semplicemente scartato invece che essere utilizzato per l’inizializzazione
della variabile.

Il valore prodotto quindi dalla valutazione di un'istanza del costrutto
\texttt{CASE} è quello prodotto dalla valutazione dell’espressione $e_i$
associata al primo pattern $p_i$ identificato come match rispetto
all’espressione discriminante $e$.\\

Durante la valutazione di un'istanza del costrutto \texttt{CASE}, valgono le
seguenti proprietà:

\begin{itemize}
\item i diversi pattern vengono verificati a partire dall’ordine di
  definizione, ossia dall’alto verso il basso, così come anche le diverse
  componenti interne a ciascun pattern, ossia da sinistra verso destra;
\item nel caso in cui i pattern non siano sufficienti a coprire l’intero
  insieme di variabilità dell’espressione discriminante, il compilatore
  aggiunge un ulteriore pattern in grado di matchare un qualsiasi valore e
  associa a questo un’espressione che porterà alla generazione di un errore a
  runtime;
\item l’espressione discriminante $e$ viene valutata solamente quanto basta per
  poter identificare il primo pattern con il quale si ha un match;
\item é ammesso che i pattern si sovrappongano tra di loro, ma non è ammesso
  che un pattern si sovrapponga completamente ad uno dei pattern successivi ad
  esso. Questo perché il pattern successivo non potrebbe mai essere matchato e
  il linguaggio considera questo come un errore di programmazione da parte
  dell'utente.\\
\end{itemize}

La scelta del linguaggio ML di definire un ordinamento totale tra i diversi
pattern, utilizzando l’ordine di definizione è una scelta arbitraria ma
condivisa da molte altre implementazioni del pattern matching e che consente di
identificare univocamente l'espressione che dovrà essere ritornata nel caso in
cui più pattern vadano a matchare l'espressione discriminante. Una verifica
compiuta in assenza di un ordinamento totale richiederebbe infatti la
definizione un altro meccanismo per la risoluzione dei conflitti nel caso in
siano presenti più pattern per il quale si ha un match con l’espressione
discriminante. Tuttavia la definizione di un meccanismo generale per risolvere
questo genere di problemi non risulta particolarmente intuitiva e lavorerebbe
necessariamente per singoli casi, richiedendo inoltre tipicamente la
definizione di una funzione di distanza che consenta di identificare il match
più specifico rispetto all’espressione discriminante in grado di agire a
prescindere dal tipo di espressione e pattern in analisi. Molte implementazioni
ritengono che questo processo agguinga un overhead significativo e
difficilmente giustificabile alla programmazione.\\

È importante sottolineare che, nonostante gran parte delle implementazioni di
strumenti per il supporto al pattern matching all’interno di diversi linguaggi
di programmazione utilizzino un’interpretazione delle idee fondamentali molto
simile a quella appena esposta, non tutte queste implementazioni hanno
necessariamente lo stesso funzionamento e/o processo di esecuzione.

Tuttavia, come accennato in precedenza, i principi del funzionamento del
costrutto \texttt{CASE} appena riportato possono essere rintracciati in maniera
del tutto analoga anche nel contesto di implementazioni dei principi alla base
del pattern matching che utilizzano linguaggi di programmazione diversi e che
si pongono obiettivo completamente diversi.

Ad esempio, nel linguaggio Haskell, sviluppato da un gruppo di sviluppatori tra
cui L. Augustsson, coautore di LML e della definizione del costrutto
\texttt{CASE} presente in questo, il processo di definizione e valutazione di
una funzione, elemento centrale al funzionamento del linguaggio, lavora in modo
molto simile a quanto appena riportato.

\begin{lstlisting}[caption=Esempio di definizione di una funzione semplificata
  per il calcolo del fattoriale utilizzando il linguaggio Haskell.]
factorial 0 = 1
factorial n = n * factorial(n-1)
\end{lstlisting}

Il precedente esempio definisce una prima versione della funzione factorial, il
cui input ammesso è il valore intero 0, e che ritorna il valore intero 1. La
seconda definizione specifica invece come input il nome di una variabile, la
quale andrà ad essere matchata in modo incondizionato e quindi a rappresentare
un qualsiasi parametro fornito in input alla funzione, e ritorna il risultato
di una computazione ricorsiva costruita partire dal valore del parametro in
input.

All’invocazione della funzione factorial vengono applicate delle regole del
tutto analoghe a quelle definite in precedenza:

\begin{itemize}
\item le diverse definizioni della funzione vengono valutare in ordine di
  definizione come anche i parametri specificati da ciascuna di queste, nel
  caso in si abbia un match tra i parametri specificati dall'invocazione e
  quelli specificati dalla definizione in analisi, il codice associato a quella
  particolare definizione viene eseguito.
\item nel caso non sia presente una definizione una della funzione in grado di
  matchare i parametri di invocazione, il runtime del linguaggio produrrà un
  errore nel quale indicherà che i pattern associati alla funzione f non sono
  sufficienti per poter valutare la chiamata (\texttt{Non-exhaustive patterns
  in function 'factorial'});
\item i parametri di input vengono valutati solamente quanto basta per poter
  identificare la prima definizione della funzione con la quale si ha un match.
\end{itemize}

Nel contesto del linguaggio Haskell, il pattern matching viene quindi
utilizzato anche come meccanismo alla base del dispatching delle chiamate ad
una funzione. Un’aspetto di questa implementazione che è significativo
sottolineare è rappresentato dal fatto che, a differenza dei più tradizionali
meccanismi per il dispatching di chiamate a funzione utilizzati da linguaggi di
programmazione più comuni, come ad esempio il linguaggio Java, il dispatching
delle chiamate viene effettuato andando a verificare l’effettivo valore dei
parametri specificati dall’invocazione, e non solamente il tipo di questi.\\

Il fatto che il pattern matching possa essere utilizzato come strumento che
consente di introdurre aspetti come l’esecuzione condizionata e dynamic
dispatch all’interno di un linguaggio di programmazione può fornire un’idea
della potenza espressivo del meccanismo

Dopo aver presentato nei precedenti paragrafi una definizione di pattern
matching, i principi alla base di questo e alcuni esempi di utilizzo di questo
meccanismo, in seguito vengono analizzati i principali vantaggi che un
linguaggio di programmazione può ottenere introducendo un supporto al pattern
matching e come il linguaggio Common Lisp si relazioni con il pattern
matching.\\

Alcuni dei vantaggi che l'introduzione di un supporto al pattern matching porta
all’interno di un linguaggio di programmazione sono legati in prima battuta al
fatto che questo consente, in maniera semplificata, l’utilizzo di un paradigma
di programmazione dichiarativo piuttosto che imperativo. Questo consente di
evidenziare in modo chiaro l’intento di un frammento di codice utilizzando una
forma leggibile. L’utilizzo di un approccio come quello descritto in precedenza
per la definizione di una funzione rende immediatamente visibili quali siano le
precondizioni e le supposizioni rispetto ai parametri di input ad una funzione;
un compito per il quale un approccio imperativo necessiterebbe di una struttura
apposita per la formulazione di asserzioni rispetto ai parametri in input, che
infatti molto spesso viene aggiunta a linguaggi di questo tipo. Strumento che,
quello delle asserzioni, comunque non è in grado di fornire le garanzie che è
in grado di fornire un approccio dichiarativo, in cui la validità delle
precondizioni, o asserzioni, rispetto ai parametri può essere garantita già in
fase di compilazione, prevenendo il potenziale generarsi di errori a runtime.\\

Il vantaggio più significativo che la disponibilità di un supporto al pattern
matching porta all’interno di un linguaggio di programmazione è però
rappresentato dal fatto che questo consente di trasferire lavoro dall’utente
del linguaggio al compilatore. Lavori, come ad esempio la validazione di
asserzioni e precondizioni e la gestione del branching, che risultano
particolarmente soggetti ad errori di programmazione, e per i quali un
compilatore è tipicamente in grado di produrre con maggior successo un codice
corretto. Oltre alla correttezza, un ulteriore aspetto che un utilizzo più
intensivo del compilatore porta ad ottenere è relativo alle prestazioni di un
programma: avendo a disposizione maggiori informazioni rispetto al
funzionamento e ad i meccanismi di rappresentazione interni del linguaggio di
programmazione, un compilatore è tipicamente in grado di produrre codice più
efficiente rispetto a quello che potrebbe essere prodotto da un utente del
linguaggio.\\

Common Lisp non fornisce un supporto diretto al pattern matching.
Tuttavia il concetto di pattern matching può essere rintracciato in diversi
aspetti del linguaggio.

Ad esempio, il costrutto \texttt{DESTRUCTURING-BIND} è una macro fornita
nativamente dal linguaggio per la destrutturazione di liste, anche
arbitrariamente innestate, all’interno delle loro componenti elementari.

\begin{lstlisting}[caption=Esempio utilizzo della macro
  \texttt{DESTRUCTURING-BIND}.]
(destructuring-bind (a (b) c)
    (list 1 (list 2) 3)
  (values a b c))             ; prints 1 2 3
\end{lstlisting}

\begin{lstlisting}[caption=Signature della macro \texttt{DESTRUCTURING-BIND}.]
destructuring-bind lambda-list expression declaration* form*
=> result*
\end{lstlisting}

\texttt{DESTRUCTURING-BIND} rappresenta una forma elementare di pattern
matching in cui viene ammesso un singolo pattern, il quale può
solamente essere composto da variabili, e che essendo unico dovrà
necessariamente matchare correttamente l’espressione in input affinché non si
generino errori. Oltre alle limitazioni appena presentate, il costrutto non
ammette l'utilizzo di costanti e wildcard. \footnote{In sostanza il costrutto
non opera un reale matching degli elementi di una struttura, ma solamente una
scomposizione di questi all'interno dell'insieme di variabili specificato.}

Tuttavia, nonostante l'assenza di un supporto nativo al pattern matching, la
forte estensibilità che contraddistingue Common Lisp rende possibile
l'introduzione di un costrutto per il pattern matching in maniera naturale.
Come verrà approfondito all'interno delle prossime Sezioni.
