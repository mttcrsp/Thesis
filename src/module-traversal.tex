\section{Modulo di traversal}

Il terzo e ultimo modulo che compone la libreria CLAST è rappresentato dal
modulo di traversal. Lo scopo di questo modulo è quello di consentire agli
utenti della libreria, la ricerca e l'ispezione delle strutture di
rappresentazione prodotte in output dal modulo di parsing ed analisi descritto
dalla sezione precedente. Si tratta quindi del modulo più critico dal punto di
vista dell'API, in quanto rappresenta la principale interfaccia tra CLAST ed
un analizzatore costruito a partire da questa.\\

\image{img/visitor-pattern}
      {Diagramma delle classi relativo al design pattern Visitor}
      {fig:visitor-pattern}
      {0.5}

% (In questa sezione presenta un diagramma delle classi che indichi la
% (struttura gerarchica utilizzata per la rappresentazione. Qua ci vuole un
% (diagramma veramente bello. Fai anche un discorso sull'ereditarietà multipla
% (e come questa venga utilizzata come mixin/trait per la rappresentazione di
% (IMPLICIT-PROGN forms e un altro tipo di forms.)

% MODULO DI TRAVERSAL

% AGGIUNGI APPROFONDIMENTO SUL DESIGN PATTERN VISITOR

% clast-element-subforms: form to list-of-subforms (clast-elements.lisp)

% (correggi questa funzione non viene specializzata per ogni possibile form,
% (viene specializzata per ogni sottoclasse di CLAST-ELEMENT in maniera tale
% (da incapsulare la nozione rispetto alle foglie di ciascun nodo,
% (rappresentato da un'istanza di CLAST-ELEMENT) Questa funzione data una form
% (restituisce l'insieme di tutte le sottoform contenute da questa. Questa
% (funzione risulta fondamentale alla porzione della libreria che si occupa
% (della visita dell'AST prodotto al termine del parsing, in quanto consente
% (la navigazione all'interno di questo. (aggiungi che parte del pattern
% (Visitor implementa questa funzione)

% map-subforms: form func result-type to form (walk.lisp)

% MAP-SUBFORMS è una funzione generica che rappresenta lo strumento a partire
% dalla quale viene implementata la visita dell'AST, accessibile dagli utenti
% della libreria tramite la funzione WALK. Ciascun metodo si occupa di
% implementare la navigazione all'interno di una specifica sottoclasse di
% CLAST-ELEMENT, lavorando a partire da CLAST-ELEMENT-SUBFORMS. Mentre la
% funzione, precedentemente illustrata, incapsula la conoscenza rispetto alla
% presenza di foglie di ciascuno nodo dell'AST, rappresentato da un oggetto di
% tipo CLAST-ELEMENT, questa funzione incapsula la conoscenza rispetto alle
% modalità di traversal di ciascun sottoalbero presente a partire dal nodo in
% analisi. (indica quale parte del pattern Visitor viene realizzata da questa
% funzione)

% walk clast-element to void (walk.lisp)

% Funzione particolarmente importante perché fornisce accesso alla
% funzionalità di traversal dell'AST prodotto in output dalla libreria. Questa
% è la funzione che opera la visita dei diversi nodi, istanze della classe
% CLAST-ELEMENT, che compongono l'AST; visita a partire dalla quale è
% possibile ispezionare la struttura del programma in analisi. (indica quale
% parte del pattern Visitor viene invece realizzata da questa funzione)

% as-string: form to string (clast-printing.lisp)

% AS-STRING è una funzione che prende in input un'istanza di CLAST-ELEMENT e
% una rappresentazione testuale dei dati relativi a questa. Oltre a
% rappresentare una funzione particolarmente utile in fase di debugging e
% costruzione della libreria, questa funzione risulta particolarmente
% importante in quanto, essendo in grado di produrre un rappresentazione
% testuale, consente l'analisi di programmi Common LISP anche da parte di
% sistemi scritti utilizzando un linguaggio di programmazione differente, come
% ad esempio sistemi di source code translation, fornendo un elemento che ad
% esempio potrebbe consentire la produzione di codice in un qualsiasi
% linguaggio di programmazione a partire dalla ricca rappresentazione di una
% programma LISP prodotta in output proprio da questa funzione.

% — dettaglio anche su query functions —
