\section{Modulo di traversal}

\image{img/visitor-pattern}
      {Diagramma delle classi relativo al design pattern Visitor}
      {fig:visitor-pattern}
      {0.5}

% (In questa sezione presenta un diagramma delle classi che indichi la
% (struttura gerarchica utilizzata per la rappresentazione. Qua ci vuole un
% (diagramma veramente bello. Fai anche un discorso sull'ereditarietà multipla
% (e come questa venga utilizzata come mixin/trait per la rappresentazione di
% (IMPLICIT-PROGN forms e un altro tipo di forms.)

% (Aggiungi una parte in cui dici che a differenza di quanto avviene in altri
% (contesti, come ad esempio quello dei compilatori, in cui si ha la
% (generazione di un parse tree e, solo a partire da questo, si ha la
% (generazione dell'AST vero e proprio, nel caso di CLAST si ha una
% (generazione diretta dell'AST. Inventa una motivazione per cui questo viene
% (fatto. Una buona motivazione iniziale potrebbe essere perché dal punto di
% (vista dell'API risulterebbe scomodo avere a che fare con questi due step e
% (la libreria cerca di sollevare lo sviluppo e la progettazione di uno
% (strumento di source code analysis da tutto ciò che non riguarda la fase di
% (analisi vera e proprio obiettivo dello strumento. Questa è una motivazione
% (per cui questo non si vede dall'esterno. Spiegare perché questo non viene
% (fatto nemmeno internamente non è altrettanto semplice. Si può fare leva sul
% (fatto che a differenza del contesto della compilazione in cui si vuole
% (operare anche una validazione del codice soggetto, al fine di identificare
% (errori logici o sintattici, questo non è uno degli obiettivi di CLAST.
% (CLAST si limita a produrre l'AST di un programma Common LISP, senza
% (effettuare una validazione di questo. Da questo punto di vista la libreria
% (potrebbe produrre in output anche un AST per un programma che non ha senso
% (dal punto di vista sintattico. Fai un esempio in cui mostri che un
% (programma che contiene una chiamata ad una funzione che non è definita
% (viene parsato senza alcun problema e viene prodotto in output un AST
% (perfettamente valido. Questo perché la validazione e la compilazione non
% (sono obiettivi dello strumento.)

% — discuti gli slot delle strutture al livello più alto della gerarchia —

% MODULO DI TRAVERSAL

% AGGIUNGI APPROFONDIMENTO SUL DESIGN PATTERN VISITOR

% clast-element-subforms: form to list-of-subforms (clast-elements.lisp)

% (correggi questa funzione non viene specializzata per ogni possibile form,
% (viene specializzata per ogni sottoclasse di CLAST-ELEMENT in maniera tale
% (da incapsulare la nozione rispetto alle foglie di ciascun nodo,
% (rappresentato da un'istanza di CLAST-ELEMENT) Questa funzione data una form
% (restituisce l'insieme di tutte le sottoform contenute da questa. Questa
% (funzione risulta fondamentale alla porzione della libreria che si occupa
% (della visita dell'AST prodotto al termine del parsing, in quanto consente
% (la navigazione all'interno di questo. (aggiungi che parte del pattern
% (Visitor implementa questa funzione)

% map-subforms: form func result-type to form (walk.lisp)

% MAP-SUBFORMS è una funzione generica che rappresenta lo strumento a partire
% dalla quale viene implementata la visita dell'AST, accessibile dagli utenti
% della libreria tramite la funzione WALK. Ciascun metodo si occupa di
% implementare la navigazione all'interno di una specifica sottoclasse di
% CLAST-ELEMENT, lavorando a partire da CLAST-ELEMENT-SUBFORMS. Mentre la
% funzione, precedentemente illustrata, incapsula la conoscenza rispetto alla
% presenza di foglie di ciascuno nodo dell'AST, rappresentato da un oggetto di
% tipo CLAST-ELEMENT, questa funzione incapsula la conoscenza rispetto alle
% modalità di traversal di ciascun sottoalbero presente a partire dal nodo in
% analisi. (indica quale parte del pattern Visitor viene realizzata da questa
% funzione)

% walk clast-element to void (walk.lisp)

% Funzione particolarmente importante perché fornisce accesso alla
% funzionalità di traversal dell'AST prodotto in output dalla libreria. Questa
% è la funzione che opera la visita dei diversi nodi, istanze della classe
% CLAST-ELEMENT, che compongono l'AST; visita a partire dalla quale è
% possibile ispezionare la struttura del programma in analisi. (indica quale
% parte del pattern Visitor viene invece realizzata da questa funzione)

% as-string: form to string (clast-printing.lisp)

% AS-STRING è una funzione che prende in input un'istanza di CLAST-ELEMENT e
% una rappresentazione testuale dei dati relativi a questa. Oltre a
% rappresentare una funzione particolarmente utile in fase di debugging e
% costruzione della libreria, questa funzione risulta particolarmente
% importante in quanto, essendo in grado di produrre un rappresentazione
% testuale, consente l'analisi di programmi Common LISP anche da parte di
% sistemi scritti utilizzando un linguaggio di programmazione differente, come
% ad esempio sistemi di source code translation, fornendo un elemento che ad
% esempio potrebbe consentire la produzione di codice in un qualsiasi
% linguaggio di programmazione a partire dalla ricca rappresentazione di una
% programma LISP prodotta in output proprio da questa funzione.

% — dettaglio anche su query functions —
