\section{Modulo di traversal}

Il terzo e ultimo modulo che compone la libreria CLAST è rappresentato dal
modulo di traversal. Lo scopo di questo modulo è quello di consentire agli
utenti della libreria, la ricerca e l'ispezione delle strutture di
rappresentazione prodotte in output dal modulo di parsing ed analisi descritto
dalla Sezione \ref{parsing}. Si tratta quindi del modulo più critico dal punto
di vista dell'API, in quanto rappresenta la principale interfaccia tra CLAST ed
un analizzatore costruito a partire da questa.

\subsection{Funzionalità traversal}

In questa Sottosezione viene fornita al lettore una presentazione ad alto
livello del funzionamento interno del modulo, per poi proseguire presentando le
diverse funzionalità che vengono offerte dal modulo di traversal agli utenti
della libreria.\\

La struttura interna del modulo ricalca molto da vicino il design pattern
Visitor, presentato da Gamma et al. in \cite{gamma1995design}. In seguito, viene
presentato il funzionamento di base del design pattern e come questo sia stato
esteso per adattarsi allo specifico contesto in analisi.\\

Si è scelto di sfruttare il design pattern Visitor principalmente per come
questo consente di ottenere una separazione tra una dato insieme di operazioni e
gli oggetti a partire dai quali queste operazioni vengono eseguite. Separazione
che, all'interno della libreria CLAST, consente di isolare all'interno di un
modulo specifico per il traversal dell'AST tutte le operazioni relative alla
visita dei nodi e isolare all'interno di un modulo differente, come presentato
nella Sezione \ref{representation}, tutte le strutture e gli oggetti che per la
rappresentazione dell'AST stesso.

\image{img/visitor-classes}
      {Design Pattern Visitor - Diagramma delle classi}
      {fig:visitor-classes}
      {0.5}

\subsubsection{Design pattern Visitor}

Come mostrato dal diagramma in Figura \ref{fig:visitor-classes}, diagramma
presentato in \cite{gamma1995design}, il pattern descrive una collaborazione tra
5 diversi elementi:

\begin{itemize}

\item \texttt{Visitor} dichiara un'operazione di visita per ciascuna classe che
sarà toccata dal processo di visita, ossia ciascun \texttt{ConcreteElement} che
compone la \texttt{ObjectStructure}.

\item \texttt{ConcreteVisitor} implementa ciascuna operazione dichiarata dal
\texttt{Visitor}. Ciascuna operazione realizza lo specifico frammento della
logica di visita associata alla particolare classe soggetto dell'operazione.

\item \texttt{Element} definisce un'operazione \texttt{accept} che riceve in
input un oggetto di tipo \texttt{Visitor}.

\item \texttt{ConcreteElement} implementa l'operazione definita da \texttt
{Element}.

\item \texttt{ObjectStructure} incapsula la logica di enumerazione dei propri
elementi, fornendo ad un visitor un'interfaccia semplificata per la visita di
ciascuno dei propri elementi.

\end{itemize}

Le modalità di interazione vengono quindi riassunte dal diagramma di attività
in Figura \ref{fig:visitor-sequence}:

\begin{enumerate}

\item l'interazione ha inizio con l'invocazione del metodo della classe
\texttt{ObjectStructure} che raccoglie la conoscenza rispetto all'ordine di
visita dei diversi elementi che compongono la propria struttura;

\item il metodo opera quindi un invocazione del metodo \texttt{accept} di
ciascun elemento di tipo (\texttt{ConcreteElement});

\item secondo le modalità specifiche della classe di appartenenza, ciascuna
elemento opera quindi un invocazione del metodo \texttt{visit}, fornendo
tipicamente sé stesso come parametro dell'invocazione.

\end{enumerate}

\image{img/visitor-sequence}
      {Design Pattern Visitor - Diagramma di sequenza}
      {fig:visitor-sequence}
      {0.5}

Dopo aver brevemente presentato il design pattern a cui la struttura del
modulo fa riferimento, le prossime sottosezioni presentano l'API del modulo e
alcuni dettagli del funzionamento interno.

\subsubsection{\texttt{CLAST-ELEMENT-SUBFORMS}}

La funzione \texttt{CLAST-ELEMENT-SUBFORMS} è una funzione generica. Questa
funzione può essere considerata come il corrispondente della classe
\texttt{ObjectStructure} e dell'interfaccia \texttt{Element} descritte dal
design pattern. Lo scopo di questa funzione è infatti quello di raccogliere
l'informazione rispetto alla struttura di ciascun elemento interessato dal
processo di visita e guidare il processo di visita all'interno di una dato
elemento; scopo che, nel contesto della libreria CLAST, si traduce
nell'identificazione della struttura interna di ciascuna form e nella guida
del processo di visita attraverso ciascuna form. La funzione opera quindi un
mapping tra una form e le sue sottoform. Dal punto di vista pratico, data una
form, produce in output una lista contenente le diverse sottoform contenute in
questa, ordinate in accordo a quanto richiesto dal processo di visita.

\begin{lstlisting}[caption=Definizione della funzione \texttt
{CLAST-ELEMENT-SUBFORMS}]

(defgeneric clast-element-subforms (form)
  (:documentation "Returns a list of 'subforms' of a given FORM.

The methods of this generic form operate on the different kinds of
AST nodes that are of class FORM.  Other Common Lisp objects have
NULL subforms and LISTs are returned as they are.

Arguments and Values:

form : an instance of class FORM or LIST or a Common Lisp object.
result : a list of 'subforms' (or NIL).
"))

\end{lstlisting}

La funzione generica \texttt{CLAST-ELEMENT-SUBFORMS} viene quindi specializzata
da uno specifico metodo per ciascun \texttt{CLAST-ELEMENT}, ossia per ciascuna
potenziale form Common Lisp nodo dell'AST. Dal punto di vista teorico, questo
significa che ciascun metodo può essere visto come l'estensione di ciascun
\texttt{ConcreteElement} per la definizione del metodo \texttt{accept}.\\

Chiaramente, essendo il metodo comune a tutte le diverse form, una data
implementazione potrebbe ritornare un certo numero di elementi, ad esempio nel
caso dell'implementazione relativa ad una form \texttt{LET*}, o una lista vuota,
ad esempio nel caso di un self-evalutating object come la form \texttt{9}.

\begin{lstlisting}[caption=Esempi di implementazione del metodo \texttt
{CLAST-ELEMENT-SUBFORMS}]

;; Simple implementation example
(defmethod clast-element-subforms ((ce constant-form)) ())

;; Complex implementation example
(defmethod clast-element-subforms ((df do-form))
  (list (form-binds df)
        (form-test df)
        (return-form df)
        (form-body df)))

\end{lstlisting}

\subsubsection{WALK}

A partire dalla funzione \texttt{CLAST-ELEMENT-SUBFORMS} lavora la funzione
probabilmente più importante a livello di API offerta dell'intero modulo, la
funzione \texttt{WALK}. Lo scopo di questa funzione è quello di consentire ad un
utente della libreria di operare il reale traversal della rappresentazione in
forma di AST prodotta dal modulo di parsing ed analisi. Da un punto di vista
pratico, data un'istanza di \texttt{CLAST-ELEMENT}, la funzione \texttt{WALK}
opera una visita dell'AST radicato in questa in modo ricorsivo, ossia
depth-first.

Per continuare il parallelo tra struttura della libreria e il design pattern
Visitor, precedentemente illustrato, la funzione \texttt{WALK} può essere
vista, con una certa approssimazione, come la classe astratta
\texttt{Visitor}, la quale può essere estesa per implementare un particolare
processo di visita.\\

% Prova a modificare la parte che segue in maniera tale che sia più coerente
% con la parte che la precede. Prova con qualcosa come "la modalità di
% estensione offerta dalla libreria è rappresentata dal meccanismo delle high
% order functions..."

La funzione \texttt{WALK} è una funzione di ordine superiore, ossia una funzione
che prende in input o restituisce altre funzioni. Proprio questa aspetto
rappresenta il meccanismo di estensione attraverso il quale un utente della
libreria può adattare il processo di visita alle proprie necessità e specificare
le operazioni che desidera vengano compiute in risposta alla visita di un nodo o
porzione dell'AST.

Le diverse funzioni che vengono fornite in input alla funzione \texttt{WALK}
possono quindi invece essere viste come classi \texttt{ConcreteVisitor}.\\

\begin{lstlisting}[
  caption=Definizione della funzione WALK,
  label={lst:walk}
]

(defgeneric walk (clast-element &rest keys
                                &key
                                key ; #'identity
                                result-type
                                map-fun
                                reduce-fun
                                initial-value
                                environment
                                op-first
                                &allow-other-keys)
  (:documentation "The 'visiting' engine used to traverse a form.

The WALK generic function methods recursively traverse the tree
corresponding to a form (i.e., CLAST-ELEMENT) using a map/reduce
scheme.

The function MAP-FUN is applied to each (sub)form and their respective
subforms are WALKed over.  WALK uses MAP-SUBFORMS internally,
therefore it generates sequences (of type RESULT-TYPE) as output. Once
the traversing of subforms is completed the function REDUCE-FUN is
applied, via REDUCE to the resulting sequence.
")
  )

\end{lstlisting}

Come il listato \ref{lst:walk} mostra, la funzione prende in input in
particolare due funzioni, una funzione chiamata \texttt{MAP-FUN} e una funzione
chiamata \texttt{REDUCE-FUN}. Come il nome stesso di questi due parametri le
funzioni vengon invocate seguendo lo schema di esecuzione map-reduce. Questo
significa che durante il processo di visita, la funzione \texttt{WALK}
applicherà il la funzione \texttt{MAP-FUN} a ciascun nodo dell'AST incontrato, e
utilizzerà la funzione \texttt{REDUCE-FUN} a partire dalla lista di valori
ritornati dall'applicazione di \texttt{MAP-FUN} per ottenere il risultato finale
della computazione.

L'utilizzo di questo schema di funzionamento risulta particolarmente per
l'implementazione di funzioni che operano un'interrogazione dell'AST allo scopo
di identificare particolari tipologie di nodi. Operazioni di interrogazione che
risultano particolarmente utili a strumenti per la source code analysis, come
verrà illustrato nel corso della Sezione \ref{free-variables-analysis}.\\

Oltre alla funzionalità di visita vera e propria esposta dalla funzione
\texttt{WALK}, la quale rappresenta il cuore del funzinamento del modulo, il
modulo di traversal offre diverse altre funzionalità. Ciascuna di queste
funzionalità viene costruita proprio a partire dall'interfaccia esposta dalla
funzione \texttt{WALK}. Nella prossima Sottosezione viene presentato un esempio
di funzionalità di questo tipo.

\subsubsection{AS-STRING}

\texttt{AS-STRING} è un'altra delle funzioni più importanti offerte dal modulo.
La funzione funzione prende in input un'istanza di \texttt{CLAST-ELEMENT} e una
rappresentazione testuale dei dati relativi a questa. Può quindi risultare come
una funzione analoga a metodi come \texttt{toString} Java e \texttt{description}
Objective-C. La particolarità di questa funzione generica è costituita dal fatto
che ciascuno dei suoi metodi agisce in modo ricorsivo, restituendo una
rappresentazione completamente componibile, strutturata e leggibile da un
sistema automatico, dell'AST radicato nel nodo specificato come input.\\

Oltre a rappresentare una funzione particolarmente utile in fase di debugging e
costruzione della libreria, questa funzione risulta particolarmente importante
in quanto, producendo un rappresentazione testuale, consente l'analisi dell'AST
programmi Common Lisp, e più in generale della rappresentazione di un programma offerta dalla libreria, anche da parte di sistemi scritti utilizzando un
linguaggio di programmazione differente.

Esempi di sistemi che vengono abilitati dalla presenza di questo metodo sono per
compilatori source-to-source, ossia sistemi per la traduzioni di codice sorgente
da un linguaggio di programmazione ad un altro, fornendo un elemento che ad
esempio potrebbe consentire la produzione di codice in un qualsiasi linguaggio
di programmazione a partire dalla ricca rappresentazione di una programma Common
Lisp prodotta in output dal modulo di parsing e analisi della libreria. Un altra
classi di sistemi la cui creazione viene abilitata dalla presenza della
funzionalità offerta dai metodi \texttt{AS-STRING} è rappresentata strumenti per
il linting di codice Common Lisp, tipicamente integrati all'interno di editor di
testo in molti casi scritti utilizzando linguaggio di programmazione diversi dal
Common Lisp.
