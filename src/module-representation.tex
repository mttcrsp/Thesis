\section{Modulo di rappresentazione}
\label{representation}

In questa sezione viene approfondita la struttura e il contenuto del modulo di
rappresentazione presente all'interno della libreria CLAST. In particolare,
vengono presentate e discusse le scelte compiute e il procedimento utilizzato
per la definizione delle diverse classi utilizzate per la rappresentazione di un
programma Common Lisp. Classi che verranno poi utilizzate dal modulo di parsing
ed analisi in fase di generazione dell'AST associato al programma target
dell'azione della libreria.\\

La struttura del modulo di rappresentazione ricalca molto da vicino il processo
di valutazione del codice sorgente di un programma Common Lisp. Questo allo
scopo di ottenere delle strutture che forniscano una rappresentazione molto
vicina a quella utilizzata internamente da un reale sistema Common Lisp, ossia
la più ricca possibile. Tutto questo mantenendo sempre il riferimento al codice
sorgente vero e proprio, in maniera tale da ottenere sia informazioni rispetto
alla semantica del programma a livello macchina, sia informazioni rispetto alla
semantica desiderata dall’autore del programma.

\subsection{Processo di valutazione}
\label{CL-valutazione}

Come affermato nel precedente paragrafo, le componenti facenti parte del
modulo di rappresentazione sono particolarmente legate al processo di
valutazione utilizzato dal linguaggio. Per questa ragione, allo scopo di
fornire al lettore una migliore comprensione degli elementi che verranno
presentati nel dettaglio in seguito, in questa sottosezione viene presentato
un breve sunto del processo di valutazione e delle strutture fondamentali di
un programma scritto utilizzando il linguaggio di programmazione Common
Lisp.\\

La struttura fondamentale alla base del processo di valutazione di un programma
Lisp è la form. Una \textit{form} viene infatti definita formalmente dallo
standard ANSI Common Lisp proprio come \textit{“an object meant to be
evaluated”}. Dal punto di vista pratico una form può essere costituita sia da un
atomo che da una lista. A partire da questo elemento particolarmente generico
vengono poi definite delle particolari tipologie di form, a seconda delle
modalità con le quali procede la valutazione operata da un sistema Common Lisp
nel momento in cui queste vengono incontrate.

\subsubsection{Form atomiche: self-evaluating objects}

Il caso in cui una form è costituita da un atomo rappresenta il caso più
semplice del processo di valutazione. Una volta identificato un atomo infatti,
il sistema di valutazione deve semplicemente verificare se questo rappresenti un
oggetto auto-valutante, \textit{self-evaluating object}, oppure un simbolo.

Nel caso di un self-evaluating object, il sistema Lisp si limita a produrre
come risultato l’oggetto stesso, come il nome stesso di questa struttura
suggerisce. Esempi di self-evaluating objects sono rappresentati da numeri,
sia interi che complessi, stringhe, pathnames e array.

\begin{lstlisting}[caption=Esempi di form di tipo self-evaluating object]

CL-USER > 3
3
CL-USER > #c(2/3 5/8)
#C(2/3 5/8)
CL-USER > #p"S:[BILL]OTHELLO.TXT"
#P"S:[BILL]OTHELLO.TXT"
CL-USER > #(a b c)
#(A B C)
CL-USER > "fred smith"
"fred smith"

\end{lstlisting}

\subsubsection{Form atomiche: simboli}

Nel caso in cui un atomo non sia un self-evaluating object, il sistema Lisp lo
identifica automaticamente come simbolo. A sua volta, un simbolo può
rappresentare o una symbol macro, una form utilizzata in sostituzione di
un’altra form, o una variabile.

La prima operazione compiuta da un sistema Common Lisp aderente allo standard in
questo caso è una verifica della presenza o meno di un binding relativo ad una
macro all’interno dell'ambiente lessicale attivo al momento della valutazione.
Se viene effettivamente identificato un binding che utilizza il simbolo come
nome, viene applicata la funzione associata dalla definizione di macro. Questo
al fine di produrre una form che verrà valutata al posto del simbolo stesso,
come descritto durante la discussione del processo di macro-espansione
illustrato nella Sottosezione \ref{macroexpansion}.

Nel caso in cui non sia presente una definizione di macro all’interno
dell’ambiente, il sistema Lisp assume che il simbolo rappresenti una variabile.
Il valore della variabile associata al simbolo viene quindi ricercato e prodotto
come output del processo di valutazione.\\

\begin{lstlisting}[caption=Esempio di form di tipo symbol]

CL-USER > *STANDARD-INPUT*
#<SWANK/GRAY::SLIME-INPUT-STREAM {1004F2D653}>

\end{lstlisting}

Come affermato, nel caso in cui il simbolo non sia stato identificato come
symbol-macro, il sistema assume che questo rappresenti una variabile, nel caso
in cui questo non lo sia, o più generalmente in cui non sia stato associato
alcun valore alla variabile referenziata dal simbolo, un sistema Common Lisp
aderente allo standard si limita a segnalare un errore di tipo \texttt
{UNBOUND-VARIABLE}.

Sia il modulo di rappresentazione che il modulo di parsing, seguono questa
stessa convenzione, il che porta ad ottenere una rappresentazione corretta del
comportamento di un programma che, a runtime, manifesterà un comportamento
scorretto e un fallimento. Questo significa però che il risultato prodotto dal
lavoro del modulo di parsing e analisi sarà una rappresentazione che presenterà
sufficienti informazioni da consentire l’individuazione e la segnalazione del un
comportamento potenzialmente scorretto; esattamente ciò che risulta più
interessante per gli strumenti di source code analysis per i quali CLAST si pone
come infrastruttura abilitante.\\

\subsubsection{Form composte: function, macro, lambda e special forms}

Nel caso invece in un una form non sia costituita da una variabile, ma piuttosto
da una lista, questa prende il nome di compound form. In questo caso il processo
di valutazione prosegue in modo significativamente più approfondito. Ciascuna
compound form viene infatti scomposta in due componenti: un operatore,
rappresentato dal simbolo di testa della lista, detto \texttt{CAR} della lista,
e una lista di parametri, rappresentata da una nuova lista contenente tutti gli
elementi della lista originale ad eccezione dell’operatore, detta \texttt{CDR}.
Il sistema Lisp procede quindi all'analisi del simbolo operatore, ricercando
eventuali associazioni tra questo ed elementi all’interno ambiente lessicale
corrente alla valutazione e all'interno dell'ambiente globale. In base al
risultato prodotto da questa ricerca la compound form viene classificata
all’interno di una delle seguenti 4 categorie: \textit{special form},
\textit{macro form}, \textit{function form} o \textit{lambda form}.

\begin{lstlisting}[
  caption=Esempio di estrazione degli elementi operatore e parametri di una
  compound form,
  label={lst:car-cdr}
]

CL-USER > (car '(+ 1 2))
+                        ; Operator: CAR is a symbol
CL-USER > (cdr '(+ 1 2))
(1 2)                    ; Params: CDR is a list

\end{lstlisting}

Il processo di valutazione si ramifica quindi a seconda di come la form sia
stata classificata al passo precedente.

\begin{itemize}

\item Nel caso in cui l’operatore venga identificato come il nome di una
funzione e quindi la compound form sia stata classificata come function form,
questa viene invocata utilizzando come input la lista dei parametri identificata
al passo precedente.

\item Nel caso in cui l’operatore venga identificato come nome di una macro e
quindi la compound form sia stata classificata come macro form, la valutazione
prosegue dando inizio al processo di valutazione delle macro illustrato nel
corso del capitolo precedente, Sottosezione \ref{macroexpansion}.

\item Nel caso in cui l’environment in cui viene eseguita la ricerca non
contenga alcuna definizione per il simbolo operatore, la compound form viene
identificata come lambda form. L’operatore viene quindi utilizzato come una
funzione e la valutazione avviene in maniera analoga a quanto riportato nel caso
di una function form.

\end{itemize}

Il caso più particolare dal punto di vista del processo di valutazione si
verifica quando, a partire dall’analisi dell’operatore, una compound form viene
identificata dal sistema come una special form. Il maggior interesse per
compound form di tipo special form è legato al fatto che form di questo tipo
possono utilizzare o una sintassi speciale, o regole di valutazione speciali, o
entrambi, oltre ad essere potenzialmente in grado di compiere modifiche
dell’environment all’interno del quale vengono valutate, oltre che più
generalmente del flusso di controllo stesso del programma.

Questo significa che la valutazione di una compound form di questo tipo può
produrre nuovi environment lessicali o dinamici all’interno dei quali verranno
valutate eventuali sotto-form innestate a questa.

Ad esempio, una compound form che utilizza l’operatore \texttt{LET}, dichiara un
nuovo environment lessicale, identico a quello di invocazione, all’interno del
quale vengono aggiunti nuovi binding rispetto a nomi di variabile. Questo
significa che form di tipo special form avranno un impatto particolarmente
significativo sul funzionamento del modulo di parsing e analisi, in quanto, ad
esempio, avendo la possibilità di utilizzare una sintassi speciale, e
rappresenterà quindi potenzialmente una nuova regola della grammatica target del
parser.\\

Questo conclude la breve panoramica del processo di valutazione di un programma
Common Lisp e delle strutture fondamentali che caratterizzano questo processo
offerta da questa sottosezione. Panoramica che ha mostrato le modalità con il
quale opera il processo di valutazione di un programma Lisp rispetto alle
principali componenti di questo.

Nel corso della prossima sottosezione verrano quindi approfonditi meccanismi che
la libreria CLAST utilizza per la rappresentazione di queste strutture al fine
di facilitare i compiti di analisi del codice sorgente di un programma.

\subsection{Strutture di rappresentazione}

Dopo aver brevemente presentato il processo di valutazione utilizzato da un
sistema Common Lisp e le strutture fondamentali utilizzate da questo processo,
si presentano i meccanismi che vengono forniti e utilizzati dalla libreria CLAST
allo scopo di rappresentare un programma Lisp.

Come precedentemente riportato, i meccanismi di rappresentazione ricalcano
molto da vicino le informazioni prodotte e raccolte da un sistema Lisp durante
il processo di valutazione. Questo perché che la rappresentazione che la
libreria vuole offrire è la più ricca possibile e quello della valutazione è
il momento in cui viene prodotto e raccolto il maggior numero di informazioni
rispetto alla semantica del programma in analisi.\\

Dal punto di vista pratico, la rappresentazione fornita dalla libreria è
strutturata all’interno di un grande insieme di classi. Classi organizzate
all’interno di una gerarchia, sfruttando il supporto offerto dal linguaggio
Common Lisp all’ereditarietà e in particolare all’ereditarietà multipla.

\image{img/clast-representation.png}
      {Modulo di rappresentazione - Diagramma delle classi (ridotto)}
      {fig:clast-representation}
      {0.5}

\subsubsection{CLAST-ELEMENT e FORM}

La classe fondamentale, al vertice della gerarchia delle strutture esposte da
CLAST, è la classe \texttt{CLAST-ELEMENT}. Questa classe ha il semplice scopo
di raccogliere le diverse strutture offerte dalla libreria all'interno di un
unico tipo, in maniera tale da facilitare l’ispezione e l’analisi di oggetti
prodotti dalla libreria. Per questa ragione la classe non dichiara alcun
attributo e nessun metodo viene specializzato rispetto a questa.\\

La prima classe concreta all’interno della gerarchia è la classe \texttt{FORM}.
Lo scopo di questa classe è quello di rappresentare i dettagli fondamentali di
qualsiasi elemento presente all’interno di un programma Common Lisp e, in
particolare ,fungere da nodo, ossia unità fondamentale, della rappresentazione
mediante AST fornita dalla libreria.\\

Per fare questo, la classe \texttt{FORM} espone quindi tre slot, attributi,
fondamentali.

\begin{itemize}

\item Un attributo \texttt{SOURCE} che riporta il codice sorgente associato al
nodo, form, in analisi. La presenza di questo attributo ha lo scopo di
facilitare il compito di un analizzatore che lavora a partire dalla libreria,
interessato in particolare modo ad aspetti testuali del codice sorgente.

\item Un attributo \texttt{TOP} indica il nodo \texttt{FORM} all’interno della
quale questa istanza è innestata. Questo attributo risulta di fondamentale
importanza, sia per gli utenti della libreria, sia per la libreria stessa, in
quanto è l'elemento che consente di rappresentare un programma all’interno di
una struttura ad albero, un AST, come anticipato dal Capitolo
\ref{abstract-syntax-tree}, e di implementare una navigazione all'interno di
questa struttura. Questo consente di ottenere una rappresentazione
universalmente nota e per la quale le operazioni di traversal risultano
particolarmente semplici.

\item Un attributo \texttt{TYPE} riporta invece il tipo dichiarato, o
potenzialmente inferito, del nodo o form in analisi. Questo allo scopo di
facilitare il lavoro di strumenti come type checkers, che cerchino di aggiungere
una tipizzazione statica ad un linguaggio di programmazione dinamico come il
Common Lisp.

\end{itemize}

\subsubsection{Mixin}

Come precedentemente affermato, il modulo fa utilizzo del supporto
all'ereditarietà multiplo offerto dal linguaggio. In particolare, questo il
meccanismo viene sfruttato allo scopo di definire un insieme di classi in grado
di agire da mixin.\\

Un \textit{mixin} o \textit{trait} viene definito come una classe che definisce
un insieme i metodi e/o attributi allo scopo di facilitare il riuso di questi da
parte di altre classi, senza però forzare la definizione di una relazione di
ereditarietà diretta tra questa e le classi che operano il riuso. Lo scopo di
soluzioni di questo tipo, implementate da diversi linguaggi di programmazione
con modalità e nomi differenti, è quindi fondamentalmente quello di facilitare
il riuso di codice, evitando allo stesso tempo i problemi legati alle ambiguità
che possono essere causate dall’impiego dell’ereditarietà multipla, legate ad
esempio al Diamond Problem\cite{martin1997}.\\

A differenza di quanto avviene nel contesto di altri linguaggi di
programmazione, come ad esempio il linguaggio Scala, un linguaggio che
specifica un costrutto dedicato esclusivamente alla definizione di mixin, in
Common Lisp un mixin viene definito attraverso una semplice definizione di
classe, in maniera del tutto analoga a quanto avverrebbe per la definizione di
una classe tradizionale.\\

Alla base della libreria vengono quindi definite altre due classi, oltre alla
classe \texttt{FORM} che, come anticipato, hanno lo scopo di agire da mixin.
La prima di queste classi è chiamata \texttt{IMPLICIT-PROGN}, la seconda è
chiamata \texttt{EXPANSION-COMPONENT}.

\subsubsection{IMPLICIT-PROGN e EXPANSION-COMPONENT}

Il costrutto \texttt{PROGN} è il costrutto fondamentale alla definizione di
codice imperativo in Common Lisp: valuta l’insieme di form fornite in input in
sequenza e ritorna il risultato prodotto dalla valutazione dell’ultima di
queste, scartando il risultato di tutte le precedenti. Lo scopo della classe
\texttt{IMPLICIT-PROGN} è quello di raccogliere attributi e metodi necessari
all’analisi di compound form che contengono una form di tipo \texttt{PROGN}
implicita. Essendo infatti l’utilizzo del costrutto \texttt{PROGN} presente in
modo implicito alla base del funzionamento di diversi altri costrutti, come ad
esempio \texttt{DEFUN} e \texttt{DEFMACRO}, si è scelto di isolare le
responsabilità e le strutture fondamentali al parsing e all’analisi di form di
questo tipo all’interno di questo mixin, allo scopo di facilitare il riuso
all’interno delle funzioni e delle strutture dedicate al parsing dei diversi
costrutti che utilizzano un \texttt{IMPLICIT-PROGN}.

\begin{lstlisting}[caption=Esempio di costrutto che fa utilizzo di del
costrutto PROGN in modo implicito]

(defun sum-and-log (x y)
  ;; Forms on these next two lines will be implicitly wrapped in a
  ;; PROGN form together and thus executed sequentially. When the
  ;; SUM-AND-LOG function the evaluation process will encounter the
  ;; first form, execute it and discard the result. Then, it will
  ;; execute the second form, and since this is the last form in the
  ;; implicit progn wrapper it will return the result of its
  ;; evaluation as the function evalutation result.
  (format t "~a plus ~a equals..." x y)
  (+ x y)
  )

\end{lstlisting}

I due slot fondamentali esposti da questa classe sono i seguenti.

\begin{itemize}

\item \texttt{IPROGN-FORMS} è un attributo utilizzato allo scopo di tenere
traccia delle form innestate all’interno di quella rappresentata da questa
istanza di CLAST-ELEMENT, ossia delle form che verranno implicitamente
eseguite all’interno di una form di tipo \texttt{PROGN}. Un secondo scopo di
questa classe, non meno importante del primo, è quello di mantenere la
relazione tra una form e le sue sotto-form. Ciò consentirà infatti la
navigazione all'interno del codice, in quanto mantiene traccia delle form
innestate all'interno dell'istanza corrente.

\item \texttt{BODY-ENV} è invece un attributo che riporta un oggetto di tipo
ambiente, il quale rappresenta l'ambiente all’interno del quale verrà eseguita
la valutazione delle form memorizzate dall’attributo \texttt{IPROGN-FORMS}. È
importante notare che, nel caso di molte special forms, questo attributo risulta
di fondamentale importanza agli scopi di uno strumento di analisi. Questo
consente infatti di osservare il reale ambiente all'interno del quale verrà
eseguita una determinata form.

\end{itemize}

La classe \texttt{EXPANSION-COMPONENT} è invece responsabile per la
definizione delle strutture e dei metodi che consentono il parsing di form
soggette al processo di valutazione tipico di una macro, presentato nel corso
dalla Sottosezione \ref{macroexpansion}.

In particolare, questa classe espone un altro attributo, chiamato \texttt
{FORM-EXPANSION}, fondamentale ad uno strumento per l’analisi di un programma
Lisp, ossia la risultato dell’espansione della macro rappresentata dal nodo.\\

\subsubsection{Strutture di dettaglio}

Le tre classi appena riportate rappresentano il substrato fondamentale della
rappresentazione offerta dalla libreria CLAST. Il livello di dettaglio offerto
dalla libreria è però molto maggiore rispetto a quello possibile utilizzando
solamente queste tre classi. La libreria dichiara infatti più di cento classi
che vengono utilizzate per rappresentare istruzioni ad un livello di dettaglio
di singolo operatore. Una discussione puntuale rispetto alla rappresentazione
offerta permessa da ciascuna di queste risulterebbe troppo estesa per essere
riportata all’interno di questa tesi e viene pertanto rimandata alla
documentazione della libreria.\\

Non sarebbe possibile e nemmeno utile, avere un classe specifica per la
rappresentazioni qualsiasi possibile operatore definito da un utente e
linguaggio. In particolare, vengono la libreria espone strutture per la
rappresentazione nel dettaglio di tutti gli operatori speciali indicati dallo
standard ANSI Common Lisp, Sezione 3.1.2.1.2 Listato 2, più tutti gli operatori
ritenuti di particolare interesse dal punto di vista dell’analisi non presentati
all’interno di quella lista.

L'insieme di questi operatori aggiuntivi consiste principalmente degli
operatori legati al CLOS, al meccanismo delle dichiarazioni e al costrutto
loop, considerati di maggiore interesse rispetto ad altri in quanto
particolarmente utilizzati dagli utenti del linguaggio Common Lisp e quindi
significativamente più rilevanti dal punto di vista dell’analisi.\\

Qualsiasi compound form viene tuttavia rappresentata con il maggiore grado di
precisione possibile, a seconda dello specifico caso in analisi. In presenza di
operatori non trattati in modo specifico dalla libreria, come un'applicazione
dell'operatore della compound form la libreria è in grado di rappresentare,
distinguendo, se si tratti dell’applicazione di una funzione o di una macro a
partire dalla ricerca dell'operatore all'interno dell'environment in uso.\\

Questa sezione ha presentato nel dettaglio la struttura e le componenti
fondamentali del modulo di rappresentazione presente nella libreria CLAST.
Nella prossima sezione verrà invece approfondito il funzionamento del modulo
dedicato a parsing e analisi del codice sorgente di un programma, mostrando
quindi come gli elementi del modulo di rappresentazione vengano realmente
impiegati dalla libreria stessa.
