\section{Macro}

Questa sezione affronta un altro dei fondamenti teorici alla base del progetto
discusso da questa tesi, il costrutto macro Common Lisp. Questo costrutto
risulta di particolare interesse in quanto particolarmente caratterizzante del
processo di valutazione di codice scritto utilizzando il linguaggio Common Lisp.

La presenza di questo costrutto all'interno del linguaggio porta infatti alla
presenza di diversi livelli di astrazione anche dal punto di vista del codice
sorgente, un aspetto che risulta particolarmente condizionante per la
progettazione e la realizzazione di una libreria come CLAST, la quale si pone
come fondamento per la definizione di strumenti di source code analysis per
sistemi scritti utilizzando il linguaggio di programmazione Common Lisp.\\

Il costrutto macro Common Lisp può essere definito come un funzionalità del
linguaggio che consente ad un utente di questo la definizione di funzioni in
grado di convertire form Lisp in differenti form Lisp prima che queste vengano
valutate e compilate.\\

A differenza differenza dell'omonimo costrutto offerto da diversi altri
linguaggi di programmazione, la conversione operata dal costrutto macro Common
Lisp viene operata a livello di espressione, e non a livello i caratteri. Questo
significa che il costrutto non consente semplicemente la possibilità di definire
una sostituzione testuale che permette ad un utente del linguaggio l’utilizzo di
una sintassi tipicamente più breve, una macro Lisp consente l'estensione in
maniera nativa il linguaggio di programmazione stesso.\\

Nel contesto del linguaggio di programmazione Common Lisp, il costrutto macro
risulta di fondamentale importanza per la scrittura di codice di qualità: la
presenza di questo costrutto consente infatti la scrittura codice semplice e di
facile lettura a livello di interfaccia, aspetto che consente avere una migliore
comprensione del sistema che si sta costruendo e di ragionare in modo più
semplice rispetto alla correttezza e alle caratteristiche di questo, andando poi
però ad eseguire una trasformazione ad versione interna di questo codice più
efficiente ma tipicamente più complessa. Tutto questo avviene tipicamente a
livello di compilazione, senza che l'implementazione incorra in penalità a
runtime.\\

La caratteristica fondamentale che distingue una form macro da form Lisp
ordinarie é rappresentata dal fatto che, come accennato in apertura di questa
sezione, una macro non produce in output un valore. Ogni macro produce come
output una nuova form, costruita a partire da quella in input, rappresentata dal
parametro body specificato in fase di invocazione della macro.\\

Proprio perché una macro consiste in un processo di generazione di una nuova
form, ossia codice Lisp, a partire da altre form, molto spesso risulta
particolarmente intuitivo pensare ad una macro come ad un programma in grado di
generare un altro programma.\\

Nel caso più semplice, una macro sostituisce una form a partire da un template,
a partire da una chiara corrispondenza, anche a livello visivo, tra il codice
generante e il codice generato, consentita dalla combinazione tra costrutto
\texttt{BACKQUOTE} e costrutto \texttt{COMMA}. Macro più complesse possono
invece essere utilizzate per accedere all’intera potenza espressiva del
linguaggio Common Lisp e poter generare codice a partire dai parametri forniti
in input a queste durante l’esecuzione di un programma.

\subsection{Definizione di una macro}

Il principale strumento per la definizione di macro in Common Lisp è
rappresentato una macro stessa, chiamata \texttt{DEFMACRO}. Una form
\texttt{DEFMACRO} ricalca molto da vicino la struttura di una form
\texttt{DEFUN}, il più comune costrutto utilizzato per la definizione di
funzioni. Infatti, sia \texttt{DEFUN} che \texttt{DEFMACRO} specificano:

\begin{itemize}

\item un nome con il quale dovranno essere internata nell'ambiente di
definizione dal sistema Common Lisp,

\item una lista di nomi dei parametri che ricevono in input,

\item un corpo, che nel caso di \texttt{DEFUN} opera una computazione generica a
partire dai parametri di input, mentre nel caso di \texttt{DEFMACRO} opera
necessariamente una trasformazione tra form.

\end{itemize}

\begin{lstlisting}[caption=Signature della macro DEFMACRO]

(defmacro name (argument ...) body)

\end{lstlisting}

A differenza di \texttt{DEFUN}, per la quale è ammesso che vengano ritornati uno
o più valori, \texttt{DEFMACRO} è vincolata a ritornare un unico elemento: una
form che rappresenti il risultato dell'espansione del parametro body fornito in
input. Un’ulteriore differenza, più complessa dal punto di vista concettuale, è
rappresentata dal fatto che i parametri che vengono forniti ad una macro non
vengono valutati, a differenza di quanto avviene nel caso dell’invocazione di
una funzione. Ad esempio, fornendo in input la form \texttt{(+ 1 2)} ad una
macro, il parametro body della macro verrà definito come la lista \texttt{(+ 1
2)} e non come il valore 3.\\

Il metodo più semplice per generare una form nel corpo di una macro è
utilizzando la backquote reader macro \texttt{`}. Questa macro si comporta in
maniera molto simile al più noto costrutto quote \texttt{‘}, ad eccezione dei
casi in cui questo viene combinato con un’occorrenza del simbolo comma
\texttt{,}.

Il costrutto \texttt{COMMA} è ammesso dalla sintassi Lisp tradizionale solamente
all’interno di una backquoted form, ossia una form prefissa dal simbolo
backquote. In caso di utilizzo del simbolo nel contesto di una quoted form,
ossia una form prefissa dal simbolo quote, Lisp segnalerà un errore al momento
della lettura della form.

In maniera analoga a \texttt{QUOTE}, \texttt{BACKQUOTE} previene la valutazione
di una form. L’utilizzo del costrutto \texttt{COMMA} consente invece di fare in
modo che la valutazione di una subform avvenga nonostante l’utilizzo del
costrutto \texttt{BACKQUOTE}.

\begin{lstlisting}

CL-USER > `(1 plus 2 is ,(+ 1 2))
(1 PLUS 2 IS 3)

\end{lstlisting}

Se si confronta il precedente esempio, che utilizza una combinazione dei simboli
\texttt{BACKQUOTE} e \texttt{COMMA}, e il seguente esempio, definito in maniera
molto simile, che utilizza però il simbolo \texttt{QUOTE}, è piuttosto semplice
comprendere il significato di quanto appena esposto.

\begin{lstlisting}

CL-USER > '(1 plus 2 is (+ 1 2))
(1 PLUS 2 IS (+ 1 2))

\end{lstlisting}

Risulta abbastanza naturale immaginare come \texttt{BACKQUOTE} e \texttt{COMMA}
possano quindi fornire uno strumento per la definizione di template che
consentano la sostituzione di elementi.\\

Quanto appena riportato raccoglie le informazioni fondamentali alla definizione
di una macro. In seguito viene invece analizzato il momento successivo alla
definizione, ossia il processo di espansione di una macro.

\subsection{Processo di macro-espansione}
\label{macroexpansion}

Quando l'esecuzione della funzione \texttt{EVAL} riceve in input una form
rappresentata da un lista il primo elemento è un simbolo, il sistema Common Lisp
procede con la verifica della presenza del simbolo tra le definizioni locali
all’esecuzione (\texttt{FLET}, \texttt{LABEL}, \texttt{MACROLET}). Se questa
ricerca non ottiene risultati, Lisp verifica invece la presenza di definizioni a
livello globale per il simbolo. Se una qualsiasi di queste due ricerche ha
successo, e quindi il simbolo utilizza un l’identificativo precedentemente
associato ad una definizione di macro, la form viene identificata come una
chiamata a macro.\\

Dopo aver identificato una chiamata a macro, il sistema utilizza una funzione
definita a livello di linguaggio, detta funzione di espansione. Questa funzione
viene invocata fornendo l’intera macro come primo parametro e un oggetto di tipo
environment come secondo parametro. Come affermato affermato dalla definizione
di macro fornita nei paragrafi precedenti, questa funzione deve ritornare una
nuova form Lisp, la quale viene chiamata espansione della chiamata a macro.

Una volta che la funzione di espansione ha prodotto in output l'espansione della
form originale, questa viene valutata al posto della form originale e il
risultato di questa valutazione viene ritornato come risultato complessivo della
chiamata a macro.\\

Il processo appena illustrato rappresenta una semplificazione del reale processo
di espansione di una macro ma rimane comunque corretto dal punto di vista
concettuale, consentendo la comprensione del funzionamento generale del
processo di macro espansione in maniera più semplice.

Tuttavia un aspetto particolarmente significativo viene trascurato dalla
precedente spiegazione. Questo aspetto riguarda il momento in cui il processo di
macroespansione stesso avviene. Il linguaggio Common Lisp prevede infatti che
sia possibile ritardare il processo di espansione delle macro anche ad un
momento successivo alla compilazione. Nel prossimo paragrafo vengono indicati i
momenti in cui è possibile che una macro venga espansa.

Come detto, una macro potrebbe essere espansa una sola volta, quando il
programma viene compilato. Altrimenti potrebbe essere espansa al primo utilizzo
di questa durante dell’esecuzione del programma, e questa espansione potrebbe
essere memorizzata per consentirne un riutilizzo più efficiente nel caso in cui
si dovessero verificare chiamate successive. Infine, una macro potrebbe essere
espansa ad ogni invocazione. Una macro definita in modo corretto dovrebbe essere
in grado di agire in modo corretto in ciascuna queste diverse situazioni, il che
significa che:

\begin{itemize}

\item Essendo che una macro potrebbe essere espansa in momenti diversi nel ciclo
di vita di un programma, questa dovrebbe essere scritta in maniera tale da
dipendere il meno possibile dall’ambiente in cui viene eseguita per ottenere una
corretta espansione.

\item Inoltre, per assicurarsi un comportamento consistente in diverse
esecuzioni, sarebbe meglio assicurarsi del fatto che tutte le definizioni di
macro siano disponibili, all’interprete o compilatore, prima che si verifichi
una qualsiasi chiamata a queste nel codice del programma.

\item Essendo che una macro potrebbe potenzialmente dover essere espansa già in
fase di compilazione, è necessario che un compilatore Lisp, abbia già internato
la definizione di questa macro prima del primo utilizzo della stessa all’interno
del codice. Se così non dovesse essere si verificherebbe infatti un errore, in
quanto il compilatore sarebbe in grado di espandere la macro al momento
corretto.

\end{itemize}

Un ulteriore aspetto particolarmente interessante rispetto al funzionamento
delle macro Common Lisp è rappresentato dal fatto che macro e funzioni in Lisp
non sono in alcun modo interscambiabili. Come presentato nei paragrafi
precedente infatti, il processo di valutazione di una macro si compone di
passaggi diversi rispetto a quelli che contraddistinguono la valutazione di una
funzione, differenza che rende impossibile il riutilizzo delle une al posto
delle altre.

Una conseguenza rilevante di questa impossibilità di sostuire macro a funzioni e
vicersa è il fatto che, nonostante possa sembrare sensato in prima battuta, una
macro non può essere utilizzata come parametri di una higher order function,
come ad esempio \texttt{APPLY}, \texttt{FUNCALL} o \texttt{MAP}. In queste
situazioni, la lista che rappresenta la chiama a macro originale infatti non
esisterà, e non può esistere, questo perché, in una certa misura, è come se i
parametri di questa fossero già stati valutati.\\

Si è scelto di sottolineare questa la particolarità del costrutto macro dal
punto di vista della valutazione, e in particolare dal punto di vista dei
diversi momenti in cui il processo di macroespansione può avere luogo, in quanto
questa particolarità risulta, non in linea con le modalità di compilazione e
valutazioni del resto del linguaggio, è risultata particolarmente critica anche
dal punto di vista della progettazione della libreria CLAST, come verrà
approfondito nel capitolo \ref{library} di questa tesi.
