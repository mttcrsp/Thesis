\section{Motivazioni del lavoro}

% QUESTA PARTE È STATA ESTRATTA DAL CAPITOLO RELATIVO ALLA SOURCE CODE ANALYSIS,
% UTILIZZALA PER FARE UN DISCORSO RIGUARDO ALL'IMPORTANZA DI UN PROGETTO COME
% CLAST E QUANTO QUESTO POSSA ESSERE ABILITANTE.

% La source code analysis rappresenta il passo preliminare necessario per lo
% svolgimento di attività che si concentrano sulla stima, sul monitoraggio e sul
% miglioramento della qualità di un sistema software. Attività come reverse
% engineering, re-engineering, testing o costruzione di modelli empirici per la
% valutazione della qualità del software prevedono tutte una fase preliminare in
% cui vengono estratte informazioni dal codice sorgente, o per la costruzione di
% un AST che consenta una più semplice interazione con il codice di un programma
% o per la costruzione di una rappresentazione analoga che funga da meccanismo
% per la raccolta e l’accesso alle informazioni espresse dal codice sorgente del
% sistema in analisi.

% Nell’ultimo decennio sono stati sviluppati diversi tipi di linguaggi e toolkits
% per la source code analysis. Alcuni di questi sono particolarmente adatti alla
% comprensione e trasformazione di un sistema <citazione> <citazione> <citazione>
% ; gli strumenti di questo tipo hanno delle potenti capacità dal punto di vista
% dell’analisi, consentono ad esempio di fornire linguaggi per il pattern
% matching o un modo per compiere ricerche all’interno di un AST o operare
% trasformazioni di questo. Altri strumenti, come ad esempio <citazione>
% <citazione> <citazione>, sono invece più orientati verso l’analisi statica e
% verso il calcolo di metriche di valutazione della qualità piuttosto che verso
% la capacità di operare trasformazioni. Ciascuno di questi strumenti per la
% stima, il monitoraggio e l’estrazione di metriche rispetto alla qualità del
% codice sorgente di un programma richiede però, come accennato in precedenza,
% che sia disponibile una qualche tecnologia che consenta il parsing.

% - LISP IS "EASY", BUT NOT "TOO EASY", THEREFORE AN AST LIBRARY IS NEEDED.

% - USEFUL TO BUILD "SOURCE BASED" TOOLS, E.G., PATTERN MATCHING AND 
%   TYPE-INFERENCE TOOLS (WHICH, FOR  COMMON LISP, ARE RATHER TRICKY).