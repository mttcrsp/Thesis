\section{Modulo di parsing}

Dopo aver approfondito, nel corso della precedente sezione, le strutture che
la libreria CLAST offre per la rappresentazione di codice sorgente, questa
sezione ha l’obiettivo di presentare e discutere la progettazione e
implementazione del secondo modulo facente parte libreria. Il modulo di
parsing ed analisi è responsabile dello svolgimento delle reali operazioni di
analisi sintattica, e in parte semantica, del codice sorgente fornito in input
alla libreria. Attività che producono come output la rappresentazione del
programma Common Lisp soggetto in forma di abstract syntax tree da cui la
libreria prende nome.\\

Dal punto di vista strutturale, il modulo consiste di un parser a discesa
ricorsiva, una particolare tipologia di parser che verrà brevemente discussa
nel corso dei prossimi paragrafi.

\subsubsection{Parser a discesa ricorsiva}

Formalmente, un parser a discesa ricorsiva è definito come un parser che opera
in modo top-down, costruito a partire da un insieme di procedure mutualmente
esclusive, o equivalenti, in cui tipicamente ciascuna procedura implementa una
delle produzioni della grammatica associata al linguaggio target dell’attività
di parsing.

Il particolare parser presente all’interno della libreria CLAST può inoltre
essere definito un predictive-parser, ossia un parser a discesa ricorsiva in
grado di svolgere la propria attività di analisi e scomposizione senza
necessità di operare backtracking o, in altri termini, senza la necessità di
svolgere e abbandonare computazioni parziali.\\

I principali vantaggi legati alla scelta di organizzare il processo di parsing
come un parser di questo tipo sono principalmente due. Da un punto di vista
computazionale, la proprietà più interessante di un predictive-parser è
rappresentata dal fatto che questo è in grado di lavorare in tempo lineare
nella dimensione del suo input.

Un secondo vantaggio legato alla scelta di organizzare le attività di parsing
come un predictive-parser è invece più legato ad aspetti di qualità del
software e manutenibilità dell’implementazione prodotta. La forte
strutturazione del parser all'interno un insieme di procedure specifiche per
ciascuna produzione della grammatica del linguaggio in analisi, guida infatti
l’implementazione nella definizione di un insieme di funzioni altamente coese
nel rispetto dei principi di single responsibility e separation of concerns
che consentono di ottenere codice particolarmente espressivo, manutenibile e
facilmente testabile.\\

È importante sottolineare che, affinché sia possibile definire un predictive-
parser per un dato linguaggio, è necessario che la grammatica di questo sia un
grammatica di tipo context-free, ossia una grammatica non ambigua e per la
quale vale la proprietà per cui è sempre possibile osservare un numero finito
di token dallo stream di input per poter determinare la corretta produzione
della grammatica da applicare. Fortunatamente, la grammatica del linguaggio
Common Lisp è una grammatica di tipo context-free.

\subsubsection{Strategie per la costruzione di un parser}

Un parser come quello appena descritto può essere costruito utilizzando in
maniera manuale o automatica. Approcci automatici alla creazione di un parser
lavorano tipicamente a partire da strumento specializzato nella generazione di
sistemi software di questo tipo. Un generatore di parser prende in input una
descrizione formale della grammatica del linguaggio target del parsing e
produce in output un programma in grado di operare il parsing del linguaggio
target oppure una struttura, solitamente chiamata parse table, a partire dalla
quale un programma driver è in grado di operare il parsing.\\

Il vantaggio dell’utilizzo di strumenti automatici è chiaramente rappresentato
dall’immediatezza con la quale questi consentono di produrre strumenti
efficienti e per i quali è possibile compiere una verifica formale di
correttezza. Lo svantaggio fondamentale dovuto al fatto che la loro estensione
al fine di supportare la raccolta di informazioni aggiuntive risulta molto
spesso difficile o impossibile. Inoltre, l’utilizzo di questi strumenti non è
possibile nel caso in cui non sia disponibile una grammatica formale relativa
al linguaggio target delle operazioni di parsing.\\

Il requisito e lo svantaggio appena citati rendono l’applicazione di un
approccio automatico impossibile nel contesto del progetto soggetto di questa
tesi. Infatti, in prima battuta, non sono disponibili definizioni formali
della grammatica completa del linguaggio Common Lisp. Inoltre, ponendosi CLAST
come una soluzione per la creazione di strumenti di source code analysis di
diverso tipo, risulta di fondamentale importanza la possibilità di estendere
il processo di parsing al fine di raccogliere il maggior numero di
informazioni possibili rispetto al codice del programma in analisi e che un
parser generato in modo automatico da uno strumento generico non è in grado di
fornire.\\

Per queste ragioni, la scelta intrapresa è stata quella di implementare un
parser in modo manuale. Da un punto di vista prettamente legato all’analisi
sintattica che deve essere operata da un parser, il linguaggio Lisp ben si
presta all’azione di strumenti di questo tipo, essendo il codice molto
semplicemente organizzato all’interno di liste. Tuttavia, questa attività ha
comunque necessitato di uno sforzo dal punto di vista dell’implementazione e
del testing particolarmente significativo dato il grande numero di costrutti
che utilizzano una sintassi speciale utilizzati dal linguaggio.\\

Come precedentemente affermato la complessità di CLAST come strumento non è
solamente rappresentata dalla fase di parsing, ma soprattutto dalla fase di
raccolta di informazioni aggiuntive svolta parallelamente al parsing.

L’output dell’azione del modulo di parsing è infatti, affinché sia possibile
raccogliere informazioni come dichiarazioni di tipo, particolarmente utili
all'analisi strumenti di type checking, sia informazioni relative all’utilizzo
di variabili non inizializzate, particolarmente utili all’analisi svolta da
strumenti di linting, è necessario che, comtemporaneamente alla
formalizzazione della struttura del programma all’interno di una
rappresentazione formale, questa rappresentazione formale venga arricchita con
informazioni relative all’ambiente di valutazione delle diversi componenti del
programma. Questo viene fatto a partire dal gestione di oggetti di tipo
environment, oggetti che consentono la memorizzazione di dichiarazioni valide
al momento della valutazione di un particolare frammento di codice. Questa
attività di gestione di tutte le possibili informazioni relative alle
dichiarazione presenti nel codice sorgente del programma in analisi ricalca
molto da vicino l’attività svolta dalla componente di valutazione di un
sistema Common Lisp e rappresentazione un’attività di molto significativa
complessità, in quanto richiede sia la comprensione dei meccanismi interni di
funzionamento di ciascun costrutto del linguaggio che la manipolazione,
utilizzando un API particolarmente ristretta e relativamente poco adatta allo
scopo, degli environment in modo parallelo al processo di navigazione del
codice sorgente operato dal parsing, aspetti che non sempre possono essere
combinati con facilità.

\subsection{Implementazione del modulo}

L’API fondamentale offerta da questo modulo è rappresentata da un’unica
funzione. La funzione PARSE è una funzione che prende in input una form Lisp e
produce in output due valori, una rappresentazione in forma di AST composta da
istanze di oggetti definiti all’interno del modulo di rappresentazione,
presentati nella sezione precedente di questo capitolo, e un oggetto
environment, il quale rappresenta l’environment prodotto dalla valutazione
della form. Il listato <> mostra un estratto del codice della libreria in cui
si mostra la definizione della funzione PARSE.\\

CODICE DELLA DEFINIZIONE DI PARSE CON DOCUMENTAZIONE\\

La funzione PARSE viene utilizzata per identificare ed applicare le regole più
generali della grammatica soggetto del parsing. La funzione verifica se la
form fornita in input alla funzione rappresenta un self-evaluating object
oppure se rappresenta un simbolo o una compound form, i tre elementi
fondamentali del processo di valutazione del Common Lisp presentato nel corso
della sezione precedente.\\

DEVI SPIEGARE IL FATTO CHE BODY-ENV, NEL CASO DI ISTRUZIONI COME LET, È MOLTO
DIVERSO DALL’ENVIRONMENT RITORNATO, IN QUANTO ISTRUZIONI COME LET SONO IN
GRADO DI PRODURRE UN NUOVO ENVIRONMENT, LESSICALE O DINAMICO, AGGIUNGENDO
NUOVE INFORMAZIONI A QUESTO, ALL’INTERNO DEL QUALE VERRANO VALUTATE LE
IMPROGN- FORMS. SE INVECE VIENE MESSA PRIMA È NECESSARIO SPIEGARE NELLA
SEZIONE RELATIVA AL PARSING CHE IN ALCUNI CASI SI HA QUESTA DIFFERENZA.\\

Nel caso in cui la form rappresenti un self-evaluating symbol, il parsing e la
valutazione risultano banali, viene immediatamente ritornata un’istanza della
classe CLAST-ELEMENT appartenente alla sottoclasse più appropriata alla
rappresentazione dell’oggetto in analisi e il parsing termina immediatamente,
riportando, insieme all’istanza appena descritta, un environment lessicale
vuoto.\\

Nel caso in cui la form rappresenti un simbolo questo viene risolto
all’interno dell’environment opzionale specificato in fase di invocazione. Nel
caso in cui il simbolo sia associato ad una variabile il parsing agisce in
maniera analoga a quanto riportato in precedenza, viene ritornato un’istanza
della classe CLAST-ELEMENT adeguata alla rappresentazione del simbolo e il
valore associato a questo identificato nell’environment, assieme ad un
environment lessicale vuoto, e il parsing ha fine. Nel caso in cui invece il
simbolo rappresenti una macro, la funzione PARSE svolge un procedimento più
complesso, andando ad operare la macro-espansione del simbolo all’interno
dell’environment opzionale fornito in input alla funzione, per poi ritornare
come primo valore un’istanza di EXPANSION-COMPONENT, che riporti il simbolo
originale, il corpo ottenuto tramite macro espansione e il valore ottenuto
dalla valutazione di questo corpo, e come secondo valore l’environment
lessicale prodotto dall’espansione del simbolo.\\

Infine, nel caso in cui invece il simbolo sia una compound form, il parsing e
l’analisi proseguono in modo significativamente più complesso. Questo in
quanto, come riportato nella sezione <> relativamente al processo di
valutazione caratteristico del linguaggio Common Lisp, questo è il caso in cui
si ha la possibilità di incontrare diversi costrutti che utilizzano una
sintassi speciale e regole di valutazione speciali.\\

Tenendo fede alla struttura precedentemente citata di definizione di un
funzione per produzione della grammatica di parsing, la funzione PARSE delega
ad un’altra funzione il processo di ricerca, all’interno dell’insieme delle
diverse regole della grammatica, della corretta produzione da applicare per
proseguire nel processo di parsing.\\

La funzione PARSE si limita quindi ad operare l'estrazione dalla compound form
identificata dell’elemento CAR e del CDR, la testa e la coda della lista che
costituisce una compond form, come mostra il listato <>, ed invocare, a
partire da questi due elementi, la funzione generica PARSE-FORM, responsabile
per la prosecuzione del parsing secondo lo schema di discesa ricorsiva.\\

ESEMPIO DI CAR E CDR DI UNA COMPOUND FORM\\

La funzione PARSE-FORM è, come anche la funzione PARSE, una funzione generica.
Il meccanismo delle funzioni generiche è uno dei meccanismi fondamentali
attraverso i quali il CLOS, ossia il Common Lisp Object System, implementa il
supporto alla programmazione object-oriented offerto dal linguaggio Common
Lisp.\\

Una funzione generica Common Lisp è una funzione il cui comportamento viene
determinato a partire dalle classi e dalle identità dei parametri che vengono
forniti in input a questa. Una funzione generica specifica un nome di funzione
e una lista di parametri ma, a differenza di una funzione ordinaria, non
specifica alcun corpo e quindi nessun comportamento che deve essere eseguito
in risposta ad una sua invocazione. La reale implementazione del comportamento
di una funzione generica può essere specificato attraverso la definizione di
metodi.\\

CODICE DI ESEMPIO PER LA DEFINIZIONE DI UNA FUNZIONE GENERICA\\

Un metodo Common Lisp è una funzione che specifica l’implementazione di una
funzione generica per una determinato insieme di specializzazioni dei
parametri di questa. Ciascun parametro può essere specializzato da un metodo
in due diversi modi: indicando la classe di appartenenza di questo, come ad
esempio NUMBER, oppure indicando l’identità di uno specifico oggetto.\\

CODICE CON DUE ESEMPI DI DEFINIZIONE DI UN METODO CHE MOSTRI I DUE DIVERSI
TIPI DI SPECIALIZZAZIONE\\

Quando una funzione generica viene invocata, il sistema Lisp analizza i
parametri forniti in input e ricerca, dall’elenco dei metodi associati alla
funzione, quale metodo riporti specializzazioni compatibili con i tipi e le
identità di questi, per poi eseguire il codice associato al metodo e
determinare così il reale comportamento della funzione generica.\\

Da un punto di vista teorico, il meccanismo delle funzioni generiche Common
Lisp rappresenta un’implementazione del concetto di polimorfismo tipico della
programmazione Object Oriented, realizzato solitamente nel contesto di altri
linguaggi di programmazione con tecniche di message passing e funzioni come
SEND in Smalltalk e objc\_msgSend in Objective-C.\\

Dal punto di vista delle performance e della complessità algoritmica, è
importante sottolineare che l’attività più significativa è rappresentata
dall’identificazione della regola corretta da applicare in risposta ad un
particolare input. L’aver strutturato il processo di parsing in maniera tale
che questo vada a delegare questa attività interamente al sistema Lisp in
utilizzo, tramite l’utilizzo di funzioni generiche in fase di implementazione,
è un aspetto che rende il modulo performante e significativamente meno
complesso.\\

QUESTA PARTE SULLE FUNZIONI GENERICHE DOVREBBE ESSERE SPOSTATA PRIMA DELLA
SPIEGAZIONE DI PARSE. IN QUESTO MODO DIVENTA POSSIBILE PRESENTARE IL CONFRONTO
TRA IL FATTO CHE PARSE LAVORA UTILIZZANDO SPECIALIZZAZIONI DI TIPO, MENTRE
PARSE-FORM CON SPECIALIZZAZIONI RISPETTO ALL’IDENTITÀ.\\

Ritornando alla funzione PARSE-FORM, si è detto che questa è una funzione
generica. Il comportamento di questa funzione viene specificato andando a
definire grande insieme di metodi, sostanzialmente uno per ciascuno operatore
Common Lisp, e quindi per ciascuna special form che utilizza una sintassi
speciale, che a sua volta identifica ciascuna delle regole della grammatica.
Ciascuno di questi metodi opera una specializzazione rispetto all’identità del
primo parametro fornito in input a questa,  in particolare una
specializzazione rispetto al particolare operatore che identifica una delle
regole della grammatica. Questo porta ad ottenere la struttura di predictive-
parser, già citata in precedenza in questo capitolo, con una funzione
mutuamente esclusiva per ciascuna produzione della grammatica del linguaggio
target. Il listato <> mostra alcuni esempi di signature dei metodi associati
alla funzione generica PARSE-FORM.\\

ESEMPI DI SIGNATURE DEI METODI PARSE-FORM\\

I METODI PARSE-FORM LAVORANO IN MODO RICORSIVO\\

OLTRE AD I METODI PARSE-FORM E PARSE IL MODULO CONTIENE UN GRANDE INSIEME DI
HELPER FUNCTIONS. UNA FUNZIONE PER CIASCUNA REGOLA DELLA GRAMMATICA RELATIVA A
COMPOUND FORM

\section{Parsing e analisi in dettaglio}

Per illustrare in maniera più approfondita il funzionamento della libreria, in
questa sezione viene brevemente mostrata il del processo di parsing e analisi
operato dalla libreria attraverso la discussione di un esempio, rappresentato
da una semplificazione di uno dei metodi PARSE-FORM. In particolare, il metodo
dedicato al parsing di compound form che utilizzano l’operatore LET.\\

In seguito viene riportato un frammento della specifica ANSI in cui viene
definita la sintassi del costrutto LET e vengono specificati i dettagli
rispetto al processo di valutazione di form che utilizzano questo operatore.\\

\begin{lstlisting}

let ({var | (var [init-form])}*) declaration* form* => result*

(1) LET and LET* create new variable bindings and (2) execute a series of
forms that use these bindings. LET performs the bindings in parallel (3) and
LET* does them sequentially.

The form

 (let ((var1 init-form-1)
       (var2 init-form-2)
       ...
       (varm init-form-m))
   declaration1
   declaration2
   ...
   declarationp
   form1
   form2
   ...
   formn)

first evaluates the expressions init-form-1, init-form-2, and so on, in that
order, saving the resulting values. Then all of the variables varj are bound
to the corresponding values; (4) each binding is lexical unless there is a
special declaration to the contrary. The expressions formk are then evaluated
in order; the values of all but the last are discarded (that is, the body of a
let is an implicit progn).

For both LET and LET*, if there is not an init-form associated with a var, var
is initialized to NIL.

\end{lstlisting}

Il listato presenta una approssimazione del metodo PARSE-FORM implementato
dalla libreria.

\begin{lstlisting}
(defmethod parse-form ((op (eql 'let*)) params top :optional env internal-env)
 ;; 1
 (destructuring-bind (bindings declarations :rest body-forms) params
    ;; 2
    (let ((body-env (clone (or internal-env body-env))))
      ;; 3
      (dolist (binding bindings)
  (setf body-env (parse-binding binding body-env)))
      ;; 4
      (when declarations
  (setf body-env (parse-declarations body-env)))
      ;; 5
      (multiple-value-bind (iprogn-forms augmented-env)
    (if body-forms
        (parse-iprogn-forms body-forms env body-env)
        (values nil env))
  ;; 6
  (values
   (make-instance 'let*-form
      :source (cons operator params) :top top :type t
      :iprogn-forms iprogn-forms :body-env body-env
      :bindings bindings)
   env)))))
1
  ;; Extracts all the various syntactic element from the provided
  ;; params of the compound form based on the syntax rules associated
  ;; with this special form denoted by the let* operator.
2
    ;; A copy of the provided environment is used in order to be able
    ;; to represent both dynamic and lexical scope and be able to have
    ;; local bindings and other bindings separated.
3
      ;; For each binding in the list of bindings invokes the
      ;; parse-binding function on it. The parse binding function
      ;; simply performs a side effect on the specified environment
      ;; adding a variable declaration to it based on the name
      ;; specified by the binding `var` portion.
4
      ;; For each declaration parse-declaration performs a side-effect
      ;; on the provided environment adding informations as specified
      ;; by the declaration.
5
      ;; Operates parsing of body forms. If a value for internal-env
      ;; was specified, each parsing methods uses it to lookup
      ;; definitions, but only augments `env`; otherwise it defaults
      ;; to using `env` for both lookup and recording definitions.
6
  ;; Returns the element representation and the provided
  ;; environment augmented only with definitions from the body
  ;; form (i.e. that does not contain any information about
  ;; local bindings).
\end{lstlisting}

Il metodo prende in input cinque valori, due dei quali opzionali. La prima
operazione che viene svolta è legata al reale parsing. Il valore PARAMS, il
quale rappresenta la lista dei parametri della compound form originale, viene
decomposto all’interno di tre diversi elementi in base a quanto specificato
dalla regola della grammatica del linguaggio associata al costrutto LET*. La
regola specifica infatti che ciascuna form che abbia LET* come operatore
prevede come primo parametro una lista di bindings, una lista di dichiarazioni
come secondo parametro opzionale e che tutte le form che seguono queste due
liste rappresentino una sequenza di form che devono essere eseguite
nell’ambiente prodotto dalla valutazione di dichiarazioni di bindings e
dichiarazioni.\\

La seconda operazione svolta dal metodo è la definizione di una copia
dell’environment fornito in input, copia che verrà utilizzata in seguito. Il
metodo opera quindi un’invocazione del metodo parse-binding a partire da
ciascun binding. La funzione PARSE-BINDING è quindi responsabile del parsing
delle regole specificate dalla grammatica in relazione alla sintassi dei
binding, ossia la regola var | (var [init-form]), e di creare un nuovo
environment a partire da quello fornito in input contenente la definizione di
variabile operata dal binding. Ogni invocazione di PARSE-BINDING ritorna
quindi un nuovo oggetto environment che viene utilizzato per aggiornare
l’environment interno al metodo.\\

Il metodo prosegue quindi con l’analisi delle dichiarazioni specificate dalla
form relativamente alle variabili dichiarate dai binding, nel caso siano
presenti. Questa analisi viene operata dalla funzione PARSE-DECLARATIONS. Come
la precedentemente citata PARSE-BINDING, PARSE-DECLARATIONS è quindi
responsabile del parsing a partire dalle regole del linguaggio relativamente
alle dichiarazioni e di restituire una nuova istanza di environment, costruita
a partire da quello in input, contenente le informazioni specificate da
ciascuna dichiarazione. L’istanza restituita da PARSE-DECLARATIONS viene
quindi utilizzata per aggiornare l’environment interno al metodo.\\

Il passo successivo è rappresentato dal parsing delle form innestate alla form
LET*, operazione che viene delegata alla funzione PARSE-PROGN-FORMS. Questa
funzione ha lo scopo di compiere la valutazione delle form specificate in
input utilizzando i due environment specificati in input e produrre in output
due valori, una lista contenente la rappresentazioni di ciascuna delle form
utilizzando le strutture dal modulo di rappresentazione e un environment
aumentato con le definizioni che occorrono all’interno delle form.\\

Questa operazione consente di mostrare uno dei principi più complessi del
funzionamento della libreria. Si può osservare infatti che la funzione prende
in input due diversi environment. Il primo environment rappresenta
l’environment che verrà aumentato con le dichiarazioni operate dalle form e
quindi ritornato, il secondo environment viene invece solamente utilizzato per
la ricerca di informazioni relativamente a variabili, simboli e funzioni
identificati durante il parsing. La presenza di questi due environment
distinti, uno per la scrittura e uno per la lettura, rappresenta una
convenzione comune all’intero modulo e di particolare importanza. Questa
convenzione consente infatti di implementare il parsing di costrutti come
LET*, in cui l’environment in cui avviene la valutazione contiene informazioni
aggiuntive rispetto quello in prodotto in output dalla valutazione stessa.\\

PROVA AD AGGIUNGERE IL DISCORSO RELATIVO ALLA GESTIONE DEL FATTO CHE POSSONO
ANCHE NON ESSERCI FORM INNESTATE.\\

MODIFICA IL PARSING DELLE DICHIARAZIONI IN MANIERA TALE DA USARE PARSE-FORM
CON ‘DECLARE. QUESTO È UN ALTRO ASPETTO CHE MOSTRA LA DISCESA RICORSIVA. FAI
VEDERE ANCHE CHE L’ELEMENTO PRODOTTO DA PARSE-FORM VIENE SCARTATO E VIENE
CONSIDERATO SOLAMENTE L’ENVIRONMENT. QUESTO PERCHÈ NEL CASO DELLE
DICHIARAZIONI DI QUESTO TIPO RISULTA RIDONDANTE MANTENERE SIA LA
RAPPRESENTAZIONE COME CLAST-ELEMENT CHE LA RAPPRESENTAZIONE COME INFORMAZIONE
INTERNA ALL’ENVIRONMENT.\\

L’ESEMPIO NON MOSTRA LA DISCESA RICORSIVA AL MOMENTO. MODIFICA IL CODICE IN
MANIERA TALE CHE SI VEDA CHE VIENE CHIAMATO PARSE SULLE FORM INNESTATE.\\

A partire da quanto prodotto dai passi precedenti, il metodo ritorna quindi i
due valori comuni a qualsiasi metodo PARSE-FORM: un primo valore costituito da
un’istanza di CLAST-ELEMENT che rappresenta il nodo radice del sottoalbero del
quale il metodo ha effettuato il parsing, ed un secondo elemento che
rappresenta l’environment prodotto dalla valutazione del sottoalbero.\\

Questa sottosezione ha presentato, attraverso un esempio, il funzionamento e
l’analisi svolta dalla libreria. Analisi che consente di produrre una
rappresentazione estremamente precisa del codice fornito in input a questa.
