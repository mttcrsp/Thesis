\section{Modulo di parsing}
\label{parsing}

Dopo aver approfondito, nel corso della precedente Sezione, le strutture che la
libreria CLAST offre per la rappresentazione di codice sorgente, questa sezione
ha l’obiettivo di presentare e discutere la progettazione e l'implementazione
del secondo modulo facente parte libreria. Il modulo di parsing ed analisi è
responsabile dello svolgimento delle reali operazioni di analisi sintattica e
semantica, del codice sorgente fornito in input alla libreria. Attività che
produrranno come output la rappresentazione del programma Common Lisp soggetto
in forma di abstract syntax tree da cui la libreria prende nome.\\

\subsubsection{Attività parsing}

Dal punto di vista strutturale, il modulo consiste di un parser a discesa
ricorsiva, una particolare tipologia di parser che verrà brevemente discussa nel
corso dei prossimi paragrafi.

Formalmente, un parser a discesa ricorsiva è definito come un parser che opera
in modo top-down, costruito a partire da un insieme di procedure mutuamente
esclusive, o equivalenti, in cui tipicamente ciascuna procedura implementa una
delle produzioni della grammatica associata al linguaggio target dell’attività
di parsing.

Il particolare parser presente all’interno della libreria CLAST può inoltre
essere definito anche come predictive-parser, dal punto di vista degli
analizzatori sintattici può formalemente essere definito come un recursive
predictive parser, un parser a discesa ricorsiva in grado di svolgere la propria
attività di analisi e scomposizione senza necessità di operare backtracking o,
in altri termini, senza la necessità di svolgere e abbandonare computazioni
parziali.\\

I principali vantaggi legati alla scelta di organizzare il processo di parsing
come un parser di questo tipo sono principalmente due. Da un punto di vista
computazionale, la proprietà più interessante di un predictive-parser è
rappresentata dal fatto che questo è in grado di lavorare in tempo lineare nella
dimensione del suo input.

Un secondo vantaggio legato alla scelta di organizzare le attività di parsing
come un predictive-parser è invece più legato ad aspetti di qualità del software
e manutenibilità dell’implementazione prodotta. La forte strutturazione del
parser all'interno un insieme di procedure specifiche per ciascuna produzione
della grammatica del linguaggio in analisi, guida infatti l’implementazione
nella definizione di un insieme di funzioni altamente coese nel rispetto dei
principi di single responsibility e separation of concerns che consentono di
ottenere codice particolarmente espressivo, manutenibile e facilmente
testabile.\\

È importante sottolineare che, affinché sia possibile definire un predictive-
parser per un dato linguaggio, è necessario che la grammatica di questo sia un
grammatica di tipo context-free, ossia una grammatica non ambigua e per la quale
vale la proprietà per cui è sempre possibile osservare un numero finito di token
dallo stream di input per poter determinare la corretta produzione della
grammatica da applicare. Fortunatamente, la grammatica del linguaggio Common
Lisp è una grammatica di tipo context-free.

\subsubsection{Strategie per la costruzione di un parser}

Un parser come quello appena descritto può essere costruito utilizzando un
approccio di tipo automatico oppure utilizzando un approccio manuale.\\

Approcci automatici alla creazione di un parser lavorano tipicamente a partire
da uno strumento specializzato nella generazione di sistemi software di questo
tipo, detti appunto generatori di parser. Un generatore di parser prende in
input una descrizione formale della grammatica del linguaggio target del parsing
e produce in output un programma in grado di operare il parsing di tale
linguaggio. Alternativamente, alcuni generatori di parser producono in output
una struttura, solitamente chiamata parse table, a partire dalla quale un
programma driver generico è in grado di operare il parsing.\\

Il vantaggio principale legato all'utilizzo di strumenti automatici è
chiaramente rappresentato dall’immediatezza con la quale questi consentono di
produrre strumenti efficienti. Un secondo vantaggio particolarmente
significativo è rappresentato dal fatto che, in molti casi, il parser generato
dallo strumento è corretto in modo formalmente dimostrabile.

Lo svantaggio fondamentale legato all'utilizzo di un generatore di parser è
dovuto al fatto che la loro estensione, al fine di supportare la raccolta di
informazioni aggiuntive, risulta molto spesso difficile se non impossibile.
Inoltre, l’utilizzo di questi strumenti non è possibile nel caso in cui non sia
disponibile una descrizione formale della grammatica del linguaggio target delle
operazioni di parsing, una descrizione che possa essere fornita in input allo
strumento.\\

Il requisito e lo svantaggio appena citati rendono l’applicazione di un
approccio automatico non applicabile al contesto del progetto soggetto di questa
tesi. Infatti, in prima battuta, non sono disponibili definizioni formali della
grammatica completa del linguaggio Common Lisp. Inoltre, ponendosi CLAST come
una soluzione per la creazione di strumenti di source code analysis di
molteplici tipologie, la possibilità di estendere il processo di parsing al fine
di raccogliere il maggior numero di informazioni possibili rispetto al codice
del programma in analisi risulta di fondamentale importanza. Informazioni che un
parser generato in modo automatico da uno strumento generico non è tipicamente
in grado di fornire.\\

Per queste ragioni, la scelta intrapresa è stata quella di implementare un
parser in modo manuale. Da un punto di vista prettamente legato all’analisi
sintattica che deve essere operata da un parser, il linguaggio Common Lisp ben
si presta alla costruzione di strumenti di questo tipo, essendo il codice Lisp
organizzato all’interno di semplici liste. Tuttavia, questa attività ha comunque
necessitato di uno sforzo particolarmente significativo dal punto di vista
implementativo e del testing. Questo soprattutto a causa del grande numero di
costrutti che utilizzano una sintassi speciale utilizzati dal linguaggio e della
mancanza di una definizione formale che potesse automatizzare, almeno in parte,
la generazione del codice di implementazione e la ricerca di casi di test.

\subsubsection{Attività analisi}

Come precedentemente affermato la complessità di CLAST come strumento non è
solamente rappresentata dalla fase di parsing, ma soprattutto dalla fase di
raccolta di informazioni aggiuntive, fase che viene svolta parallelamente al
parsing.

Infatti, molte delle informazioni utili ed interessanti a strumenti di source
code analisys non sono immediatamente ottenibili dalla rappresentazione in forma
di abstract syntax tree prodotta dalla sola attività di parsing di un programma
e realizzata, nel contesto di CLAST, come descritto dalla precedente
sottosezione.

Esempi di informazioni di particolare utilità e interesse per uno strumento di
source code analysis che non possono essere ricavate da un solo AST sono le
informazioni relative all'utilizzo da parte del programma di variabili libere,
particolarmente interessanti per la realizzazione di strumenti per il linting, o
informazioni relative al tipo delle form che rappresentano i nodi dell'AST,
particolarmente interessati per la realizzazione type checkers.\\

Per questa ragione, parallelamente all'attività di parsing vera a propria, il
modulo descritto da questa sezione opera un'attività di analisi e raccolta di
informazioni riguardo all'ambiente di valutazione del codice del programma
soggetto.

Questa attività di analisi viene svolta a partire dal gestione di oggetti di
tipo ambiente, oggetti che consentono la memorizzazione di definizioni e
dichiarazioni valide al momento della valutazione di un particolare frammento di
codice, e che ricalca molto da vicino l’attività svolta dalla componente di
valutazione di un sistema Common Lisp.\\

La progettazione e realizzazione dell'attività appena descritta rappresenta uno
degli aspetti di maggiore complessità svolta dalla libreria, in quanto richiede
sia la comprensione del meccanismo interno di valutazione caratteristico del
linguaggio Common Lisp e del funzionamento di ciascun costrutto del linguaggio
sia la manipolazione, utilizzando un API particolarmente ristretta e
relativamente poco adatta allo scopo, di un oggetto di tipo ambiente. Tutto
questo in modo parallelo al processo di navigazione del codice sorgente operato
dal parsing, aspetti che non sempre possono essere combinati con facilità.

\subsection{Implementazione del modulo}

I precedenti paragrafi hanno descritto, ad alto livello, le principali
funzionalità e il funzionamento del modulo di parsing. La prossima sottosezione
entrerà nel dettaglio dell'API offerta dal modulo agli utenti della libreria e
del funzionamento interno al modulo, mostrando come le attività di parsing ed
analisi siano state realizzate all'interno di CLAST.

\subsubsection{La funzione \texttt{PARSE}}

L’API fondamentale realizzata dal modulo di parsing e analisi viene esposta agli
utenti della libreria attraverso un'interfaccia rappresentata da un’unica
funzione, la funzione \texttt{PARSE}. La funzione \texttt{PARSE} è una funzione
che prende in input una form Lisp e produce in output due valori.

\begin{itemize}

\item Il primo valore ritornato consiste nel nodo radice della rappresentazione
del programma Common Lisp, rappresentato dalla form fornita in input, in forma
di AST. Un AST in cui nodi sono oggetti definiti all’interno del modulo di
rappresentazione, presentati nella Sezione \ref{representation} di questo
capitolo.

\item Il secondo valore ritornato consiste invece di un oggetto ambiente, il
quale rappresenta il particolare ambiente prodotto dalla valutazione della form
e arricchito durante l'attività di analisi svolta dal modulo in modo parallelo
al parsing.

\end{itemize}

La coppia prodotta come output rappresenta il risultato delle attività parallele
di parsing e analisi precedentemente descritte e il massimo insieme di
formazioni che risulta possibile raccogliere, rispetto ad un dato frammento di
codice Common Lisp, utilizzando la libreria CLAST. Il Listato \ref
{lst:parse-function} mostra un estratto del codice della libreria in cui viene
presentata la definizione della funzione \texttt{PARSE}.

\begin{lstlisting}[
  caption=Definizione e documentazione relative alla funzione \texttt{PARSE},
  label={lst:parse-function}
]

(defgeneric parse (form &rest keys
                        &key
                        enclosing-form
                        macroexpand
                        environment
                        &allow-other-keys)
  (:documentation "Parses a form in a given 'environment'.

The methods of this generic function return a AST 'node' (a
CLAST-ELEMENT) and the - possibly modified - ENVIRONMENT.

The methods of PARSE dispatch on 'atomic' and on 'compound' (i.e.,
CONS) forms.  Atomic forms - numbers, string, arrays, and symbols -
are dealt with directly.  Compound forms are dispatched to PARSE-FORM.

Arguments and Values:

form : the form to be parsed.
keys : the collection of key-value pairs passed to the call.
enclosing-form : the form that \"contains\" the form beling parsed.
environment : the environment in which the form is being parsed.
element : a CLAST-ELEMENT representing the AST node just parsed.
environment1 : the environment resulting from parsing the FORM.
")
  )

\end{lstlisting}

I precedenti paragrafi hanno discusso la funzione \texttt{PARSE} dal punto di
vista della funzionalità che viene esposta da questa, ossia dal punto di vista
dell'utente della libreria. Dal punto di vista del funzionamento interno alla
libreria, la funzione \texttt{PARSE} rappresenta il punto di inizio del processo
di parsing e analisi operato dalla libreria. In particolare, la funzione ha la
responsabilità di applicare le regole più generali della grammatica del
linguaggio Common Lisp.

Questo significa che la funzione si limita a verificare se la form fornita in
input rappresenti un self-evaluating object, un simbolo o una compound form: i
tre elementi fondamentali della grammatica e del processo di valutazione del
Common Lisp presentato nel corso della Sezione \ref{representation}.\\

Nel caso in cui la form rappresenti un self-evaluating symbol, il parsing e la
valutazione risultano relativamente semplici. Viene infatti immediatamente
ritornata un’istanza della classe \texttt{CLAST-ELEMENT} appartenente alla
sottoclasse più appropriata alla rappresentazione del self-evaluating object in
analisi e il parsing termina immediatamente, riportando, insieme all’istanza
appena descritta, un ambiente lessicale vuoto. Questo in quanto la valutazione
di self-evaluating objects non produce alcuna variazione rispetto al contenuto
dell'ambiente.\\

Nel caso in cui la form rappresenti un simbolo questo viene risolto all’interno
dell’oggetto ambiente opzionale specificato in fase di invocazione. A questo
punto, il processo di parsing e analisi prendono due possibili percorsi diversi.

\begin{enumerate}

\item Nel caso in cui il simbolo sia associato ad una variabile, il parsing
agisce in maniera analoga a quanto riportato in precedenza. Viene ritornato
un’istanza della classe \texttt{CLAST-ELEMENT} adeguata alla rappresentazione
del simbolo e il valore associato a questo identificato nell’ambiente, assieme
ad un ambiente lessicale vuoto. Il processo ha quindi fine.

\item Nel caso in cui invece il simbolo rappresenti una macro, la funzione
\texttt{PARSE} svolge un procedimento più complesso, andando ad operare la
macro-espansione del simbolo all’interno dell’ambiente opzionale fornito in
input alla funzione, per poi ritornare, come primo valore un’istanza di \texttt
{EXPANSION-COMPONENT} che riporti il simbolo originale, il corpo ottenuto
tramite macro espansione e il valore ottenuto dalla valutazione di questo corpo,
e come secondo valore l’ambiente lessicale prodotto dall’espansione del simbolo.

\end{enumerate}

Infine, nel caso in cui invece la form fornita in input alla funzione
\texttt{PARSE} sia una compound form, il parsing e l’analisi proseguono in modo
significativamente più complesso. Questo in quanto, come riportato nella
Sottosezione \ref{CL-valutazione} relativamente al processo di valutazione
caratteristico del linguaggio Common Lisp, questo è il caso in cui si ha la
possibilità di incontrare diversi costrutti che utilizzano una sintassi speciale
e regole di valutazione speciali.\\

Tenendo fede alla struttura precedentemente citata, in cui si ha una definizione
di un funzione per ciascuna produzione della grammatica del linguaggio target e
il parsing viene organizzato costruendo un predictive parser, la funzione
\texttt{PARSE} delega ad un’altra funzione il processo di ricerca, all’interno
dell’insieme delle diverse regole della grammatica, della corretta produzione da
applicare per proseguire nel processo di parsing.\\

La funzione \texttt{PARSE} si limita quindi ad operare l'estrazione dalla
compound form identificata dell’elemento \texttt{CAR} e del \texttt{CDR}, la
testa e la coda della lista che costituisce la compond form, come mostra il
Listato \ref{lst:car-cdr}, ed invocare, a partire da questi due elementi, la
funzione generica \texttt{PARSE-FORM}, responsabile per la prosecuzione del
parsing e per l'avvio dello schema di discesa ricorsiva.\\

La funzione \texttt{PARSE-FORM} è, come anche la funzione \texttt{PARSE}, una
funzione generica. Il meccanismo delle funzioni generiche è uno dei meccanismi
fondamentali attraverso i quali il CLOS, ossia il Common Lisp Object System,
implementa il supporto alla programmazione object-oriented offerto dal
linguaggio Common Lisp. Come verrà presentato in seguito, le funzioni generiche
rappresentano anche uno degli strumenti che la libreria CLAST utilizza in modo
più intensivo. Per questa ragione, la prossima sezione illustra brevemente il
meccanismo delle funzioni generiche Common Lisp.

\subsubsection{Funzioni generiche}

Formalmente, una funzione generica Common Lisp può essere definita come una
funzione il cui comportamento viene determinato a partire dalle classi e dalle
identità dei parametri che le vengono forniti in input.\\

Una funzione generica specifica un nome di funzione e una lista di parametri ma,
a differenza di una funzione ordinaria, non specifica alcun corpo e quindi
nessun comportamento che deve essere eseguito in risposta ad una sua
invocazione. La reale implementazione del comportamento di una funzione generica
viene specificata attraverso la definizione di metodi.

\begin{lstlisting}[caption=Confronto tra definizione di funzioni generiche e
ordinarie]

;; A generic function definition. Note that no body for the function
;; needs to be specified.
(defgeneric id (x))

;; An ordinary function definition. A body needs to be specified,
;; otherwise an empty body is implicit nil return form is assumed.
(defun id (x)
  x)

\end{lstlisting}

Un metodo Common Lisp è una funzione che specifica l’implementazione di una
funzione generica per una determinato insieme di specializzazioni dei parametri
di questa. Ciascun parametro può essere specializzato da un metodo in due
diversi modi:

\begin{itemize}

\item indicando la classe di appartenenza di questo, come ad esempio
\texttt{NUMBER},

\item indicando l’identità di uno specifico oggetto, come ad esempio il valore
9.

\end{itemize}

\begin{lstlisting}[caption=Esempi definizione di metodi Common Lisp]

CL-USER > (defgeneric some-method (x))
#<STANDARD-GENERIC-FUNCTION COMMON-LISP-USER::SOME-METHOD (0)>
CL-USER > (defmethod some-method ((x number))
     "NUMBER")
#<STANDARD-METHOD COMMON-LISP-USER::SOME-METHOD (NUMBER) {10037A9953}>
CL-USER > (defmethod some-method ((x (eql 42)))
     "This method is eql specialized")
#<STANDARD-METHOD COMMON-LISP-USER::SOME-METHOD ((EQL 42)) {1003843553}>

\end{lstlisting}

Quando una funzione generica viene invocata, il sistema Lisp analizza i
parametri specificati dall'invocazione e ricerca, dall’elenco dei metodi
associati alla funzione, quale metodo riporti specializzazioni compatibili con i
tipi e le identità di questi, per poi eseguire il codice associato al metodo e
determinare così il reale comportamento della funzione generica.

\begin{lstlisting}[caption=Esempi di invocazione di metodi Common Lisp]

CL-USER > (some-method 9)
"NUMBER"
CL-USER > (some-method 42)
"This method is eql specialized"
CL-USER > (some-method "invalid")
;; Evaluation aborted on #<SIMPLE-ERROR "~@<There is no applicable
;; method for the generic function ~2I~_~S~~I~_when called with
;; arguments ~2I~_~S.~:>" {1003B3E8F3}>.

\end{lstlisting}

Da un punto di vista teorico, il meccanismo delle funzioni generiche Common Lisp
rappresenta un’implementazione del concetto di polimorfismo tipico della
programmazione Object Oriented, realizzato solitamente nel contesto di altri
linguaggi di programmazione con tecniche di message passing e funzioni come
\texttt{SEND} in Smalltalk e \texttt{objc\_msgSend} in Objective-C.\\

Dal punto di vista delle performance e della complessità algoritmica, è
importante sottolineare che una delle operazioni più significative nel contesto
di un predictive parser è rappresentata dall’identificazione della corretta
regola da applicare in risposta ad un particolare input. L’aver strutturato il
processo di parsing in maniera tale che questo vada a delegare l'attività di
ricerca di questa regola interamente al sistema Lisp in utilizzo, sistema già
largamente ottimizzato e studiato da questo punto di vista, tramite l’utilizzo
di funzioni generiche in fase di implementazione, è un aspetto che rende il
modulo performante e significativamente meno complesso di quanto sarebbe
altrimenti necessario nel caso in cui il processo di ottimizzazione della
ricerca dovesse essere svolto internamente alla libreria stessa.

\begin{lstlisting}[caption=Signature della funzione \texttt{PARSE-FORM}]

(defgeneric parse-form (op form &rest keys
                        &key
                        enclosing-form
                        macroexpand
                        environment
                        &allow-other-keys)
  (:documentation "Parses a form in a given 'ENVIRONMENT' given its 'op'.

The methods of PARSE-FORM descend recursively in a form, by
dispatching on the form 'operator'.  Each sub-form is passed,
recursively to PARSE.

Arguments and Values:

form : the form to be parsed.
keys : the collection of key-value pairs passed to the call.
enclosing-form : the form that \"contains\" the form beling parsed.
environment : the environment in which the form is being parsed.
element : a CLAST-ELEMENT representing the AST node just parsed.
environment1 : the environment resulting from parsing the FORM.
")
  )

\end{lstlisting}

\subsubsection{La funzione \texttt{PARSE-FORM}}

Ritornando alla funzione \texttt{PARSE-FORM}, si è detto che questa è una
funzione generica. Il comportamento di questa funzione viene quindi specificato
andando a definire un certo insieme di metodi. In particolare, la libreria
riporta sostanzialmente la dichiarazione di un metodo per ciascuno operatore
Common Lisp, ossia uno per ciascuna special form che utilizza una sintassi
particolare, la quale a sua volta identifica una particolare regola della
grammatica.

Dal punto di vista pratico, ciascuno di questi metodi opera una specializzazione
rispetto all’identità del primo parametro che riceve in input, il parametro
\texttt{OP} o \texttt{OPERATOR}, in particolare una specializzazione rispetto al
particolare operatore che identifica una delle regole della grammatica. Questo
porta ad ottenere la struttura di predictive- parser, già citata in precedenza
in questo capitolo, in cui si ha la definizione di una funzione mutuamente
esclusiva per ciascuna produzione della grammatica. Il Listato
\ref{lst:metodi-parse-form} mostra alcuni esempi di signature dei metodi
associati alla funzione generica \texttt{PARSE-FORM}.

\begin{lstlisting}[
  caption=Esempi di specializzazione operata dai metodi \texttt{PARSE-FORM},
  label={lst:metodi-parse-form}
]

(defmethod parse-form ((op (eql 'block)) form |# ... #|)
  ;; ...
  )

(defmethod parse-form ((op (eql 'declare)) form |# ... #|)
  ;; ...
  )

(defmethod parse-form ((op (eql 'let*)) form |# ... #|)
  ;; ...
  )

\end{lstlisting}

\subsection{Parsing e analisi in dettaglio}

Per illustrare in maniera più approfondita il funzionamento della libreria, in
questa sezione vengono discusse nel dettaglio le attività di parsing e analisi
operate dalla libreria attraverso la discussione di un esempio, rappresentato da
una semplificazione di uno dei metodi \texttt {PARSE-FORM} descritti nel corso
dei paragrafi precedenti. In particolare, viene analizzato il funzionamento e
l'implementazione del metodo \texttt{PARSE-FORM} responsabile del parsing e
dell'analisi di compound form che utilizzano l’operatore \texttt{LET*}.\\

In seguito viene riportato un frammento della specifica ANSI in cui viene
definita la sintassi del costrutto \texttt{LET*} e vengono specificati i
dettagli rispetto al processo di valutazione di form che utilizzano questo
operatore.

\begin{lstlisting}[caption=Estratto della documentazione relativa al costrutto
\texttt{LET*}]

let ({var | (var [init-form])}*) declaration* form* => result*

(1) LET and LET* create new variable bindings and (2) execute a
series of forms that use these bindings. LET performs the
bindings in parallel (3) and LET* does them sequentially.

The form

 (let ((var1 init-form-1)
       (var2 init-form-2)
       ...
       (varm init-form-m))
   declaration1
   declaration2
   ...
   declarationp
   form1
   form2
   ...
   formn)

first evaluates the expressions init-form-1, init-form-2, and so
on, in that order, saving the resulting values. Then all of the
variables varj are bound to the corresponding values; (4) each
binding is lexical unless there is a special declaration to the
contrary. The expressions formk are then evaluated in order; the
values of all but the last are discarded (that is, the body of a
let is an implicit progn).

For both LET and LET*, if there is not an init-form associated
with a var, var is initialized to NIL.

\end{lstlisting}

Il Listato \ref{lst:parse-form-let} presenta invece una approssimazione del
metodo \texttt{PARSE-FORM} implementato dalla libreria specializzato rispetto
all'operatore \texttt{LET*}. I dettagli rispetto alle diverse operazioni
compiute dal metodo verranno approfondite nei prossimi paragrafi.

\begin{lstlisting}[
  caption=Semplificazione del metodo \texttt{PARSE-FORM} per il parsing del
  costrutto \texttt{LET*},
  label={lst:parse-form-let}
]

(defmethod parse-form ((op (eql 'let*)) params top
                       :optional env internal-env)

  ;; 1. Extract all the various syntactic elements from the compound
  ;; form based on the syntax rules associated with the special form
  ;; denoted by the let* operator.
  (destructuring-bind (bindings declarations :rest body-forms) params

    ;; 2. A copy of the provided environment is used in order to be
    ;; able to reproduce both the environment that is produced by the
    ;; evaluation of the form and the environment `body-env` in which
    ;; the form is evaluated. For LET* forms this difference is very
    ;; significant, since the evaluation environment will contain
    ;; variable definitions specified by the bindings, but the returned
    ;; environment will not.
    (let ((body-env (clone (or internal-env body-env))))

      ;; 3. For each binding in the list of bindings invoke the
      ;; parse-binding function on it. The parse binding function
      ;; simply returns a new environment object, based on the
      ;; invocation one, that records the addition of variable
      ;; declaration with the name specified by `var` portion of
      ;; the binding.
      (dolist (binding bindings)
        (setf body-env (parse-binding binding body-env)))

      ;; 4. For each declaration parse-declaration generete a new
      ;; environment object, based on the provided one, adding
      ;; informations as specified by the declaration.
      (when declarations
        (multiple-value-bind (declaration-form body-env-with-declarations)
            (parse 'declare (cdr declarations) body-env)

          ;; 5. Update the body-env with information from all
          ;; declarations. Since the declarations themselves are not
          ;; tracked directly the representation object, a CLAST-ELEMENT
          ;; instance is discarded.
          (declare (ignore declaration-form))
          (setf body-env body-env-with-declarations)))

      ;; 6. Execute parsing of body forms. If a value for
      ;; internal-env was specified, each parsing methods uses it to
      ;; lookup definitions, but only augments `env`; otherwise it
      ;; defaults to using `env` for both lookup and recording
      ;; definitions. This step represents the recursive descent
      ;; portion of the parsing algorithm.
      (multiple-value-bind (iprogn-forms augmented-env)
          (if body-forms
              (parse-iprogn-forms body-forms env body-env)
            (values nil env))

        ;; 7. Return a CLAST-ELEMENT representation of the input
        ;; form and the provided environment augmented only with
        ;; definitions from body forms (i.e. that does not contain
        ;; any information about local bindings).
        (values
         (make-instance 'let*-form
            :source (cons operator params) :top top :type t
            :iprogn-forms iprogn-forms :body-env body-env
            :bindings bindings)
         env)))))

\end{lstlisting}

Il metodo prende in input cinque valori, tre dei quali opzionali. Il parametro
\texttt{OP} rappresenta il parametro rispetto al quale il metodo specializza la
funzione generica \texttt{PARSE-FORM}. Questo in maniera tale da consentire
l'identificazione della particolare regola nella grammatica Common Lisp di cui
il metodo di occupa di compiere il parsing.

Il parametro \texttt{PARAMS} rappresenta invece la lista dei parametri che.
insieme all'operatore, formava la compound form originale che a partire dalla
quale è iniziato l'ultimo passo della discesa ricorsiva operata dal parser.

Il parametro opzionale \texttt{ENCLOSING-FORM} rappresenta invece un puntatore
alla form all'interno della quale la form attualmente in analisi è contenuta.
Questo consente, in fase di inizializzazione del nodo che rappresenta la form
all'interno dell'AST prodotto in output dalla libreria, di specificare un valore
per l'attributo \texttt{TOP} del nodo. Come specificato in precedenza, questo
attributo risulta di fondamentale importanza in quanto rappresenta un arco
all'interno dell'AST e consente di organizzare i nodi, oggetti del modulo di
rappresentazione, all'interno di un albero.

Il parametro \texttt{ENV} rappresenta l'ambiente principale utilizzato dal
metodo, ossia l'oggetto ambiente a partire dal quale verranno aggiunte le
informazioni prodotte dalla valutazione della form e identificate durante
l'attività di analisi. Si tratta di un valore opzionale, in fase di analisi
della radice dell'AST non potrà infatti presente alcun ambiente precedente che
dovrà essere arricchito. In questo caso il comportamento di default è quello di
definire un ambiente lessicale nullo.

Infine, il parametro \texttt{INTERNAL-ENV} rappresenta un ambiente secondario,
il quale viene utilizzato per la ricerca delle informazioni necessarie al
processo di analisi. Questa divisione risulta particolarmente utile all'interno
dell'intero modulo in quanto consente di mantenere l'ambiente di valutazione
della form distinto dall'ambiente che verrà arricchito con le informazioni
prodotte dalla valutazione stessa e ritornato. In molti casi infatti, come ad
esempio nel caso del costrutto \texttt{LET*} analizzato dall'esempio presentato
in questa Sezione, l'ambiente di definizione presenta infatti informazioni
aggiuntive, non accessibili esternamente alla form, e che quindi non dovranno
fare parte dell'ambiente ritornato dal parsing.

La prima operazione svolta dal metodo è legata all'attività di parsing.
\textit{1.} Il valore \texttt{PARAMS}, il quale rappresenta la lista dei
parametri della compound form originale, viene infatti decomposto all’interno di
tre diversi elementi in base a quanto specificato dalla regola della grammatica
del linguaggio associata al costrutto \texttt{LET*}. La regola specifica infatti
che ciascuna form che abbia \texttt{LET*} come operatore prevede:

\begin{itemize}

\item come primo parametro, una lista di bindings;

\item come secondo parametro opzionale, una lista di dichiarazioni;

\item che tutte le form successive rappresentino il corpo della compound form
complessiva, una sequenza di form che dovranno essere eseguite nell’ambiente
prodotto dalla valutazione di bindings e dichiarazioni.

\end{itemize}

La seconda operazione svolta dal metodo è \textit{2.} la generazione di una
copia dell’oggetto ambiente fornito in input, copia che verrà utilizzata in
seguito. Questa copia risulta fondamentale in quanto è necessario che la
libreria mantenga due oggetti ambiente distinti durante le attività di analisi:

\begin{itemize}

\item un ambiente che verrà arricchito con le informazioni relative ai binding
specificati localmente dalla form e all'interno del quale avverrà la valutazione
del corpo di questa,

\item un ambiente all'interno del quale verranno aggiunte dichiarazioni e
definizioni presenti nel corpo della form relativamente a dichiarazioni di
variabili, funzioni, macro e altro, e che verrà ritornato al termine
dell'attività di parsing.

\end{itemize}

La presenza di questi due environment distinti, uno per la scrittura e uno per
la lettura, rappresenta una convenzione comune all’intero modulo e di
particolare importanza e uno dei concetti più complessi alla base del
funzionamento della libreria. Questa convenzione consente infatti di
implementare il parsing di costrutti come \texttt{LET*}, in cui l’environment in
cui avviene la valutazione di una form di questo tipo contiene informazioni
aggiuntive rispetto quello in prodotto in output dalla valutazione stessa.\\

Il metodo opera quindi \textit{3.} un’invocazione del metodo \texttt
{PARSE-BINDING} a partire da ciascun binding. La funzione \texttt
{PARSE-BINDING} è responsabile del parsing delle regole specificate dalla
grammatica in relazione alla sintassi dei binding, ossia la regola \textit
{var | (var [init-form])}, e della creazione un nuovo environment, a partire da
quello fornito in input, contenente la definizione di variabile operata dal
binding in analisi. Ogni invocazione della funzione \texttt{PARSE-BINDING}
ritorna quindi un nuovo oggetto environment che viene utilizzato per aggiornare
l’environment interno al metodo.\\

Il metodo prosegue con \textit{4.} l’analisi delle dichiarazioni specificate
dalla form relativamente alle variabili dichiarate dai binding. Chiaramente,
solo nel caso in cui queste siano presenti. Questa analisi viene operata
procedendo in modo ricorsivo e invocando la funzione \texttt{PARSE} a partire
dalla form di specifica delle dichiarazioni. Una particolarità di questa
operazione è rappresentata dal fatto che, in questo caso, \texttt{5.} la
rappresentazione della form \texttt{DECLARE} viene scartata e viene considerato
solamente l'environment prodotto dalla valutazione di questa come aggiornamento
dell'environment interno al metodo. Questo perchè le informazioni prodotte dalle
dichiarazioni vengono già conservate all'interno dell'environment ritornato.
Sarebbe quindi ridondante e relativamente poco utile tenere traccia della
presenza di dichiarazioni in due modi differenti ma equivalenti.

Il passo successivo è rappresentato dal parsing delle form innestate alla form
\texttt{LET*}, operazione che viene delegata alla funzione \texttt
{PARSE-IPROGN-FORMS}. Questa funzione ha lo scopo di compiere la valutazione
delle form specificate in input utilizzando i due environment specificati in
input e produrre in output due valori, una lista contenente la rappresentazioni
di ciascuna delle form utilizzando le strutture dal modulo di rappresentazione e
un environment aumentato con le definizioni che occorrono all’interno delle
form. Questa funzione rappresenta l'effettiva operazione di discesa ricorsiva
operata dal modulo. Dal punto di vista pratico, la funzione si limita infatti ad
operare una chiamata ricorsiva alla funzione \texttt{PARSE} per ciascuna form
innestata, aggiornando ad ogni passo l'environment di scrittura.\\

Infine, a partire da quanto prodotto dai passi precedenti, il metodo si limita
a \texttt{7.} ritornare i due valori descritti in apertura di questa
Sottosezione e comuni a qualsiasi metodo \texttt{PARSE-FORM}: un'istanza di
\texttt{CLAST-ELEMENT} che riporta tutti dettagli relativi all'utilizzo della
form \texttt{LET*} fornita in input, che rappresenta il nodo radice del
sottoalbero del quale il metodo ha effettuato il parsing, e un secondo valore
rappresentato dall'oggetto environment contenente tutte le informazioni
raccolte durante la discesa ricorsiva attraverso la form operata dal parser.\\

Questa Sottosezione ha presentato, attraverso un esempio, il funzionamento e
l’analisi svolta dalla libreria. Analisi che consente di produrre una
rappresentazione estremamente precisa del codice fornito in input a questa.
Questo conclude la discussione rispetto al modulo di parsing e analisi offerto
dalla libreria CLAST.

% AGGIUNGI UNA PARTE IN CUI DICI CHE A DIFFERENZA DI QUANTO AVVIENE IN ALTRI
% CONTESTI, COME AD ESEMPIO QUELLO DEI COMPILATORI, IN CUI SI HA LA
% GENERAZIONE DI UN PARSE TREE E, SOLO A PARTIRE DA QUESTO, SI HA LA
% GENERAZIONE DELL'AST VERO E PROPRIO, NEL CASO DI CLAST SI HA UNA
% GENERAZIONE DIRETTA DELL'AST. INVENTA UNA MOTIVAZIONE PER CUI QUESTO VIENE
% FATTO. UNA BUONA MOTIVAZIONE INIZIALE POTREBBE ESSERE PERCHÉ DAL PUNTO DI
% VISTA DELL'API RISULTEREBBE SCOMODO AVERE A CHE FARE CON QUESTI DUE STEP E
% LA LIBRERIA CERCA DI SOLLEVARE LO SVILUPPO E LA PROGETTAZIONE DI UNO
% STRUMENTO DI SOURCE CODE ANALYSIS DA TUTTO CIÒ CHE NON RIGUARDA LA FASE DI
% ANALISI VERA E PROPRIO OBIETTIVO DELLO STRUMENTO. QUESTA È UNA MOTIVAZIONE
% PER CUI QUESTO NON SI VEDE DALL'ESTERNO. SPIEGARE PERCHÉ QUESTO NON VIENE
% FATTO NEMMENO INTERNAMENTE NON È ALTRETTANTO SEMPLICE. SI PUÒ FARE LEVA SUL
% FATTO CHE A DIFFERENZA DEL CONTESTO DELLA COMPILAZIONE IN CUI SI VUOLE
% OPERARE ANCHE UNA VALIDAZIONE DEL CODICE SOGGETTO, AL FINE DI IDENTIFICARE
% ERRORI LOGICI O SINTATTICI, QUESTO NON È UNO DEGLI OBIETTIVI DI CLAST.
% CLAST SI LIMITA A PRODURRE L'AST DI UN PROGRAMMA COMMON LISP, SENZA
% EFFETTUARE UNA VALIDAZIONE DI QUESTO. DA QUESTO PUNTO DI VISTA LA LIBRERIA
% POTREBBE PRODURRE IN OUTPUT ANCHE UN AST PER UN PROGRAMMA CHE NON HA SENSO
% DAL PUNTO DI VISTA SINTATTICO. FAI UN ESEMPIO IN CUI MOSTRI CHE UN
% PROGRAMMA CHE CONTIENE UNA CHIAMATA AD UNA FUNZIONE CHE NON È DEFINITA
% VIENE PARSATO SENZA ALCUN PROBLEMA E VIENE PRODOTTO IN OUTPUT UN AST
% PERFETTAMENTE VALIDO. QUESTO PERCHÉ LA VALIDAZIONE E LA COMPILAZIONE NON
% SONO OBIETTIVI DELLO STRUMENTO.
