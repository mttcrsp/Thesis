\section{Ambienti}
\label{environments}

Questa sezione descrive un altro elemento fondamentale alla comprensione del
funzionamento della libreria. Gli ambienti Common Lisp rappresentano infatti il
principale meccanismo per ottenere accesso alle informazioni generate da un
sistema Common Lisp durante il processo di valutazione di una form. Questo
elemento risulta di fondamentale importanza per la libreria CLAST, che si pone
come obiettivo quello di raccogliere ed esporre ad un utente il maggior numero
di informazioni possibili rispetto ad un programma, in quanto semplifica la
memorizzazione e l'introspezione rispetto alle informazioni di un dato programma
Lisp, cercando di fornire un'interfaccia il quanto più possibile uniforme tra le
diverse implementazioni del linguaggio.\\

Formalmente la definizione di ambiente viene fornita a partire dalla definizione
di binding. Un \textit{binding} viene definito dallo standard ANSI Common Lisp
come un'associazione tra un nome e ciò che questo denota. Il linguaggio consente
la definizione di binding all'interno di un ambiente a partire da un particolare
insieme di operatori specifici per operazioni di questo tipo, come ad esempio
\texttt{DEFMACRO} o \texttt{DEFPARAMETER}.

A partire da questa definizione di binding, un \textit{ambiente} viene definito,
sempre dallo standard ANSI, come \enquote{\textit{a set of bindings and other
information used during evaluation (e.g., to associate meanings with names)}}.

Un ambiente raccoglie un insieme di binding all'interno di diversi namespace,
in maniera tale da consentire il l'utilizzo simultaneo di uno stesso nome per
identificare una funzione, una variabile, una macro o altro. Tuttavia, affichè
non si generino ambiguità, il linguaggio consente l'utilizzo di un dato nome
solamente una volta all'interno di un dato namespace.\\

Vengono inoltre specificate diverse specializzazioni del concetto di ambiente,
in particolare di: ambiente globale, ambiente dinamico e ambiente lessicale.

L'ambiente globale, \textit{global environment}, è quella parte di un ambiente
che contiene binding che hanno scope e extent indefinito. Tra i diversi
elementi, l'ambiente globale contiene anche i binding relativi a variabili
dinamiche e costanti, binding relativi a funzioni, macro e operatori speciali,
oltre che informazioni relative alle proclamazioni che occorrono all'interno di
un programma.

Un ambiente dinamico, \textit{dynamic environment}, rappresenta invece quella
parte di ambiente che contiene binding il cui momento di validità si limita alla
durata dell'esecuzione della form all'interno della quale questi sono stati
definiti. Tra gli altri elementi, un ambiente dinamico contiene i binding
relativi a variabili dinamiche, informazioni rispetto ai punti di ritorno
dall'esecuzione definiti utilizzando \texttt{UNWIND-PROTECT}, oltre che
informazioni relative agli handler e restart attivi durante un determinato
momento del processo di valutazione.

Un ambiente lessicale, \textit{lexical environment}, per la valutazione di un
particolare frammento di codice è invece definito come quella parte
dell'ambiente che contiene le definizioni che hanno scope lessicale nel contesto
delle form contenute dal frammento. Tra i diversi elementi registrati da un
ambiente lessicale si trovano binding relativi a variabili lessicali, binding
relativi a funzioni e macro, binding relativi a tag \texttt{GO} e informazioni
relative alle dichiarazioni valide rispetto a variabili, funzioni, macro e altro
contenute dall'ambiente.\\

Le informazioni contenute in un ambiente vengono utilizzate a diversi scopi
anche nel contesto del sistema Common Lisp stesso, tra cui per la ricerca della
funzione di macro-espansione associata ad una particolare macro, per
l'identificazione delle dichiarazioni relative ad una variabile, durante il
processo di ottimizzazione del codice e, più in generale, durante tutto il
processo di valutazione e compilazione delle form che compongono un programma.
Anche dal punto di vista dell'API offerta dalla libreria standard del
linguaggio, l’operatore \texttt{MACROEXPAND}, ad esempio, prende in input un
parametro opzionale che rappresenta l’ambiente all’interno del quale una macro
verrà espansa.
\footnote{http://franz.com/support/documentation/current/doc/environments.htm}\\

Dal punto di vista pratico, un ambiente viene ovviamente rappresentato,
all'interno del linguaggio, mediante uno specifico oggetto. Tuttavia, lo
standard ANSI Common Lisp delega a ciascuna implementazione il compito di
definire la struttura e le caratteristiche di questo.

L'unico aspetto formalizzato dallo standard riguarda il fatto che il valore
\texttt{NIL} possa essere utilizzato, ovunque sia possibile specifica un oggetto
ambiente allo scopo di denotare l'ambiente lessicale vuoto.\\

Questo particolare rende significativamente più difficile l'interazione tra un
programma e oggetti di tipo ambiente. Ciò risulta particolarmente critico nel
contesto della libreria CLAST, e più in generale nel contesto di qualsiasi
sistema per l'analisi di codice sorgente Common Lisp, in quanto questa richiede
un utilizzo particolarmente centrale di oggetti di questo tipo, allo scopo di
tenere traccia delle informazioni relative al contesto di definizione di una
data form e allo stesso tempo richiede la massima portabilità da
un'implementazione all'altra.

\subsection{Standardizzazione ed ambienti}

Come affermato nel corso della precedente sottosezione, lo standard ANSI Common
Lisp non definisce in alcun modo le modalità di rappresentazione ed interazione
con oggetti di tipo ambiente. Tuttavia, diverse implementazioni forniscono un
interfaccia, relativamente comune, che consente la creazione e la manipolazione
di ambienti Common Lisp. Nei prossimi paragrafi si presentano il processo e le
ragioni che hanno portato alla definizione di questa API, strumento a partire
dal quale la libreria CLAST compie gran parte delle proprie analisi e che, per
questa ragione, rappresenta una importante dipendenza del progetto.\\

L’ANSI X3J13 fu il comitato tecnico, nato nel 1986, responsabile per la
definizione dello standard ANSI Common Lisp, pubblicato nel 1994. Gran parte del
lavoro di formalizzazione svolto dal comitato negli anni, si è basata sul
contenuto del libro \enquote{\textit{Common Lisp: the Language}} di G.L. Steele
\cite{steele1984common}, spesso abbreviato nella letteratura con la sigla
\textit{CLtL1}. Prima che cominciassero i lavori del comitato infatti, il libro
pubblicato nel 1984 ha per anni rappresentato uno standard de facto per le
diverse implementazioni del linguaggio. Ad oggi, la più recente riedizione del
libro \cite{steele1990common}, datata 1990, rappresenta ancora uno dei
riferimenti di maggiore importanza per la comunità, tanto che anche lo lo
standard definitivo suggerisce agli implementatori del linguaggio l’utilizzo dei
simboli \texttt{:ctlt1} e \texttt{:ctlt2} per consentire l’interoperabilità tra
implementazioni ANSI Common Lisp e altri dialetti che fanno invece maggiore
riferimento alle diverse edizioni del libro.\\

Rispetto alla seconda edizione del libro, il documento di standardizzazione
definitivo riporta delle significative variazioni, sia in termini additivi e che
sottrattivi. Data la rilevanza che il libro ha all’interno della comunità Common
Lisp, la gran parte delle implementazioni del linguaggio più diffuse supportano
l’API descritta dal libro. Tuttavia, la mancanza di una formalizzazione
all’interno dello standard ha portato all’introduzione di variazioni e
peculiarità tipiche di ciascuna implementazione, variazioni che complicano un
lavoro che voglia mantenere una portabilità da implementazione a
implementazione.\\

Tra gli elementi non approvati dallo standard ANSI per l’aggiunta al linguaggio
ma presenti all’interno di \textit{CLtL2} e per questa ragione supportati dalle
principali implementazioni del Common Lisp, si trova proprio la definizione di
oggetti ambiente Common Lisp e la proposta di un'API specifica per l’interazione
con  oggetti di questo tipo, uno degli elementi più importanti per la creazione
della libreria CLAST.\\

Le funzioni che compongono questa API sono quindi l'unico strumento, comune alle
principali implementazioni, a partire dal quale è possibile creare e manipolare
ambienti Common Lisp e quindi a partire dal quale la libreria CLAST è in grado
di svolgere il proprio lavoro di analisi di un programma. Per questa ragione,
nella prossima sezione, se ne presenta un approfondimento.

\subsection{Environments API}

Uno degli aspetti comuni alle diverse implementazioni dello standard ANSI Common
Lisp è rappresentato dal fatto che un utente tipicamente non è in grado di
accedere o compiere modifiche direttamente ad un oggetto ambiente in alcun modo.
Per questa ragione, molte implementazioni utilizzano degli oggetti immutabili
come meccanismo di rappresentazione di un ambiente. Questo aspetto risulta
perfettamente in linea con l’API descritta da CLtL2, la quale descrive un
insieme di funzioni che consentono l’accesso ai contenuti memorizzati da un
oggetto di tipo ambiente e funzioni pure per l’aggiunta controllata di
informazioni a questo.\\

Le funzioni che compongono la citata API, alla quale nel seguito di questa tesi
si farà riferimento con il nome \enquote{\textit{Environments API}}, sono sette.
In particolare, CLtL2 descrive un insieme minimo di funzioni quattro funzioni
che consentono l’interazione con oggetti environment: tre funzioni che
consentono l’accesso alle informazioni presenti all’interno di un oggetto
environment, \texttt{VARIABLE-INFORMATION}, \texttt{FUNCTION-INFORMATION} e
\texttt{DECLARATION-INFORMATION}, e un costruttore
\texttt{AUGMENT-ENVIRONMENT}. Le altre tre funzioni che compongono l’interfaccia
suggerita dal libro sono chiamate \texttt{PARSE-MACRO}, \texttt{ENCLOSE} e
\texttt{DEFINE-DECLARATION}. In seguito viene presentata un breve descrizione
relativa al funzionamento di ciascuna delle funzioni indicate, in maniera tale
da consentire al lettore di comprendere il supporto che tipicamente viene
offerto da ciascuna implementazione del linguaggio per la creazione di in grado
di compiere un analisi di programmi Common Lisp, come libreria soggetto di
questa tesi.\\

Le funzioni \texttt{VARIABLE-INFORMATION}, \texttt{FUNCTION-INFORMATION} e
\texttt{DECLARATION-INFORMATION} forniscono accesso alle informazioni relative
alle dichiarazioni attualmente presenti all’interno di un environment,
memorizzate all’interno dell’oggetto che rappresenta questo attraverso
l’utilizzo della funzione \texttt{AUGMENT-ENVIRONMENT}, o automaticamente
aggiunte da interprete e compilatore del linguaggio. Ciascuna di queste funzioni
può essere eseguita fornendo in input un parametro environment opzionale. Nel
caso in cui questo non venga specificato infatti, la funzione utilizzerà
semplicemente l’environment lessicale vuoto come valore di default, convenzione
precedentemente citata e suggerita dalla standard ANSI. In seguito viene
presentato il funzionamento di ciascuna di queste funzioni ad un livello di
dettaglio maggiore.

\subsubsection{Funzioni di base}

La funzione \texttt{VARIABLE-INFORMATION} ritorna le informazioni riguardanti
l’interpretazione del simbolo fornito in input come simbolo associato ad una
variabile, ossia le informazioni associate al binding che utilizza quello
specifico simbolo come nome, operando una ricerca all’interno dell'ambiente
lessicale fornite in input. Queste informazioni sono rappresentate da un valore
che indica il tipo binding, presente per la variabile all’interno dell’ambiente
(speciale, lessicale, costante, \dots), un valore che indica se sia stato
effettivamente trovato o meno un binding relativo al simbolo ed un valore che
riporta informazioni aggiuntive rispetto alla variabile in analisi, come ad
esempio il tipo e se questa sia stata dichiarata utilizzando
\texttt{DYNAMIC-EXTENT} o \texttt{IGNORE}.\\

La funzione \texttt{FUNCTION-INFORMATION} ritorna informazioni relative
all’interpretazione del simbolo fornito in input come nome di funzione, nel caso
in cui questo appaia nel contesto di un binding relativo ad una funzione
all’interno dell’ambiente in input. Anche questa funzione ritorna tre valori: un
primo valore utilizzato per indicare il tipo di definizione o binding presente
relativamente simbolo all’interno dell’ambiente (funzione, macro, form speciale
o assente), un secondo valore che specifica se la funzione sia locale o globale
ed un terzo valore che specifica informazioni aggiuntive rispetto alla funzione,
identificate dal sistema Common Lisp nell’ambiente, come ad esempio se questa
sia dichiarata utilizzando \texttt{DYNAMIC-EXTENT}, se sia stato richiesto
l’inling o meno della funzione ed il tipo di questa.\\

La funzione \texttt{DECLARATION-INFORMATION} ritorna le informazioni associate
alle dichiarazioni che utilizzano come nome il simbolo fornito in input presenti
all’interno dell'ambiente in input. Questa funzione consente di analizzare tutte
le definizioni relative ad elementi diversi da variabili e funzioni che possono
essere presenti all’interno di un ambiente.\\

La funzione \texttt{AUGMENT-ENVIRONMENT} rappresenta lo strumento fondamentale
per l’analisi necessaria alla libreria CLAST. \texttt{AUGMENT-ENVIRONMENT}
consente sia di produrre un nuovo oggetto environment, sia di aggiungere
informazioni ad un'istanza di environment già esistente. La funzione lavora a
partire da un input rappresentato da un environment opzionale, presente nel
caso in cui si desideri aggiungere informazioni ad un environment piuttosto
che crearne uno nuovo, e da una o più liste di definizioni di variabili,
funzioni, macro e dichiarazioni. La funzione produce quindi in output un
oggetto environment che raccoglie le informazioni contenute dall'environment
fornito in input, nel caso in cui questo sia stato utilizzato un parametro di
questo tipo in fase di invocazione, e le informazioni generate dall'analisi
delle dichiarazioni specificate.\\

\subsubsection{Macro accessorie}

A differenza delle funzioni appena presentate, che hanno come obiettivo quello
di consentire ad un utente l'interazione con un oggetto ambiente, in questo
paragrafo vengono presentate una serie di macro, descritte dall'Environments
API. Macro che hanno il particolare obiettivo di fornire un'implementazione di
alcune delle funzionalità fondamentali per la creazione di programmi in grado di
analizzare programmi Common Lisp e per l'estensione dell'interfaccia composta
dalle funzioni descritte nel corso della precedente sottosezione.\\

\texttt{DEFINE-DECLARATION} é una macro che consente di estendere e introdurre
un supporto a nuove tipologie di dichiarazione per la memorizzazione attraverso
l'utilizzo della funzione \texttt{AUGMENT-ENVIRONMENT}; operazione che viene
compiuta ad esempio dall’implementazione Allegro Common Lisp, la quale introduce
due nuove ulteriori definizioni relativamente ai costrutti \texttt{BLOCK} e
\texttt{TAG}. In generale, questa funzione risulta utile soprattutto agli
implementatori del linguaggio stesso, o a quegli utenti che desiderano estendere
il linguaggio attraverso l'introduzione nuovi costrutti.\\

La funzione \texttt{PARSE-MACRO} consente di destrutturare la lista di parametri
di una macro, un compito molto spesso necessario e generalmente realizzabile a
partire dalla macro \texttt{DESTRUCTURING-BIND}, la quale consente utilizzare
gli elementi di una lista per inizializzare un insieme di variabili. L’autore
G.L. Steele indica come ragione per cui si è scelto di aggiungere questa macro
il fatto che qualsiasi programma desideri analizzare del codice Lisp dovrà molto
probabilmente definire una funzione analoga e, nonostante la presenza di \texttt
{DESTRUCTURING-BIND}, l’implementazione di questa funzionalità non è del tutto
banale. \cite{steele1990common} Dal punto di vista pratico,
\texttt{PARSE-MACRO} è strutturata in maniera del tutto analoga ad una
\texttt{DEFMACRO}, accetta infatti i medesimi parametri nel medesimo ordine. A
livello di output invece ritorna una lambda-expression che prende in input due
valori, una form ed un oggetto ambiente, e che potrà essere utilizzata per
l'aggiunta di informazioni ad un ambiente fornito in input alla funzione
lambda.\\

Infine, la funzione \texttt{ENCLOSE} consente di espandere funzioni definite
all’interno di un ambiente lessicale e consente quindi l’analisi, da parte di un
programma Common Lisp, di un programma che fa utilizzo di una form del tipo
\texttt{(eval-when (:compile-toplevel) …)} attraverso l’esecuzione all’interno
dell’ambiente che contiene questa. Dal punto di vista pratico, \texttt{ENCLOSE}
è una funzione che lavora a partire da un input rappresentato da una coppia
formata da una lambda-expression e da un oggetto ambiente e ritorna un oggetto
di tipo funzione, equivalente a quello che si otterrebbe valutando la form
\texttt{`(function ,lambda-expression)} nell'ambiente sintattico fornito in
input.
