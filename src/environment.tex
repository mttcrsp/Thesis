\section{Environments}

Un \textit{environment} Common Lisp è definito come un oggetto che raccoglie
un insieme di binding, e più generalmente informazioni riguardo a variabili,
funzioni, simboli e altri elementi utilizzati da un programma Lisp. Le
informazioni contenute in un environment possono essere utili a diversi scopi
tra cui espansione di macro e, più in generale, valutazione e compilazione di
form. L’operatore macroexpand, ad esempio, prende in input un parametro
opzionale che rappresenta l’environment all’interno del quale una macro verrà
e spansa.\footnote{http://franz.com/support/documentation/current/doc/environments.htm}\\

La definizione riportata dal paragrafo precedente fa riferimento alla
documentazione di una delle più note ed impiegate implementazioni dello
standard Common Lisp. Si è scelto di riportare questa particolare definizione
in quanto risulta essere sufficientemente generica ed articolata da riassumere
e approssimare in modo ragionevole il funzionamento generale degli
environments a prescindere dalla particolare implementazione. Purtroppo, non è
possibile citare una definizione standard di environment, in quanto la
standard Common Lisp non tratta in alcun modo questa tematica. In seguito
vengono spiegate le motivazioni di questa lacuna.\\

L’ANSI X3J13 è il comitato tecnico, nato nel 1986, responsabile per la
definizione dello standard ANSI Common Lisp. La formalizzazione operata del
comitato si è basata in gran parte sul contenuto del libro “Common Lisp: the
Language” di G.L. Steele \cite{steele1984common}, spesso abbreviato nella
comunità Lisp con la sigla \textit{CLtL1}. Prima che cominciassero i lavori
del comitato, il libro, pubblicato nel 1984, ha infatti per anni rappresentato
uno standard \textit{de facto} per le diverse implementazioni del linguaggio.
Ad oggi, la più recente ri-edizione del libro \cite{steele1990common}, datata
1990, rappresenta uno dei riferimenti di maggiore importanza per la comunità,
tanto che la versione definitiva dello standard, pubblicata nel 1994,
attribuisce tanta importanza al libro da suggerire agli implementatori
l’utilizzo dei simboli \texttt{:ctlt1} e \texttt{:ctlt2} per consentire
l’interoperabilità tra implementazioni ANSI Common Lisp e altri dialetti che
fanno invece maggiore riferimento alle diverse edizioni del libro.

Rispetto alla seconda edizione del libro, lo standard definitivo riporta delle
significative variazioni, sia in termini additivi e che sottrattivi. Data la
rilevanza che il libro ha all’interno della comunità Common Lisp, la gran
parte delle implementazioni del linguaggio più diffuse supportano l’API
descritta dal libro. Tuttavia, la mancanza di una formalizzazione all’interno
dello standard ha portato all’introduzione di variazioni e peculiarità tipiche
di ciascuna implementazione, variazioni che complicano un lavoro che voglia
mantenere una portabilità da implementazione a implementazione e che verranno
approfondite nelle sezioni a seguire.\\

Tra gli elementi non approvati dallo standard ANSI per l’aggiunta al
linguaggio ma presenti all’interno di \textit{CLtL2}, si trova proprio la
definizione di environment Common Lisp e la proposta di un'API per
l’interazione con questo tipo di oggetto, uno degli elementi più importanti
per la creazione della libreria CLAST. Nei prossimi paragrafi viene presentato
il concetto di environment in Common Lisp a partire da quanto descritto da
CLtL2 e dalla documentazione disponibile in accompagnamento alle diverse
implementazioni del linguaggio.\\

ESSENDO CHE OGNI TANTO SI DICE CHE LO STANDARD NON FA ALCUN RIFERIMENTO AGLI
ENVIRONMENT, IN ALTRI CONTESTI PERÒ SE NE PARLA E SI DICE CHE UN UTENTE NON
DEVE ESSERE IN GRADO DI ACCEDERE DIRETTAMENTE.

\subsection{Environments API}

Uno degli aspetti formalizzati dallo standard ANSI Common Lisp rispetto agli
environment specifica che un utente del linguaggio non debba essere in grado
in alcun modo di accedere o modificare direttamente un oggetto environment.
Per questa ragione, molte implementazioni utilizzano degli oggetti immutabili
come meccanismo di rappresentazione di un environment. Questo aspetto risulta
perfettamente in linea con l’API descritta da CLtL2, la quale descrive un
insieme di funzioni che consentono l’accesso ai contenuti memorizzati
dall’environment e una funzione per l’aggiunta controllata di informazioni a
questo.

Le funzioni che compongono l’API appena citata sono sette, in particolare,
CLtL2 descrive come insieme minimo di funzioni quattro funzioni che consentono
l’interazione con oggetti environment: tre funzioni che consentono l’accesso
alle informazioni presenti all’interno di un oggetto environment, \texttt
{VARIABLE-INFORMATION}, \texttt{FUNCTION-INFORMATION} e \texttt
{DECLARATION-INFORMATION}, e un costruttore \texttt{AUGMENT-ENVIRONMENT}. Le
altre tre funzioni che compongono l’interfaccia suggerita dal libro sono
chiamate \texttt {PARSE-MACRO}, \texttt{ENCLOSE} e \texttt{DEFINE-
DECLARATION}. In seguito viene presentata un breve descrizione relativa al
funzionamento di ciascuna delle funzioni indicate, in maniera tale da
consentire al lettore di meglio comprendere il supporto tipicamente offerto
dal linguaggio alla creazione delle libreria soggetto di questa tesi.\\

Le funzioni \texttt{VARIABLE-INFORMATION}, \texttt{FUNCTION-INFORMATION} e
\texttt{DECLARATION-INFORMATION} forniscono l’accesso alle informazioni
relative alle dichiarazioni attualmente presenti all’interno di un
environment, memorizzate all’interno dell’oggetto che rappresenta questo
attraverso l’utilizzo della funzione \texttt{AUGMENT-ENVIRONMENT}, o anche
automaticamente aggiunte dall’interprete o dal compilatore del linguaggio.
Ciascuna di queste funzioni può essere eseguita fornendo in input un parametro
environment opzionale. Nel caso in cui questo non non venga specificato
infatti, la funzione utilizzerà semplicemente l’environment lessicale vuoto
come valore di default. In seguito viene presentato il funzionamento di
ciascuna di queste funzioni ad un livello di dettaglio maggiore.

\subsubsection{Funzioni di base}

\texttt{VARIABLE-INFORMATION} ritorna le informazioni riguardanti
l’interpretazione del simbolo fornito in input come simbolo associato ad una
variabile presente all’interno dell'environment lessicale in input. Queste
informazioni sono rappresentate da un valore che indica il tipo della
definizione, o del binding, presente per la variabile all’interno
dell’environment (speciale, lessicale, costante, \dots), un valore che indica
se sia stato trovato o meno un binding per la variabile ed un valore che
riporta informazioni aggiuntive rispetto alla variabile in analisi, come ad
esempio il tipo e se questa sia stata dichiarata utilizzando \texttt
{DYNAMIC-EXTENT} o \texttt{IGNORE}.\\

\texttt{FUNCTION-INFORMATION} ritorna informazioni relative
all’interpretazione della simbolo fornito in input come nome di funzione, nel
caso in cui questo appaia come funzione all’interno dell’environment in input.
Anche questa funzione ritorna tre valori: un primo valore utilizzato per
indicare il tipo di definizione o binding presente per la funzione all’interno
dell’environment (funzione, macro, form speciale o assente, un secondo valore
che specifica se la funzione sia locale o globale ed un terzo valore che
specifica informazioni aggiuntive rispetto alla funzione, identificate da Lisp
all’interno dell’ambiente, come ad esempio se questa sia dichiarata
utilizzando \texttt {DYNAMIC-EXTENT}, se sia stato richiesto l’inling o meno
della funzione e il tipo di questa.\\

\texttt{DECLARATION-INFORMATION} ritorna le informazioni relative ad una
dichiarazione che utilizza come nome il simbolo fornito in input presenti
all’interno del environment in input. Questa funzione consente di analizzare
tutte le definizioni relative ad elementi diversi da variabili e funzioni che
possono essere presenti all’interno di un environment.\\

La funzione \texttt{AUGMENT-ENVIRONMENT} rappresenta lo strumento fondamentale
per l’analisi necessaria alla libreria CLAST. \texttt{AUGMENT-ENVIRONMENT}
consente di produrre un nuovo oggetto environment a partire da un input
rappresentato da un environment opzionale, a partire dal quale si desidera che
venga generato il nuovo oggetto, e da una o più liste di definizioni di
variabili, funzioni, macro e dichiariazioni. La funzione produrrà quindi in
output un oggetto environment che raccoglierà le informazioni contenute
dall'environment fornito in input, nel caso in cui questo sia stato utilizzato
un parametro di questo tipo in fase di invocazione, e le informazioni generate
della nuove dichiarazioni specificate.\\

La funzione \texttt{AUGMENT-ENVIRONMENT} rappresenta lo strumento fondamentale
per l’analisi necessaria alla libreria CLAST. \texttt{AUGMENT-ENVIRONMENT}
consente sia di produrre un nuovo oggetto environment, sia di aggiungere
informazioni ad un'instanza di environment già esistente. La funzione lavora a
partire da un input rappresentato da un environment opzionale, presente nel
caso in cui si desideri aggiungere informazioni ad un environment piuttosto
che crearne uno nuovo, e da una o più liste di definizioni di variabili,
funzioni, macro e dichiariazioni. La funzione produce quindi in output un
oggetto environment che raccoglie le informazioni contenute dall'environment
fornito in input, nel caso in cui questo sia stato utilizzato un parametro di
questo tipo in fase di invocazione, e le informazioni generate dall'analisi
delle dichiarazioni specificate.\\

\subsubsection{Macro accessorie}

A differenza delle funzioni appena presentate, che hanno come obiettivo quello
di consentire ad un utente l'interazione con un oggetto environment, in questo
paragrafo vengono presentate delle macro che hanno il particolare obiettivo di
fornire un'implementazione di alcune delle funzionalità fondamentali per la
creazione di programmi in grado di analizzare programmi Lisp e per
l'estensione dell'API offerta dalle funzioni descritte dal paragrafo
precedente.\\

\texttt{DEFINE-DECLARATION} é una macro che consente di estendere e introdurre
un supporto a nuove tipologie di dichiarazione per la memorizzazione
attraverso \texttt {AUGMENT-ENVIRONMENT}; operazione che viene compiuta ad
esempio dall’implementazione Allegro Common Lisp, la quale introduce due nuove
ulteriori definizioni \texttt{BLOCK} e \texttt{TAG}. In generale, questa
funzione risulta utile soprattutto internamente agli implementatori del
linguaggio, o a quegli utenti che desiderano estendere il linguaggio con nuovi
costrutti.\\

La funzione \texttt{PARSE-MACRO} consente di destrutturare i parametri di una
macro, un compito molto spesso necessario e generalmente realizzabile a
partire dalla macro \texttt{DESTRUCTURING-BIND}, la quale consente utilizzare
gli elementi di una lista per inizializzare un insieme di variabili. L’autore
indica come ragione per cui si è scelto di aggiungere questa funzione il fatto
che qualsiasi programma desideri analizzare del codice Lisp dovrà molto
probabilmente definire una funzione analoga e, nonostante la presenza di
\texttt{DESTRUCTURING-BIND}, l’implementazione di questa funzionalità non è
del tutto banale. \cite{steele1990common} Dal punto di vista pratico, \texttt
{PARSE-MACRO} è strutturata in maniera del tutto analoga ad una
\texttt{DEFMACRO}, accetta infatti i medesimi parametri nel medesimo ordine, a
livello di output invece ritorna una lambda- expression che prende in input
due valori, una form ed un environment, e che potrà essere utilizzata per
aggiungere informazioni ad un ambiente fornito in input alla funzione
lambda.\\

Infine, la funzione \texttt{ENCLOSE} consente di espandere funzioni definite
all’interno di un lexical environment e consente quindi l’analisi, da parte di
un programma Lisp, di un programma Lisp che fa utilizzo di una form del tipo
\texttt{(eval-when (:compile-toplevel) …)} attraverso l’esecuzione all’interno
dell’environment che contiene questa. Dal punto di vista pratico,
\texttt{ENCLOSE} è una funzione che lavora a partire da un input,
rappresentato da una coppia lambda-expression ed environment, e ritorna un
oggetto di tipo funzione equivalente a quello che si otterebbe valutando la
form \texttt{`(function ,lambda-expression)} nell'environment sintattico
fornito in input. Affinchè la valutazione possa avere successo è necessario
che la form faccia riferimento solamente a simboli definiti all'interno
dell'environment fornito in input assieme ad essa.
