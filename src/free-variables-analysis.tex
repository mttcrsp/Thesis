\section{Strumenti di linting}
\label{lint}

Lo scopo di questa sezione è quello di presentare come la libreria CLAST
consenta e faciliti la costruzione di strumenti di source code analysis
relativamente a programmi scritti utilizzando il linguaggio Common Lisp.

La Sezione \ref{sca-architecture} ha presentato come la libreria possa
integrarsi all'interno dell'architettura di uno strumento di source code
analysis, andando a fornire due delle tre componenti fondamentali per la
creazione di uno strumento di questo tipo, un parser e il generatore di una
rappresentazione del programma che consenta lo svolgimento di analisi,
componenti comuni ad ogni strumento di source code analysis per il particolare
linguaggio target dell'analisi.

L'avere a disposizione queste due componenti off-the-shelf rende la creazione di
uno strumento di source analysis significativamente più semplice, consentendo ad
un utente di potersi focalizzare sulla creazione del solo analizzatore,
responsabile per lo studio delle proprietà di interesse.

Questa sezione presenta quindi la definzione di free variables analysis,
un'analisi mirata all'indentificazione di variabili libere all'interno di un
frammento di codice, e mostra come un'analisi di questo tipo possa essere
operata in modo estremamente semplice a partire dalla libreria. Si discute
quindi come la capacità di svolgere questa analisi possa portare alla
costruzione di uno strumento di analisi statica che consenta di prevenire
un'intera categoria di errori nel contesto dell'utilizzo del linguaggio Common
Lisp.

La sezione si conclude presentando come, la possibilità di identificare
l'utilizzo di particolari costrutti o la presenza di dati elementi all'interno
di un programma consente la creazione di una particolare categoria di strumenti,
chiamati linter, per il linguaggio Common Lisp. Strumenti attualmente non
disponibili per il supporto alla programmazione utilizzando questo linguaggio ma
particolarmente diffusi ed utilizzati con ottimi risultati nel contesto di altri
linguaggi di programmazione. \cite{DBLP:conf/sac/PotocnikCS14}

\subsection{Free Variables Analysis}
\label{free-variables-analysis}

Formalmente, una variabile libera è una notazione che specifica una posizione
all'interno di un'espressione in cui una sostituzione potrebbe avere luogo.
L'idea è collegata al concetto di placeholder, un simbolo che verrà rimpiazzato
in seguito da un certo valore, o di carattere wildcard, ossia un simbolo che
rappresenta un qualsiasi simbolo, aspetto che verrà approfondito maggiormente
dalla Sezione \ref{pattern-matching}, relativamente al concetto di pattern
matching.

In particolare, nel contesto della programmazione e dei linguaggi di
programmazione, il termine variabile libera viene utilizzato per riferirsi a
variabili che vengono utilizzate all'interno di una funzione senza che queste
siano definite all'interno di questa o specificate come parametro. In questo
ambito, il termine risulta essere molto spesso sinonimo di variabile non
locale.

Per contrasto, viene definita variabile occupata una variabile che è
precedemente stata libera ma è attualmente è associata ad uno specifico valore o
insieme di valori.

Dal punto di vista logico, il valore utilizzato come nome di una particolare
variabile non risulta di particolare importanza. Tuttavia, il riutilizzo dello
stesso nome associato ad una variabile attualmente occupata potrebbe risultare
contraddittorio o, più generalmente, generare confusione. Per questa ragione,
nel momento in cui una variabile libera viene occupata, in un linguaggio di
programmazione tradizionale come anche nel contesto del linguaggio Common Lisp,
il nome associato a questa viene ritirato dallo spazio dei nomi validi per
l'associazione ad una qualsiasi variabile.

L'analisi di un frammento di codice mirata ad identificare la presenza e
l'utilizzo di variabili libere all'interno di un dato frammento di codice viene
definita \textit{free variables analysis}.

I prossimi paragrafi presentano come la libreria CLAST sia in grado di agire
come fondamento per lo svolgimento di free variables analysis.

\subsubsection{Free Variables Analysis e CLAST}

Nel corso della Sezione \ref{traversal}, si è mostrato come la libreria CLAST
consenta ad un utente di operare una navigazione all'interno della
rappresentazione offerta da questa di un programma Common Lisp. In particolare,
nella Sottosezione \ref{walk}, si è mostrato come questa operazione possa essere
estesa modo molto semplice allo scopo di ricercare e raccogliere informazioni da
un particolare frammento di codice. In seguito si presenta quindi come questa
potenzialità della libreria possa essere sfruttata allo scopo di svolgere free
variable analysis.

La libreria modella l'utilizzo di una variabile libera all'interno di una data
form utilizzando la classe \texttt{FREE-VARIABLE-REF}. Il diagramma in Figura
\ref{fig:free-variable-ref} mostra la porzione del modulo di rappresentazione
interessata dalla definizione di questa classe.

\image{img/free-variable-ref.png}
      {Diagramma della classi per la rappresentazione di variabili libere}
      {fig:free-variable-ref}
      {0.5}

Dal punto di vista del processo di valutazione, si ha un'occorrenza di una
variabile libera nel caso in cui, una volta che il sistema Common Lisp incontra
una symbol form e che questo identifica la symbol form come l'utilizzo di una
variabile, una ricerca all'interno dell'ambiente lessicale attivo non produce
alcun risultato relativamente al simbolo.

Come illustrato nel corso della Sottosezione \ref{parsing}, la libreria CLAST
esegue le attività di parsing e analisi in modo del tutto coerente al processo
di valutazione e, in particolare, anche relativamente ai passaggi appena
descritti.

% AGGIUNGI FLOWCHART DEL PROCESSO DI IDENTIFICAZIONE DI UNA VARIABILE LIBERA

Questo significa che, analizzando l'AST prodotto dalla libreria allo scopo di
rappresentare un programma, è sufficiente ricercare la presenza di nodi di tipo
\texttt{FREE-VARIABLE-REF} in maniera tale da identificare l'utilizzo e la
presenza di variabili libere all'interno di un programma.

Per svolgere questa ricerca la libreria offre la funzione \texttt{WALK},
esportata dal modulo di traversal e presentata nel corso della Sezione
\ref{traversal}. Questa funzione consente l'esecuzione del traversal di un AST
prodotto dalla libreria secondo lo schema map-reduce, dando quindi la
possibilità ad un utente di definire operazioni di ricerca e raccolta di
informazioni all'interno dell'AST.

Il Listato \ref{lst:free-variables} mostra quindi come la funzione \texttt{WALK}
possa essere estesa, in maniera particolarmente semplice, allo scopo di svolgere
free variables analysis su di un particolare frammento di codice.

\begin{lstlisting}[
  caption=Definizione di una funzione per la ricerca di variabili libere all'interno di una form,
  label={lst:free-variables}
]

(defun free-variables (form)
  "Returns all the 'free' variables present in FORM.

Arguments and Values:

form   : a CLAST-ELEMENT.
result : a LIST of SYMBOLS."
  (walk form
        :map (lambda (e)
               (typecase e
                 (free-variable-ref (list (form-symbol e)))))
        ))

\end{lstlisting}

La funzione \texttt{FREE-VARIABLES} opera una semplice specializzazione della
funzione \texttt{WALK}. Come descritto dalla Sottosezione \ref{walk},
\texttt{WALK} prende in input due funzioni che vengono eseguite secondo uno
schema map-reduce. \texttt{FREE-VARIABLES} specifica quindi una funzione che
implementa la componente di mapping della funzione. In particolare, la funzione
specificata viene applicata ad ogni nodo e ritorna il nodo stesso, nel caso in
cui il nodo in input sia di tipo \texttt{FREE-VARIABLE}, e il valore
\texttt{NIL} in qualsiasi altro. Non specificando un valore per la componente di
reducing della funzione \texttt{WALK}, la funzione \texttt{FREE-VARIABLES}
sfrutta del comportamento di default di questa, ossia quello di ritornare la
lista degli elementi prodotti in output da ciascuna applicazione della funzione
di mapping.

La Free variable analysis è quindi una delle tipologie di analisi che viene
significativamente semplificato dalla possibilità di avere a disposizione una
libreria per la generazione di AST come CLAST.

Dal punto di vista della programmazione, la disponibilità di uno strumento di
questo tipo risulta molto significativa. In particolare, nel contesto di un
linguaggio non tipizzato e fortemente dinamico come il Common Lisp, questo la
possibilità di definire strumenti come questo, e come potenzialmente molti altri
strumenti di analisi statica, porta alla produzione di codice più sicuro e alla
prevenzione di diversi potenziali errori a runtime, come ad esempio gli errori
\texttt{UNBOUND-VARIABLE} generati dall'accesso ad una variabile libera.

La prossima sezione presenta alcuni esempi di strumenti source di code analysis
abilitati dalla presenza funzionalità appena citata sono rappresentati da linter
per il linguaggio di programmazione Common Lisp.

\subsection{Linting}

Un linter è uno strumento che ha come scopo quello di segnalare ad uno
sviluppatore potenziali errori ed inconsistenze all'interno del codice sorgente
di un programma. Questa categoria di strumenti prende il proprio nome da un
strumento di questo tipo specifico per il linguaggio C, chiamato Lint, e
disponibile come parte del sistema operativo Unix. \cite{johnson1977lint}

Generalmente, viene definito \textit{strumento di linting} un qualsiasi
programma in grado di svolgere un'attività di analisi statica del codice
sorgente di un programma alla ricerca di potenziali inconsistenze a livello
funzionamento, potenziali errori nell'utilizzo dei costrutti offerti dal
linguaggio di programmazione o frammenti di codice non aderente alle particolare
linee guida stilistiche utilizzate dal team di sviluppo, producendo a partire da
quanto identificato segnalazioni di errori o warning, potenzialmente già in fase
di scrittura del codice.

Esempi di potenziali errori sono rappresentati dall'utilizzo di variabili
libere, descritto dalla precedente sottosezione, divisioni per zero, l'utilizzo
di condizioni costanti all'interno di istruzioni di branching e potenziali
overflow aritmetici. A differenza delle ottimizzazioni e delle segnalazioni di
errori che vengono tipicamente operate da un compilatore. Gli strumenti di
linting si riferiscono molto spesso ad utilizzi del linguaggio consentiti da un
compilatore e dalla sintassi del linguaggio ma spesso sono indice di
imprecisioni da parte di uno sviluppatore. Inoltre, in alcuni casi, strumenti di
questo tipo fanno riferimento ad livello di superiore un livello superiore
rispetto ad un compilatore, andando a segnalare potenziali problemi di
portabilità di un programma nel passaggio dall'utilizzo di un compilatore ad un
altro.

La verifica compiuta da un linter è quindi tipicamente più stringente rispetto a
quella operata da un compilatore e consente di evidenziare l'utilizzo di bad
practice nello sviluppo, di costrutti che portano ad uno spreco di risorse, o in
generale di operazioni che sono tipicamente problematiche.

È importante sottolineare il fatto che molti degli errori che risulta impossibile
identificare a livello di compilazione risultano impossibili da identificare
anche a livello di linting. Problemi di questo tipo sono rappresentati
da tutti quei problemi che richiedono l'utilizzo di tecniche di analisi
dinamica. Questo rappresenta il limite più significativo delle analisi che è
possibili svolgere a partire dalla libreria CLAST e in particolare a partire da
qualsiasi strumento per la source code analysis. Non risulta infatti possibile
identificare tutti quegli errori che possono essere identificati solamente in
seguito all'esecuzione del codice.

Inoltre, l'approccio utilizzato da strumenti di linting è quello accettare un
compromesso e cerca di segnalare errori rispetto alla cui presenza si ha più
confidenza. Ad esempio, un linter a cui viene richiesto di verificare l'assenza
di chiamate ad una particolare funzione potrebbe lavorare assumendo che in
assenza di occorrenze di una chiamata alla funzione questa non possa essere mai
chiamata, mentre nel aso in cui sia presente almeno un'occorrenza che questa
possa essere effettivamente invocata. Si tratta di un approccio ragionevole ma
che produce una quantità significativa di falsi positivi. Per questa ragione i
linter sono tipicamente strumenti con un alto grado di configurabilità.

Alla scrittura di questa tesi, non è disponibile alcuno strumento di linting
compatibile con codice Common Lisp. Questo a causa dell'assenza di strumenti che consentano di operare un'analisi del codice sorgente di programmi Common Lisp.

La libreria CLAST rappresenta quindi infrastruttura particolarmente abilitante
da questo punto di vista, consentendo possibilità di definire strumenti di
questo tipo, come mostrato  questa sezione nel particolare caso della free
variable analysis.

La libreria rende infatti immediata la ricerca di un qualsiasi specifico
elemento all'interno di un programma Common Lisp. Aspetto che a sua volta
consente la definizione di strumenti di linting che compiano un monitoraggio e
una validazione del codice, potenzialmente già nel momento in cui questo viene
scritto, andando semplicemente verificare particolari condizioni in seguito ad
ogni modifica ad un file, andando a verificare, ad esempio, che un predicato
simile a \texttt{FREE-VARIABLES} ritorni sempre una lista vuota.
