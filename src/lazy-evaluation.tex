\section{Supporto alla Lazy Evaluation}
\label{lazy-evaluation}

All'interno di questa sezione viene presentato un terzo caso di studio relativo
alle principali applicazioni della libreria CLAST. Questa sezione mostra quindi
come la libreria possa essere utilizzata per definire una seconda estensione al
linguaggio di programmazione Common Lisp. In particolare, un'estensione che
consenta ad un utente del linguaggio di utilizzare un modello di valutazione
lazy, al contrario del tradizionale modello adottato dal linguaggio, detto di
tipo eager o strict.\\

A questo scopo viene dapprima presentato il modello di valutazione lazy,
discutendo le principali applicazioni, attraverso un esempio, e riportando
vantaggi e svantaggi legati all'utilizzo di questo.

La Sottosezione \ref{lazy-eval-cl} riporta quindi come sia possibile introdurre
un supporto a questo modello di valutazione nel contesto del linguaggio di
programmazione Common Lisp, discute le principali problematiche che si
incontrano durante un'operazione di questo tipo e le soluzioni presenti nella
letteratura rispetto relativamente a questi problemi.

Infine, si riporta come la libreria CLAST possa essere utilizzata per rendere
l'utilizzo di un modello di valutazione lazy in Common Lisp sostanzialmente
indistinguibile, dal punto di vista sintattico, dall'utilizzo del modello
originale adottato dal linguaggio stesso, attraverso la definizione di
operazioni di riscrittura automatica del codice.

\subsection{Lazy Evaluation}

Ciascun linguaggio di programmazione, funzionale o meno, può essere classificato
o come linguaggio strict o come linguaggio lazy, in base alla modalità di
valutazione che adotta.

La fase di passaggio dei parametri ad una funzione rappresenta il momento in cui
le differenze tra questi due modelli di valutazione risultano più evidenti:

\begin{itemize}

\item in un linguaggio che adotta un modello di valutazione strict, i parametri
di una funzione vengono valutati subito prima dell'esecuzione del corpo di
questa e tipicamente già in fase di invocazione.

\item In un linguaggio che adotta un modello di valutazione lazy, i parametri
vengono invece valutati \textit{on-demand}.

\end{itemize}

In un modello di valutazione lazy, i parametri vengono quindi inizialmente
passati al corpo della funzione come espressioni non valutate e vengono valutati
solamente nel momento in cui la computazione necessita realmente del risultato
della loro valutazione per poter proseguire. Questo significa anche che, se la
computazione non dovesse mai avere necessità del valore prodotto dalla
valutazione di una data un'espressione, questa espressione non verrà mai
effettivamente valutata.\\

Linguaggi di programmazione funzionali puri che adottano questo secondo modello
di valutazione molto spesso combinano a questo l'utilizzo di tecniche di
memoizzazione \cite{Geyer-Schulz:1989:MEM:379209.379211}. Nel contesto di un
linguaggio di questo tipo quindi, una volta che un dato parametro viene
valutato, il valore prodotto dalla valutazione viene memorizzato. Questo in
maniera tale che, nel caso in cui il programma richieda nuovamente la sua
valutazione dell'espressione associata al parametro, il sistema possa limitarsi
a restituire il valore precedentemente memorizzato piuttosto che operare la
stessa computazione nuovamente. Si tratta di un'ottimizzazione che in molti casi
può portare ad ottimi risultati, ma che può essere applicata in modo sicuro
solamente all'interno di un linguaggio puro, in cui la valutazione di una data
espressione produce sempre lo stesso risultato.\\

Per meglio illustrare la differenza tra i modelli di valutazione appena
descritti, il Listato \ref{lst:si} mostra un frammento di codice il cui
comportamento varia in modo significativo nel passaggio dall'utilizzo di un
modello all'altro.

\begin{lstlisting}[
  caption=Codice di esempio,
  label={lst:si}
]

(defun si (condicio ergo alternatio)
  (if condition
      ergo
      alternatio))

\end{lstlisting}

Se si considera l'invocazione \texttt{(si t 9 (loop))} della funzione riportata
come esempio dal Listato \ref{lst:si}, relativamente ad un linguaggio che adotta
un modello di valutazione strict, si osserva che l'invocazione produce un loop
infinito. Prima dell'esecuzione della funzione infatti, il linguaggio opera una
valutazione dei parametri forniti a questa e, in particolare, la valutazione
dell'espressione \texttt{(loop)}. La valutazione di questa espressione causa
quindi la non terminazione del programma. Se si considera invece la medesima
invocazione all'interno di un linguaggio che adotta un modello di valutazione
lazy, si osserva invece che l'esecuzione termina, producendo come output il
valore \texttt{9}. Questo perché i parametri forniti in input alla funzione non
vengono valutati fino al momento in cui il valore associato a questo risulta
realmente necessario all'esecuzione ed, essendo che il terzo parametro non
risulta mai effettivamente necessario a questa, non sia avrà mai una valutazione
del parametro \texttt{(loop)} né quindi il generarsi di un loop infinito.\\

Il vantaggio fondamentale introdotto dalla disponibilità di un supporto alla lazy
evaluation è principalmente legato a questa ottimizzazione. A sua volta, la
presenza di questa ottimizzazione introduce però nuove possibilità, come ad
esempio la possibilità di rappresentare strutture dati infinite o circolari e
operazioni su sequenze di valori particolarmente lunghe in maniera
significativamente più efficiente rispetto a quanto si otterrebbe con un
linguaggio che adotta un modello di valutazione strict.

Lo svantaggio fondamentale del modello di valutazione lazy è rappresentato dal
fatto che, a differenza di quanto avviene nel contesto del modello di
valutazione strict, risulta particolarmente complesso ragionare rispetto alle
performance di un dato programma. Nei linguaggi che utilizzano un modello
strict, ciascuna sotto-espressione viene valutata nel momento stesso in cui
questa compare a livello sintattico. Questo semplifica in maniera sostanziale il
processo di ragionamento rispetto alle risorse di tempo e spazio utilizzate da
un programma di questo tipo in un dato istante. Nel contesto di un linguaggio
che utilizza un modello di valutazione lazy, anche utenti esperti del linguaggio
possono avere difficoltà nel predire quando e se una data sotto-espressione
verrà effettivamente valutata.

Entrambi i modelli di valutazione hanno quindi i propri vantaggi e svantaggi.
Per questa ragione, un linguaggio di programmazione ideale dovrebbe consentire
ad un utente un supporto per l'utilizzo di entrambi questi modelli, senza
introdurre overhead significativi dal punto di vista sintattico e della
complessità di un sistema, consentendo quindi ad un utente di scegliere quale
modello adottare in relazione alle necessità del sistema in sviluppo.\\

Linguaggi di programmazione tradizionali, come ad esempio il linguaggio C e il
linguaggio Java, adottano un modello di valutazione strict, e non prevedono, a
livello di linguaggio stesso, un supporto alla definizione di meccanismi che
consentano di influenzare il processo di valutazione in maniera tale da ottenere
un comportamento lazy.

Esempi di linguaggi di programmazione che adottano un modello lazy come default
sono rappresentati dal linguaggio Haskell \cite{DBLP:conf/hopl/HudakHJW07}, dal
linguaggio Miranda \cite{DBLP:journals/eatcs/Turner87} e dalla variante del
linguaggio ML chiamata LML \cite{DBLP:conf/lfp/Augustsson84}.

Altri linguaggi invece utilizzano un modello di valutazione strict come default
ma consentono all'utente di specificare che un particolare frammento di codice
debba essere eseguito utilizzando secondo il modello lazy attraverso l'utilizzo
di una sintassi speciale. Esempi di linguaggi appartenenti a quest'ultima
categoria sono il linguaggio Scheme, con i costrutti \texttt{delay} e
\texttt{force} e OCaml, con \texttt{lazy} e \texttt{Lazy.force}.

\subsection{Lazy Evalutation e Common Lisp}
\label{lazy-eval-cl}

Il linguaggio di programmazione Common Lisp fa parte della categoria dei
linguaggi che adottano un modello di valutazione strict e non forniscono alcun
costrutto che consenta l'utilizzo del modello lazy.

Esistono tuttavia diverse strategie che possono essere impiegate allo scopo di
introdurre un supporto all'utilizzo di un modello di valutazione lazy nel
contesto di un linguaggio di programmazione.

Tra queste strategie, la più nota e più adatta al contesto del linguaggio Common
Lisp è probabilmente quella illustrata da Abelson et al. in \cite{Abelson1996} e
parallelamente da Okasaki in \cite{DBLP:conf/afp/Okasaki96}, basata sul concetto
di stream e sulle primitive \texttt{delay} e \texttt{force}.\\

La primitiva \texttt{delay} genera a partire da una data espressione una
funzione che può essere invocata ottenendo come risultato il valore prodotto
dalla valutazione dell'espressione. Nel contesto dei linguaggi funzionali, la
funzione prodotta dalla primitiva \texttt{delay} viene spesso chiamata
\textit{thunk}. Il Listato \ref{lst:delay} mostra un'implementazione della
primitiva \texttt{delay} in Common Lisp.

\begin{lstlisting}[
  caption=Implementazione della primitiva delay in Common Lisp,
  label={lst:delay}
]

(defmacro delay (expr) `(lambda () ,expr))

\end{lstlisting}

La primitiva \texttt{force} consente, dato il risultato di una funzione prodotta
utilizzando la primitiva \texttt{delay} ossia un thunk, di ottenere il risultato
della valutazione dell'espressione a partire dalla quale questa è stata creata.

\begin{lstlisting}[
  caption=Implementazione della primitiva force in Common Lisp,
  label={lst:force}
]

(defun force (thunk) (funcall thunk))

\end{lstlisting}

Il limite di un sistema all'interno del quale un modello di valutazione lazy
viene implementato solamente a partire da queste due primitive è rappresentato
dal fatto che la definizione di una qualsiasi funzione, che adotta il modello
lazy, prevederà un utilizzo particolarmente intenso di questi due costrutti. Ad
esempio, l'operatore discusso dalla precedente sottosezione e presentato dal
Listato \ref{lst:si} potrebbe essere implementato in modo lazy come mostrato dal
Listato \ref{lst:si-delay-force}, combinando il meccanismo delle macro Common
Lisp alle due primitive.

\begin{lstlisting}[
  caption=Implementazione della primitiva force in Common Lisp,
  label={lst:si-delay-force}
]

(defmacro si (condicio ergo alternatio)
       `(if (force ,condition)
            (force (delay ,ergo))
            (force (delay ,alternatio))))

\end{lstlisting}

Si può osservare che, nonostante la semplicità dell'esempio, l'implementazione
dell'operatore risulta quasi nascosta dall'utilizzo delle due primitive. Mentre
l'implementazione originale dell'operatore utilizzava 67 caratteri, la nuova
implementazione ne richiede 115.

Oltre a questo aspetto relativo alla leggibilità e manutenibilità del codice,
un'altra problematica particolarmente significativa è rappresentato dal costante
overhead introdotto alla fase di sviluppo del codice vero e proprio. Per un
utente del linguaggio diventa infatti importante tenere sempre presente quando e
dove sia necessario ritardare l'esecuzione di ciascun elemento, un aspetto
significativo soprattutto all'aumentare della complessità.

Una terza problematica è invece rappresentata dal fatto che, avendo utilizzato
il costrutto macro, il prodotto della definizione non è una funzione. Questo
significa che l'operatore \texttt{SI}, come qualsiasi altro operatore definito
in maniera analoga, non potrà essere utilizzato come una funzione e quindi
utilizzato in combinazione a funzioni come \texttt{FUNCALL}, \texttt{MAP} e in
generale qualsiasi funzione di ordine superiore; aspetto che limita in maniera
significativa l'utilità di questa definizione.\\

Per risolvere questi due problemi è possibile introdurre un nuovo costrutto,
discusso nel dettaglio nei prossimi paragrafi.

\subsubsection{DEFLAZY}

Data la definizione di una qualsiasi funzione è sempre possibile produrre una
nuova versione di questa, del tutto equivalente ma che utilizza un modello di
valutazione lazy, attraverso l’utilizzo di una macro Common Lisp. Una macro di
questo tipo dovrebbe:

\begin{enumerate}

\item utilizzare la definizione di funzione fornita dall'utente per definire la
normale versione strict della funzione all'interno dell'ambiente;

\item generare un thunk per ciascuna parametro della funzione;

\item generare una versione della funzione fornita che lavora secondo il modello
lazy a partire dai thunk prodotti per ciascun parametro e dal corpo della
versione originale della funzione.

\end{enumerate}

Un'implementazione della macro appena descritta viene fornita dalla libreria
CLAZY \cite{DBLP:journals/corr/Antoniotti14} e viene riportata nel listato
\ref{lst:deflazy}.

\begin{lstlisting}[
  caption=Definizione della macro \texttt{DEFLAZY},
  label={lst:deflazy}
]

(defmacro deflazy (name args &body body)
  "Defines a function while ensuring that a lazy version exists as well."
  (declare (type cons args body)
           (type symbol name))
  (multiple-value-bind (renamed-arglist _ vars-thunk-names)
      (rename-lambda-vars args)
    (declare (type list renamed-arglist)
             (ignore _))
    (let ((new-name (lazy-name name)))
      (declare (type symbol new-name))
      `(progn
         (defun ,new-name ,renamed-arglist
           (symbol-macrolet ,(mapcar #'create-var-thunk-call-expansion
                                     vars-thunk-names)
             ;; (format t "Calling the lazy version...~%")
             ,@body))
         (setf (get-lazy-version ',name) (function ,new-name))
         (cl:defun ,name ,args ,@body)
         ))))

\end{lstlisting}

Questo primo costrutto, chiamato \texttt{DEFLAZY}, risulta essere una soluzione
ai problemi descritti in precedenza.

\begin{itemize}

\item Consente di mantenere uno stile di definizione del tutto analogo a quello
che si utilizzerebbe per la definizione di una funzione secondo il modello di
valutazione strict.

\item Produce funzioni vere e proprie e che pertanto possono essere utilizzate
come parametri in qualsiasi ambito del linguaggio lo richieda.

\item Contente di ottenere due definizioni della funzione, una per ciascun
modello di valutazione.

\end{itemize}

\begin{lstlisting}[
  caption=Esempio di utilizzo della macro \texttt{DEFLAZY},
  label={lst:deflazy-use}
]

(deflazy si (condicio ergo alternatio)
       (if condicio ergo alternatio))

\end{lstlisting}

L'avere a disposizione di entrambe le versioni di una stessa funzione permette
ad un utente di poter scegliere il modello di esecuzione più adatto a seconda
dallo specifico contesto in analisi.\\

La libreria affronta quindi il problema delle definizione di una funzione lazy a
partire dalla soluzione appena presentata. Dal punto di vista dell'invocazione
di una funzione, la scelta operata dalla libreria CLAZY è invece la seguente.

Il nome specificato dall'utente in fase di definizione tramite \texttt{DEFLAZY},
viene utilizzato dalla libreria per identificare la versione strict della
funzione. Questo consente infatti di mantenere un comportamento di default di
una chiamata a funzione del tutto analogo a quello che si avrebbe operando una
normale definizione della funzione, rimanendo quindi in linea con il resto del
linguaggio di programmazione.

Per quanto riguarda invece la versione lazy della funzione, come mostrato dal
Listato \ref{lst:deflazy}, questa non viene invece definita all'interno
dell'ambiente utente ma piuttosto all'interno di un ambiente specifico alla
libreria. La libreria prevede quindi che un'utente utilizzi un particolare
operatore, chiamato \texttt{LAZY:CALL}, in fase di chiamata per indicare la
volontà di utilizzare la versione lazy della funzione. In risposta all'utilizzo
di questo operatore, la libreria estrae la versione lazy della funzione definita
all'interno del proprio ambiente ed riflette la chiamata operata dall'utente su
di questa.

Queste due modalità di invocazione vengono presentate dal Listato
\ref{lst:strict-lazy-calls}.

\begin{lstlisting}[
  caption=Confronto tra invocazione delle versione strict e lazy di una funzione,
  label={lst:strict-lazy-calls}
]

;; Strict invocation
(si 42 (loop))

;; Lazy invocation
(lazy:call 'si 42 (loop))

\end{lstlisting}

\subsection{Lazy Evalutation e CLAST}

La soluzione realizzata dalla libreria CLAZY risulta particolarmente appropriata
al contesto del linguaggio Common Lisp ma risulta ancora soggetta ad un
problema.

Tipicamente, nel caso in cui si voglia adottare un modello di valutazione di
tipo lazy, si vuole che questo venga utilizzato all'interno di un intero modulo
e non semplicemente nel contesto di un numero ristretto di chiamate a funzione.

Infatti, combinare utilizzo di entrambi i modelli all'interno della stessa
porzione di sistema può portare a risultati inattesi, sia in termini di
performance che in termini correttezza. Questo specialmente nel contesto di un
linguaggio non puro, come il Common Lisp, in cui è possibile che una funzione
produca cambiamenti di stato del un sistema complessivo come effetto
collaterale.

Tipicamente quello che si desidera è quindi avere determinate parti all'interno
di un sistema in cui il modello di valutazione adottato è un modello lazy, e
altre parti in cui si adotta un modello di valutazione strict.

In un contesto come quello appena descritto, quello che si ottiene utilizzando
la soluzione descritta nel corso della precedente sezione, è che in alcuni
moduli del sistema, quelli che adottano il modello di valutazione lazy, ogni
chiamata a funzione sarà mediata dall'utilizzo del costrutto \texttt{LAZY:CALL},
attraverso il quale l'utente specifica ogni volta la volontà di applicare la
definizione lazy di una data funzione.

Questo porta ad un problema analogo a quanto presentato dal Listato
\ref{lst:si-delay-force}, in cui si aveva una situazione in cui il codice che
consente l’utilizzo di un modello di valutazione lazy nasconde la semantica vera
e propria del programma e introduce un pesante overhead all'attività di
sviluppo, come viene mostrato dall'esempio riportato nel Listato
\ref{lst:messy-lazy-call}.

Anche in questo caso il problema potrebbe essere risolto automatizzando la
sostituzione delle chiamate a funzioni lazy attraverso un processo di
riscrittura del codice sorgente della data porzione di programma per la quale
l'utente del linguaggio desidera adottare un modello di valutazione lazy.

I passi previsti da questo processo di riscrittura automatico sarebbero i
seguenti:

\begin{enumerate}

\item identificare tutte le chiamate a funzione presenti all'interno di un dato
frammento di codice

\item identificare per quali di queste funzioni è disponibile una definizione
lazy;

\item Operare un riscrittura di ciascuna delle chiamate a funzione identificate
al passo precedente utilizzando il costrutto \texttt{LAZY:CALL}.

\end{enumerate}

Come riportato, risulta necessario identificare l'elenco di tutte le chiamate a
funzione all'interno di un frammento di codice, operazione che, prima della
definizione della libreria CLAST, risultava impossibile a causa della mancaza di
uno strumento in grado di compiere analisi di codice sorgente scritto
utilizzando il linguaggio Common Lisp.

Come presentato nel corso capitoli precedenti, la libreria CLAST è però in grado
di compiere questa operazione di ricerca in modo particolarmente semplice ed
efficiente. La libreria consente quindi la definizione del processo di
riscrittura appena descritto attraverso la definizione di una macro, risolvendo
il problema discusso nel corso di questa sottosezione.\\

\begin{lstlisting}[
  caption=Confronto tra definizione di funzione con chiamate strict e con chiamate lazy,
  label={lst:messy-lazy-call}
]

(defun mean-confidence-interval (data confidence)
  (let* ((array  (fill-array data 1))
   (count          (length data))
   (mean           (mean data))
   (standard-error (standard-error array))
   (percent-point  (percent-point (/ (1+ confidence) 2.0)
          (- count 1)))
   (h              (* standard-error percent-point)))
    (values mean (- mean h) (+ mean h))))

(deflazy mean-confidence-interval (data confidence)
  (let* ((initial-array  (lazy:call 'fill-array data 1))
   (count          (lazy:call 'length a))
   (mean           (lazy:call 'mean a))
   (standard-error (lazy:call 'standard-error initial-array))
   (percent-point  (lazy:call 'percent-point
            (/ (lazy:call 'plus-one
              confidence)
               2.0)
            (lazy:call 'subtract count 1)))
   (h              (lazy:call 'multiply
            standard-error
            percent-point))
   )
    (values mean
      (lazy:call 'subtract mean h)
      (lazy:call 'sum mean h))))

\end{lstlisting}

Questo da quindi ad un utente del linguaggio di programmazione Common Lisp la
possibilità di adottare un modello di valutazione lazy per un certa porzione di
codice, in maniera particolarmente semplice e nativa rispetto al linguaggio di
programmazione, come mostrato nel Listato \ref{lst:lazily}.

In questa sezione si è quindi come la libreria CLAST possa essere utilizzata
come strumento per definire un estensione nativa del linguaggio Common Lisp,
estensione che consenta ad un utente di scegliere ed utilizzare in maniera
semplice e naturale rispetto al linguaggio stesso il modello di valutazione lazy
durante la definizione del proprio sistema.

\begin{lstlisting}[
  caption=Modalità di utilizzo della lazy evalutation abilitata da CLAST,
  label={lst:lazily}
]

(lazily
 (defun mean-confidence-interval (data confidence)
   (let* ((initial-array  (fill-array data 1))
    (count          (length a))
    (mean           (mean a))
    (standard-error (standard-error initial-array))
    (percent-point  (percent-point (/ (1+ confidence) 2.0)
           (- count 1)))
    (h              (* standard-error percent-point)))
     (values mean (- mean h) (+ mean h)))))

\end{lstlisting}
